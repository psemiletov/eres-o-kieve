\bibitem{akty}
\emph{Акты, относящиеся к истории Южной и Западной России}. Санкт-Петербург, 19 век. 

\bibitem{akty02}
\emph{Акты Юго-Западной России}. Киев, 19 век. 

\bibitem{akty03}
\emph{Акты, относящиеся к истории Западной России}. Санкт-Петербург, 19 век. 

\bibitem{antonovich01}
\emph{Археологическая карта Киевской губернии}. В. Б. Антонович. Москва, 1895. 

\bibitem{gulyaev01}
\emph{Былины и песни Алтая из собрания С. И. Гуляева}. Барнаул, Алтайское книжное издание, 1988.

\bibitem{perevalova01}
\emph{Войны и миграции северных хантов (по материалам фольклора)}. Е. В. Перевалова.  Уральский исторический вестник. №8 (Древние и средневековые культуры Урала в евразийском культурном пространстве). – Екатеринбург: «Академкнига», 2002. С. 36-58. 

\bibitem{vostistnovosel}
\emph{Восточные источники о восточных славянах и Руси VI - IX вв.}. А. П. Новосельцев. 1965.

\bibitem{konovalova01}
\emph{Восточная Европа в сочинениях арабских географов XIII-XIV вв}, Коновалова И. Г. Москва,  2009.

\bibitem{strabon01}
\emph{География}. Страбон. Москва, 1879.

\bibitem{boplan01}
\emph{Гийом Левассер-де-Боплан и его историко-географические труды относительно Южной России}. Перевод В. Г. Ляскоронского. Киев. 1901. 

\bibitem{gramoty01}
\emph{Грамоты великих князей Литовских с 1390 по 1569 год}. Под редакцией Владимира Антоновича и Константина Козловского. Киев, 1868.

\bibitem{fotiy01}
\emph{Две беседы святейшего патриарха константинопольского Фотия по случаю нашествия россов на Константинополь}. Е. Ловягин. Журнал «Христианское чтение, издаваемое при Санкт-петербургской Духовной Академии». Санкт-Петербург, 1882.

\bibitem{maxdnepr01}
\emph{Днепр и его бассейн}. Н. И. Максимович. Киев, 1901.

\bibitem{hvoyka02}
\emph{Древние обитатели Среднего Приднепровья и их культура в доисторические времена (по раскопкам)}. В. В. Хвойка. Киев, 1913.

\bibitem{beylisdelo}
\emph{Дело Бейлиса, стенографический отчет}. В трех томах. Киев, 1913.

\bibitem{mihdocs}
\emph{Документальна спадщина
Свято-михайлівського Золотоверхого
монастиря у Києві ХVІ–ХVІІІ ст.
з фондів Національної бібліотеки
України імені В. І. ВЕРНАДСЬКОГО}. Киев, 2011.

\bibitem{karger01}
\emph{Древний Киев}. М. К. Каргер. Москва – Ленинград, 1958.

\bibitem{procopius02}
\emph{Древние славяне в отрывках греко-римских и византийских писателей по VII в. н. э.}. Вестник древней истории. 1941. № 1, стр. 230.

\bibitem{ivanov01}
\emph{Жизнь и поверья крестьян Купянского уезда, Харьковской губернии}, П. В. Иванов. XV Сборник Харьковского Историко-Филологического общества. Харьков, 1907.

\bibitem{zavit01}
\emph{Замок князя Семена Олельковича и летописный Городец под Киевом}. Завитневич В. З. // Чтения в историческом обществе Нестора Летописца. – К., 1891. – Кн. 5. – Отд. 2. – С. 134-141.

\bibitem{herber}
\emph{Записки о Московии}. Сигизмунд Герберштейн. Переводы А. И. Малеина, А. В. Назаренко. Москва, 2008.

\bibitem{zapvas01}
\emph{Записки чина св. Василія Великого}, Жовква, 1924.

\bibitem{gruneveg}
\emph{Записки от торговой поездке в Москву в 1584-1585 гг.}. Мартин Груневег (отец Венцеслав), духовник Марины Мнишек. Перевод Хорошкевича А. Л. Москва, Памятники исторической мысли, 2013.

\bibitem{kamanin01}
\emph{Зверинецкие пещеры в Киеве}. Каманин И. Киев, 1914.

\bibitem{zeleninrusalki}
\emph{Избранные труды. Очерки русской мифологии: Умершие неестественною смертью и русалки}. Зеленин Д. К. Индрик, Москва, 1995.

\bibitem{skifizv}
\emph{Известия древних писателей греческих и латинских о Скифии и Кавказе}. Собрал и издал В. В. Латышев. Санкт-Петербург, 1900-е.

\bibitem{grinchenko01}
\emph{Из уст народа. Малорусские рассказы, сказки и пр.}. Б. Д. Гринченко. Чернигов, 1901.

\bibitem{onuchkov01}
\emph{Из Уральского фольклора}. Н. Е. Ончуков, сборник «Сказочная комиссия в 1927 г. Обзор работ». Издание Государственного Русского Географического общества. Ленинград. 1928.

\bibitem{korabl01}
\emph{Из рассказов о древнеисландском колдовстве и Сокрытом Народе}. Леонид Кораблев. Москва, «София», 2003.

\bibitem{izbornik01}
\emph{Изборник (Сборник произведений литературы Древней Руси)}. Москва, Художественная литература, 1969.

\bibitem{izbornik02}
\emph{Изборник славянских и русских сочинений и статей, внесенных в хронографы русской редакции. Собрал и издал Андрей Попов}. Москва, 1869.

\bibitem{jackson01}
\emph{Исландские королевские саги о Восточной Европе (с древнейших времен до 1000 г.)}. Джаксон Т. Н. 1993.

\bibitem{diakon01}
\emph{История}. Лев Диакон. Перевод М. Копыленко. Москва, 1988.

\bibitem{herodotus01}
\emph{История}. Геродот. Перевод Ф. Г. Мищенко. Москва, 1888.

\bibitem{tatishev01}
\emph{История Российская}. Василий Татищев. Москва, 1768.

\bibitem{petrov01}
\emph{Историко-топографические очерки древнего Киева}. Н. Петров. 1897.

\bibitem{histmatkiev}
\emph{Исторические материалы из архива Киевского губернского правления}. Андреевский А. Киев, 1880-е.

\bibitem{kazanletop}
\emph{Казанский летописец. Полное собрание русских летописей, том 19}. 1903.

\bibitem{anuchin01}
\emph{К истории искусств и верований у приуральской чуди, чудския изображения летящих птиц и мифических существ}. Д. Анучин. 1899.

\bibitem{max}
\emph{Киев явился градом великим}. М. О. Максимович, Киев, Лыбедь, 1994.


\bibitem{teodorov01}
\emph{Кирил и Методи}. А. Теодоров-Балан. София, 1920.

\bibitem{sinopsis}
\emph{Киевский Синопсис, или краткое собрание от различных летописей о начале славенороссийского народа}. Под редакцией Гизеля. 1836.

\bibitem{bilbasov}
\emph{Кирилл и Мефодий по документальным источникам, том 1}. В. А. Бильбасов. СПб., 1868 

\bibitem{fadlan01}
\emph{Книга Ахмеда ибн-Фадлана о его путешествии на Волгу в 921-922 гг.}. А. П. Ковалевский. Москва, 1956.

\bibitem{chert}
\emph{Книга большому чертежу}. под редакцией К. Н. Сербиной, Ленинград, 1950.

\bibitem{amartol01}
\emph{Книги временные и образные Георгия Мниха. Хроника Георгия Амартола}. Истрин В. М. Петроград, 1920.

\bibitem{berl01}
\emph{Краткое описание Киева}. Максим Берлинский. Санкт-Петербург, 1820.

\bibitem{sofonovich01}
\emph{Кройника з летописцев стародавних [...]}. Софонович Ф. Киев, 1992.

\bibitem{snorry01}
\emph{Круг Земной}. Снорри Стурлусон. Москва, 1980.

\bibitem{zaharch01}
\emph{Киев теперь и прежде}. Захарченко М. Киев, 1888.

\bibitem{sement01}
\emph{Киев, его святыня, древности, достопримечательности (7-ое издание)}. Сементовский Николай. Киев, 1900.

\bibitem{kieparhprib01}
\emph{Киевские епархиальные ведомости. 1861. Прибавления}. 1861.

\bibitem{sitkareva01}
\emph{Киевская крепость XVIII0XIX вв.}. Ситкарева О. В. Киев, 1997.

\bibitem{muhin01}
\emph{Киево-Братский училищный монастырь}. Мухин Н. Киев, 1893.

\bibitem{arhsved01}
\emph{Краткие археологические сведения о предках славян и руси}. И. А. Хойновский. Киев, 1896.

\bibitem{zinoviev01}
\emph{Мифологические рассказы русского населения Восточной Сибири}. Зиновьев В. П. Новосибирск, 1987.

\bibitem{pohilmon}
\emph{Монастыри и церкви Киева}. Похилевич Л. Киев, 1865.

\bibitem{olbia01}
\emph{Надписи Ольвии (1917-1965)}. Ленинград, 1968. 

\bibitem{rudskazki}
\emph{Народныя южнорусские сказки}. Рудченко И. Киев, 1869-1870.

\bibitem{narty01}
\emph{Нарты. Ирон адамы героикон эпос}. Москва, 1990.

\bibitem{nazarenco01}
\emph{Немецкие латиноязычные источники IX–XI веков}. Назаренко А. В. Москва, 1993.

\bibitem{nizami01}
\emph{Низами Гянджеви. Собрание сочинений в 5 томах}. Москва, 1985.

\bibitem{kbagr01}
\emph{Об управлении империей.}. Константин Багрянородный. Москва, Наука, 1991, серия «Древнейшие источники по истории народов СССР».

\bibitem{litvin}
\emph{О нравах татар, литовцев и москвитян}. Михалон Литвин. Перевод В.И. Матузовой. Москва, 1994.

\bibitem{ochernignamest}
\emph{О Черниговском наместничестве. Черниговского наместничества топографическое описание, сочиненное Афанасием Шафонским в Чернигове, 1786 года}. Киев, 1851.

\bibitem{gilder01}
\emph{Онежские былины, записанные А. Ф. Гилдерфингом лето 1871 года}. Санкт-Петербург, 1873.

\bibitem{zakr01}
\emph{Описание Киева}. Закревский Николай. Москва, 1868.

\bibitem{sofiasobor01}
\emph{Описание Киево-Софиевского собора и Киевской Епархии}. Болховитинов Евгений. Киев, 1825.

\bibitem{bolhlavr}
\emph{Описание Киевопечерской Лавры}. Болховитинов Евгений. 1847.

\bibitem{polevoy01}
\emph{Очерки русской истории в памятниках быта}. Полевой П. Санкт-Петербург, 1880.

\bibitem{rybnikov01}
\emph{Песни, собранные П.Н. Рыбниковым}. Рыбников П. Москва, 1861.

\bibitem{tolstaya01}
\emph{Полесский народный календарь}. Толстая С. М. Индрик, Москва, 2005.

\bibitem{gilyarov01}
\emph{Предания Русской Начальной летописи (по 969 год). Приложения}. Гиляров Ф. Москва, 1878.

\bibitem{sharl01}
\emph{Природа и люди Киевской Руси}. Шарлемань Н. В. Киев, 2014.

\bibitem{antpublect01}
\emph{Публичные лекции по геологии и истории Киева, читанные профессорами П. Я. Армашевским и Вл. Б. Антоновичем в Историческом обществе Нестора-Летописца в марте 1896 г.}. Киев, 1897.

\bibitem{demiranda}
\emph{Путешествие по Российской Империи}. Франциск де Миранда. Пер. с исп. Москва, Наука, Интерпериодика, 2001.

\bibitem{karpini}
\emph{Путешествие в восточные страны Плано де Карпини и Рубрука}. Москва, 1957.

\bibitem{yakushin01}
\emph{Путевыя письма из Новогородской и Псковской Губерний}. Павел Якушин. Санктпетербург, 1860.

\bibitem{sbornikletug}
\emph{Сборник летописей, относящихся к истории Южной и Западной Руси, изданный Комиссией для разбора древних актов, состоящей при Киевском , Подольском и Волынском генерал-губернаторе}. 1888.

\bibitem{pohilskaz}
\emph{Сказания о населенных местностях Киевской губернии}. Л. Похилевич. Киев, 1864.

\bibitem{slavhron}
\emph{Славянские хроники: Деяния Архиепископов Гамбургской церкви, Славянская Хроника, Славянская хроника}. Адам Бременский, Гельмольд из Босау, Арнольд Любекский. Москва, 2011. 

\bibitem{lasota}
\emph{Путевые записки}. Эрих Ласота. Одесса, 1873.

\bibitem{mikorsky01}
\emph{Разрушение культурно-исторических памятников в Киеве в 1934-1936 годах}. Мюнхен, 1951.

\bibitem{sbornikmat}
\emph{Сборник материалов для исторической топографии Киева и его окрестностей, изданный Комиссией для разбора древних актов, состоящей при Киевском , Подольском и Волынском генерал-губернаторе}. Киев, 1874.

\bibitem{saharov}
\emph{Сказания русского народа, собранные И. П. Сахаровым}. Москва, 1990 (по изданию 1885 года).

\bibitem{hrabr}
\emph{Сказание черноризца Храбра о письменах славянских}. С. Г. Вилинский. Одесса, 1901.

\bibitem{afanchujb}
\emph{Собрание сочинений Афанасьева (Чужбинского)}. Санкт-Петербург, 1893.

\bibitem{fundstat}
\emph{Статистическое описание Киевской губернии,  изданное Иваном Фундуклеем, в трех частях}. Санкт-Петербург, 1853. 

\bibitem{ivancov}
\emph{Стародавній Київ}. Иванцов И. О. Киев, Феникс, 2003.

\bibitem{telegin01}
\emph{Там, где вырос Киев}.
Телегин Д. Я. Киев, Наукова думка, 1982.

\bibitem{novickiy01}
\emph{Твори у 5 томах}. Яків Новицький. Запорожье, Тандем-У, 2007.

\bibitem{titmar01}
\emph{Титмар Мерзебургский. Хроника}. Москва, Русская панорама, 2009. 

\bibitem{troyaskaz}
\emph{Троянские сказания}. Ленинград, Наука, 1972. 

\bibitem{trudy-chub}
\emph{Труды этнографическо-статистической экспедиции в Западно-русский край, снаряженной Императорским Русским Географическим обществом. Юго-Западный отдел. Материалы и исследования, собранные П. П. Чубинским}. 19 век.

\bibitem{trudy-aknauk-otdel-drevsuslit}
\emph{Труды Отдела древне-русской литературы Института русской  литературы}. Академия наук СССР.

\bibitem{cherndoc01}
\emph{Чернігівській губернії – 210 років. Збірник документів і матеріалів}. Чернигов, 2012.

\bibitem{chtenianestora01}
\emph{Чтения в историческом обществе Нестора летописца. Книга 1, 1873-1877}. Киев, 1879.

\bibitem{pohyluezd}
\emph{Уезды Киевский и Радомысльский}. Леонтий Похилевич. Киев, 1887.

\bibitem{vozn01}
\emph{Українські перекази}, М. Возняк, Львов, Краков, 1944.  

\bibitem{petrdusburg}
\emph{Хроника земли Прусской}. Петр из Дусбурга. Москва, 1997.

\bibitem{malala01}
\emph{Хроника Иоанна Малалы в славянском переводе}. Истрин В. М. Москва, 1994.

\bibitem{etnoboz3}
\emph{Этнографическое обозрение}, книга III, Москва, 1889 год.

\bibitem{grinetnochern}
\emph{Этнографические материалы, собранные в Черниговской и соседних с ней губерниях}. Гринченко Борис Дмитриевич. Чернигов, 1895.

\bibitem{wilde01}
\emph{Ancient Legends, Mystic Charms, and Superstitions of Ireland}. Lady Wilde. Boston, 1887.

\bibitem{annals4mast}
\emph{Annals of The Kingdom of Ireland, by the Four Masters}. 7 volumes, Dublin, 1856.

\bibitem{amartol02}
\emph{Chronicon breve}. Georgius Monachus. 

\bibitem{cathmaigetuired}
\emph{Cath Maige Tuired, The Second Battle of Mag Tuired}. Translated by Elizabeth A. Gray.

\bibitem{procopius01}
\emph{Corpus scriptorum historie Byzantinae.Pars II. Procopis. Volumen II}. Bonnae, MDCCCXXXIII.

\bibitem{adambrem}
\emph{Gesta Hammaburgensis ecclesiae pontificum}. Adamus. Hannoverae, 1876. 

\bibitem{orb01}
\emph{Il regno de gli Slavi}. Mavro Orbini, M.DCI.

\bibitem{saxon01}
\emph{Journal of the British Astronomical Association, Vol. 115, No. 5, p.261}.

\bibitem{strykron}
\emph{Kronika Ploska, Litewska, Zmodska i Wszystkiej Rusi Macieja Stryjkowskiego}. Warszawa, 1846.

%\bibitem{hajek01}
%\emph{Pověsti o počátcích českého národu a o českých pohanských knížatech. Příprava vydání Jan Kočí}. Hájek z Libočan, Václav. Praha: Kočí, 1917.

\bibitem{mabinogion}
\emph{The Mabinogion}. Translation by Lady Charlotte Guest, 1877.

\bibitem{obs1880v3}
\emph{The Observatory, Vol. 3, p. 590-592 }, 11/1880.

\bibitem{siem01}
\emph{Podania a Legendy polskie, ruskie i litowskie}. Lucian Siemienski. Poznan, 1845.

\bibitem{polychro}
\emph{Polychronicon}. Randulphi Hidgen. English translation by John Trevisa and Unknown writer of 15-th century. London, 1865.

\bibitem{panoptikon}
\emph{Samuel Bentham's Panopticon}. Philip Steadman. Bartlett School (Faculty of the Built Environment), University College London.

\bibitem{leibn}
\emph{Accessiones Historica quibus potisfimum continentur scriptores rerum germanicarum}. Godefridi Guillielmi Leibnitii. Hanovere, Anno MDCC.

\bibitem{alepp02}
\emph{The travels of Macarius : Patriarch of Antioch}. Paul of Aleppo. London, Printed for the Oriental Translation Fund of Great-Britain and Ireland, 1836.

\bibitem{tafur01}
\emph{The Travels Of Pero Tafur (1435-1439)}. Pero Tafur. Translated and edited with an introduction by Malcolm Letts. New York, London: Harper and brothers, 1926.

\bibitem{titmar00}
\emph{Thietmari Merseburgensis
Episcopi Chronicon}. Hannoverae, Impensis Bibliopolii Hahniani, 1889.

\bibitem{tafur00}
\emph{Viajes de Pero Tafur por diversas partes del mundo avidos. 1435-1439}. Madrid, 1874.
