\chapter{Стол на Угорьском}

Я должен отвлечься и порассуждать об этом, прежде чем продолжить про Аскольда и Дира. Заодно дополню сказанное в прошлой главе. Постоянно забываю, что описываемая местность знакома киевлянам, однако жители других городов могут плохо себе представлять, о чем идет речь.

Когда едешь на метро с левого берега на правый, от Гидропарка до станции «Днепр», то видишь как приближается могучий зеленый холм, вид коего испорчен уже высотными зданиями. Подножия этих круч над рекой изъедены кирпичными заводами, работавшими тут в тридцатых годах 19 столетия при строительстве Киевской крепости, окружившей Лавру звездообразным укреплением, совершенно бесполезным при обстреле артиллерией с высот Батыевой горы. Ведь достаточно подвести туда пушки, и ляжет всё в пределах досягаемости.

Наверху – Лавра, Парк Вечной славы и местность Аскольдова могила, под которой почему-то подразумевают обычно находящуюся там церквушку. А северная граница Лавры, у церкви Спаса на Берестовом\footnote{Археологи раскопали возле церкви остатки древнего поселения, датируют его 4 веком до нашей эры.} – судя по названию, накладывается на летописное урочище Берестовое (Берестово), где были известны сельцо Берестовое и княжий двор. Сохранившаяся по наши дни церковь Спаса, судя по плану в Тератургиме, восстановлена из древней при Петре Могиле.

Берестовое граничило с Варяжскими пещерами, при коих возник пещерный, Печерский монастырь, позже ставший наземной Лаврой. Пещеры Лавры делятся на Ближние и Дальние, Дальние смежны с Варяжскими – историки обходят вопрос использования Варяжских пещер монахами.

Варяжскими пещерами слывут закрытые (обоснование сотрудников Лавры – там опасно, всё может обвалиться) пещеры под нынешним Ландшафтным парком, в районе Певческого поля. Через вход, находящийся в самой Лавре, в западной части подземной Благовещенской церкви на Дальних пещерах, можно попасть в коридор длиной более ста метров, с пятью – а по другим данным, девятью – боковыми коридорами, кельей и нишами. Коридоры имеют два метра в высоту (кельи ниже, там только согнувшись можно) и довольно широки – «два человека могут разойтись».
% Связаны ли они с пещерой «у Цепного моста», я не знаю.

В урочище Берестовое можно включить, наверное, и парк Вечной славы. А северной стороной он выходит к Аскольдовой могиле. Гора же с нею в летописях называется Угорьским. По холму спускается Днепровский спуск, бывший Никольский (он начинался от Дворца Пионеров, где прежде стоял Никольский военный собор).

Таковы самые краткие сведения о местности, многократно искаженной фортификационными работами да в трактовках событий истории.

Представления эти грубо рублены топором. Вот, под Угорьским в давнее время останавливались шедшие военным походом Угры, потому так названо урочище. На Берестовом пресвитер Иларион ископал пещерку малую, и где-то около нее в пещере поселился затем святой Антоний, положивший начало пещерному монастырю. 

С Антонием вообще любопытно. Нам обычно говорят, что он – основатель древнейшего на Руси монастыря, Лавры. Верно, что основатель. Но древнейшего ли? 

Открываем летопись, начинаем читать.

Антоний, родился в Любече. Отправился в Царьград, на Афоне стал монахом. Был послан оттуда в Киев нести слово веры. И что же он делает, прибыв в Киев? Ходит по монастырям!

Ипатьевский список:

\begin{quotation}
Антоний же приде к Кыеву, и мышляше кде жити, и походи по манастырем, и не возлюби, Богу не хотящу, и поча ходити по дебрем и по горам, ища, кде бы ему Бог показал, и приде на холм, идеже бе Ларион печеру ископал, и взьлюби мьстьце се, и вселися в онь [...]
\end{quotation}

Значит, тут уже были монастыри, и Лавра вовсе не древнейший. Но Антонию не понравился ни один из монастырей. Поэтому Антоний удалился в пещеру, выкопанную Ларионом, и начинает ее копать дальше. За благословением приходили к нему люди, а после смерти Ярослава, в правление сына его Изяслава, постриг от Антония приняли 12 человек. Так, согласно Повести временных лет, было положено начало Печерскому монастырю, будущей Лавре.

В Патерике Печерском (а оттуда – в Житии Антония из Четий-Миней Димитрия Ростовского) история эта углублена в прошлое и расширена – Антоний пришел из Афона в Киев еще при Владимире, посещал тут монастыри, потом на Берестовом «обрете пещеру, юже некогда ископаша Варяги» и поселился в ней, затем при сыне Владимира, Святополке, снова отправился на Афон, а при Ярославе вернулся в Киев, и тогда уже обосновался в пещере Лариона\footnote{Тот же Патерик, в рассказе о святых преподобных отцах Феодоре и Василии (слово 33-е) сообщает, что монах Феодор долгие годы жил в Варяжской пещере, и во сне ему в виде ангела явился бес и указал место, где копать сокровище. Феодор взялся за дело и нашел сокровища – злато и сребро, сосуды многоценные. Незадолго до этого и после, к Феодору под видом другого монаха, Василия, тоже являлся бес и распалял в пещернике дух стяжательства и сребролюбия. 

И вот Феодор, собирался было уже, нагрузив добром ящики и возы, покинуть пещеру и монастырь, как пришел настоящий Василий и сказал, что впервые слышит, будто дух сребролюбия распалял и подзуживал уйти из монастыря.

Тогда Феодор куда-то закопал сокровища, а во всяком входящем к нему в пещеру подозревал двойника-беса, и заставлял на проверку молиться.

Однако бес, приняв привычный облик Василия, отправился к знакомому Василия, боярину, и поведал о сокровище Феодора. Боярин свёл Василия с князем Мстиславом Святополчичем.

Почуяв сребро, Мстислав с отрядом воинов отправился в монастырь и привез Феодора к себе в дом. Поначалу просто беседовал. Было ли сокровище? Коли было, давай поделимся. Феодор отвечал учёно, со ссылкой на житие святого Антония:

\begin{quotation}
В житии святого отца нашего Антония поведаеться Варяжьскыи поклажа есть, понеже съсуды Латиньскии суть; сего ради Варяжьская печера зоветься и доныне. Злата же и сребра бещисла много.
\end{quotation}

На вопрос же, где оно всё, сказал так:

\begin{quotation}
И ныне Господь взя память от мене сребролюбия и не вем, где скрых.
\end{quotation}

Пытки не вернули ему память. Князь послал за Василием, а тот стал говорить, что прежде с князем общался не он, Василий, а бес. 

Утром пытки обоих монахов должны были продолжиться, но Феодор и Василий ночью скончались в застенках. Пришли иноки и взяли тела их, и погребли в пещере Варяжской, одетыми в кровавых смертельных своих одеждах.}.

На Берестовом, как считают ученые, у князя Владимира было нечто вроде загородного дома. Княжий двор, но загородный. %А под церковью «Аскольдовой могилой», мол, был похоронен князь Аскольд, вероломно убитый Олегом.

Возникает много вопросов. Например, если около Берестового уже существовали Варяжские пещеры, зачем монахи копали новые? А насколько безопасно было там селиться? На последнее – безопасно, если местность не стояла на отшибе, а входила в городскую черту.

%На первый вопрос предположу, что Варяжские пещеры поначалу кем-то использовались и монахов туда эти «кто-то» не пускали, либо монахи сами боялись туда лезть.

Про княжий двор. С ним неразрывно связано понятие «стола» – престола, трона. Так же, как слово «трон», «стол» употребляется в прямом и переносном смысле.

Очевидно, что там, где княжий двор, там и престол. Но было бы странным, если таковой находился где-то на отшибе, вне городских стен, подставляясь под удар не то что любой армии, но даже разбойникам. 
 
Ученые говорят нам, что княжий двор Владимира был в сердце нынешнего Киева – пресловутом пятачке наверху около фуникулера, Михайловской площади, улицы Большой Житомирской. Они называют это «городом Владимира». Туда же помещают все княжеские дворы – Ольги, потом Владимира, и так до Ярослава – тот мол, построил «город Ярослава», расширив киевскую крепость, и правил уже изнутри этого «города».

А двору Владимира на Берестовом наука отводит роль загородного гарема – и там в самом деле таковой был. Однако на основе каких источников ученые полагают, будто существовал какой-то другой, «основной» двор Владимира в центре современного Киева? Что вообще говорят источники?

Давайте изучим их, памятуя, что Берестовое можно расценивать как местность наверху Угорьского, либо смежную с Угорьським местность, в непосредственной близости на юг и восток от Угорьского.

Ипатьевская летопись, за 6406 (898 – как толкует официальная хронология) год:

\begin{quotation}
Идоша Угре мимо Киев горою, еже ся зовет ныне Угорьское, и пришедше к Днепру, сташа вежами; беша бо ходяще яко и Половци.

И пришедше от въстока и устремишася черес горы великыя, иже прозвашася горы Угорьскыя, и почаша воевати на живущая ту. 

Седяху бо ту преже Словене, и Волохове переяша землю Словенскую; посем же Угре\footnote{Угре Белии – уточняется на другой странице летописи.} прогнаша Волохы, и наследиша землю ту, и седоша со Словеньми, покоривше я под ся, и оттоль прозвася земля Угорьска.

И начаша воевати Угоре на Грекы, и пополониша землю Фрачьскую и Македоньску доже и до Селуня, и начаша воевати на Мораву и на Чехы: бе бо един язык Словенеск. Словене же седяху по Дунаю, их же прияша Угре, и Морава, и Чеси, и Ляхове, и Поляне, яже ныне зовемая Русь.
\end{quotation}

Обсудим.

Угры шли мимо Киева горой. В Хлебниковском списке нет «горой», там просто – мимо Киева.

И добравшись к Днепру, стали, подобно Половцам, «вежами». В украинском языке «вежа» означает «башня», но в старину у этого слова были и другие значения. Откроем словарь Даля:

\begin{quotation}
ВЕЖА, вежа ж. стар. намет, шатер, палатка; кочевой шалаш, юрта, кибитка; | башня, батура, каланча; это значение осталось в зап. губ. | арх. шалаш, будка, сторожка, балаган, лопарский шалаш, сахарной головою, сложенный остроконечно из жердей и покрытый хворостом, мохом и дерном, чем и отличается от крытых шкурами или берестою чума и юрты; он одевается пластами дерна; юрта, кибитка, переносное, кочевое жилье. Татарские племена, а за ними и русские зовут кошемную кибитку свою так, как и не напишешь: ближе всего будет эй, или немецк. oei, дом. Иде в вежю и сварши зелие... стар. клеть? кухня? | Кур. вежка, полевой шалаш, балаган, сторожка. | Новг. межа, грань, рубеж?
\end{quotation} 

Тут кстати узнаем и любопытное – «Татарские племена, а за ними и русские зовут кошемную кибитку свою так, как и не напишешь: ближе всего будет эй». Наверняка речь идет об известном слове из трех букв! 

Не отнесемся к этому легкомысленно и запомним на будущее – пригодится. 

Итак, Угры разбили лагерь у Днепра, и – как понимаю я летописца, около горы, что потом назвалась Угорьским.

Но откуда пришли эти Угры, с какой стороны света?

\begin{quotation}
И пришедше от въстока
\end{quotation}

Ага, с востока. Это с левой стороны Днепра. А гора на правой.

Странно. Перебравшись где-то с левого берега на правый через Днепр, со своими лошадьми, может и верблюдами, а также походными кибитками, Угры, идя по правому берегу мимо Киева горой или не горой, спустились к Днепру и стали вежами. Как перебрались, вопрос интересный. Может была зима и по льду перешли, а может лето и через брод какой. Или на плотах. Либо мост навели. Я не знаю как в то время пересекали большую воду.

%Зачем? Переправиться снова на левый берег, под Угорьским, там был брод? Я не знаю. В части про Городки будет – предположение, что именно в том месте раньше находилось устье Десны, а возможно таки и брод.

А для чего разбивать лагерь под горой? Это невыгодно с военной точки зрения. А глядя на современный рельеф, не совсем ясно, где разместить там армию, разве что у правого берега около нынешнего моста Метро, на восток от крутой горы лежала суша, подобная Подолу. Там-то и могли расположиться Угры своими вежами.

Еще вопрос, зачем Уграм становиться вежами под какой-то горой, если там только березовая роща? А не княжий ли там двор, который надо взять осадой? И не двор ли это в пределах города?

Шествие Угров из Скифии описано в давнем латинском сочинении Анонима «Gesta Hungarorum»\footnote{Деяния Хунгаров, книга датируется 12-13 веками.}, отрывок из которого даю в вольном переводе. Угры в сочинении этом – Хунгары, как именовали их иноземцы, а самоназвание было Магары\footnote{Посему население нынешней Венгрии говорит о себе – Мадьяры, а на украинском Венгрия – Угорщина.}.

В 884 (D. CCCC. V. XXX. IIII.) году от рождества Христова семеро скифских вождей, называемые Хэтумогэр (Hetumoger), собрались переселиться со своими людьми из Скифии на запад. Среди вождей был князь Алмос (может быть просто «Алм», если имя было латинизировано), сын Угека, из рода короля Магога, вместе с женой и сыном Арпадом, и двумя сыновьями своего дяди Хулэка – Зуардом и Кадусой, и с великим множеством народу.

Они шли пустынными местами, переправились через Итиль (Волгу) на кожаных мешках (tulbou), и двигались тех пор, пока не достигли Русциама (Rusciam) который именовался Сусудалом (Susudal).

Страна, куда они прибыли, называется Анонимом Рутэнией (Rutenia). По Рутэнии Угры, не встречая сопротивления, добрались до Киэва (Kyeu, Kieu). Преодолели Дэнэпэр (Deneper) и прошли через Киэв, властвовавший над Рутэнами (в Рутэнах легко угадываются «русены»). Это место в повествовании неясно, поскольку, как вы прочтете далее, оказывается, что местные защитники Киева отступают внутрь его, под защиту стен.

Князья Рутэнов, узнав о вторжении, испугались, зная, что князь Алмос, сын Вгэка, был из роду короля Аттилы, которому их праотцы платили ежегодную дань.

Князь Хуэва (dux de hyeu) и все предводители собрались на совет и решили, что должны дать бой князю Алмосу, и лучше умереть в бою, чем покориться Алмосу. Князь Киева послал гонцов и попросил о помощи семь князей Куманов, своих самых преданных друзей.

Эти семь князей, имена их – Эд, Эдум, Эту, Бунгэр, Оусад, отец Урсуура, Бойта, и Кэтэл, отец Олуптулма (Ed, Edum, Etu, Bunger, Ousad, pater ursuur, Boyta, Ketel pater oluptulma) – вместе с конницей вскоре прибыли на подмогу и присоединились к войску киевского князя.

Князь Алмос произнес перед своим войском воодушевляющую речь, в которой назвал Рутэнов и Куманов «нашими псами», да привел исторические примеры, в которых Скифы побеждают короля Персов Дария, короля Персов Кира, и способны были драться даже с Александром Македонским.

Начинался бой между Скифами-Уграми с одной стороны, и Рутэнами и Куманами – с другой. Последние стали проигрывать и укрылись за стенами Киева. Воины Алмоса преследовали их до города, разбивая бритые налысо головы Куманов как свежие тыквы. Защитники города, запершись в нем, затихли.

Неделю Скифы Алмоса грабили окрестности, на вторую же неделю стали искать подступы к Киеву. Начали подводить к стене лестницы (et dum scalas ad murum). Это обеспокоило киевлян. Поняв, что не смогут сопротивляться, князья Рутэнов и Куманов отправили послов ко князю Алмосу и его старшинам заключить мир и просить не высылать князей киевских из мест своего обитания. 

Посовещавшись, Алмос дал ответ – чтобы киевские князья и их глава предоставили своих сыновей в заложники, а также платили ежегодную дань в 10000 марок (decem milia marcarum), а еще пищу, одежду и другие предметы дани.

Князья Рутэнов с неохотой согласились на эти условия, но сказали князю Алмоса, дабы тот, покинув землю Галиции (terra galicie), спускался дальше на запад через лес Хоуос (Houos)\footnote{Созвучно с Киевом и вариантом его названия на букву «х».} в землю Паннонии, которая раньше была землей короля Аттилы, и советовали ему эту землю как изобильную чрезмерно. Там протекают Данубий и Тисция (danubius et tyscia) и другие большие реки, полные хорошей рыбы. В тех краях живут Склавии, Булгарии и Блахии (sclauij, Bulgarij et Blachij) и пастухи Романов (pastores Romanorum). После смерти короля Аттилы, Романы (Римляне) заявили, что земли Паннонии были их пастбищами, потому что там паслись их (Романов) стада. Ну а теперь пусть на владениях Романов попасутся Хунгары. 

Князь Алмос и его старшины держали совет,  согласились с предложением киевских князей идти в Паннонию и заключили с ними мир.

Затем князья Рутэнов, а именно Киева и Суздали (er se consilio pete Kyeu, et Sudal), дабы не быть изгнанными, послали своих сыновей к Алмосу в заложники, а с ними 10000 марок в придачу и тысячу лошадей с седлами и уздечками, украшенными по-рутенски, сто мальчиков Куманов и 40 верблюдов для поклажи, а также бесчисленное количество шкур и прочего добра.

Затем князья Куманов присягнули на верность князю Алмосу и собираются следовать с ним. Перечислены такие предводители Куманов: Эд, Эдумен, Эту, Бунгэр отец Борсу, Оусад отец Врсууру, Бойта, который из рода потомства Бруксы, и Кэтэл отец Олуптулма (Ed, Edumen, Etu, Bunger, pater Borsu, Ousad pater Vrsuuru, Boyta, a quo genus Brucsa descendit, Ketel, pater Oluptulma).

Далее Куманы, а заодно часть Рутэнов отправились с Алмосом в Паннонию, и поныне (при жизни Анонима) в Венгрии живут их потомки. Алмоса проводили туда киевские Рутены. Алмос пришел в город Лодомер (в котором угадывается имя «Володимер»), и так 
далее – но сие уже Киева не касается. 

Возможно именно эти события отражены в киевской летописи как «идоша Угре мимо Киев горою». Сочинение Gesta Hungarorum, насколько мне известно, в качестве источника для истории Киевской Руси не привлекалось, хотя в нем затрагиваются те самые времена, по рубеж которых русские летописи как бы обрезаны и крайне противоречивы.

Но и здесь, в Gesta Hungarorum, множество странных пробелов. Подробно перечислены имена Куманов, а правитель Киева остается безымянным, равно как остальные князья Рутенов. Может мы бы узнали в первом, допустим, Вещего Олега, либо услыхали про вообще доселе неведомых князей.

Рассмотрим теперь летописные сообщения, связанные с местностями Угорьское и Берестовое.

Об Угорьском, Олеге, Оскольде с Диром мы уже говорили. Олег почему-то приплывает именно под Угорьское, на Угорьском хоронят Аскольда, и там же, при Несторе, оказывается двор Олма, Ольга, Ольги. Ежели двор князя Олега или княгини Ольги, то значит, княжий двор был на Угорьском. А на Угорьском или рядом с ним – Берестовое.

Листаем страницы летописи дальше. Нестор, рассказывая о мести княгини Ольги Древлянам за смерть ее мужа, князя Игоря, дает подробное описание старинного уже для него, Нестора, Киева – и летописец толкует современникам, где, в понятных им ориентирах, находился княжий двор (князя Игоря, надо полагать), отдельный двор княгини Ольги, и так далее.

Ипатьевский список, 6453 (945):

\begin{quotation}
И послаша Деревляне лучьшии мужи свои числом 20, в лодьи к Ользе, и приста под Боричевом в лодьи. бе бо тогда вода текущи возле горы Кьевьскыя и на Подоле не седяхуть людье, но на горе;

город же бяше Киев, идеже есть ныне двор Гордятин и Никифоров, а двор кьняж бяше в городе идеже есть ныне двор Воротиславль и Чюдин, а перевесище бе вне града двор теремный и другый идеже есть двор демьсников за святою Богородицею над горою, бе бо ту терем камен.

[...]

Ольга же повеле ископати яму велику и глубоку, на дворе теремьском, вне города. И заутра Ольга, седящи в тереме, посла по гости [...]
\end{quotation}

Классическая научная трактовка этого крутится возле будущего «града Владимира» на известном пятачке близ фуникулера и Михайловской площади. Все описанные объекты ученые помещают рядышком. С чем? А с Десятинной церковью. Она же посвящена Богородице, вот и полагают, что раз в летописи сказано «за святою Богородицею», то значит за Десятинной. А Десятинную полагают около Исторического музея, хотя разве в летописи сказано, где была построена эта церковь? Не сказано.

Теперь отложим научную трактовку в сторону и порассуждаем сами.

Из сказанного выходит, что князь Игорь жил отдельно от Ольги, у нее был свой двор и терем каменный. Но мы знаем, со слов Нестора же (по ряду списков), что Ольгин двор находился в Угорьском, у Аскольдовой могилы.

Ученые этого не замечают и помещают «двор Ольги» на всё тот же пресловутый пятачок в «историческом центре Киева».

Церковь святой Богородицы. Но и в Лавре с ней связаны два храма. Рождества Пресвятой Богородицы, выстроенный в 17 веке на месте деревянного. И знаменитый, разрушенный и воссозданный Успенский собор Киево-Печерской Лавры, полное название коего – Собор Успения Пресвятой Богородицы.

Нестор монашествовал именно в Лавре. Кто, как не он, знал окрестности и потому уточнил «идеже есть двор демьстиков за святою Богородицею» – а деместик это церковный главный певчий.

Нестор в своем «Житии Феодосия Печерского» описывает, как взамен Феодосия выбирают нового игумена, деместика Стефана: «То же слышавше, братия в велику печаль и плачь впадоша, и по сих излезше вън и сами в себе совет сотвориша, и якоже с совета вьсех Стефана игумена в себе нарекоша быти, деместика суща цьрковьнаго». Возможно, о дворе нового, последующего деместика и шла речь в уточнении положения «двора теремного и другого» – ведь Нестор прибыл в Лавру уже при игуменстве Стефана.

Теперь про несторовы «вне града» и «в городе». Я совершенно не понимаю, к какому времени относятся эти слова – ко времени Нестора или к описываемому им времени, княгини Ольги? Двор «теремный и другый» вне града – вне града Киева при Ольге, или при Несторе? Другый двор – это двор теремный или отдельный двор?

Двор теремный, где Ольга сидит, подчеркивается – каменный. А княж двор, игорев – деревянный?

Об этом дворе теремном Лаврентьевская летопись за 6488 (980) год пишет, что Владимир 

\begin{quotation}
въшед в двор теремный отень, о нем же преже сказахом, седе ту с дружиною своею
\end{quotation}

Отень – отцовский, Святослава – сына Ольги с Игорем. Ранее по летописи двор теремный закреплен за Ольгой.

И далее:

\begin{quotation}
И нача княжити Володимер в Киеве един, и постави кумиры на холму вне двора теремнаго: Перуна древяна, а главу его серебрену, а ус злат, и Хърса, и Дажьбога, и Стрибога, и Симаргла, и Мокошь. И жряху им, наричуще я богы [...]

На том холме ныне церкы стоить, святаго Василья есть
\end{quotation}

Холм – по летописям, у Боричева спуска. Вне двора теремного. Разве сказано, что возле двора теремного? Просто – вне его. А где двор теремный, здесь не сказано.

%Еще летопись говорит, что на холме том, где был Перун, построили затем церковь святого Василия. Но этому противоречит Синопсис:

%Повеле же Владимир поставити каменную Церковь в Киеве Святаго Спаса великую, и на том месте, идеже кумир Перун бяше, церковь

Про Владимира, что у него много наложниц, Ипатьевский список говорит за 6488 (980) год:

\begin{quotation}
и наложьниц у него 300 в Вышегороде, 300 в Белегороде, а 200 на Берестовем в сельци, еже зовуть и ныне Берестовое.
\end{quotation}

А при княжем дворе сколько? Поскольку более не перечислено, остается предположить, что в одном из указанных мест и находился княжий двор. Вышегород и Белегород (современная Белгородка под Киевом) – далековато. А вот на Берестовом – в самый раз.

Откуда Владимир правил, там и умер. Ипатьевский список, 6523 (1015):

\begin{quotation}
Умре же Володимир, князь великый, на Берестовем
\end{quotation}

При жизни же, как повествует Синопсис,

\begin{quotation}
Повеле же Владимир поставити каменную Церковь в Киеве Святаго Спаса великую
\end{quotation}

А где еще ставить, как не на Берестовом? Известна поныне церковь Спаса на Берестовом! Наука же считает, что последнюю возвели при другом Владимире, Мономахе. 

Рассказывая о создании Печерского монастыря, Нес\-тор обращается к самому началу его, при Ярославе Мудром. Ипатьевская летопись, 6559 (1051):

\begin{quotation}
боголюбивому князю Ярославу любяще Берестове, и церковь сущую святых апостол, и попы многи набдящу, и в них же бе прозувитер, именем Ларион, мужь благ и книжен и постник, и хожаше с Берестового на Дьнепр на холм, кгде ныне ветхий монастырь Печерьскый, и ту молитвы творяше; бе бо лес ту велик. Ископа ту печерку малу, 2 саженю, и приходя с Берестового, отпеваше часы и моляшся ту Богу втайне.
\end{quotation}

Прозувитер – пресвитер, по-гречески πρεσβύτερος – значит старец, старейшина. Это чин после диакона, но перед епископом. Иные названия пресвитера: иерей, священник. У князя Ярослава был приближенный пресвитер, начальник над попами (которые Ярослав «многи набдящу»).

После строительства Софии Киевской, сего Лариона назначили туда «митрополитом Руси». А прежде времени всего этого большого строительства в нынешнем центре города (когда возвели и новые крепостные укрепления вместе с Золотыми воротами, и монастыри святого Георгия да «святая Ирины»), оказывается, что Ярослав любит Берестовое, а при князе есть Ларион. И Ларион ходит с Берестового на лесистый холм у Днепра.

Где же обитает Ярослав в то время? Выходит, что на Берестовом. Как и отец его Владимир.

Читаем Ипатьевский список далее:

\begin{quotation}
В лето 6581 (1073). Взжвиже дьявол котору в братьи сей Ярославичих. И бывши распре межи ими, быста сь себе Святослав со Всеволодом на Изяслава; и изииде Изяслав ис Кыева, Святослав же и Всеволод внидоста в Кыев, месяца марта в 22, и седоста на столе на Берестовом, преступивша заповедь отню. [...]

Того же лета основана быть церковь Печерьская Святославом князем, сыном Ярославлим, игуменом Федосьем, епископом Михаилом, митрополиту Георгиеви тогда сущю в Грецех, а Святославу в Кыеве седящю.
\end{quotation}

Между сыновьями Ярослава Мудрого – распря. Некогда, еще в 6562 (1054) году Ярослав, умирая, завещал сыновьям княжить в определенных городах – Изяславу в Киеве, Святославу в Чернигове, Вячеславу в Смоленске, Всеволоду в Переяславле. 

И вот Изяслава прогоняют из Киева, а «Святослав же и Всеволод внидоста в Кыев, месяца марта в 22, и седоста на столе на Берестовом, преступивша заповедь отню». Нарушив заповедь отца, Святослав и Всеволод входят в Киев и – седоста на столе на Берестовом.

Стол на Берестовом. Прямо сказано.

Уже давно построен любимый наукой в качестве пристанища княжьего двора «город Ярослава», однако по летописям стол всё еще – на Берестовом!

Сын Всеволода, Владимир Мономах (1053-1125), сам будучи уже киевским князем, вносит изменения в свод законов «Роуськую правду». В ее уставе номер 48 сохранилась такая запись:

\begin{quotation}
Оустав Володимира Всеволодича (Мономаха)

48. Володимир Всеволодичь, по Святополце, созва дружину свою на Берестовем: Ратибора Киевьского тысячьского, Прокопью Белогородьского тысячьского, Станислава Переяславьского тысячьского, Нажира, Мирослава, Иванка Чюдиновича Олгова мужа, и оустави до третьяго реза, оже емлеть в треть куны; аже кто возьметь два реза, то то ему исто; паки ли возьметь три резы, то иста ему не взяти.
\end{quotation}

Мономах утверждает этот закон на Берестовом, при перечисленных свидетелях.

%Как же так? Наука утверждает, что тогда князья свои государственные дела вершили из нынешнего центра города, разве что на фуникулере не катались. А летописи упорно твердят про Берестове. Досадная неувязка.

Но вот княжему двору на Берестовом приходится туго. Ипатьевский список, за 6696 (1096) год:

\begin{quotation}
В се же время прииде Боняк с Половьце к Кыеву, у неделю, от вечера, и повоевоша окол Кыева, и пожьже на Берестовом двор княжь.

Того же месяца [...] (Святополк убивает Половца Тугоркана, своего тестя, под Переяславлем)

и взя и Святополк цьстя своего и аки врага, и привезше Киеву и погребоша и на Берестовом на могыле, межи путем грядущим на Берестовое, а другым идущим в монастырь
\end{quotation}.

Отсюда узнаём, что Боняк с Половцами пожог княжий двор на Берестовом. В какой степени «пожьже», неясно. 

В том же месяце Святополк, убив Тугорхана, привез его тело в Киев и похоронил на Берестовом на кургане, между дорогами на Берестовое и в монастырь (Печерский, Лавру). Может речь идет о распутье, перекрестке. Однако и Аскольда похоронили в кургане где-то в окрестностях!

А что известно про курганы в тех краях? Нетрудно ответить.

Археологам известен курган во дворе за Введенским монастырем (стык Панаса Мирного и Московской улиц), около дома на Московской улице, 8/10. Но более любопытно поле, где с 1915–1916 по 1950-е находился Печерский Ипподром\footnote{50°26'10"N 30°32'56"E, за Лейпцигской, 1.}, что был между улицами Цитадельной, Суворова и Январского восстания (Лаврской). Ныне под главным полем ипподрома – хранилище пресной воды Киевводоканала.

К сожалению, я не помню, откуда выписал эти сведения, а именно – там были найдены каменные бабы, и могилы, как полагают, 11-13 веков. Каменные бабы соседствовали обычно с курганами. На начало 20 века они, по словами Шероцкого, были помещены в ботсад (что на нынешнем бульваре Шевченко),

А место бывшего Иппордрома – всего в двухста с копейками метрах от лаврской стены и лежит на одной широте с церковью Спаса на Берестовом. Быть может, именно на будущем ипподроме и был погребен Тугорхан – если хоронить, то где, как не на тогдашнем кладбище, уходящем корнями в языческие времена?

Однако продолжим чтение летописи. После упоминания о сожжении княжего двора на Берестовом, он более не появляется на страницах летописи, но остается Угорьское.

Ипатьевский список за 6654 (1146) год:

\begin{quotation}
Всеволод же пришед в Киев разболися, и посла по брата своего по Игоря и по Святослава, и бысть велми болен, и ста под Вышегородом в острове;

и Всеволод же призва к собе Кияне и нача молвити: «аз есмь велми болен, а се вы брат мой Игорь, иметесь по нь». Они же рекоша: «княже! ради, ся имемь».

И пояша Игоря в Киеве, иде с ними под Угорьский, и съхав Кияне вси; они же вси целоваше к нему крест, рекуче: «ты нам князь», и яшася по нь лестью;

заутрии же день еха Игорь Вышегороду, и целоваше к нему хрест Вышегородце.

%[после смерти Всеволода]

%Игорь же еха Киеву, и созва Кияне вси на гору на Ярославль двор, и целовавше к нему хрест; и пакы скупишася вси Кияне у Туровы божьницы, и послаша по Игоря, рекуче: «княже, поеди к нам».

%Игорь же, поем брата своего Святослава, и еха к ним, и ста с дружиною своею, а брата своего Святослава посла к ним у вече.
%И почаша Кияне складывати вину на тиуна на Всеволожа, на Ратьшу, и на другаго тивуна Вышегородьского, на Тудора [...] «а ныне, княже Святославе, целуй нам хрест, и з братом своим».
\end{quotation}

Всеволод, возвращаясь в Киев, заболел и послал по братьев своих Игоря и Святослава, сам же остановился на острове под Вышгородом. Позвал также к себе Киян – Киевлян, мол, берите себе князем Игоря.

Под Угорьское все Кияне съезжаются с Игорем и целуют ему крест, признавая за князя – «ты нам князь». Следовательно, Угорьское, близкое к Берестовому, еще имело какое-то значение в качестве престола.

Целование креста повторяется в Вышегороде. И далее, уже после смерти Всеволода, 

\begin{quotation}
Игорь же еха Киеву, и созва Кияне вси на гору на Ярославль двор, и целовавше к нему хрест; и пакы скупишася вси Кияне у Туровы божьницы, и послаша по Игоря, рекуче: «княже, поеди к нам».
\end{quotation}

Игорь едет в Киев, уже на гору, на Ярославль двор, и часть Киян там целовали ему крест, а часть собралась у Туровой божницы и послала за Игорем, дабы он приехал к ним. Новый князь выбрал себе новый двор-престол на Ярославле дворе? 

%Но не так всё просто. До того, смертельно больной Всеволод, представлял наследника следующим образом:

%\begin{quotation}
%пояша Игоря в Киеве . иде с ними
%под Оугорьскии и съзва Киянѣ
%\end{quotation} 


Ипатьевский список, 6659 (1151) год:

\begin{quotation}
а Коуеве и Торчи и Печенези туда сташа от Золотых ворот до Лядьских ворот, а оттоле оли и до Клова и до Берестового и до Угорьских ворот и до Днепра [...]

Изяслав же с Вячеславом седе в Киеве; Вячеслав же на Велицем Дворе, а Изяслав под Угорьским, а сына Мьстислава посади Переяславли.
\end{quotation}

Тут два важных дела. 

Первое – среди прочих ворот Киева упомянуты Угорьские. Следовательно, крепостная стена Киева простиралась и по эти Угорьские ворота! Ведь не будут же они стоять сами по себе, отдельно от стены.

Так ли мал и убог был верхний Киев того времени, как говорит наука? По одному упоминанию Угорьских ворот его границы существенно расширяются по Аскольдову могилу и Берестовое!

И далее, Изяслав со стрыем – братом своего отца, Вячеславом, «садится» в Киеве княжить, причем Вячеслав поместился на Велицем (большом) дворе, а Изяслав где-то под Угорьским – вероятно, на месте Берестового. При этом Изяслав далее по летописи играет роль более весомую, нежели Вячеслав, который как бы страховал в Киеве тыл племянника во время его военных походов.

Последнее упоминание Угорьского находим в Ипатьевском списке, за 6669 (1161) год. 

Изяслав с Половцами нападает на Киев (где княжит Ростислав) с севера – мы уже рассматривали сообщение про «загорожено бо бяше столпием от горы оли и до Днепра». Далее происходит вот что:

\begin{quotation}
И начала одолевати Изяслав: уже бо Половци въездяху в город, просекаюче столпие, и зажгоша двор Лихачев попов, и Радьславль, и побегоша Берендиче к Угорьскому, а друзии к Золотым воротам.
\end{quotation}

Берендичи, одни из защитников Киева, бросив Ростислава, побежали к Угорьскому – возможно к тамошним воротам, ибо некие «друзии» (другие) дали стрекача к Золотым воротам.

После в дошедших до нас летописях ни Берестовое, ни Угорьское не упоминаются, не встречал я этих названий и в давних земельных грамотах времен Великого Княжества Литовского и позднее.

Память о Берестовом осталась, привязавшись к названию церкви Спаса, причем Лавра, разрастаясь, дошла пределами своими до этой, некогда стоящей отдельно, церкви. Да она и сейчас находится в самом северном углу лаврского заповедника, в переулке Лаврском.

Угорьское же удается соотнести с Аскольдовой могилой только по летописным сведениям, а в свою очередь Аскольдова могила привязана к местности испокон веков в народной молве, как Олеговой горкой называют Щекавицу или часть ее. 

%Летописи и «правда Роуська» красноречиво освещают рассвет и закат княжего двора на Берестовем. Почему наука в упор этого не видит – вопрос к науке, мы же двинемся дальше.
