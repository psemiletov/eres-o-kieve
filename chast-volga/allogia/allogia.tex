\chapter{Аллогия}

После летописной смерти своей, Вольга снова возвращается на страницы истории как Ольга, княгиня, жена Владимира Красна-Солнышка. Но ведь она, по нашим летописям, его покойная бабушка!

Однако существует исландская сага об Олафе Трюггвасоне (Olafs saga Tryggvasonar)\cite{jackson01}, ее хорошо знают ученые, которые рассматривают сагу по кусочкам в силу своих научных интересов, принимая одни кусочки за истинные, другие за ложные, ибо неудобны они, не вписываются в общепринятое.

%\begin{center}
%\includegraphics[width=\linewidth]{volga/olav_tryggvasson_mynt.jpg}

%textit{Монета с изображением Олава.}
%\end{center}

Есть несколько вариантов этой саги, наиболее известный сложен монахом бенедиктинского Тингейрарского монастыря Оддом Сноррасоном, умершем в 1200 году. Латинский подлинник, из коего до нас дошло лишь несколько слов, написан примерно в 1190 году. Чуть позже появился исландский перевод, сохранившийся в трех списках.

Сага рассказывает об Олаве, сыне конунга Трюггви, Волей судьбы Олав попадает в Гардарики или Гарды (Русь), в Хольмгард к конунгу Вальдамару Старому (Владимир Красно Солнышко) и Аллогии. Наши историки имя сие не любят и предпочитают писать туманно «княгиня». В подлиннике так – Allogia.

С Хольмгардом современные ученые связывает Новгород. Прежде чем перейти к изложению частей саги, относящихся к Аллогии, я хочу разобраться, в какое время происходит её действие и в самом ли деле Хольмгард, в рамках именно этой саги – Новгород, а не Киев.

Известно, что Владимир поначалу княжил в Новгороде. Обратимся к любимой наукой общепринятой хронологии!

Согласно ей, по летописям, в 970 (6478) году Святослав «садит» сыновей – Ярополка в Киеве, а Ольга (Олега) «в Деревех», в Овруче. В то же время являются люди новгородские и просят себе князя, иначе выберут оного со стороны. Святослав спрашивает своих детей, но Ярополк и Олег в Новгород не хотят. Тут Добрыня предлагает:

\begin{quotation}
«просите Володимера». Володимер бо бе от Малуши, ключницы Ользины; сестра же бе Добрыни, отец же бе има Малк Любечанин, и бе Добрыня уй\footnote{Уй или вуй значит – дядя.} Володимеру.
\end{quotation}

Это по Лаврентьевской летописи. А вот Ипатьевская:

\begin{quotation}
«просите Володимира». Володимир бо бе от Малуши милоситьнице Ольжины. сестра же бе Добрыня. отец же бе има Малко Любчанин. и бе Добрыня оуи Володимиру.
\end{quotation}

То есть дядя Владимира, Добрыня, советует на княженье в Новгороде своего племянника Володимира, мать которого – Малуша, ключница либо «милоситьница» Ольги. Отец Добрыни и Малуши – Малко Любечанин.

А что такое «милоситьница»? Одни полагают, что нечто вроде приятельницы, приживалки. Другие – мол, так называли приближенных к князю или княгине бояр. Третьи – что было такое сословие.

Стрыйковский в «Кронике польской, литовской» предлагает следующий вариант родственных отношений упомянутых имен:

\begin{otherlanguage}{polish}
\begin{quotation}
A Swentoslaw po smierci matki swojej Holhy albo Heleny rozdzielit xiestwa Ruskie trzem synom swoim, Jaropolkowi dal Kijow; Oldze albo Olhowi Drewlany z zamkiem Choroscienem i z Pereslaw\-em; Vlodimirzowi Nowogrod Wielki, bo Nowogrod\-zanie z namowy niktorej niewiasty Dobrynie, Wlodi\-mirza za xiaze sobie uprosili, bowiem byl w Nowog\-ridzie Wielkim mieszczanin jeden Kalufcza albo Kaluscza Malec przezwiskiem; ten mial dwie corce, Dobryne i Maluske; Maluska u xiezny Holhy byla w fraucim\-erze klucznicza, s ktora mial Swantoslaw Wlodimi\-ra.
\end{quotation}
\end{otherlanguage}

Мой перевод:

\begin{quotation}
А Свентослав по смерти матери своей Олхи или Елены разделил княжества русские трем сынам своим, Ярополку дал Киев; Олдже или Олхови Древлян с замком Коростенем и с Переславем; Влодимиру Новгород Великий, ибо новогородцы, подговоренные некоторой женщиной Добрыни, Влодимира за князя себе упросили, потому что был в Новгороде Великом один мещанин Kalufcza или Kaluscza по прозвищу Малец, он имел двух дочерей, Добрыню и Малушку; Малушка у княгини Олхи была в ключницах, от которой Святослав имел Влодимира.
\end{quotation}

«Fraucimerze klucznicza» – два слова, второе ясно – ключница, а первое я в польском словаре не нашел, его вариант есть в чешском, «fraucimerka», заимствованное из немецкого. Оно значит – камеристка, горничная, служанка, но и нечто вроде «приживалки». Одним словом, Малуша была доверенной компаньонкой-служанкой, заведовавшей ключами.

Новгородцы принимают предложение, чтобы Владимиру быть князем. Тот с Добрыней отправляется в Новгород, а Святослав – в свой Переславец, это ныне, как некоторые считают, село Преслава на берегу Георгиевского Дуная.

Итак, в 970 году Владимир начинает княжить в Новгороде. Историки любят писать, что Красну Солнышку было тогда 10 лет. Ладно, допустим, всем управлял более взрослый Добрыня, как регент.

По редкой «Летописи Панцерного и Аверки», в ее части, выписанной Стефаном Гавриловичем Аверкой из книжки Михала Панцерного, мещанина витебского 13 мая 1768 года, в 974 году Ольга основывает Витебск\footnote{Подлинник:
\begin{otherlanguage}{polish}
Dziele miasta Witebska, ktore sie działy na swiecie dawnych czasow. Wypisane tu z xiengi pisaney renku i [ego] m [ości] pana Michała Pancernego, mieszczanina witebskiego słowo w słowo przez Stefana Hawriłowicza Awierke, mieszczanina
wittebskiego w roku 1768, miesionca mli, 13 dnia.

Zaczynaion sie tak pisane w xiendze Pancernego:

Roku 974. Zbiwszy Olha iaćwingow i piczyngow y, przeprawiwszy sie przez rzeke Dzwina zanocowawszy z woyskiem, y upodobawszy gure założyła zamek drewniany, nazwała od rżieki Widiby Wittebskiem, wmurowała cerkicu w Wysznim zamku swienteho Michała, a w Niżnim Zwiastowanie. Dwa roki zmieszkawszy odiechała do Kiowa.\end{otherlanguage}}:  

\begin{quotation}
Победив языгов и печенегов (iaćwingow i piczyngow), и переправившись с войском через реку Двину, Ольга на понравившейся горе 
заложила деревянный замок, назвала его от реки Видибы Виттебским, и построила в Вышнем замке  каменную церковь святого Михаила, а в Нижнем – Благовещения. Два года прожив тут, отбыла в Киев.\end{quotation}

Не буду заниматься толкованием этого известия и давать ему оценку, привел как есть. 

Вернемся к русским летописям. Уже после гибели Святослава, Владимир упомянут за 977 (6485) год в связи с бегством из Новгорода, когда Владимир проведал, что брат его, Ярополк, убил брата Олга. Ярополк не преминул воспользоваться бегством Владимира и назначил посадника своего в Новгороде, таким образом подчинив себе управление всем княжеством.

Владимир, проживший в Новгороде семь лет, удрал за море, где три года отсиживался и, надо полагать, договаривался с Варягами. Кстати, что им мог посулить беглец?

И вот в 980 году (6488) Владимир, уже с Варягами возвращается в Новгород и отсылает к Ярополку его посадников, вызывая брата сражаться. Сам же княжит в Новгороде и задумывает жениться. На примете Рогнеда (Рохмида), дочь Рогволда Полоцкого. Тот спрашивает у дочери – хочешь ли за Владимира. Она отвечает: «не хочу розути робичича, но Ярополка хочю».

Ученые говорят – был такой обряд при свадьбе, как разувание мужа. Робичич – обычно переводят как «сын рабыни», но можно и как – «сын служанки». Полагаю, Рогнеда не подразумевала Малушу рабой, а отозвалась пренебрежительно к более низкому, чем Рогнеды, происхождению Владимира. Дескать, не хочу с сына служанки сапоги стаскивать\footnote{Я надумал еще одну трактовку. Если вдруг этот сюжет был взят из греческого источника, то в подлиннике «робичич» могло быть записано как нечто вроде «ο γιος ενός σκλάβου», то есть «сын склава», а последним словом Греки обозначали как Славян, так и рабов. И переводчик вместо «сын славянина, славянки» переложил – «сын раба, рабыни», «робичич».}!

Далее Красно Солнышко, добрый князь, берет штурмом Полоцк, насилует Рогнеду на глазах её родных\footnote{У Владимира это не единственное деяние такого рода.}, затем убивает последних, Рогнеду нарекает Гореславой, и отправляется брать Киев, вышибать оттуда брата Ярополка. Получается хитростью, при помощи предательства Блуда – ярополкова воеводы – выманить того из города. Ярополк драпает в город Родню, в устье Роси.

Владимир же захватывает Киев, а заодно берет жену Ярополка. Последнего же убивают, не без помощи дружеского совета Блуда. Судя по всему, Ярополк легко попадал под влияние. То Свенельд им крутил, то Блуд.

Теперь пора Владимиру расплачиваться с нанятыми Варягами. Они глаголют: «се град наш; мы прияхом и, да хочем иметь окуп на них, по 2 гривне от человека». Владимир просит обождать месяц, пока налоги соберет, и ничего не дает. По какой-то причине Варяги не бунтят против Владимира и просят показать им путь в Греки. Владимир наделяет Варягов другими городами и пропускает к Царьграду, но отправляет туда гонцов предупредить.

Так что с 980 года Владимир княжит в Киеве, покидая оный для военных походов. Эта дата нужна, чтобы судить о времени действия саги об Олафе Трюггвасоне.

Мы выяснили, что 7 юношеских лет Владимир княжил в Новгороде с 970 по 977 годы, а потом с 980 – княжил в Киеве (кстати, в древнем житии Владимира указан 978 год).

Это значит, что временной отрезок, где Олав живет вместе с Владимиром и Ольгой, лежит в пределах с 970 по 977 год, если принять, что место действие Хольмгард – это Новгород. Если же в саге под Хольмгардом подразумевается не Новгород, а Киев, то действие происходит начиная с 978 или 980 года.

Но Олав жил при Владимире и Ольге 9 лет, а Владимир княжил в Новгороде 7 лет. Можно сделать один из выводов:

1. Под Хольмгардом в саге подразумевают Киев, а не Новгород.

2. В саге ошибка в количестве лет.

3. В саге речь не о том, что все 9 лет Олав был в Хольмгарде-Новгороде, а подразумевается, что Олав перемещался вместе с Владимиром и Ольгой в Киев.

Сейчас «общепринято», что Хольмгард (Hólmgarðr) – это Новгород. Раньше считалось, что не Новгород, а Холмгоры, древнее селение в низовьях Северной Двины. В некоторых сагах говорится об островах и мысах у Хольмгарда. Какие острова и мысы возле Новгорода? А какие зимовки флота Олава Святого могут быть в Новгороде, если на реке Волхов – пороги, непроходимые для морских судов, и грузы с них перекладывали на речные корабли с небольшой осадкой? 

Скажу то, что академическая наука не хочет рассматривать. Под Хольмгардом в разных сагах назывались разные города. Вот и всё. И я не буду описывать различные трактовки, что означает слово Хольмгард – историки тут ломают копья. В этой книге мне важно, что по совокупности данных из саги и русских летописей – как мы увидим чуть ниже – вычисляется место действия Киев, а не Новгород.

Я выпишу из саги куски, касающиеся большей частью непосредственно Аллогии (Ольги) и Вальдамара (Владимира). Сама же сага повествует об Олаве, отцом которого был конунг Трюггви (его убили), а матерью Астрид (Астрида), чей брат Сигурд служил у Владимира. 

Астрид с трехлетним Олавом спасается от врагов. Она бежит на трех кораблях на шотландские острова Оркни. В «Истории Норвегии» изложено несколько иначе – беременная Астрид скрывается на Оркни, и там рожает.

Но и на островах она ощущает опасность. Поручает Торольву Вшивобородому перевезти сына в Хольмгард. Однако по пути нападают викинги и мальчик становится рабом в Эйстланде, нынешней Эстонии. Владимир поручает Сигурду собрать дань в Эйстланде, тот встречает племянника, вызволяет его и привозит в Хольмгард. И так далее.

Сага существует в нескольких вариантах. Наиболее полная их подборка в подлиннике и переводе дана в книге Татьяны Джаксон «Исландские королевские саги о Восточной Европе»\cite{jackson01} – по этому изданию я и буду цитировать переводы. Сага включена также в «Круг земной» Снорри Стурлусона. В ходу как короткие версии, так и «Большая сага об Олаве Трюггвасоне», составленная из нескольких источников в 13-14 веках.

По исландским анналам, или хроникам (записывались с конца 13 века) даты из жизни Олава выглядят примерно так:

968 или 969 год – рождения Олава.

971 – пленение Олава в Эйстланде.

978 или 979 – Олав в Гардарики (жизнь при Владимире и Ольге).

986 или 987 – Олав покидает Гардарики.

993 – крещение Олава на островах Сюллингах (около Ирландии).

995 – начало правления Олава в Норвегии.

999-1000 – смерть Олава.

Как сие перекликается с нашими летописями?

Даже быстрый взгляд на цифры дает понять, что пребывание Олава в Гардарике, с 978 либо 979 года близко к началу княжения Владимира в Киеве с 978-980 года. А может всё напутано и речь идет о другом князе Владимире? Нет, указано же – Вальдамар Старый, отец Ярицлейва (Ярослава).

Другие сравнения. В 989-м, по летописям, Владимир «замыслил» церковь Богородицы (Десятинную), в 996 году по завершении строительства выкопал гроб с Ольгой. Значит, согласно общепринятым датам и сюжету, до 996 года Ольга лежит в гробу, не потревоженная внуком своим Владимиром.

Как мы узнаем из саги, Олав способствует крещению и Владимира, и Ольги. Но знаменитое летописное крещение Руси историки относят к 988 году. Иногда отодвигают к первым годам девятого десятилетия. Теоретически, мог ли Олав «крестить» (приводит епископа) Ольгу? Если принять во внимание те летописные источники, где говорится, что Ольга была тайной христианкой, то наверное да.

Вспоминается и список, где Ольга, а не Владимир, строит церковь Богородицы, Десятинную – сообщение казалось мутным и ошибочным, если не рассматривать его в рамках саги об Олаве. Потом, традиционно князей хоронили в церквях, ими возведенных. Ольга могла начать строительство, умереть до его окончания, но затем быть таки помещенной в Десятинную.

Самого Олава, по одним сагам крестили несколько позже, нежели Владимира и Аллогию, а по другим – до князя и княгини.

Ниже я начинаю приводить свод кусков из вариантов саги, взяв за основу «Большую сагу» и дополняя теми, где сюжет имеет расширенное описание. По ходу буду делать примечания и рассуждать.

\begin{quotation}
В то время, когда сыновья Гуннхильд пришли к власти в Нореге, правил в Гардарики тот конунг, которого звали Вальдамар. Жена его звалась Аллогия. Она была умной и доброжелательной, хотя и была в то время язычницей.
\end{quotation}

«Норег» значит «Норвегия». Отметим – написано, что Вальдамар правит во всей Гардарики, не в одном Новгороде. Значит, он уже захватил власть над всей Русью, это уже классический князь Владимир Красно Солнышко времени своего сидения в Киеве. Аллогия названа язычницей. Это возможно и в случае, если Ольга скрыла своё крещение в Царьграде.

Далее идет пророчество от матери Вальдамара. Мать – в наших летописях Малуша, по имени здесь не названа. Она пророчица, и возможно это причина «подбора кадров», когда Вольга взяла Малушу к себе. Я бы еще предположил, что сама Вольга иногда играла роль дряхлой матери-вещуньи, но это будет слишком.

\begin{quotation}
У конунга Вальдамара была мать, очень старая и дряхлая, так что она лежала в постели. И все же обладала она даром провидения от духа фитона, как многие те язычники, о которых говорили, что они предсказывают еще не наступившие и сомнительные события.

И таков был там постоянно обычай, что в первый вечер йоля, когда люди сядут по местам в палате конунга, приносили старую мать конунга к его высокому сиденью. Говорила она тогда, не предстояло ли какого опасного немирья конунгу или его государству и о другом подобном, о чем ее спрашивали. 
\end{quotation}

Йоль – праздник зимнего солнцеворота у скандинавов и германских народов, 19-20 декабря. Слово «йоль», так похожее на «ёлку», обозначает полено, которое медленно сжигали 12 дней, а уголь собирали для ритуалов.

На Руси празднику Йоля по времени соответствует Коляда (с 24 на 25 декабря по старому стилю), с которой совместилось Рождество. В украинском названии Рождества – «Різдво» сохранилось буквальное обозначение действия – забой скота. «Різдво» – от «різати», а «Коляда» – от «колоть». Известно, что в селах под Коляду, под Рождество колят или колют, её не спросив, свинью к праздничному столу. Отсюда и слово «колдовство» – это искаженное «колядовство», ибо колдовство подразумевало и жертвоприношение, убиение жертвы, закалывание оной.

Но вернемся к саге:

\begin{quotation}
Так случилось одной зимой, что, по обычаю, старая женщина была принесена в палату. Тогда спросил конунг, не видит ли она какого иностранного конунга или воина, который бы претендовал на его государство.

Она отвечает: «Не знаю я, сын мой, ни о какой губительной войне, которая угрожала бы тебе или твоему государству. Однако вижу я видение большое и очень хорошее. На севере в Нореге родился некоторое время назад сын конунга, который будет воспитываться здесь в Гардарики, пока он не сделается знаменитым хёвдингом\footnote{Тут слово «хёвдинг» можно трактовать как «полководец».}. Он не нанесет никакого урона твоему государству. Напротив, он и мир сохранит, и всячески увеличит твою честь. Наконец, он вернется в свое отечество, когда он будет в расцвете лет, и завладеет тем государством, на которое он имеет право по рождению. Он будет сиять ярким светом и достоинством, и многим народам станет он спасителем в северной части мира. Но короткое время продержится его власть над Норегом. Отнесите меня теперь прочь, – говорит она, – поскольку слишком много и хорошо рассказала я об этом человеке». 

Унесли ее тогда прочь.
\end{quotation}

На сцену выходит дядя Олава – Сигурд:

\begin{quotation}
Многие молодые люди и сыновья могущественных людей бежали из Норега из-за тирании сыновей Гуннхильд. Некоторые искали себе почета у иностранных хёвдингов, как Харальд Гренландец, о котором говорится раньше, что он был в Свитьод у Скёглар-Тости. Сигурдом звали сына Эйрика в Опростадире, брата Астрид. Он был на востоке в Гардарики у конунга Вальдамара и имел от него большой почет и власть. 
\end{quotation}

В иных вариантах саги дополняется, что Сигурд получил от Вальдамара «большие владения и большой лен», «его повеления должны были иметь силу во всем государстве конунга», и должен был «вести дела конунга и собирать дань конунга по всем областям». Сигурду также поручалось ведение суда конунга и выбор величины дани. Что тут правда, неясно, но Сигурд – важная птица! А в русских летописях о нем ни слова.

\begin{quotation}
Захотела тогда Астрида, его сестра, поехать туда к нему. К тому времени она пробыла в Свиавельди у Хакона Старого два года. Олаву, сыну ее и конунга Трюггви, было тогда три года. Хакон Старый отправил ее в сопровождении неких купцов и дал ей и ее людям в изобилии все, что им требовалось иметь. И когда они поплыли на восток в море, на них напали викинги. Это были эйсты. Захватили викинги в качестве добычи и добро, и тех людей; некоторых [они] убили, а некоторых поделили между собой в качестве рабов.

Там разлучился Олав со своей матерью. Олава и его воспитателя Торольва и Торгильса, сына Торольва, взял тогда себе тот человек, которого звали Клеркон.
\end{quotation}

Клеркон убивает Торольва, а Олава и Торгильса продает в рабство. Оба мальчика проходят через цепь перепродаж и наконец оказываются в Эйстланде, у Реаса, где оседают на шесть лет, причем с Олавом Реас обращается как с сыном, а Торгильса заставляет работать наряду с прочими рабами. Эйстланд – Эстония (не путать с нынешним Эстландом – юго-восточной Норвегией). 

\begin{quotation}
Тогда приехал в Эйстланд Сигурд Эйрикссон, брат матери Олава. Он был послан Вальдамаром, конунгом Хольмгарда,чтобы собрать там в стране подати для конунга; ехал Сигурд с большой пышностью, у него было с собой очень много людей.

Однажды Сигурд въехал со своими спутниками в усадьбу Реаса, когда мальчик Олав играл с другими молодыми людьми. 

[Сигурд узнаёт в Олаве племянника и выкупает его и Торгильса] 

Тогда купил Сигурд Торгильса за марку золота, а Олава за 9 марок золота. Привез Сигурд обоих мальчиков с собой в Хольмгард и никому из людей не открыл происхождения Олава, и обращался с ним хорошо во всех отношениях. Тогда было Олаву 9 лет.
\end{quotation}

9 марок золота – большие деньги. В другом варианте саги вместо Реаса – Эрес. Олав был младше Торгильса, и хозяин вообще не хотел его продавать, поэтому заломил цену заведомо большую, но Сигурд все же уплатил.

Далее рассказывается, как Олав на торгу встречает убийцу Торольва. В саге используется слово torg, и ученые чего-то полагают, что это новгородский торг. Будто в Киеве не было торга!

\begin{quotation}
Случилось однажды, что Олав Трюггвасон был на торгу. Там было очень много народа. Там он узнал Клеркона, который убил его воспитателя Торольва Вшивобородого. У Олава был в руке маленький топор, он подошел к Клеркону и ударил его топором по голове так, что разрубил ему мозг.
\end{quotation}

В иной версии Олав, узнав Клеркона, вернулся к Сигурду и рассказал об этом. Просил дядю помочь наказать врага. С отрядом воинов идут на торг, хватают там Клеркона и уводят за город. Дают Олаву в руки большой топор. Олав бьет им Клеркона по шее и отрубает голову. В этой версии саги нет последующей просьбы Сигурда к Ольге уберечь Олава от кары за преступление, а Ольга берет Олава под своё крыло посредством загадочных массовых смотрин, определяя человека, про которого пророчествовала мать Владимира.

Но вернемся к «Большой саге»:

\begin{quotation}
тотчас же побежал Олав домой и сказал Сигурду, своему родичу. А Сигурд сразу же отвел его в покои княгини Аллогии и рассказал ей новости, и попросил ее помочь мальчику. 

Она поглядела на мальчика и сказала: «Не следует убивать такого красивого мальчика». Велела она тогда всем своим людям прийти туда в полном вооружении. В Хольмгарде была такая великая неприкосновенность мира, что следовало убить всякого, кто убьет неосужденного человека. 

Вот бросился весь народ, по обычаю своему и законам, бежать за Олавом, куда он скрылся. Хотели они лишить его жизни, как требовал закон. Говорили, что он во дворе княгини и что там собрался отряд людей в полном вооружении, чтобы охранять его.
\end{quotation}

Обычно ученые, комментируя эту сагу, не верят, что у Аллогии была своя дружина и отдельный двор. Твердят – быть такого не может, чтобы княгиня имела столь странную обособленность и независимость. Но как мы помним из русских летописей, именно такой независимостью обладала княгиня Ольга. 

Ранее сага называет Сигурда человеком Владимира. А теперь, когда надо укрыть племянника, Сигурд почему-то прибегает к помощи Ольги. Ольга жестко попирает закон – обеспечивает охрану для Олава, пока Владимир не прибывает со своей дружиной и разруливает положение. У Ольги была некая веская причина, чтобы спасти Олава.

\begin{quotation}
Затем это дошло до конунга. Пошел он тогда быстро со своей дружиной и не хотел, чтобы они бились. Устроил он тогда мир, а затем и соглашение. Назначил конунг выкуп за убийство, а княгиня заплатила.

С тех пор был Олав у княгини, он был очень любим ею, и весь народ был к нему очень привязан. В Гардарики были законы, что там не могли находиться люди королевского рода, кроме как с разрешения конунга.

Тогда сказал Сигурд княгине, какого рода был Олав, и о том, по какой причине он туда приехал, что он не мог жить в своей стране из-за немирья и вражды своих недругов; просил Сигурд ее поговорить об этом с конунгом. Она сделала так и попросила конунга помочь этому сыну конунга, с которым так сурово поступили; привели к тому ее уговоры, что конунг обещал ей то, о чем она просила; взял он тогда Олава под свое покровительство и обращался с ним прекрасно, как и положено было обращаться с сыном конунга.
\end{quotation}

Именно Ольга, а не Сигурд, уплачивает выкуп за убийство. Ольга заручается поддержкой Владимира, чтобы Олава никто не трогал длительное время. Что это со стороны Ольги? Небывалая благотворительность? Свои четкие планы на Олава – например, захват власти в Норвегии, свой человек в Норвегии? Нечто иное? Несколько позже я сделаю предположение, которое объяснит многие странности и нестыковки.

Кстати, на протяжении всей саги Владимир поступает лишь так, как хочет Ольга. Но двинемся дальше.

\begin{quotation}
Пробыл Олав там девять лет в Гардарики у конунга Вальдамара. Он был всех людей красивее, выше и сильнее и превосходил в искусствах всех северных людей, о которых рассказывается.
\end{quotation}

Сцена «массовых смотрин», где Ольга ищет человека из пророчества, в «большой саге» опущена, вместо нее стоит урезанный текст, к тому же иначе, чем ранее, объясняющий благосклонность Аллогии к Олаву. Прежде причиной тому была открытая Сигурдом родословная Олава, а теперь – принятие Олава за человека из пророчества, но не матери Владимира, а просто неких многих безымянных вещунов:

\begin{quotation}
В то время, когда Олав приехал в Гардарики, было в Хольмгарде много тех людей, которые предсказывали многие будущие события. Они все говорили в своей мудрости одно, что духи-хранители одного чужестранца, молодого по возрасту, пришли в эту страну, обещающие такое счастье, что ни у одного человека не видели они духов более прекрасных. Но не знали они, где или кто был тот человек, и все же они многими словами, доказывали что тот ясный свет, который сиял над ним, рассеивался по всему Гардарики и повсюду в восточной части мира. 

А поскольку княгиня Аллогия была умнейшей из всех женщин, то поняла она тотчас по внешнему виду Олава, как только она взглянула на него в первый раз, что этот мальчик должен был обладать тем большим счастьем, на которое перед этим указывали прорицания, что он завоюет большую честь Гардарики. Поэтому пользовался он наибольшим благорасположением конунга и княгини и большим поклонением людей мудрых и благородных. 

Вырос Олав там в Гардарики и скорее развивался по уму и силе и зрелости, нежели по годам. Конунг Вальдамар любил Олава так, словно он был его собственным сыном, и приказал обучить его хорошему владению оружием и рыцарскому делу, и всякого рода искусствам, и всему, необходимому для хёвдинга. Он скорее приобретал физическое и духовное совершенство, чем многие другие люди.
\end{quotation}

Ускоренное развитие Олава сходно с подобным у самой Вольги. Ольга, возможно, чувствует близкого по крови (12 лет отроду, Олав уже просит Владимира выделить ему воинов и корабли для военного похода).

В другом варианте саги этот эпизод описан подробнее – впрочем, здесь опущена сцена, где Сигурд прибегает к помощи Ольги для спасения Олава (убившего Клеркона). Ольга желает найти человека из пророчества, и устраивает большие смотрины: 

\begin{quotation}
В это время было в Гардарики много прорицателей, тех, которые знали о многом. Они говорили в своих пророчествах, что в эту страну пришли духи-хранители какого-то благородного человека, хотя и молодого. И никогда раньше они не видели ни у одного человека духов более светлых либо более прекрасных, и доказывали они это многими словами, но не могли узнать они, где он был. И таким, говорили они, замечательным был его дух, что тот свет, который сиял над ним, рассеивался по всему Гардарики и повсюду в восточной части мира. 

А потому что, как ранее было сказано, княгиня Аллогия была умнейшей из всех женщин, то посчитала она все это очень важным. Вот она конунга в красивых просит словах, чтобы он велел созвать тинг\footnote{Вроде нашего вече, собрание народа.}, чтобы люди пришли туда из всех близлежащих местностей; она говорит, что она придет туда и распорядится «так, как мне хочется». 
\end{quotation}

Конунг откликается согласием на все затеи Аллогии. А затевается ведь странное. Аллогия по некоему своему соображению намерена выбрать человека из толпы.

\begin{quotation}
Вот делает конунг так; приходит туда огромное множество людей. Вот приказывает княгиня, чтобы образовали круг из людей, из всей толпы, «и должен каждый стоять рядом с другим, так чтобы я могла видеть внешность каждого человека и выражение, и особенно глаза, и я надеюсь, что я смогу почувствовать, кто владеет этим духом, если я увижу зрачки его глаз, и никто тогда не сможет скрыть,если такова его природа». 
\end{quotation}

По зрачкам глаз. Вот признак, по коему Аллогия определит нужного человека. Как не вспомнить тут ирландские да осетинские сказания с их героями, у которых было по нескольку зрачков в глазу? А седых с рождения людей из скифских народов, также с несколькими зрачками и с завораживающим взглядом, про коих писал Исигон Никейский?

\begin{quotation}
Послушался тогда конунг ее речей. И длится этот многолюдный тинг два дня. А княгиня подходит к каждому человеку и осматривает внешность каждого человека, и не находит никого, кто показался бы ей похожим на человека, которому выпал такой великий жребий. И когда тинг продолжался два дня и настал третий день, то увеличился тинг.

Шли тогда туда все по его приказу, а иначе их сочли бы виновными. 
\end{quotation}

Не просто так собрание. Из-под палки, многолюдно. Два дня Аллогия кого-то упорно высматривает среди становящихся в круг людей. Значит, ей очень нужно.

\begin{quotation}
Вот образовал весь народ круг, а эта славная женщина и знаменитая княгиня осмотрела внешний вид и выражение каждого человека.

Подходит она через некоторое время туда, где перед ней стоял юный мальчик в плохой одежде; он был в широком плаще, и капюшон был откинут на плечи. Она посмотрела в его глаза, и поняла она тотчас, что это у него было такое большое счастье, и ведет она его к конунгу, и стало тогда ясно всем, что нашелся на этот раз тот человек, которого она долго искала.

Вот взял конунг этого мальчика под свою власть. Открыл он тогда конунгу и княгине род свой и достоинство, что он не был рабом, но открылось теперь, что он был украшен королевским происхождением. 

С тех пор стали конунг и княгиня воспитывать Олава любовно, с большой лаской. Одарили они его многими дорогими вещами, как своего собственного сына. Этот мальчик вырос в Гардах, рано достиг совершенства по силе и уму и развивался не по годам, так что через немного лет был он далеко впереди своих сверстников во всем том, что может украсить хорошего вождя.

И как только он начал проявлять себя и свое физическое и духовное совершенство, то выглядело это очень хорошо со многих сторон, и через короткое время научился он всему рыцарскому обычаю и военной мудрости, как те люди, которые были самыми искусными и доблестными в той области.
\end{quotation}

Понятно, почему «смотрин» нет в «большой саге» – ибо противоречат сцене, где Сигурд прячет Олава у Аллогии. Подробная сцена смотрин и сцена «Сигурд – Олав – Аллогия» содержат взаимоисключающие описания приближения Олава к Аллогии и Вальдамару.

Но я догадываюсь, как увязать в один сюжет обе сцены. Полагаю, что они были перепутаны местами. Вначале были «смотрины». Ольга выбирает Олава среди множества кандидатов на человека из пророчества. Приближает его к себе. Затем, спустя какое-то время, Олав на торгу встречает злодея Клеркона и убивает его. У кого искать защиту? Конечно у покровительницы. Вот почему Сигурд сопровождает Олава во двор Ольги и та устанавливает охрану до прибытия Владимира.

Только так логично раскладываются обе сцены, и попутно проясняется, почему Сигурд укрывает Олава у Аллогии и она платит выкуп. 

Вернемся к «большой саге». Об Олаве говорится, что даже будучи юным, не одобрял он исполняемые Владимиром ритуалы:

\begin{quotation}
Но одно было, что не любил конунг в нем, что он никогда не хотел восхвалять языческих идолов и противился всяческому жертвоприношению.

Постоянно ходил он с конунгом к храму, но никогда не входил внутрь, стоял он снаружи у дверей храма, в то время как конунг приносил жертвы богам. Конунг говорил часто о том, что ему не следует так поступать, что он вызовет гнев богов и погубит этим цвет своей молодости. «И о том прошу я тебя, – говорит конунг, – чтобы ты славил богов и относился к ним со смирением, поскольку в противном случае я боюсь, что они обрушат на тебя некий ужас своего бурного гнева и зла, такого большого, какому ты себя подвергаешь». 

Олав отвечает: «Никогда не испугаюсь я богов, которым ты поклоняешься, потому что у них нет ни речи, ни зрения, ни слуха, и у них нет никакого разума. И потому, думается мне, я скорее всего могу различить, какова, вероятно, их природа, что мне всегда представляется твое конунгское достоинство, воспитатель мой, и твое лицо с милым и ясным выражением, кроме как когда ты ходишь в храм и приносишь жертвы богам. Тогда видишься ты мне с темным лицом и в то же время несчастным. И после этого знаю я, что эти боги, которым ты служишь, должно быть, правят мраком. И потому я никогда не буду им поклоняться. Но я не буду бесчестить их, так как я не хочу оскорбить тебя».
\end{quotation}

Не приемля богов, правящих мраком, Олав вовсе не миролюбив.

\begin{quotation}
Так говорится, что, когда Олаву Трюггвасону было 12 лет, попросил он своего воспитателя дать ему боевые корабли и войско. Конунг тотчас выполнил его просьбу. 
\end{quotation}

Не навеяна ли эта поспешность княгиней Ольгой? Всё-то ладно получается. Захотелось боевые корабли и войско пацану – пожалуйста! Причем Олав отправился не просто убивать и грабить, но восстанавливая государственность.

\begin{quotation}
Снарядил он тогда свои корабли и войско. И прежде чем он отправился прочь, спросил он у конунга, нет ли там каких-нибудь городов или округов, которые прежде были под властью конунга Гардов, а теперь бы ушли из-под его власти. 

Конунг сказал, что были большие и сильные государства, которые долго находились под властью Хольмгарда, а теперь другие хёвдинги и воины покорили их силой и войной. Как только конунг разъяснил Олаву это дело, Олав впервые вывел свои боевые корабли из Гардов. У него было прекрасное войско, но небольшое. Таков был обычай у викингов, что если сыновья конунга предводительствовали войском, то их называли конунгами, хотя они и не управляли землями. Поэтому воины дали Олаву имя конунга.
\end{quotation}

Тут неясно. То ли Олав открыто всем рассказывал, что является сыном норвежского конунга, то ли его принимали за сына Владимира. 

\begin{quotation}
Обнаружилось тогда вскоре, как хорошо он умел управлять войском, хотя и был юным по возрасту, поскольку он убил некоторых хёвдингов, а некоторых изгнал прочь, из тех, что несправедливостью и пиратством утвердились в землях, обязанных данью конунгу Вальдамару. Имел он с ними большие битвы и во всех одержал победу.

И вернул назад в первое лето все те государства и земли, обязанные данью, которые раньше принадлежали конунгу Вальдамару. Вернулся он назад в Хольмгард осенью, и были у него многие и редкие сокровища, которые он привез конунгу и княгине, из золота и драгоценных камней, и великолепные одежды. Приняли его конунг и княгиня и весь народ с радостью и весельем. Так длилось некоторое время, что Олав по летам был в военном походе и охранял Гардарики смело и стойко от викингов, на него нападали, и подчинил которые он конунгу Вальдамару многие города и округа в Аустрвеге. 
\end{quotation}

Аустрвег – в переводе значит «восточный путь». Считается, что это другое название Руси, наряду с Гардарики, более раннее – и по смысловому употреблению так и получается. Однако зачем тогда в сагах употребляют рядом и Гардарики, и Аустрвег?

В поздних сагах, название Аустревег «смещено» на юго-восток Балтики – Эйстланд, Винладн, Финнланд, Курланд и так далее.

\begin{quotation}
Он часто бывал по зимам в Хольмгарде и имел достойный прием от конунга и любовь княгини. Имел он тогда собственный большой отряд воинов на свои средства, те, что давал ему конунг. Олав был щедр на имущество со своими людьми, поэтому был он любим.
\end{quotation}

Владимир дает Олаву достаточно денег – тот может содержать за них небольшую дружину.

\begin{quotation}
И тогда случилось так, как часто может произойти, если чужеземцы достигают большого могущества или большей славы, чем люди в этой стране, что многие стали завидовать тому, как Олав был любим конунгом и не меньше княгиней. Говорили эти люди конунгу, что он должен остерегаться слишком возвышать Олава, «ибо, – говорили они, – такой человек для тебя всего опаснее, если он захочет посвятить тебя тому, чтобы нанести вред тебе или твоему государству, особенно поскольку он лучше других людей владеет искусствами, любим, обладает физическими и духовными совершенствами. И мы не знаем, о чем они с княгиней постоянно разговаривают». 
\end{quotation}

Не только известность и возвышение Олава вызывают у людей недовольство и подозрение. Спевка Олава с Аллогией. О чем они постоянно разговаривают?

А далее мы увидим странное государственное устройство на Руси при Владимире и Ольге. Хотя составитель саги упоминает «обычай», этот обычай более никогда в исторических источниках не возникает. Вероятно, описанное ниже положение дел показалось странным и составителю саги, а он решил пояснить сие обычаем. Во всей летописной истории Руси, из княгинь только Ольга обладала подобной независимостью и силой.

\begin{quotation}
Таков был обычай могущественнейших конунгов, что княгиня должна была владеть половиной дружины и содержать ее на собственные средства и для этого собирать требуемые дань и налоги. Вот так было и у конунга Вальдамара, что княгиня владела не меньшей дружиной, чем конунг, и они постоянно соперничали из-за родовитых людей. Каждый хотел заполучить их себе. Вот случилось так, что конунг поверил советам этих людей, которые оклеветали Олава. 
\end{quotation}

В чем же его оклеветали? Сага умалчивает.

\begin{quotation}
Сделался конунг несколько раздражительным и сдержанным по отношению к нему. А когда Олав это заметил, сказал он княгине среди прочего и о том, что он хочет отправиться в Нордрлонд\footnote{Норвегия.}, сказал, что его родичи владели там государством. 
\end{quotation}

Запахло жареным, и Олав вдруг вспоминает о Нордрлонде. Струхнул именно Олав, не Аллогия.

\begin{quotation}
«Мне кажется, – раньше говорит он, – что там я всего больше преуспею». Княгиня пожелала ему счастливого пути. Она сказала, что он везде будет считаться благородным, где бы он ни находился. [...]

Вскоре снарядил Олав свои корабли и войско и держал путь из Гардов в Эйстрасальт\footnote{Эйстрасальт, Eystrasalkt – «Восточное море» – сопоставляют с Балтийским морем.}. Корабли те были защищены щитами с обоих бортов, быстроходные и хорошо слушающиеся ветра. И когда конунг Олав плыл с востока, подошел он к Боргундархольму\footnote{Остров Борнхольм в Дании.}, высадился там на берег и грабил. А жители острова подошли к побережью и дали ему битву. Олав одержал победу и захватил много добра.

[Олав плывет на юг, к берегам Винланда. Женитьба Олава на Гейре. Три года в Винланде. Смерть Гейры от болезни.]
\end{quotation}

Виндланд (страна Вендов) – север Германии и Польша. Был населен Вендами (Вандалами) и другими Славянами, которые потом местами были вытеснены, смешались с германскими народами либо онемечились.

\begin{quotation}
Отправился он тогда на корабли и поплыл сначала в Данмарк, а оттуда собирался он на восток в Гарды. Вполне можно думать, что с его горем мог он сперва повернуть туда, где он раньше долго был и образом жизни своим был более всего доволен. [...]

После того события направил Олав свои корабли на восток в Гарды. Был он там неплохо принят конунгом Вальдамаром и княгиней Аллогией. Провел он зиму в Хольмгарде со своими людьми.
\end{quotation}

Какой это примерно год в общепринятой хронологии? Как видим, Аллогия вполне жива еще. Гардарику Олав покинул в 986 или 987 году, три года в Виндланде, туда-сюда пара лет. Ну, скажем, самое начало 990-х. 

Далее, по саге, Олав во сне зрит каменный столб со ступенями. Поднимается на небо, а там красота райская да благоухание, и люди светлые в белых одеждах. Люди эти радостны. Олав слышит над собой голос, одобряющий неприятие Олавом поклонение идолам. Голос вещает – мол, ты хочешь стать «божьим человеком», но еще не вполне готов и не крещен. Отправляйся-ка в Грикланд\footnote{Грикланд – Византия. Примечательно, что наши летописцы называли Византию похоже – «Греки».} и просветись насчет истинной веры. Олав глядит со столба вниз – там разворачивается страшная картина ада, где мучаются язычники, в том числе Вальдамар и Аллогия. Олав плачет и просыпается. Впечатлённый, отправляется на кораблях в Грикланд, где  

\begin{quotation}
отыскал он многих превосходных проповедников, тех, что открыли ему имя господа Иисуса Христа, и говорят, что тогда Олаву было дано prima signatio. 
\end{quotation}

Prima signatio («первая отметка крестом») было ритуалом, при котором христианский священник, под молитву об изгнании злых духов дышал на посвящаемого, а затем рисовал на его лбу крест. Этот ритуал предшествовал крещению, не подменяя его.

Потом сага рассказывает, как Олав способствовал крещению Руси, а поначалу Ольги и Владимира. Привожу почти без примечаний, ибо вопрос крещения Руси выходит за рамки моей книги. Но любопытно, что миссионерская деятельность Олава примерно, плюс-минус половина десятилетия, совпадает с летописным временем крещения Руси при Владимире.

\begin{quotation}
Затем просил он епископа, того, что звали Палл\footnote{В подлиннике – Pall.}, чтобы он поехал в Гардарики и провозгласил там божественное крещение языческим народам. Епископ Палл был великий друг Божий. Он сказал, что он поедет в Гардарики, если Олав поедет вперед и разъяснит его дело, чтобы хёвдинги не противились тому, что он будет насаждать там Божественное крещение. [...]

Отправился тогда конунг Олав назад в Гарды и проповедовал там святую веру, сначала тайно конунгу и княгине. Конунг сначала сильно противился, а княгиня была менее непреклонной. Но все же случилось так со временем, что по побуждению княгини конунг велел созвать многолюдный тинг.

И когда там собралось много знатных людей и множество народа, и был начат тинг, встал Олав и сказал так: 

[Олав рассказывает о преимуществах христианства перед язычеством.]

Конунг Вальдамар ответил на его речь и сказал так:

«Из тех малых лучей, которые сияют моему разуму из твоих прекрасных увещеваний, я думаю, что мораль христиан лучше нашей. Но долговременная привычка к прежней вере удерживает меня, так что я плохо понимаю, что более достойно. И кроме того, мой разум говорит, что было бы безрассудством покинуть ту веру, которую мои родичи и предки сохраняли всегда во всю свою жизнь, один после другого. Поэтому я хочу об этом трудном деле послушать мнение сначала княгини, которая гораздо умнее меня, потом всех других хёвдингов и наших советников».
\end{quotation}

Конечно, Владимир поступит только как скажет Ольга. Видно, кто принимает самые важные решения. Ей слово! 

\begin{quotation}
Была тогда большая овация после речи конунга. И когда наступила тишина, княгиня начала свою речь так: 

«Этот муж Олав пришел к тебе, конунг, тогда, когда он был ребенком по возрасту, когда он только что вернулся из изгнания и большой неволи; принял ты его, чужеземного и незнакомого, в твою милость, так что ты взрастил и воспитал его с такой любовью, как своего собственного сына. Он так сумел употребить это в свою пользу, что укрепил и увеличил твое государство со всем благорасположением, как только смог что-то предпринять по своему возрасту и силам; поэтому он был любим всеми добрыми людьми. Теперь он отсутствовал некоторое время и сделался другом и добрым советником тем хёвдингам, которым он не так должен был отплатить добром, как тебе. Кажется также, что он с великим и большой усердием серьезностью предлагает то приличествующее дело, о котором он печется и которое всякому разумному человеку покажется спасительным. Поэтому моя совесть дает мне понять, что твоя мать, конунг, предвидела некогда этого самого мужа и что многие другие предсказатели и мудрые люди этого государства предсказали, что один чужестранный человек будет здесь воспитан, тот, что не только украсит это государство ярким светом своей мудрости и знания, но чья благость прекрасно расцветет повсюду в других местах. Я видела это прежде на его лице и полюбила его тотчас и навсегда с тех пор более, нежели других юношей. Теперь это справедливее, чем то, что люди пытаются подозревать, будто нечто порочное может скрываться под нашей любовью».
\end{quotation}

Насколько эта речь принадлежит княгине, а насколько – монаху, составителю саги – сказать трудно. Кажется, Аллогия открывает здесь тайну своих тесных отношений с Олавом – мол, мы сошлись на религиозной почве, а люди подозревали нечто порочное.

\begin{quotation}
Княгиня кончила свою речь так, что все похвалили ее и красноречие мудрость. Тем закончился тинг, что все Божьей помощью и убеждением княгини обещали принять истинную веру. В это время прибыл из Грикланда епископ Палл по просьбе конунга Олава, и крестил конунга Вальдамара и княгиню Аллогию, и весь народ их, и утвердил их в святой вере.
\end{quotation}

Тинг завершается обещанием «всех» принять истинную веру. Вчитаемся в текст. Божьей помощью и убеждением княгини. Ни Владимир, ни Олав не оказали влияния на тинг. Убеждение княгини! И более о ней в саге – ни слова!

\begin{quotation}
То, что было сейчас рассказано о христианской проповеди Олава Трюггвасона в Гардарики\footnote{Без помощи Аллогии слова Олава прозвучали бы вхолостую.}, не является невероятным, потому что превосходная и достойная веры книга, которая зовется Imago mundi, говорит ясно, что те народы, которые зовутся руссы, полавы, унгарии, крестились во дни Оттона, того, который был третьим императором с этим именем. Некоторые книги говорят, что император Оттон ходил со своим войском в Аустрвег и повсюду там привел народ к христианству, а с ним Олав Трюггвасон.

Вслед за тем снарядил конунг Олав свои корабли и войско с востока из Гардарики. Он поплыл сначала в Данмарк, а оттуда на запад за море. Так говорит Халлар-Стейн: 

«Все корабли князя, весьма многочисленные, полностью нагруженные вооружением, после этого сразу отправились с щедрым князем из Гардов. Олав, прекрасный наследник, Трюггви, напал на Вестрлонд\footnote{Британские острова.} со своими кораблями и порубил людей своими мечами».
\end{quotation}

Олав воюет в Энгланд и Вестлонд, затем посещает прорицателя на Сюллингах. Скрывает своё имя, называясь Оли из Гардов. Собирается вернуться в Норвегию, но у него мало войска, а в Норвегии – друзей. Несколько ранее Лодин выкупает из рабства в Эйстланде мать Олава, Астрид, и женится на ней. Ярл Хакон подозревает, что тот, кого называют Оли – сын Трюггви. Наводит справки. Поручает своим хёвдингам разобраться с Олавом – пленить и привезти в Норвегию или немножко убить. Торир, человек Хакона, находит Олава Гардского в Дублине у конунга Олава Кварана. Гибель Хакона. Тинг – Олава хотят избрать конунгом.

Речь Олава на тинге – про Аллогию умалчивает, упоминая только доброту «хорошего хёвдинга». Важно с точки зрения чисел:

\begin{quotation}
«Затем, когда мне было три года, отправились мы с моей матерью из Свитьод на корабле и направлялись на восток в Гардарики к Сигурду, ее брату. Тогда встретили мы викингов и были взяты в плен, и проданы в рабство, а некоторые из спутников были убиты; расстались мы тогда с моей матерью, так что я ее никогда с тех пор не видел. Я был тогда продан за ту же цену, что и другие люди. Я провел 6 лет в этом рабстве в Эйстланде, до того времени, пока Сигурд, брат моей матери, не выкупил меня оттуда, и не привез меня оттуда с собой на восток в Гардарики. Тогда было мне 9 лет. Другие 9 лет был я в Гардах, еще в изгнании, хотя, благодаря доброте хорошего хёвдинга, обо мне там прекрасно заботились. Затем я провел 3 года в Виндланде и 4 года за Вестанхав в военном походе».
\end{quotation}

В другом месте саги, рассказ Олава Сигмунду Брестисону о себе – снова умалчивая Аллогию:

\begin{quotation}
«Вскоре были мы оба схвачены викингами. И расстался я тогда со своей матерью, так что я ее никогда с тех пор не видел. Был я тогда трижды продан в рабство, пробыл я, никому неизвестный, в Эйстланде, пока мне не исполнилось 9 лет. Тогда приехал туда один мой родственник, который узнал, какого я рода, выкупил он меня из рабства и взял меня с собой на восток в Гарды. Там я провел другие 9 лет, еще в  изгнании, хотя меня и называли тогда свободным человеком. Там я возмужал и получил много больше чести и достойного отношения от конунга Вальдамара, чем можно было бы представить по отношению к чужестранцу».
\end{quotation}

Сага про Олава – большая, я привел, за редким исключением, лишь куски, касающиеся Аллогии. 

Что про Аллогию говорят ученые? Не верят. Мол, скандинавы напутали, надо было вместо «Аллогия» всюду писать «Рогнеда» – им же, историкам нынешним, из наших веков видней! Другие возражают, что скандинавы ничего не напутали, а намеренно придумали. Горазды на выдумки! Те ученые, что поосторожнее, вообще имени Аллогии не употребляют письменно или всуе, заменяя сухим «княгиня».

Никто не размышляет над тем, что Аллогия предстает в саге точно такой, как летописная княгиня Ольга – говорю в первую очередь об её редком положении, заключающемся в независимости и равноправном участии во власти, а по сути обладании большей власти, чем у князя. Говорю также про ум, перед которым преклоняется что составитель саги, постоянно подчеркивая «умнейшая из женщин», так и наши летописцы. Когда личность описана одинаково – и притом личность яркая, не спутаешь с другой – и личность носит в одном случае имя Ольга, а в другом Аллогия, только очень упорный человек будет отрицать, что речь идет о разных личностях.

Ученые закрывают глаза на то, что в саге Аллогия играет более важную роль, чем Владимир. Владимир, словно Игорь при Ольге – официальное представительное лицо, князь, но решения принимаются Аллогией, а выполняются Владимиром.

Увязывать сагу с русскими летописями я не берусь – слишком много разночтений, нет четкой опоры. То, что я посчитал важным, отметил ранее. Думаю, что Вольга успешно дожила до «эпохи» Владимира, тем или иным образом – я не буду предполагать, как это соотносится с лежанием нетленной в гробу, и как соотносится с саркофагом, где было некое чудесное окошко.

%Любопытное дело получается еще с монетами. Считается, что в Киевской Руси первым начал чеканить монеты Владимир Красно Солнышко, золотые и серебряные. Первые назывались «золотниками» – один золотник был того же веса, что византийский солид, и даже названия их похожи, даже и корень «золото» – «жолото» – хорошо обозначает желтый цвет употребляемого металла.

%Так вот, на русских монетах изображались правители, современные выпуску монет. На монетах времен Владимира, наряду с характерным тризубцем, самим Владимиром и Иисусом Христом, встречаются монеты, где на одной стороне – тот же тризубец, на другой – княгиня Ольга! Даже подписано – «Олга».

%И Владимир, и Ольга изображены с нимбами, как византийские императоры – указание на высокий сан. Ведь святыми Владимир и Ольга стали почитаться позже, по крайней мере не при жизни самого Владимира.

Иногда вместо Аллогии жену Владимира источники, так или иначе связанные с Олавом, называют Арлогией. Есть историческая книга на латыни «Scripta historica Islandorum de rebus gestis veterum Borealium, latine reddita et apparatu critico instructa, curante Societ\-te regia antiquariorum septentrionalium» – свод пересказов на латынь исторических саг, изданный в 1842 году. В первом томе речь идет и про Олава, а также Валдамара и жену его Аллогию. Дается примечание с вариантами имени Allogia – по крайней мере два источника передают это имя как Arlogia.

В книге на латинском «Historia rerum Norvegicarum in qua primordia gentis, instituta, mores, incrementa, successi\-ones, genealogia, chronologia, etc., exponuntur». 1711 года, об истории Норвегии, в известном сюжете об Олаве тоже упомянута Арлогия. В издании «Саги о конунге Олаве Трюггвасоне» 1825 года дается вариант Арлогия в придачу к основному Аллогия.

У иностранцев издавна страсть к изучению своих корней, происхождения. По тем же сагам и различным хроникам составлены огромные, ветвистые деревья – кто и когда кого породил, и где сочетались браком, и где умерли. И что вы думаете?

В шотландских генеалогических работах мелькает уже дочь Владимира Красно Солнышко, с легко узнаваемым именем – Арлогия, но теперь это графиня Оркни (Arlogia Vladimirovna, Arlogia Orkney, Arlogia Brusesson, Countess Arlogia (Ardogia) of Russia).

Острова Оркни (Orkney). Ныне принадлежат Шотландии. В «Истории Норвегии» говорится, что прежде они были заселены Пиктами, ростом чуть выше пигмеев, искусными в строительстве городов. Днем они скрывались в подземных жилищах, ибо в это время теряли силы. Ранее острова назывались Пиктланд, а затем – Оркни по имени правителя Оркана. Вместе с Пиктами жили Папары, носившие шапочки «как у священников». Средневековые хронисты, видевшие рукописи этих Папаров, думали, что те были Африканцами и исповедовали иудаизм. Папаров уничтожили пришлые на острова викинги во время норвежского короля Харалда Прекрасноволосого (Haraldr Harfagri). Северные конунги совершили немало набегов на острова Ирландии и Шотландии, а Оркни вообще стали одним из оплотов северян. На Как мы помним, на Оркни бежала мать Олава, Астрид.

На главном острове Оркни существует округ St Ola, то бишь «Святая Ола». Про Ольгу ученые конечно же не задумываются, и выводят Олу от Святого Олава – дескать, стояла в его имя церковь, построенная в 1035 году. Ну может и так. 

Аллогия и Арлогия. Кроме сходства имен, связь с Оркнейскими островами через Олава, вероятно рожденного там.

Что же можно узнать про эту, «дочернюю» Арлогию? 

Ничего кроме дат, а также имен родителей, мужа и детей. Даты вычислены невесть каким образом, невесть когда и кем. Но добро бы, путаница была только на уровне дат. Путаница на уровне даже родителей Арлогии.

Кто отец? По большинству источников – наш Владимир. По меньшинству – Валдемар де Оркни (Waldemar de Orkney, умер в 1015). Наш Владимир, Красно Солнышко, тоже умер в 1015. Полагаю, речь идет таки об одном человеке.

Кто же мать? А на выбор – Анна Лекапене Порфирородная (жена Владимира из Византии), Рогнеда Полоцкая, Ингебёрг Арнесон (Ingebiorge Arneson, родилась в 1032 году в Osteraat, Yrje, Норвегия, умерла возможно 1069 в Шотландии). Эта Ингебёрг Арнесон присутствует только в связке с супругом Waldemar de Orkney.

Год рождения Арлогии – тоже на выбор: 980, 1005, 1011, 1013, 1015, 1025. Место рождения – Русь (Киев) или Шотландия (Оркни). Год смерти – 1046 (10 декабря) либо 1062, 1100, или неизвестно. Похоронена – если – в Северном Оркни (Norse Orkney).

С 1029 или 1034 или 1040 года была женой ярла Оркни, Рагнвалда II Брусессона (Ragnvald II Brusesson, Earl of Orkney, убит вроде бы в декабре 1046, в Papa Stronsay, Оркни). Ярл происходил из того же рода Брюс, что Снорри Стурлуссон и упомянутый уже Яков Вильгельмович Брюс, ученый, ближайший сподвижник Петра I, прослывший чародеем. Отметим еще сходство звучания Брус и Прус\footnote{Брус также имя одного из сыновей готского короля Сенубалда из Книги Готской (Славянского королевства) попа Дуклянина. У короля было три сына – Брус, Тотила, Остроил. Имена их потомков славянские: Сенудслав, Силимир, Ратомир, Бладин, Сарамир, Светоплек (последний относится ко времени Кирилла и Мефодия). Готы прибыли «с севера» в земли Балкан.}.

У Арлогии и Рагнвалда родились дети Robert de Brusse, Earl of Annandale (1030-1080/1098), Waldemar de Orkney (1035), Hamiliana de Orkney (1037), Tora Ragnvaldsdottir (1045-1048), Arlogia de Orkney (1038). 

То бишь, по генеалогическим изысканиям, у дочери Владимира, Арлогии, в 1038 году появилась дочь, тоже Арлогия. Годы смерти всех детей, кроме Роберта и Торы, неизвестны. Один из сыновей Арлогии, Волдемар де Оркни – муж Ингебёрг Арнесон, и было бы логично думать, что сын Арлогии Волдемар женился на Ингебёрг, если бы Арлогия Оркни не была записана дочерью Ингебёрг.  

Что мы имеем для рассуждений? В Шотландии, Оркни, появляется некая дочь киевского князя Владимира, Арлогия. Кто ее мать, точно сказать нельзя. Когда родилась – сказать нельзя. Когда умерла – тоже сказать нельзя. Да и приходится ли она Владимиру дочерью? По нашим летописям та же Ольга – бабушка Владимира, а по сагам – супруга.

Но что наши летописи? Говорят ли о дочерях Владимира? Упоминают Предславу. А о жене Владимира – Анне? Ведь Анну прочат в одну из матерей Арлогии.

Анна Порфирородная (Anna Lekapene, Anna Porphyro\-genita) была, как считается, дочерью Романа II и Феофано (той самой любовницы Цимисхия). Братьями Анны были Василий II и Константин VIII. 

Согласно Ипатьевской летописи, когда Владимир в 988 году взял Корсунь (сейчас – Севастополь) в осаду, некий корсунянин послал стрелу, на которой было написано, где расположен водопровод, по коему в город поступает вода. Владимир раскопал трубу, перекрыл, и в Корсуне наступила жажда\footnote{По более подробному житию Владимира, стрелу с запиской пускает корсуньский Варяг Жберн (Ижберн, Ждьберн), сообщая о подземном ходе для доставки припасов. Воеводой же Владимира при осаде Корсуня назван князь Олг (Ольг). Взяв Корсунь, Владимир свергает местного князя и княгиню – насилует на глазах у них дочь, а князя и княгиню убивает, поруганную дочку же ихнюю выдает замуж за Жберна. Это перекликается с тем, как Владимир поступил относительно Рогнеды.}. Пришлось Владимира впустить. 

Из Корсуня он послал Василию и Константину в Царьград угрозу – выдайте за меня замуж сестру свою, иначе и столицу вашу возьму, как Корсунь. Царственные братья выдвинули ответное условие – крестись, тогда выдадим. И посылают сестру со священниками.

Корсуняне принимают гостью, вводят ее во град и «посадиша ю в полате». Имя сестры василевсов – Анна – появляется дважды, далее всюду только «царица», причем не «княгиня».
 
И вот что любопытно – Владимир-то, по летописи, уже выбрал веру, и до нападения на Корсунь стремился креститься, а взяв Корсунь, требует не крещения, но жены василевского роду. В Корсуне есть представители духовенства, казалось бы – крестись, никто не мешает. Не крестится. Царица же сидит в полате.

И тут Владимир неожиданно слепнет. 

\begin{quotation}
Не видяше ничтоже, и тужаше велми, и не домышляше, что сотворити; посла к нему цариця, рекуще: «аще хочеши болезни сия избыти, то вьскоре крестися; аще ли ни, то не имаеши избыти сего». 
\end{quotation}

Владимир соглашается: «аще се истина будет, по истине велик Бог крестьянеск».

Крестят Владимира епископ корсуньский (вероятно Настас, как везде далее) и «попы царицыны». По возложении руки епископа Владимир прозревает. Увидев сие чудо, принимают крещение и многие из дружины Владимира\footnote{В некоторых списках жития Владимира, он не слепнет, но покрывается струпьями. Струпья отпадают при крещении.}.

Затем, по летописи, Владимир собирает царицу, Настаса, попов корсуньских, берет мощи святого Климента и ученика его Фива, сосуды церковные да иконы «на благословенье себе», насыпает посреди города холм и ставит на нем церковь Иоанна Предтечи. После, присваивает два медных «капища» и четырех медных коней, «иже и ныне стоять за святой Богородицею; яко иже не ведуще мняться мраморны суща». И вот Владимир:

\begin{quotation}
Вдасть же вено Корсунь Греком царице деля, а сам прииде Кыеву. И яко приде, повеле кумиры испроврещи, овыи сещи, а другыя огньви предати; Перуна же повеле привязяти к коневи хвосту и вещи с горы по Боричеву на ручай, и 12 мужа пристави бити жезлием. [...]

Велий еси, Господи, чюдная дела твоя! вчера честим от человек, а днесь поругаем.
\end{quotation}

Что же значит «Вдасть же вено Корсунь Греком царице деля, а сам прииде Кыеву»? Вено (похоже на латинское venum – «продажа») – приданое со стороны жениха. Владимир отдает взамен невесты Корсунь. «Царице деля» значит «царицы ради», «царицы для». Можно трактовать – отдает Корсунь Грекам за царицу. Так вот зачем он город брал. Чтобы не своё потом отдать.

А ежели трактовать иначе – отдает Корсунь для царицы? В ее владение?

Крестить Киев Владимир прибывает с попами царицыными и корсуньскими, они в летописи разделяются. Анна возникает в летописи снова лишь в 1011 (6519) году по поводу своей смерти:

\begin{quotation}
Преставися царици Володимиря Анна.
\end{quotation}

Насчет Анны сведения имеются путаные. Согласно устоявшимся в науке представлениям, годы ее жизни – с 962-963 по 1011. Но в разных генеалогических списках есть и другие: 954-982, 955-990, 976-1056. Допустим, ошибаются списки.

Но вот например по датам из летописей можно предположить, что Ярослав Мудрый – сын Владимира и Анны, а не Рогнеды (последнее прямо утверждается в летописи). 

Вычисляется, что родился Ярослав в 978 году – умер в 1054 году в возрасте 76 лет. Отнимем и получим 978. Но историки смотрят на летописную дату взятия Корсуня, 988-й, читают тоже летопись и говорят – не может быть, чтобы Анна была матерью Ярослава! Матерью была Рогнеда! Вот же Нестор такоже пишет – 28-летний Ярослав правит в Новгороде, в 1016 году... Тоже вычитаем, получаем 988. Взятие Корсуня, женитьба на Анне. Сын! 

И получается, что в летописях заложено противоречие, у Ярослава две матери, Анна и Рогнеда.

Об Анне Порфирородной мы ничего не знаем, кроме отрывочных сведений из «житий» Владимира и Стефана Сурожского. В первом Анна, в пределах летописных сведений, помогает Владимиру насаждать христианство, в другом – некая царица Анна заболевает на пути из Корсуни.  Причем нет ни Владимира, ни привязки к дате, просто сказано следующее:

\begin{quotation}
Об исцелении царицы корсуньской 

И Анна царица, от Корсуня в Керчь идучи, разболелась смертным недугом среди пути на Черней Воде. На ум ей пришел (святой Стефан), и она сказала: «Святой Стефан, если меня избавишь (от этой болезни), я воздам тебе многими дарами и че­стью».

И в туже ночь явился ей святой Стефан и сказал: «Христос, истинный Бог наш, велит через меня, своего слугу, так: завтра встань здорова и иди своим путем с миром». И тотчас недуг отступил, и была она здорова, как будто никогда и не болела вовсе. 

Она же, почувствовав исцеление, вознесла добрые хвалы Богу и святому Стефану. И все те, кто были с нею, встав утром, с великой радостью пустились в путь свой, и проповедовали дела Божьи, бывшие (явлены) его угодником.
\end{quotation}

Царица Анна возникает в житии Сурожского ниоткуда, проездом, а затем со свитой исчезает в никуда.

Византийские источники об Анне молчат, кроме упоминания Иоанна Скилицы в «Обозрении истории» о рождении дочери Романа, Анны, за два дня до его смерти: «Приняли царскую власть Василий и Константин, его сыновья, с матерью Феофано, а также осталась после него родившаяся за два дня до его кончины дочь, которую назвали Анной». Отмечу, не указано от какой матери дочь, хотя по смыслу можно полагать, что от Феофано.

Про царицу по имени Анна – всё. А вот немец Титмар Мерзебургский (975-1018) пишет о Владимире\cite{nazarenco01}:

\begin{quotation}
Продолжу рассказ и коснусь несправедливости, содеянной королем Руси Владимиром. Он взял жену из Греции по имени Елена, ранее просватанную за Оттона III, но коварным образом у него восхищенную.

По ее настоянию он (Владимир) принял святую христианскую веру, которую добрыми делами не украсил, ибо был великим и жестоким распутником и чинил великие насилия над слабыми данайцами. [далее идет про то, как Владимир заточил епископа колобжегского, Рейнберна, в темницу, а также сообщается еще много чего нелестного про Красно Солнышко]
\end{quotation}

Ученые упорно понимают тут «Елена» как «Анна». Хотя мы знаем и они знают, что Елена – христианское имя княгини Ольги. Титмар сообщает также, что Владимира похоронили в Киеве, гробы его да супруги стояли посередине храма, церкви мученика папы Климента.

В 892-899 годах, в Прумском монастыре жил да был аббат Регинон. Потом его из монастыря прогнали, он поселился в городе Трире, где и провел остаток дней до своей кончины в 915 году. За этот промежуток времени Регинон составил на латыни хронику, имевшей потом ход в Германии и Франции. Сам Регинон завершил написание хроники годом 906, а от 907 и последующие 60 лет ея продолжил неизвестный сочинитель, в коем некоторые предполагают магдебургского архиепископа Адальберта. 

Как бы ни было, именно в этом продолжении\cite{nazarenco01} излагается странная история о том, что послы Ольги (Елены) просили у Оттона I епископа и священников, и что из этого вышло:

\begin{quotation}
В лето от воплощения Господня 959-е. Король\footnote{Оттон I.} снова отправился против Славян, где гибнет Титмар. Послы Елены, королевы Ругов, крестившейся в Константинополе при императоре константинопольском Романе, явившись к королю, притворно, как выяснилось впоследствии, просили назначить их народу епископа и священников. [...]

960. Король отпраздновал Рождество Господне во Франкфурте, где Либуций из обители святого Альбана посвящается в епископы для народа Ругов достопочтенным архиепископом Адальдагом. [...]

В то же лето король снова выступил против Славян. [...]

961. [...] Либуций, отправлению которого в прошлом году помешали какие-то задержки, умер 15 февраля сего года. На должности его сменил, по совету и ходатайству архиепископа Вильгельма Адальберт из обители святого Максимина, [который,] хотя и ждал от архиепископа лучшего и ничем никогда перед ним не провинился, должен был отправляться на чужбину. С почестями назначив его [епископом] народу Ругов, благочестивейший король, по обыкновенному своему милосердию, снабдил его всем, в чем тот нуждался. […]

962. [...] В это же лето Адальберт, назначенный епископом к Ругам, вернулся, не сумев преуспеть ни в чем из того, чего ради он был послан, и убедившись в тщетности своих усилий. На обратном пути некоторые из его [спутников] были убиты, сам же он, после больших лишений, едва спасся. Прибывшего к королю [Адальберта] приняли милостиво, а любезный Богу архиепископ Вильгельм в возмещение стольких тягот дальнего странствия, [которого] он сам был устроителем, предоставляет ему имущество и, словно брат брата, окружает всяческими удобствами. В его защиту [Вильгельм] даже отправил письмо императору, возвращения которого [Адальберту] было приказано дожидаться во дворце.
\end{quotation}

Под Ругами (Rugis), как считают, подразумеваются Русы\footnote{В Анналах Ламберта Херсфельдского сюжет с посольством к Оттону и Адальбертом датирован 960 годом, имен со стороны посланцев нет, названы они просто «legati Rusciae» и сказано, что отправленный по запросу католический епископ Адальберт потом едва унес оттуда ноги.}. В подлиннике, Елена-Ольга именуется так: Helenae reginae Rugorum. Написано, что крестилась она при Романе, но каком, не уточняется. Между Константином Багрянородным и Иоанном Цимисхием были Роман I Лекапенос (правление: 920-944), Роман II Багрянородный (правление: 959-963) и Никифор II Фока. Кстати на царствование Романа II выпадает и часть патриаршества Полиевкта, а с 949 по 956 Роман II был вдовцом, после чего женился на уже упоминаемой Феофано, которая потом была женой Никифора и любовницей Цимисхия.

Так вот, получается, что теоретически вопреки именам василевсов в русских летописях, если Ольгу таки крестили в 955, Роман II таки мог к ней свататься.

У меня же душа больше лежит к тому, что Ольгу крестили Цимисхий с Полиевктом в 969 году – году, который в летописях указан как год смерти Ольги. Совпадение этих дат не может быть случайностью. Как всё увязать – я не знаю. 

Могу предположить, почему в ряде летописей Цимисхий заменен на другого василевса, а имя патриарха то вовсе не указано, то сказано «Фотий» (как наиболее известный греческий патриарх, носящий этот сан с перерывами от 858 по 886 годы). По разным причинам летописцам не нравился Цимисхий в качестве крестного отца, и подыскали ближайшего по времени и известности. Посему и писали – то Константин, то Михаил.

Что до Адальберта, то выглядит довольно странным, что крестившись в Византии, Ольга отправляет за латинскими епископом и священниками. Оттенборейские анналы, в отличие от «Продолжения Регинона», относят сие посольство к 960 году. О посольстве к Оттону русские летописи молчат.

В «Саксонских анналах Квендлибургских» (Saxonicae Annal\-es Quedlinburgenses) прибытие послов от русского народа к Оттону с просьбой направить на путь христианства и прислать пастыря датируются 960 годом (по имеющемуся у меня изданию) – и король посылает католического епископа Альберта, который потом едва спасся от коварства сих язычников, которые, по словам хрониста, солгали во всем. Что именно затеяла тогда Ольга, остается неясным.

Магистр Адам Бременский в «Деяниях архиепископов Гамбургской церкви» пишет о тех же, времен Отона I, временах, и рассказывает об архиепископе Адальдаге (не Адальберте), что проповедовал среди славянских народов, не уточняя каких именно. Одновременно с ним, среди Скифов (Адам Бременский называл народами Скифии, кроме прочих, Славян и Сембов) был миссионером другой архиепископ, Унни. 

Представление о том, что Ольга была женой Владимира, и что именно Ольга, а не Владимир, распространяла на Руси христианство, находим в записках Павла Алеппского\cite{sbornikmat}: 

\begin{quotation}
Свет христианской веры пришел сюда с востока во времена императора Василия Македонянина\footnote{811-886 годы жизни, император Византии в 867-886 годах} за шестьсот пятьдесят лет до настоящего времени, как можно видеть по числам на дверях церквей и монастырей.

Это произошло вследствие супружества русского короля Владимира с сестрою императора Olikha\footnote{Г. Муркос в своем переводе с арабского пишет это имя как «Олиха».}; она прибыла в страну с митрополитами и епископами, которые крестили русского монарха и весь его великий народ, который, как говорят историки, не имел никакого понятия о святом законе, не исповедовал никакой религии; новая царица построила множество церквей и монастырей, созданных руками константинопольских мастеров; по этой причине все надписи на них на греческом языке.

В это время все народы, обитавшие около Киева, были язычники, то были: поляки, москвитяне, татары и пр., и находились в постоянной войне с царицею; но она одержала над всеми победу и распространила свет христианской веры между ними. Все они уверовали, исключая татар.
\end{quotation}

Вот снова Владимир играет второстепенную роль. Olikha и религию принесла, и в постоянной войне с окружающими народами была. Алеппский указывает, что Ольга – сестра императора, то же говорят русские летописи, но меняют имя на Анну.

В наших летописях, кого связывает родство по отцу Роману II и матери Феофано? Василия, Константина и Анну. И если к Ольге при крещении сватался Роман II, а не Цимисхий, то Роман II был Ольге крестным отцом, а Василию и Константину – родным. При истинности этого условия, кто же тогда летописная княгиня Ольга по отношению к Василию и Константину? Крестная сестра. Можно ли сказать тогда, что Ольга – сестра императора Василия? С натяжкой можно.

Это рассуждения. Не могу сбрасывать совпадение годов, время когда Цимисхий бе вдов и имел свободу свататься к Ольге, с летописным общепринятым временем смерти Ольги.

Но вернемся к Арлогии и Брусам. Основателем рода Брусов считается Brusi Sigurdsson, ярл островов Оркни, в Шотландии. Родился, как считается, в 987 году, умер в 1031. Это его сын, Ragnvald (Regenwald, Ronald) II Brusesson, записан в мужья Арлогии. И мы знаем, что Robert de Brusse (1030-1080/1098) отмечен по генеалогическим деревьям как сын Арлогии.

Этот Роберт Брус Первый (умерший до 1098 года) построил замок Брус (Bruis) около Шербурга, в нынешней Франции. Там сейчас есть поселение, вернее община Brix (название образовалось через цепочку местных искажений Бруса –  Brucius, Bruce, Bruys, Bris, Brix). В 13 веке замок Брусов был разрушен, и камни от него пошли на постройку церкви и домов селения. Теперешний замок в Бриксе, потихоньку посещаемый туристами – не тот, древний, а моложе.

После Роберта Бруса Первого, по генеалогическим спискам проходит с десяток Робертов Брусов, прослеживаемых с 1078 по 1306 год, когда умер Роберт Брус, король Шотландии, оторвавший свою страну от подчинения Англии. Сего короля Роберта Бруса\footnote{Современное произношение – нечто среднее между Брус и Брюс.}, умершего в 1306 году, тоже называют Робертом Брусом Первым, и так он вошел в историю. 

К чему я подбираюсь?

К Рюрику и его братьям. К знаменитым по варяжскому вопросу Русам или Прусам.

Соберем воедино отрывочные сведения. Вещий Олег – родич Рюрика, возможно племянник. Я уверен, что Олег это Ольга, она же Аллогия, вероятно и Арлогия. А Арлогия породнена с Брусами.

А Брус звучит так же, как Прус.

Вспомним один из списков, о призвании Варягов княжить:

\begin{quotation}
В лето 6405-го приидоша словяня из нова града великого  торговати за море к варягам в немецкую область, во град, нарицаемый прусы и рекоше словяне князем варяжским [...]
\end{quotation}

Во град, нарицаемый Прусы. Град это же крепость. Не замок ли Брус имеется в виду?

Но как быть с датами? И с тем, что Прус это некий брат императора Августа? Тот же вопрос можно отнести и к «граду Малборку», еще одной летописной родине Рюрика – Малборк также не вписывается в общепринятые годы появления Рюрика на историческом поприще.

Не утверждая действительную связь Бруса и Пруса, я не могу пройти мимо сходства и предлагаю – а если так? Беря разные совокупности источников, мы невольно получаем разные варианты развития событий в прошлом, и варианты порой весьма правдоподобные. Но какой предпочесть? На главный вопрос – как было на самом деле – нельзя дать ответ, не покривив душой и не закрыв глаза на другие варианты. 

Эта нечеткость, двойственность истории напоминает отношение, скажем, жителей Исландии к нагромождениям камней, к их толкованию, что это – просто камни или жилище эльфов. Можно так, а можно эдак. Однако нам привычно мыслить, принимая из истинный лишь один вариант и полагая другой ложным. Сколь применимо это всегда и везде?

%Чушь горожу, кажется! Если у меня в руке карта, допустим семерка пик, она не может одновременно быть семеркой треф или дамой бубен. Так же и развалины где-то посреди поля или на холме не могут в то же время быть дворцом эльфов.
