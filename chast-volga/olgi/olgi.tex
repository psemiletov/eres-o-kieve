\chapter{Вещий Олег и княгиня Ольга}

Как вы отнесетесь к мысли, что Вещий Олег и княгиня Ольга – одна личность?

Эта часть книги не зря названа «Вольга». Точное именование личности, что сначала выступила на исторической сцене Вещим Олегом, а затем  княгиней Ольгой. Вольг\'а – ударение ставится на последней букве. Как у дерева «ольха». По сути, это одно слово, ведь имя было прозвищем.

В летописях и былинах разнобой: Олег, Ольг, Олг, Ольга, Вольга. Княгиню Ольгу часто называли Вольга. Но Вольга – былинный богатырь, оборотень, сын «змея». Букву «г» в старину вероятно произносили мягко – как мы говорим теперь «мяхко» вместо «мягко». Вольга и Ольга летописная звучали как «Вольха» и «Ольха». Поставьте ударение как в дереве и будет счастье.

А на арабском дерево ольха звучит как «яр альм\'а». Это совпадает с летописной «Олмой» или «Альмой», которая поставила на Аскольдовой могиле церковь!

Кстати ольха относится к березовым. Некоторые виды ольхи похожи на обычную белую березу, в связи с этим название Берестове, возможно происходит от Ольхи. Слишком много совпадений, и легко отождествить место теремного двора Ольги, Ольмина двора и княжего двора на Берестовом.

%Предположение выглядит поначалу фантастической чепухой, но если рассмотреть множество источников, приобретает странную связность. Да и 

Однако я предлагаю отождествление и более странное. Князь Олег и княгиня Ольга. При беглом знакомстве даже с историей в традиционном изложении выстраивается любопытная цепочка.

Судите сами.

Вот Олег княжит при малолетнем Игоре. Олег ведет более чем успешные войны, устанавливает дани подчиненным землям. Игорь взрослеет. Олег находит ему жену Ольгу. Игорь женится, затем умирает Олег. Вот тут происходит летописная заковырка с датами. 

По одним спискам, принятым академической наукой за правильные, разница между женитьбой Игоря и смертью Олега – с десяток лет, то бишь временные отрезки жизней Вещего Олега и княгини Ольги частично накладываются взаимно. По другим спискам, свадьба Игоря и смерть Олега происходят чуть ли не в один год.

Я не верю летописным датам и не рассматриваю их всерьез. Ученые – те верят, ну да каждый верит во что вздумается. Я вот в следующее.

Вещий Олег держал власть крепко. Уже давно Игорь стал совершеннолетним, а регент всё княжит! Доколе? Вероятно, такое положение по каким-то причинам не могло продолжаться. Но Вольга и от власти отступиться не мог. Он (она?) должен был оставаться владетелем. 

И разыгрывается чудное дело. Вольга оставляет свой образ князя Олега и является в образе уже княгини Ольги, жены Игоря. Как член правящей семьи, Вольга в образе Ольги остается при власти, снова рядом с законным князем-наследником, Игорем. Вещий Олег же – за ненадобностью, «умирает». Либо замена Олега на Ольгу произошла чисто на бумаге, одного человека раздвоили при правке очередного списка, ибо деяния Вольги показались слишком дивными и писец решил, что это два разных человека, хотя с одинаковыми именами, но один мужчина, а другой женщина.

«Любопытно, даже логично, но чушь», – возразит любой человек, знакомый с историей по учебнику и книгам, выдержанным в ключе общепринятых представлений.

Хорошо. Двинемся дальше. Вот Игорь воюет, собирает дани, решает с Древлян взять дань повторно, Древляне привязывают его к двум деревьям и разрывают пополам. Известный сюжет.

И тут вдруг. Княгиня Ольга, доселе мышка тихая, жестоко мстит Древлянам, затем возглавляет победоносный военный поход, а после идет с воинством до Новгорода, утверждаясь во власти, восстанавливая структуры управления, созданные Олегом, и заново подчиняя некогда им же подчиненные земли.

Почему княгиня смогла вообще удержать власть? Это первый вопрос. Второй вопрос. Откуда у Ольги появились умения и опыт всё это провернуть?

Таким опытом она могла обладать, лишь будучи той же личностью, что Вещий Олег. Так Ольга-Вольга, после смерти Игоря, с маленьким Святославом на руках (опять выигрышное регенство!), снова производит подчинение соседей Киеву, как он делал это в роли Олега-Вольги с маленьким же Игорем. 

Вольга как Олег и Ольга, под одинаковыми именами, и ведут себя совершенно одинаково – жесткость, жестокость, изобретательность в войне (корабли на колесах у Олега, пожар от птиц в городе древлян – у Ольги), беспроигрышность войн.

Кем был или была Вольга – мужчиной или женщиной? Была ли это женщина, которая поначалу играла мужскую роль, в образе князя Олега? Загадка.

Как сие предположение соотносится с летописями, какими-либо другими источниками? И что насчет дат? Можно ли сопоставить возрасты и так далее?

У нас есть два источника про Вольгу – это летописи и былины. Источники мутные и противоречивые. Но вероятно, мы должны взять уважаемые летописи вроде Ипатьевской и Лаврентьевской, и сделать оттуда выжимку из общепринятых дат и связанных с ними событий. В год такой-то случилось то-то. Сейчас неважно, по какому летосчислению – от рождества Христова, от сотворения мира. Просто чтобы можно было подсчитать – вот начал княжить Олег, вот столько лет прошло, он умер, а на таком-то году впервые упоминается Ольга.

Вроде бы простая задача. Знай себе скрипи пером, стучи по клавиатуре. Берем старейшую из известных летописей – Лаврентьевскую, в издании ПСРЛ, где текст подготовлен учеными для удобного чтения – сплошной массив букв разбит на слова, поставлены знаки препинания, церковнославянские числа переведены в арабские, да еще в «от рождества Христова», и так далее. Душа радуется. Сейчас всё разложим по полочкам.

879 (6387 от «сотворения мира») – по смерти Рюрика, молодой Игорь и княженье переходит на попечение Олега. Сколько же Игорю лет? Летопись умалчивает. В списках находим: 2, 4, 17 лет. Как быть? Допустим, Игорю два, и год его рождения – 877. Здесь и далее берем наименьшие значения возраста – то же относительно Ольги, то есть мы сознательно омолаживаем героев летописи.

882 (6390) – поход Олега на Смоленск, Любеч, Киев, убиение Аскольда и Дира. Игорь, по Ипатьевской и Лаврентьевской летописям, всё еще дитя – Олег носит его на руках. Но как знаем из других списков, Игорь управляет конницей Олега. Кому верить? В этом же году Олег переселяется княжить в Киев, оставив в Новгороде наместника. Сооружает всюду грады, устанавливает налоги.

С 883 (6391) по 885 (9393) годы Олег подчиняет себе окрестные земли, устанавливает дани.

И тут оказывается, что в подлиннике Лаврентьевского списка вынуты страницы! Согласно общепринятой хронологии, с 899 (6407) по 929 (6437) год. От руки стоит, на верху десятого листа, примечание: «много листов нет». Листов нет от рассказа, как Мефодий посадил попа-борзописца помогать в переводе священных книг, по слова «ту бо есть Илюрик», и до «Приде Семевон на Царьград». 

То бишь исчезли страницы, где, по более новым спискам, развиваются едва не сказки о колёсном флоте Олега, пророчестве Олегу о гибели от любимого коня, что позже исполнится, когда из черепа коня выползет ядовитая змея. Там же изложены все походы Олега на Царьград, даны тексты олеговых договоров Руси с Византией, кратенько описано появление Ольги – мол, когда Игорь подрос, Олег привел ему из Пскова жену Олгу. Наконец, после смерти Олега начинает княжить Игорь.

Подготовленный учеными текст Лаврентьевской летописи невозмутимо продолжает повествование, заменяя пропавшие страницы содержимым за указанные годы из списков новейших относительно Лаврентьевского – Радзивилловского и Ипатьевского. Их датировка вопрос спорный, коего я не буду касаться, довольно знать, что Лаврентьевский написан раньше.

Даже если не шатать хронологию вообще, уверовать в «научные» даты и общепринятое изложение истории, что получается? Из древнейшего известного нам списка Повести временных лет изъяты события в пределах тридцати годов.

Почему в Лаврентьевской летописи они изъяты, а в Радзивилловской и Ипатьевской – нет? Насколько содержимое пропавших страниц было сходно с тем, что в тех же временных рамках говорят Радзивилловская и Ипатьевская летописи?

Более прямой вопрос – что такое было сказано в Лаврентьевской, и что переиначили в Радзивилловской да Ипатьевской? Что отличает последние от остатков Лаврентьевской?

Уход Олега, приход Ольги. Я же полагаю, что Олег и Ольга – одно лицо.

Давайте поглядим, что нам предлагают взамен упомянутые новейшие списки.

903 (6411) – женитьба Игоря на Ольге. Сколько ей лет? Не указано. По редким спискам – бывает что 10 лет. Тогда возьмем за основу, что год рождения Ольги – 893. 

907 (6415) – хождение Олега на Греков. Устрашение Царьграда кораблями на колесах. Первый мир с Царьградом, василевсами Леоном и Александром.

911 (6419) – Явися звезда велика на западе копейным (копиным) образом. Вычисляем возраст героев. Ольге 18 лет. Игорю 34.

912 (6420) – второй договор Олега с Царьградом, при царе Леоне и Александре, о мире. 5 лет назад Олегу предрекли смерть от коня. Осуществляется. Летопись насчитывает у Олега 33 года княженья. То есть княжил 912-879=33 года. И снова вычисляем возраст героев: Игорю 35, Ольге 19.

913 (6421) – Поча княжити Игорь по Ользе – начал княжить Игорь после Олега. В се же время поча царьствовати (в Царьграде) Костянтин, сын Леонтов, зять Романов. И Деревляне заратишася от Игоря по Олгове смерти. Игорю 36, Ольге 20. Хорошо Олег регентствовал!

C 929 (6437) года Лаврентьевская летопись возобновляется.

945 (6453) – смерть Игоря. 945–913=32 года княжил. Игорю 68 лет. Ольге 52.

Игорь погибает, оставив на руках Ольги сына маленького, Святослава (в летописях его имя часто писали сокращенно, «Стослав» с чертой, титлом над «С»). Сколько ему лет? А черт знает, но еще ребенок. Ольге 52. Сколько лет в супружестве с Игорем? 42. За 42 года наследников не прижили с Игорем, что ли?

955 (6463) – Ольга идет в Греки, крестится. Сватовство к ней василевса Константина, сына Леонтова, «видев ю добру сущю зело лицем и смыслену». Ольге 62 года.

969 (6477) – смерть Ольги. Ольге 76 лет. 56 лет проходит между смертью Вещего Олега и смертью Ольги.

Таковы принятые наукой числа по – именно «по» – Ипатьевскому и Радзивилловскому спискам. В главе про летосчисление я показывал, чего стоят такие числа. Но от них никуда не денешься. Их дают в учебниках.

Но даже школьнику, которого кормят этими хрестоматийными датами, должно хватить основных знаний математики, на уровне вычитания, чтобы засомневаться в указанных годах. Напомню, что за основу мы брали наименьшие возрасты Игоря и Ольга. Возьми я числа из других списков – Игорь и Ольга, люди с завидными на склоне лет здоровьем и деятельностью, превратились бы вообще в глубоких стариков.

И неслучайно основные списки осторожны с уточнениями, которые можно найти в других списках – опускают подробности. Почему? В противном случае стройный, хотя натужный ряд невесть кем и когда выстроенных дат будет  невозможен.

Ученых устраивает этот ряд, впрочем они порой позволяют себе сомневаться в какой-то одной его составляющей, вроде даты крещения Ольги – чтобы иметь повод написать большую зачетную работу. Но в целом даты принимаются как должное, хотя указывают на нелепицу.

Если же привлекать к делу другие списки и насыщать историю подробностями, то получается, например, такое.

То ли за 5, то ли за 7 лет до смерти Олега, Игорь женится на Ольге – противоречие «хрестоматийной» дате. И на пиру, Олег, будучи навеселе, хвастается боярам: «Новый есмы александр, царь македонский, надеяся обладати светом; да еще хто бы могл угодати, от чево мне будет смерть, то дал бы ему много имения».

Другой вариант: «яко аз новый есмь царь александр макидонский мудростию и храбростию надеяся обладати всем светом; да аще могл кто угонуть, от чего мне будет смерть, тогда бы тому много имения дал». 

Это начало известной, но часто сокращаемой для втискивания в традицию, истории про двух волхвов, которые предрекают Олегу смерть от его любимого коня.

Вставляем этот пир в историю – и разрушается академическая датировка свадьбы Игоря с Ольгой. Поэтому ученые в подробностях не заинтересованы. Меньше знаешь, крепче спишь. И пример с пиром – только один из многих.

Несколько дат хочу еще уточнить. По «радзивиллов\-ско-ипа\-тьевской» основе. 

Сколько времени прошло от начала княжения Олега и до конца княжения Ольги? 90 лет. Сколько лет было Олегу по смерти Рюрика? Неведомо. Вроде был Рюрику племянником. Могло и двадцать. Мог человек физически прожить 110 лет, частью в роли Олега, а потом в роли Ольги? Мог. А какие даты на самом деле, вычислить невозможно. Даже по общепринятым видно, как Игоря и Ольгу настойчиво «омолаживают», чтобы хоть как-то втиснуть в академические датировки событий. 

Потому что если Игорь не будет младенцем, а Ольга ребенком, временной промежуток пришлось бы увеличивать, и тут бы оказалось, что совсем глубокие старцы рожают детей, выбивают налоги, прельщают царьградских правителей и так далее.

Теперь о звезде с запада образом копейным, за 911 год. Морозов в книге «Христос» полагает – а его любят цитировать – что эти сведения выписаны из «Хроники» Георгия Амартола. У Амартола, в царствование Александра, который правил, как считается, с мая 912 по июнь 913, предшествуя Константину Багрянородному, «звезда явися велия с запада, копииника его нарицахоу с си злии. та звезда кровопролитие прознаменует в Костянтине граде, глахоу».

Впрочем к звезде никакое событие, связанное с Вольгой и иже с ним, не привязано. В 912 году, как считают астрономы, с Земли в который раз наблюдалась комета Галлея. Но в летописи указан 911 год, не 912. Может ошибка. Или речь идет о совсем другом явлении, произошедшем именно в 911 году. Я не могу просто взять и принять 911 и 912 за одинаковые числа. Нет на то должных оснований. По некоторым спискам звезда является еще на год раньше, то есть в 910.

Хорошо, а существует ли способ проверить летописные даты и описываемые события? Причем в рамках общепринятой хронологии. Нужен какой-то внешний источник, кроме летописей. Про княгиню Ольгу такой источник есть – византийский, и он больше задает вопросы, чем дает ответы.

Константин VII Багрянородный правил с 913 по 959 годы, порой совместно с регентами. Прозвище Константина – Багрянородный – возникло так. Законнорожденные дети византийских императоров, василевcов, рождались в особом, отделанном пурпурным камнем порфирием, Пурпурном павильоне при дворце. Хотя мать Константина, Зоя, не была замужем за отцом его Леоном, рожала она в этом императорском роддоме. Прозвище Константина как бы подтверждало его законность как правителя.

Он написал для своего сына Романа II (правил после смерти отца, в 959-963 годах) сочинение «О церемониях византийского двора»\footnote{Περί τῆς Βασιλείου Τάξεως, De cerimoniis aulae Byzantinae.}. Сразу отмечу, что сей Роман – не тот, что в летописи за 920 год «поставлен царь Роман в Греках». В 920 году – Роман Первый. А это Второй.

Подлинник «О церемониях» на греческом известен ученому миру по публикациям 18-19 веков единственного уцелевшего списка. В издании 1751 года был дан параллельный перевод на латыни. Кажется, это же издание в репринте известно как «боннское издание» 1820 года, сразу ставшее порицаемым историками за неточности.

Так вот, в книге II «О церемониях», главе 15, есть пример – описание приема Эльги Росены (Ελγας Ρωσενης). Нетрудно догадаться, о ком идет речь.

Датирован прием хитро – «девятого сентября, в четвертый день». После дождичка в четверг! Время написания самой книги ученые основывают на... известном сообщении русских летописей о крещении Ольги. Но в сочинении Константина ничего не сказано о крещении. В отношении вычислений дат, книга «О церемониях» мне не подходит.% Я не ученый, чтобы по щучьему велению, по моему хотению награждать события датами.

Ольгу принимают Константин Багрянородный и сын его Роман II Багрянородный. Присутствуют также «люди Святослава» – сына Игоря и Ольги. %Ольга названа своим же языческим именем, искаженным в Эльга, значит, вероятно, действие происходит до крещения ее, слывшей после крещения Еленой. С другой стороны, Владимира тоже не величали христианским его именем Василий.

Сочинение Константина настолько любопытно, что приведу перевод отрывка об Ольге полностью, дав цитату по работе Геннадия Литаврина «О датировке посольства княгини Ольги в Константинополь» (История СССР, 1981, № 5). Существует несколько переводов, и этот (на основе перевода В. В. Латышева?) точнее, чем из «Истории русской церкви» Голубинского.

У Голубинского перевод более приближен к русскому языку, но искажен. Например, название главы у него – «Второй прием Ольги Русской», между тем как в подлиннике стоит просто «Другой прием. Эльга Росенис». Ибо до сего описан прием сарацинского посла. По Голубинскому же получается, что Константин описал ранее еще какой-то первый прием Ольги, а теперь за второй, тоже Ольги, взялся.

\begin{quotation}
Другой прием – Эльги Росены. Девятого сентября, в четвертый день [недели], состоялся прием... по прибытии Эльги архонтиссы Росии. Сия архонтисса вошла с близкими, архонтисс\-ами-родственницами и наиболее видными из служанок. Она шествовала впереди всех прочих женщин, они же по порядку, одна за другой, следовали за ней. Остановилась она на месте, где логофет обычно задает вопросы. За ней вошли послы и купцы архонтов Росии и остановились позади у занавесей. Все дальнейшее было совершено в соответствии с вышеописанным приемом. Выйдя снова через Анадендрарий и Триклин кандидатов, а также триклин, в котором стоит камелавкий и в котором посвящают в сан магистра, она прошла через Онопод и Золотую руку, т.е. портик Августия, и села там. Когда же василевс обычным порядком вступил во дворец, состоялся другой прием следующим образом.

В Триклине Юстиниана стоял помост, украшенный порфирными дионисийскими тканями, а на нем – большой трон василевса Феофила, сбоку же – золотое царское кресло. За ним, позади двух занавесей, стояли два серебряных органа двух партий, ибо их трубы находились за занавесями. Приглашенная из Августия, архонтисса прошла через Апсиду, ипподром и внутренние переходы самого Августия и, придя, присела в Скилах (строение, примыкающее к Триклину Юстиниана. – прим. Литаврина). Деспина между тем села на упомянутый выше трон, а ее невестка – в кресло. И [тогда] вступил весь кувуклий, и препозитом и остиарием были введены вилы: вила первая – зост, вила вторая – магистриссы, вила третья – патрикиссы, вила четвертая – протоспафариссы-оффикиалы, вила пятая – прочие протоспафариссы, вила шестая – спафарокандида-тиссы, вила седьмая – пафариссы, страториссы и кандидатиссы.

Итак, лишь после этого вошла архонтисса, введенная препозитом и двумя остиариями. Она шла впереди, а родственные ей архонтиссы и наиболее видные из ее прислужниц следовали за ней, как и прежде было упомянуто. Препозит задал ей вопрос как бы от лица августы, и, выйдя, она [снова] присела в Скилах.

Деспина же, встав с трона, прошла через Лавсиак и Трипетон, и вошла в Кенургий, а через него в свой собственный китон (покои императрицы). Затем тем же самым путем архонтисса вместе с ее родственницами и прислужницами вступила через [Триклин] Юстиниана, Лавсиак и Трипетон в Кенургий и [здесь] отдохнула.

Далее, когда василевс с августой и его багрянородными детьми уселись, из Триклина Кенургия была позвана архонтисса. Сев по повелению василевса, она беседовала с ним, сколько пожелала.

В тот же самый день состоялся клиторий в том же Триклине Юстиниана. На упомянутой выше трон сели деспина и невестка. Архонтисса же стояла сбоку. Когда трапезит по обычному чину ввел архонтисс и они совершили проскинесис, архонтисса, наклонив немного голову, села к апокопту на том же месте, где стояла, вместе с зостами, по уставу. Знай, что певчие, апостолиты и агиософиты присутствовали на этом клиторий, распевая василикии. Разыгрывались также и всякие театральные игрища.

А в Хрисотриклине [в то же время] происходил другой клиторий, где пировали все послы архонтов Росии, люди и родичи архонтиссы и купцы. [После обеда] получили: анепсий ее – 30 милиарисиев, 8 ее людей – по 20 милиарисиев, 20 послов – по 12 милиарисиев, 43 купца – по 12 милиарисиев, священник Григорий (папас Грегориус) – 8 милиарисиев, 2 переводчика – по 12 милиарисиев, люди Святослава (Сфендослава, Σφενδοσλαβου) – по 5 милиарисиев, 6 людей посла – по 3, переводчик архонтиссы – 15 милиарисиев.

После того как василевс встал от обеда, состоялся десерт в Аристирии, где стоял малый золотой стол, установленный в Пентапиргии. На этом столе и был сервирован десерт в украшенных жемчугами и драгоценными камнями чашах.

Сидели [здесь] василевс\footnote{Сам Константин.}, Роман – багрянородный василевс\footnote{Роман – сын Константина.}, багрянородные их дети, невестка и архонтисса. Было вручено: архонтиссе в золотой, украшенной драгоценными камнями чаше – 500 милиарисиев, 6 ее женщинам – по 20 милиарисиев и 18 ее прислужницам – по 8 милиарисиев.

Восемнадцатого сентября, в воскресенье, состоялся клиторий в Хрисотриклине. Василевс сидел [здесь] с росами. И другой клиторий происходил в Пентакувуклии св. Павла, где сидели деспина с багрянородными ее детьми, с невесткой и архонтиссой. И было выдано: архонтиссе – 200 милиарисиев, ее анепсию – 20 милиарисиев, священнику Григорию – 8 милиарисиев, 16 ее женщинам\footnote{Подругам? Приживалкам?} – по 12 милиарисиев, 18 ее рабыням – по 6 милиарисиев, 22 послам – по 12 милиарисиев, 44 купцам – по 6 милиарисиев, двум переводчикам – по 12 милиарисиев.
\end{quotation}

Вот так без даты описывается посещение Эльгой царей Константина и Романа. Кстати, куда подевали князя Игоря? Вот же – есть Ольга, есть некие люди Святослава, а Игорь побоку!

Многие слова не переведены, а даны в греческом виде. Так лучше, ибо значение их туманно. Толкование некоторых да рассеет мглу.

Текст насыщен византийскими титулами, смысл которых не всегда ясен – за подробностями отсылаю вас к статье Ф. И. Успенского «Византийская табель о рангах», опубликованной в «Известиях Русского археологического института в Константинополе» (том III, стр. 98). В Византии существовали табели о рангах – тактиконы. Пускаться в их разбор, тем более что дело даже для знатоков в этом вопросе остается изученным мало, я не хочу. Одних только высших чинов (руководителей ведомств) в Византии было более 60! Посему вместо расшифровки чинов и титулов я обращу внимание на некоторые другие слова. 

Начнем со слова архонтисса, «αρχοντισσα» – знатная особа, правительница. Архонт – правитель.

Логофет – министр. Логофет секретов – глава правительства, логофет геникона – министр казны, логофет идика – министр императорских поместий, и так далее. О каком логофете тут идет речь, неясно.

500 милиарисиев – много или мало? Я решил плотно заняться этим вопросом и потратил немало времени, чтобы мало выяснить. Милиарисием называли серебряную монету в Римской империи, ну и в искусственно отделенной от нее учеными Византийской. Весила одна такая монета от 2,5 до 3, а позже и около 5 граммов, вроде немного была в ходу и на Руси. По ценности составляла 1/1000 литра, то бишь фунта (по весу) золота. «Miliarensis» в переводе с латыни значит «тысячный». 500 милиарисиев – сумма приличная. Не помню уж кому, сумму налога снизили с двух милиарисиев до одного – и была великая радость. Кажется, за 500 милиарисиев выкупали некоего важного христианского деятеля. Я рад бы сообщить вам, сколько булок можно было купить в Царьграде за 500 милиарисиев во время пребывания там Ольги, но для этого надо знать цену одной булки и перевести её из более мелкой монеты в милиарисии. Первое для меня загадка, поэтому второе отпадает само собой.

Анепсий (ανεψιος) – племянник, родственник. С Эльгой был некий загадочный анепсий. Кир Булычев предположил, что анепсий это сын Ольги, Святослав, и княгиня приехала сватать его за дочь василевса. В известном латинском переводе 1751 года «анепсий» переведено как «дядя», думаю ошибочно. Для проверки – английское nephew, близкое по произношению к анепсию, тоже значит «племянник».

«Люди Святослава». Что за люди, какую роль они играли – неведомо. Его представители, словом.

И вот этот отрывок из «Церемоний» многие историки и следующие им люди относят ко крещению Ольги. Мол, греческие источники свидетельствуют – и ссылка на «Церемонии». А я в этом разобраться не могу за скудостью данных. 

Но всё же! В тексте присутствует деспина – императрица, жена Константина? Да. А при крещении Ольги, сватался ли император к Ольге? Сватался, потому что, как говорит летопись, «бе вдов». Однако в «Церемониях» еще не вдов. Значит, времена разные, василевс не мог тогда крестить Ольгу. Ученые путают!

Но довольно о «Церемониях» – как источник об Ольге они исчерпаны. Тайное не стало явным, загадок прибавилось.

Попробуем по другим источникам выяснить, что вообще известно о князе Олеге и княгине Ольге.
