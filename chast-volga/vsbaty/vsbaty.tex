\chapter{Ольга и Шолудивый Буняк}

Познакомимся с новым героем нашего повествования. Его зовут Боняк по прозвищу Шолудивый.

Невесть почему историки считают его «половецким ханом», хотя давние источники нигде так его не величают. Да, порой Боняк выступает в составе войска Половцев, и однажды, посмертно, его называют великим князем.

Пролистаем летописи. Первое появление Боняка на их страницах – за год 6604 (толкуется как 1096 нашей эры). Ипатьевский список сообщает о нападении Боняка на Киев:

\begin{quotation}
иуля [...] в 20 день пяток, в час 1 дня, прииде второе Боняк безбожный шолудивый отай\footnote{Тайно.} хыщник Кыеву внезапу и мало в город не вогнаша половци, и зажгоша по песку около города, и увратишася на монастырь, и пожгоша монастырь Печерьскый [...]
\end{quotation}

Шолудивый или шелудивый означает покрытый коростой либо сыпью.

Сей Шолудивый Боняк был человеком не простым, он гадал по волчьему вою. Ипатьевская летопись за год 1097 (6605):

\begin{quotation}
Ярослав, сын Святополч, прииде с Угры, и король Коломан и 2 пискупа, и сташа около Перемышля по Вягру, а Володарь затворися в граде. Давыд же в тъ пришед из Ляхов, и посади жену свою у Володаря, а сам иде в Половце; и усрете и Боняк\footnote{Давыд встретил Боняка.}, и воротися Давыд, и поидоста на Угры. 

Идущима же има, и сташа ночьлегу и яко бысть полунощи, и востав Боняк отъеха от рати, и поча выти волчьски и отвыся ему волк, и начаша мнози волци выти; Боняк же приеха поведа Давыдови, яко «победа ны есть на Угры».
\end{quotation}

%И завтра Боняк исполчив вои свои, Давыдово 100, а Боняк у 300 стех, и раздели на 3 полкы, и поиде ко Угром.

И далее Боняк относительно часто появляется, силой грозною, на страницах летописи. Вот за 1105 (6613) год: 

\begin{quotation}
пришед Боняк зиме на Зарубе
\end{quotation}

В 1107 (6615) году Боняк, Шарукан старый «и инии князи мнози» сражаются около Лубен против Святополка, Володимера и других русских князей. При этом гибнет брат Боняка, Таз. И вот спустя 60 лет, в той же Ипатьевской летописи, за 1167 год:

\begin{quotation}
том же лете бися Олег Святославич с Боняком и победи Олег Половци
\end{quotation}

Первое упоминание Боняка – за 1096 год, здесь 1167. Промежуток времени 71 год. Допустим, в 1096 Боняк, уже предводитель нехилого войска, был молод, лет двадцати отроду. Тогда в 1167-м ему 91 год. Вполне приемлемо.

И хотя в записи за 1167 год не сказано прямо о гибели Боняка, после этого Боняк как самостоятельный деятель исчезает со страниц летописи. Лишь за 1185 год, Кончак говорит:

\begin{quotation}
пойдем на киевскую сторону, где суть избита братья наша и великий князь наш Боняк \end{quotation}

Вот сие, возможно, о схватке 1167 года, когда Боняк был побежден. Или о другом событии – не совсем ясно. Но с 12 века наши летописи про Боняка молчат. Любопытно, сколь долга народная память?

В 19 веке в ходе работы Этнографическо-статистичес\-кой экспедиции в Западно-русский край, в Луцком уезде было записано предание\cite{trudy-chub}:

\begin{quotation}
происхождение курганов приписывают какому-то воинству, под названием «Буняк», которое прошло всю вселенную и на местах своего отдыха делало курганы, насыпая при этом землю своими башмаками.
\end{quotation}

Предание выражено учеными словами – в народе не говорили «вселенная», а курганы называли могилами. Но суть любопытна. Что же, семь веков из уст в уста передавалась легенда о Буняке? И почему Луцк?

%Может быть, на Волыни, где находится Луцк (летописный Лучецк), про Буняка ходили предания, относящиеся ко времени ближе к 19 веку, нежели век 12-й?

Луцк это летописный Луческ, окрестные жители именовались Лютичами. Быть может, на Волыни про Буняка ходили еще какие-то предания?

Во тьме веков затерялась рукописная книга на латинском языке «Annales revolutionum Regni Poloniae et rerum notabilium civitatis Leoburgicae ab an 1614-1700. a Joh. Thoma Josefowicz, canonico Leopol. et. caet.» – «Анналы переворотов в Королевстве Польском и важные события, относящиеся к городу Львову в 1614-1700 годах» сочиненная львовским каноником (католическим священником кафедрального собора) Йозефовичем. В 1854 году во Львове вышел польский перевод ее под заглавием «Kronika miasta Lwowa 1634-1690 napisana w iezyku Lacinskim przez X. J. Tomasza Jozefowica, teraz przelozona przez M. Piwackiewgo».

%В рассказе за 1663 год про поход короля Яна Казимира на Украину, за 1648 год (во время волнений при Юрии Хмельницком):

На странице 266, в рассказе за 1663 год про поход короля Яна Казимира на Украину сказано:

%Во время этого похода, или, может быть, во время других, немного давнейших, пал, как говорят, в борьбе с нашими и предводитель козаков, Шолудивый Буняка, величайший неприятель поляков и поджигатель ненависти к ним, страшилище, по всему дьявол в человеческом виде.

\begin{quotation}
Во время того похода или во время иных, может давнейших, пал смертью, как говорят, в борьбе с нашими\footnote{То бишь с Поляками.}, известный предводитель Козаков, Солодивый Буняк, величайший неприятель Поляков и разжигатель ненависти к ним, страшилище, судя по всему дьявол в человеческом обличьи.
 
%А так как имя его очень известно на Украине, и много могил насыпано на Полесьи, к Киеву, покрытых лесами сосновыми и дубовыми, которые носят его имя, то будет будет некстати, если я расскажу о нем следующую повесть, которая и до сих пор держится у козаков, как найправдивейшая история...

А поскольку его имя распространено на Украине и множество курганов, насыпанных по Полесью к Киеву, и прикрытых лесами сосновыми да дубовыми, носят его имя, нелишним будет рассказать связанное с ним предание, у Козаков как наиправдивейшая история до сего дня держащееся.

%Некоторые козаки рассказывают, что Шолудивый Буняка, названный так ради паршивой головы, прибыл неизвестно откуда, к неприязненным Польше козакам и тем самым приставши к войску, так отличился жестокостью не только против поляков, но и против козаков, что в скором времени выбран был начальником последних.

Рассказывают некоторые Козаки, что Солодивый Буняк, от паршивой головы так названный, неведомо откуда прибыл к неприязненным Польше Козакам, присоединился к их войску, и отличился такой большой жестокостью не только к Полякам, но и против Козаков, что вскоре был избран у последних главным.
 
%Этот злодей ходил раз в месяц купаться и всегда брал с собою одного из ряда реестровых воинов, которого сам же после купанья убивал, для того, чтобы тот не рассказал, что он, кроме лица, рук и ног и смрадных внутренностей, не имел на себе ни мяса, ни тела живущих людей, а только кожу и кости, как видим на трупах, или на картинах представляющих смерть.

Этот злодей раз в месяц купался и каждый раз брал с собой одного реестрового воина (жолнера), которого сразу после купания убивал, для того, чтобы тот никому не поведал, что кроме лица, рук, ног и смрадных внутренностей, ни мяса, ни тела живых людей Боняк на себе не имел, только кожу и кости, как на трупах видно, или как смерть изображена на картинах.
\end{quotation}

Далее – передаю сокращенно – мать одного такого жолнера, который должен был присутствовать при купании Буняка, печет своему сыну пляцек (картофельный дерун), замешанный на молоке из собственной груди. Козак-жолнер угощает Буняка, тот ест и понимает, что стал молочным братом козаку и не может его теперь убить. Буняк говорит козаку: «ты сейчас избегнул своей смерти, но моей есть причина, потому что оба молоком одной матери кормились».

Козак перебегает к Полякам и рассказывает им секрет чудовища – что обычным способом его убить нельзя, ибо оно носит кости трупа вместо человеческого тела, и железо его не возьмет\footnote{Подлинник:\begin{otherlanguage}{polish}
\begin{quotation}
I tak powidaja, ze ow uszkodzeniu nie podlegajacy, trupie kosci zamiast ciala ludzkiego noszacy potwor, ktory widzialnem zelazem raniony byc niemogl, takim sposobem zginac chyli raczej przepasc musial. Jezeli to za prawde uchodzie niemoze, chciaz Kozacy wierza w takie czary, niech sluzy czytajacym przynajmniej ku rozrywce.\end{quotation}\end{otherlanguage}}.

Как же так? По нашим летописям, Шолудивый Буняк умер в 12 веке, а тут – предводитель козаков. Причем выглядит он в высшей степени странно, будто живой мертвец. И оружие его не берет.

Можно отмахнуться – мол, сказка! Но вот Костомаров в «Богдане Хмельницком» (изданном в 1857 году) о бунте на Волыни за 1648 год пересказывает то же самое и кое-что прибавляет:

\begin{quotation}
злодеяния волынских возстанцев [...] остались в народной памяти в дико-фантастическом образе Шолудивого Буняка: это имя древняго хана половецкого предание поместило в эпоху Хмельницкого.

Говорят, что так назывался начальник одного загона: он, по преданию, был мертвец, вставший из гроба, имел человеческое лицо, снаружи казался живым существом, но внутренность его была наполнена гнилыми костями и это было видно, когда он раздевался. Он каждый месяц ходил в баню и брал с собой козака, которого потом убивал, чтоб тот не рассказывал, кто он такой.

Пришла очередь пойти одному козаку, которого мать была колдунья: она дала сыну пирог, испеченный на молоке груди своей. Сын предложил чудовищу в бане этот пирог и тот догадался, когда съел его. «Ты ушел от смерти: я теперь брат твой, потому что мы питались от груди одной матери, но я погиб». Названный брат перебежал к полякам, открыл им, что слышал, и Шолудивый Буняка погиб в первой стычке.

Память о нем до сих пор сохраняется у волынских поселян. Между Кременцем и Дубном, близ местечка Вербы\footnote{Ныне Верба, Ровенской области, 50°16'18.3"N 25°36'19.1"E}, показывают курган, где будто бы погребен Буняк. Злые духи гнездятся там, где только чудовище обитало в жизни.
\end{quotation}

Помимо сведений о кургане близ Верб, Костомаров излагает предание, несколько отличное от изложенного Юзефовичем – Буняк гибнет! Я впрочем не понимаю в обоих вариантах, каким образом бежавший от Буняка козак мог способствовать его смерти, если чудовище было непобедимо.

Одно время Костомаров сам жил на Волыни, в Ровно, работал учителем, и наверняка собирал материалы о Хмельницком да сопутствующие местные предания, но легенду о Буняке он взял, на что дает ссылку, из книги фон Энгеля «История Украины и украинских козаков» 1796 года\footnote{von Engel, Geschichhte der Ukraine und der Ukraininschen Kosaken wie auch des Konigreich Halitsch und Wladimir, Halle, 1796, страницы 155-157.}. Энгель тоже дает ссылку, откуда взял – из книжки Йозефовича 1648 года, которую мы уже знаем по польскому переводу! Но в польском переводе не говорилось о гибели Буняка. Немецкий перевод Энгеля проясняет дело\footnote{Подлинник, вероятно передаю с ошибками по невежеству своему, далее под «новым братом в Сатане» подразумевается козак, ставший молочным братом Буняку: Bald darauf sey die ser neue Bruder im Satan zu den Pohlen übergelaufen, und hätte alles verrathen, was seine Augen gesehen hätten; und so wäre denn der unüberwindliche Solodiwy Buniak durch die Macht der Pohlnischen religiösen Beschwörungen im nächsten Treffen gefallen.} – когда козак-предатель поведал тайну Буняка Полякам, Буняк на следующей встрече был убит «Pohlnischen religiösen Beschwörungen» (буквально «польскими религиозными заклинаниями»).

Это место либо пропущено в польском переводе, либо придумано Энгелем (и кратко пересказано Костомаровым).

Обратимся к другим источникам. Михаил Драгоманов, который под псевдонимом «П. Кузмичевский» напечатал исследование «Шолудивый буняка в украинских народных сказаниях» в Киевская старине 1887 года в номерах, 8 и 10\footnote{Страницы 676-717 и  233-271.}, а в номере восьмом за 1891 обратившейся к той же теме уже за подписью «Р. Л. Н.»\footnote{Страницы 299-304, «К рассказам о Шелудивом Буняке».}, еще ранее в сборнике своем  «Малорусские народные предания и рассказы» (1876) поместил два волынских предания про  Буняка. Относятся они к окрестностям Мирополя\footnote{Житомирская область, 50°06'24.7"N 27°42'29.1"E. Чуть севернее Мирополя есть село Ольха.}.

Первое:

\begin{quotation}
Буняково замчище

Про це замчище багатого де чого розсказують. [...] говорять, що там колись мисто було, нибы на тому городищеви; говорять, що и церква була и що тую-б то церкву перенесено в Деревичи\footnote{Возможно, нынешние Великие Деревичи. Есть еще Малая Деревичка. В первых есть городище (49°58'27"N 27°34'29"E), будто бы 10-13 веков. И неподалеку в другом селе, Колодежном, тоже существует городище, Колядяжное (50°4'13"N 27°41'52"E).}. [...] нибы колись давно, давно сидив у тому городищеви якийсь лицарь, звався вин Шолудивый Буняка.

И був той лицарь не абы який, навить и тилом не такий, як повинно бути людыни; мало того, що страшенно велыкий, ще, выбачайте, у ёго печинки и легке то були на верси, отак от и стремили за плечима [...]

жив вин соби у городищеви, и дуже ёго вси боялись; настрашнийший же вин був ось через що: йв людей. Отак справди йв; оце, кажут, звелить було, щов ёму привели найкращого хлопця, та взьме та и ззисть.
\end{quotation}

Далее следует подробный рассказ, как мать печет сыну – очередной жертве – пирожки на своем молоке. Буняк и хлопец становятся молочными братьями. Хлопец ночью посещает Буняку и отрезает ему наружные печенки, после чего Буняка «зараз и пропав!».

Второе предание, Настина могила:

\begin{quotation}
В сёму миропильскому займищи, от що зовуть Буняковым, в гору по Случи, за мистечком у лиси, жив колись то давно поганый Буняка, ворожбит татарин. Коло того займища давно колись був перевиз через Случь и йшов шлях на Бердичев; того лиса, що навкруги ёго теперечки росте, зовсим не було.

От той Буняка дуже любив Настю, тутешню таки дивку, чи молодыцю, и взяв ии до себе за жинку. А вона ёго не любила. Та ще як стала з ним жити, то й доглядилася, що у его починки на верси, так просто за плечима, и побачила вона, що то вин не абы з ким знаеться.

От та Настя якось одкралася, тай утекла от Буняки; а вин доганяти! Наздонав ии у степу, що тепер за Миропильям, туда до Гордиевки\footnote{50°3'18"N 27°52'29"E}, вговорюе вернутися, а вона каже, що через те й те не хочу.

Тоди Буняка бачить, що вона про ёго усе знае, та й убив ии. Ота ж сама Настя сказала була, щоб на ии кошт поставили у степу корчму. То от, де ии вбито, высыпали могилу, так вона и зветься Настина могила, а на степу за Миропольем постановили корчму, вона и теперь стоить и теж зветься Настиною\footnote{Слышано от крестьянина в местечке Миропольи и передано Ольгой Косачевой.}.
\end{quotation}

Количество урочищ, связанных с Буняком, увеличивается. Четко проступает образ странного Буняка, строго блюдущего свою тайну.

В сборнике Biblioteka naukowego zakladu Ossolinskich, том 6, 1843, помещена статья Вагилевича «Берда в Урычи»\footnote{D. J. Wagilewicz, «Berda w Uryczy».}. Слово «берда» или «бердо» обозначает скалистую гору, а Берда в Урыче, по данным статьи, находится в 4 милях до Дрогобыча (Druhobych), и 7 милях от Стрыя. 

Ныне Урыч\footnote{49°10'58"N 23°24'18"E} – небольшое село вдоль единственной улицы, да речки Урычанки с каменистым дном. На северо-восток оттуда, за поворотом у кладбища, идет дорога к Урычским скалам, где, как полагают археологи, была крепость Тустань. Сюда впускают туристов, сделаны деревянные лестницы и ограждения. То, что пишут о Тустани ученые, вы можете прочесть в других книгах, я же рассматриваю сведения из статьи.

Судя по ним, окрестности в 19 веке еще хранили память о Шелудивом Буняке. 

\begin{quotation}
Основное скопление берд навалено на отдельном отроге Хостенка (Hostenca), на юг спускается широкий яр с дорогой и потоком Церковным (Cerkiewnym), где некогда был став Шелудивого Буняка.
\end{quotation}

Затем Вагилевич сообщает сведения из некой хроники:

%Вагилевич (D.J. Wagilewicz, Berda w Uryczy; помещено в Biblioteka naukowego zakladu Ossolinskich, t 6, 1843, стр. 163) говорит:

\begin{quotation}
Когда Татаре заняли княжество галицко-владимирское, во главе их стоял тот страшный Шелудивый Буняк (Szeludowe Buniak), полудемон с паршами (шелудями) на голове, с веками длиной до земли, которые челядь его по приказу поднимала,  золотыми вилами, и с открытым брюхом.\end{quotation}

Далее перескажу близко к тексту. 

Татары с Буняком подступили под город Быч (Bycz) над рекой Тысменицей\footnote{Tysmienica, в 19 веке Тустановица, Tustanowice. Приток Днестра, начинается километрах в пяти на северо-запад от Сходницы, и на север от Урычи, затем течет через Борислав и Дрогобыч. Не путайте Тысменицу с одноименным поселком – и последний не лежит на этой реке.}. Жители города крепко сдерживали осаду. Шелудивый Буняк отправил к осажденным послов с предложением мира, если они заплатят небольшую дань – с каждого дома по одному голубю и воробью.

Когда Бычане согласились и уплатили дань. Буняк распорядился привязать к голубям под крылья запалы, обвитые нитками, пропитанными серой, и отправил птиц назад в город. Конечно же вспоминается уничтожение Искоростеня Ольгой.

Итак, птицы возвратились в Быч. Начался пожар. Воспользовавшись этим, Татары вошли в город. Часть горожан была перебита, части удалось бежать – они основали Дрогобыч (Druhobych – Drugi-bycz).

Затем Боняк обосновался во дворце в Урыче, где воздвигнул стены из камня и в них ворота для въезда, и перегородил насыпью от южного берда, где у Буняка был пруд, а вторую насыпь сделал до северного, где остановилось его войско обозом. В замке Буняк сидел среди золота и роскоши, и держал тысячи пленников.

Однажды шли по верху Старого горба (Stary-horb) музыканты из Сходницы\footnote{49°13'49"N 23°21'6"E, ныне поселок Сходницы известен своей минеральной водой «Нафтуся».}, где играли на свадьбе. Возвращались домой в Крушельницу\footnote{Она же Кружельница, поселок, 49°6'20"N 23°29'12"E. От Сходницы до Кружельницы около 15 километров, по пути находится и Урыч.}. Их увидели из замка и дали знать Шелудивому, а он приказал позвать их к нему. 

Буняк был в свадебном зале, началась пьянка и танцы. Музыканты разглядели среди сокровищ бочки с порохом, и решили унести всё, что можно, а замок подорвать. Буняк, гости и слуги так напились, что попадали без сознания, а трезвые музыканты взяли сокровища, приладили свечи к бочкам с порохом и, заперев дворец, отправились своей дорогой. Отойдя на порядочное расстояние, услышали громкий взрыв, аж земля сотряслась – дворец взорвался, похоронив под руинами всех обитателей.

Один только Шелудивый Буняк уехал на золотой колеснице в Венгрию (Wegry), однако на границе его колесница врезалась в бук, перевернулась и вросла – некоторые добавляют, что вросла вместе с Буняком.

Однако музыканты не ушли от наказания, за ними устроили погоня, и в назидание ворам отрезали уши. Потомков их в Крушельнице – фамилию Чулевичей (Czule\-wicze) – поныне (в 19 веке) в память об этом называют безушками.

С того времени в развалинах бердского дворца никто больше не поселился, кроме, как некоторые прибавляют, беса хромого (diabla kulawego), которого до недавних пор еще прогуливался там с зажженной трубкой.

Почти в трехста километрах на восток от Урыча лежит село Великая Кужелева. В Киевская старине, том XXXVII за июнь 1892, помещена

\begin{quotation}
Подольская легенда о Буняке.\\

В 1873 году, по распоряжению центрального статистического комитета, собран был по волостям богатый материал для приведения в известность памятников древности, рассеянных в каре.

Пересматривая данные, относящиеся к территории подольской губернии, я встретил в донесении Миньковецкой волости (Ушицкого уезда) оригинальную легенду, записанную волостным писарем «от старожилов», относимую ими к городищу, сохранившемуся у села Великой Кужелевой\footnote{В Дунаевецком районе Хмельницкой области, 48°53'7"N 27°5'47"E.}. [...] Выписываю это легенду дословно в том виде, в каком она записана была [...]

Из рассказов старожилов оказывается, что городище в с. Великой Кужелевой, называемое «Мистысько», в древние времена занято было городом, называвшимся Кружель; город этот разорил в древности козак Солудывый Буйняк, который в одно и то же время разорил местечко, на месте которого теперь существует село Тышков\footnote{Возможно Тимков, Хмельницкой области, 48°47'15.7"N 27°05'10.9"E – он лежит ниже по течению, реки Ушицы, от Великой Кужелевой.}, и город, бывший там, где теперь находится село Городиска\footnote{Городиская, 48°49'50"N 27°4'59"E, население 185 человек на 2001 год.}.

Буйняк окончил свои военные разбойнические похождения у местечка Городка (Грудка) Каменецкого уезда\footnote{Grudek, ныне Городок, Хмельницкая область, 49°10'21.1"N 26°34'19.9"E. Стоит на реке Смотрич. По ощущению чем-то похож на Киев, возможно каштанами, сиренью и перепадами высот.}, где его победил кожемяка Самсон Кирилович; кожемяка этот был настолько силен, что если бывало рассердится на кого-нибудь, то в минуты вспыльчивости в одно мгновение раздерет двенадцать воловьих кож.

Когда Буйняк подступил к Городку со своим войском, кожемяка Кириллович в то время спал; жители, разбудив его, просили у него защиты.

Кирилович приказал себе сделать из шелку толстую тесьму, на которую нацепил мельничный жернов и кругообразным движением его уничтожил всё войско Буйняка и победил его самого;

это же случилось следующим образом: козак Буйняк имел веки до самой земли, которые, когда он подходил к какому-нибудь городу, два козака поднимали ему особо устроенными вилами;

тогда он, осмотрев город, отдавал приказ своему войску, откуда начинать приступ; веки эти имели такое свойство, что если кто-либо из неприятелей Буйняка окажется настолько сильным, что приподымет их, то тому суждено будет победить Буйняка.

На долю Кириловича выпало воспользоваться этим таинственным свойством век – он приподнял их, победил Буйняка и прекратил дальнейшие его разрушительные завоевания; таким образом местечко Городок спаслось от разрушения и уцелело до настоящего времени.

В. А.
\end{quotation}

Козак Буйняк! В этом рассказе появляется новая необычная особенность Буняка – веки до самой земли. От предания к преданию разнится действие, производимое открытыми глазами Буняка.

Множество историй о Буняке связано с урочищами Погоней и селением Заваловым\footnote{Между ними около 40 километров.}.

В приложениях к книге Кольберга «Покутье: этнографическая картина»\footnote{Kolberg, Pokucie: Obraz etnograficzny, Krakow, 1882, т. I, стр. 344-346.} перепечатана из альманаха Lwowianin\footnote{1837, т. II, страница 69.}  статья ксендза Харасевича (M. Harasiewicz):

\begin{quotation}
Завалов, в недавние времена усадьба князей Маковцких и Яблоновских (ныне собственность Можинской), предместье Галича, и ради положения своего за валами названное Заваловым\footnote{Три вала над городом на горе, как сообщает «Сводная галицко-русская летопись с 1600 по 1700 год» А. Петрушевича.} [...]

Лежит в бедной местности в округе Бежанском (Brzezanskim) над рекой Золотая Липа (Lipa Zlota) [...]

До наших времен дошло такое предание:

Буняк, презренно именуемый «солодивым грабителем» (zboj solodywy), нападал на все селения в околице и беспокоил население. Наиболее старался прибрать к рукам имущество погубленных им жителей, за которое содержал себя и свое войско, кочующее между выдающимися холмами со стороны южной Завалова, черным лесом покрытыми.

За рекой Липой Золотой, от востока, на том самом месте, где позже на холму князья Маковецкие поставили замок, стоящий по сей день, для обороны башнями и стеной обнесенный, из которых только одна осталась, так вот в том самом месте, русский князь, господин и наследник этой страны, выставил своё войско на отпор Буняку.

Однако понимая, что военной силы у него мало, стоял и не решался напасть на своего противника, пока во сне седой старик не надоумил князя к нападению и способу, как и с какой стороны его произвести. Утром князь смело пошел в наступление. 

Бой с обеих сторон было упорным, наконец Буняка был побежден и войско его частью рассеялось, частью утонуло в Липе Золотой.

Самому начальнику Буняку отняли голову, которая с места сражения катясь уходила, и едва была догнана на том поле, что за местечком Тисменицей (Tysmienica)\footnote{48°53'53.3"N 24°51'13.1"E} – потому то место названо «Погоня».

Вот что я слышал из уст тутошнего посполитства. 
\end{quotation}

Завалов\footnote{49°12'39"N 25°1'25"E} – село в Подгаецком районе Тернопольской области Украины. Монастырь\footnote{49°11'48"N 25°1'44"E} при нем разрушен, там пустое место. Рядом находится село Яблуновка и еще несколько селений со сходными названиями, что вероятно указывает на владельцев Яблоновских.

Погоня\footnote{48°53'00.7"N 24°53'00.2"E} – село в Тысменицком районе Ивано-Франко\-вской области, на восточной окраине села есть возрожденный в 1992-м, базилианский Погонский монастырь Успения Божьей Матери.

В годах 1665, 1677 и 1699 во Львове, Чернигове и Могилеве вышло три издания книги ректора Киево-Могилян\-ской коллегии, архимандрита Черниговского монастыря Иоанникий Галятовского «Небо новое з новыми звездами сотворенное, то есть Преблагословенная Дева Мария Богородица з чудами своими» – сборник описаний христианских чудес. В их числе упомянуто предание про Завалов:

%. – Могилев: Тип. Максима Вощанки, 1699

\begin{quotation}
Чудо семнадцатое.\footnote{В одном из разделов.}\\

В малой России, в Повете Галицком есть место Завалов, названное от валов давно высыпанных, которых есть три над тым местом на горе, межи тымы трома валами знайдется Монастыр при церкве святого Архиерея Николая, поведают люде старыи Духовныи и свецкии, которыи от дедов и Прадедов своих чували, же Буняк (яко они мовять) але рачей Батий Царь Татарский\footnote{То бишь деды и прадеды называли врага Буняком, но Галятовский предполагает, что скорее всего это был не Буняк, а Батый, царь Татарский.} егда воевал землю Рускую, на той час еден Воевода на имя Роман, з земле Рускои вышол з войсками своими против ордам татарским, а видячи великую потугу поганскую, а свое малое войско христианское, ископался трома валами, й фрасовался, й фрасуючися заснул, которому в сне показался святый Архиерей Николай, й казал ему ити смеле на орды бесерманскии, и где бы нагонил поганов и звытяжил их, казал ему на том месце церковь збудовати, на честь Пресвятой Богородицы, за которой помочь мел Махометанов звытяжити, межи тыми зась трома валами, казал збудувати церков на честь святому Архиерею Николаю, очнувшимcя от сна Роман Воевода, решил смеле з войсками своими на Татаров й погнал их, и оутекаючих гонил, и догнал на тых полях, которые недалко за место Тысменницею знайдуются, и славное над ними одержал звытяство\footnote{Победу.}, по котором звытястве на тым полю збудовал церковь Успения Пресвятой Богородицы, за которой предстателством звытяжил орды поганскии, в той церкве есть наместный образ Пресвятой Богородицы чудотворный, при той церкве знайдуется монастыр который называется Погоня, для того, же там Роман Воевода погнал и догнал й розогнав татаров неприятелей своих, и могл мовити слова Дедовы: Пожену враги моя и постигну я, и невозврашеся донжеде скончаются, отскорбл. их и невозмогут стати, падут пред ногами моима, по зветятве же межи трема валами збудова церков святого Архиерея Николая, который ему там в сне показася, и каза едну церков на честь Пресвятои Богородицы другую на имя свое збудовати.
\end{quotation}

Харасевич приводит следующие данные из Krajowej galyiciskiej Tabulie, ex libro fond. 109, страница 109 – даю в своем пересказе. Все временные отсылки «теперь», «сейчас» относятся ко времени написания Krajowej galyi\-ciskiej Tabulie.

Когда-то князь Роман из удельных князей русских, роду Острогских, и брат родной найяснейшего короля, его милости, Данила Галицкого, набожный, живший в обряде единоверном с церковью римского обряда, купно с родичами своими, князьями Острогскими, а именно братом Димитром и племянником Василем, с их войсками, победил поганых неприятелей родной церкви. Языческие неприятели – это солодивый Буняк, иначе Селодиво (Selodywo), или wsiow dziwowiska, главарь и князь Половцев, то есть Подолянов (Podolanow).

И на том месте, где стоит монастырь, расположились князья, из-за неравенства сил загородившись окопами. С другой стороны Золотой Липы, где сейчас видно замок, также просматриваются давние валы, через которые враг пробивался три дня.

Во время утренней молитвы Роману явился, в лучах света, святой епископ Николай: «Встань, выйди из окопов, начни битву и победишь, ибо услышана просьба (молитва) твоя».

По сему приказу, войско Романа вышло на ту сторону, где теперь стоит старая часовня на горе монастырской, и начало на голову громить неприятеля, как свидетельствуют и следы курганов, расположенных по склону.

Роман гнал врагов к Днестру, то бишь Тирасу, где немало взял в плен. Перебравшись через Днестр, Роман поймал самого вождя, приказал отрубить ему голову, тело сжечь и прах высыпать в Днестр, где в память об этом место под Тисменицей слывет Погоней (Pohonia).

Еще дальше за Днестр загнал Роман остатки врагов, в горы, и одержал победу. Это место называется Перкиньско (Perkinsko), давнее владение пресветлого дома Яблоновских (за этим следует запись монастырю св. Василия В.).

Катящаяся голова Буняка и погоня за ней отразились среди местных жителей в песне. Драгоманов предполагал, что песня вовсе не народная, а фальшивка, возможно сочиненная самим Харасевичем. Так это или нет, привожу ее здесь, только переделанную из латиницы в кириллицу:

\settowidth{\versewidth}{Може Буняк соладивы?}
\begin{verse}[\versewidth] 
Щож то за дивы!\\
Може Буняк соладивы?\\
Кажуть его череда\\
Пришла нас выгнати.\\
Лише дадут ему знати,\\
Куда буде утикати,\\
Хей на на,\\
Буняка!\\

Эй берут ся до него\\
Князь Роман, бояры,\\
А щож будет з него?\\
Гды пиде в прегоны?\\
От втикает голова,\\
За ним бижит череда.\\
Голова Буняка,\\
Здраст вам Боже\\
Вже наша.
\end{verse}

%Из Krajowej galyiciskiej Tabulie Харасевич выписывает:

%\begin{quotation}
%Тот же самый Буняк, или Баты-хан татарский, в 1180 году нападал также на Плениско (Plenisko), где был замок, владение княжны Елены (Heleny). Множество бед, причиненные им в Руси Червоной, и варварское обращение с населением дали ему имя шелудивого Буняка Солодивего (solodywego).\end{quotation}

%Тут мы видим, как предположение Галятовского о Батые вместо Буняка уже введено в оборот, а откуда взялись княжна Елена и 1180 год, поведаю чуть позже.

Тема Буняка не отпускала Михаила Драгоманова, и в 8-м номере «Киевской старины» за 1891 год, на страницах 299-304 он печатает работу «К рассказам о Шелудивом Буняке», с выдержками из которой мы сейчас познакомимся. Тут мы встречаем, в предании, скомканный рассказ о том, как отрубленная голова Буняка, обладавшая некой самостоятельностью, взглядом уничтожает целое селение.
%34.pdf

\begin{quotation}
Д-р Теоф. Окуневский\footnote{Вероятно, Теофил Ипполитович Окуневский (Teofil Okuniewski), 1858-1937.} посетил в 1887 г. местности в Галиции, около Завалова, к которым приурочены устные и письменные рассказы о Буняке, разобранные в I главе работы г. Кузьмичевского. [...]

Я был в Погони, – пишет г. Окуневский, говорил там с двумя стариками и с тамошним игуменом монастыря Василевского Коржинским о Солодовом Буняке и узнал от селян следующее: 

«Цему уже дуже давно, як прыйшов з татарвою якись их паша, недовирок, Буньо. Прыйшов вин у наши края з великими ордами, та пид Завиловым стяв му князь Роман Галицкий голову. 

Голова упала в один бик, а тулуб у другий. Тоди стала голова котити ся. 

Брови у той голови були таки довги, що очи скризь них дивити ся не могли. Та як раз пидняла желизними вилами ти брови на стятий голови, и голова подивилася из Могилок на Городище, то циле те мисто запало ся.

Тогди ще Тисьменици (миста) не було\footnote{По крайней мере в 1448 году город Тысменица получил Магдебургское право от Польского короля Казимира IV.} и аж вибудовали, як городище запало ся. Голова котила ся усе дальше та дальше. 

Козаки гнали за нев. В Товмачи\footnote{Вероятно, село Тлумач, 48°51'49"N 24°59'53"E. Насколько я знаю, вариант Товмач не шибко распространенный, хотя «товмач» это украинское слово, обозначающее то же, что «толмач» – «переводчик». Толмач упомянут в Ипатьевской летописи в 6721 (1213) году.} товмачили еи 70-ма язиками, з видти мисто Товмач, – та не могли еи розтовмачити. 

Голова докотила ся аж не далеко до нашого села теперишнего; тут пасла баба товар\footnote{Скотину.} та пряла куделю. Голова каже до баби: «сховай мене!». Баба сила на голову, а тим часов козаки перелитили. Лишився лишень один, якийсь пьяничка, за полком позаду, та зигнав бабу з мисца, а голова покотила ся з пид бабы. Догонив козак голову аж у лиси, та тут еи на мак розсик. 

Поховали потим козак еи у могилу, де стояв давнище монастирь, бо инакше була би лыха багато по свиту творила. Тай тому то и назвали наше село Погонев, бо тут голову дигнали».

«Ми ж, – сказал под конец селянин, – тому таки бидни, що така сатана до нас залетила!».

Игумен о. Коржинский показал г. Окуневскому несколько старых рукописей и между ними нашлась одна переплетенная тетрадь со следующею запискою:\\

SERIES HISTORIARUM.\\
Monasterii Pohonensis\\
Ex Mandato Per Jllustris Reverendissimi\\
Domini Josafat Steblecki\\
Odrinis Sancti Basilii Magni Abbatis\\
Ovrucensis Provinciae tituli Protectionis\\
Beatissimae virginis Mariae Provincialis\\
Per Vnblem Ardr. Rudum Patrem\\
Pancretium Dziubinski. O. S. B. Magni\\
Superiorem ejuisdem Monasterii\\
Confecta\\

Die 15 aji 1766-to anno.\\

De origine Mpnasterii Pohone, er under hic\\
Locis sortitus est Nomen Pohonia.
\end{quotation}

Дальнейший текст написан по-польски. Пересказываю.

Сначала идет отсылка к «Новому небу» Гялятовского, затем прибавлены некоторые подробности – около местечка Завалов на одной горе был замок, а напротив его другая гора, где кляштор Погоня. Монастырь поставили на месте войска неприятелей. В замке оставалась горстка людей во главе с князем Романом Яблоновским (Roman Jablonowski). Трижды на полях тетради этот Яблоновский, или просто Роман, другой рукой подписан как Галицкий (Halicki). 

Ему является святой Николай, Роман разбивает вражеское войско, кое-кого взял в плен, а предводителя преследовал до Днестра и за Днестр по лесам Тисменицким между горами, и отрубил ему голову.

И в месте, где лежит гроб его (непонятно, князя или вожака), могила округлая небольшая в саду нашем старом, который от теперешнего нашего кляштора за горой на юг. Возле той могилы была часовня маленькая в честь Богоматери, поставленная по приказу князя Романа – от нее только след остался, заросшие чащей бугры. При часовне жили наши монахи. Там же растет старое дерево, в том же саду, где криница под горой на юг недалеко за садом. На том месте был простенький клаштор названный Погоня от памятной погони за Солудивым Буняком.

На полях также по-польски были исторические примечания – выписки либо размышления, что речь шла про Романа Галицкого, коего убил король венгерский Владислав, «читай Сентября 20 житие Михала герцога Черниговского извлеченное из Синопсиса Synopsy Puczarskiego лист 179. Тот Михаил князь Черниговский убит в 1245-м».

От Погони на восток в ста километрах – Голенищево. Посмотрим, что пишет о нем 
Владимир Даль в «Картинах из русского быта: Червоно-русские предания» (1856):

\begin{quotation}
На том же Сбруче, на крутом каменистом берегу, есть селение Голенищево, с ясными остатками небольшого каменного укрепления, среди коего находится чистый и холодный родник, снабженный богатою жилою воды. Крестьяне это место называют Забытком, но помнят и поныне предание, в котором частица истины смешана со странною сказкою:

В глубокой древности стоял здесь замок, доставшийся по наследству одинокой и сирой княжне, произнесшей обет девства. 

О ту же пору властвовал по соседству какой-то сильный и страшный владетель, бывший, сверх того, еще и чародеем: никто и даже ничто не переносило его взгляда, он побеждал и разрушал глазами всё, на что бы ни обращал взор; но зато благая природа взяла некоторые предосторожности: веки у чародея этого всегда были сомкнуты, за что он и получил прозвание сонливого Баняка; он даже не мог сам открыть глаза, а для этого нужны были два помощника, которые, став осторожно на плечи его, чтобы не встретить страшного, губительного взора, осторожно подпирали веки сонливого Баняка золотыми вилочками.

Баняк-сонливый был человек самовластный, самовольный и свирепый: всё должно было ему покорствовать; малейшее противоречие возбуждало в нем неукротимый гнев и месть, а страшная чародейская сила, которою он владел, наводила ужас на всех, – и вся страна безусловно ему покорялась.

Услышав о молодой, прекрасной княжне в Голенищеве, он тотчас же вспыхнул: одной молвы о девственном обете ея было уже достаточно для возбуждения в своевольном Баняке неодолимого хотения подчинить княжну своей власти, заставить ее нарушить обет свой и выйти за него замуж. Самый обет казался ему личным для него оскорблением, как нарушение безусловной его власти.

Баняк-сонливый послал к княжне посольство, приказав в довольно гордой, высокомерной речи требовать руки княжны. Она отказала. Он послал в замок ратных людей; но они ничего не могли сделать, потому что замок снабжен был съестным припасами в изобилии и, сверх того, ключевой водой.

Нетерпеливый, своевольный Баняк не стал выжидать конца этой продолжительной осады: он поднялся с наперсниками своими, прибыл в стан под Голенищево, приказал себя поставить на таком месте, откуда виден был весь замок, и два человека из ближних его, достав роковыя золотыя вилочки, приподняли ими обвислые веки Баняка до самых бровей: одного этого взгляда на замок княжны было достаточно для конечного его разрушения; стены рухнулись, дружина Баняка побила не только дружин княжны, но и всех жителей городка Дивича, стоявшего под замком княжны на берегу реки.

Опустошение было таково, что ни замка, ни городка с той поры не стало; но зато в темныя, дождливыя осенния ночи тени побитых жертв и поныне еще скитаются по долине, на которой стоял город, и долина эта называется Дивич;

над городищем по воздуху проносятся тени бывшего замка и в толпе их светлый лучезарный образ самой княжны. Она спускается в долины и утешает сетующие тени. Тогда слышится вокруг топот конский, звуки неизвестных ныне рогов, клики и вопли и какие-то военные песни с припевом: идем на Дивич.

Долина Дивич расстилается по обе стороны Сбруча, и русской и австрийской стороне, местами скалиста и лесиста и на запад упирается в отроги Карпат.
\end{quotation}

Голенищево\footnote{49°08'21.2"N 26°12'09.3"E} поныне есть село в Чемеровецком районе Хмельницкой области, на левом берегу реки Збруч, на самой границе Хмельницкой и Тернопольской областей. Рядом – села Романовка, Ольховчик, Ольховцы, Сатановский лесной заказник, поселок Лысогорка, и городок Гусятин, где в речке Збруч в 1848 году нашли каменного «збручского идола». Как знать, быть может ученые грядущих веков сочтут идолами современные скульптуры?

Известные на 21 век урочища в Голенищево – родник Кадуб, Звенигора (которое некоторые голословно отождествляют с летописным Звенигородом), тоже с родником, и поблизости Княже замчисько, часть коего называется Дивичем. Там шестиметровой высоты земляной вал.

В предании действует некая княжна, и будь ее имя Ольга, я бы предположил, что ей принадлежали и села Ольховчик да Ольховцы, однако не буду заниматься домыслами.

Убийственный взгляд Буняка и поднятие ему век перекликается с ирландским сказанием о второй битве при Маг Тьюрэд, где Туаха Дэ Дананн противостояли Фомойри\footnote{Даю в своем переводе с английского по Ancient Irish Tales. ed. and trans. by Tom P. Cross \& Clark Harris Slover. NY: Henry Holt \& Co., 1936.}:

\begin{quotation}
Луг и Балор с Поражающим Глазом начали сражение. Фомойр Балор имел злой глаз. Этот глаз открывался только во время битвы. Четверо людей поднимали веко (крышку) глаза при помощи полированной ручки, продетой через веко. Если армия глядела в этот глаз, как бы ни было их много тысяч, они не могли устоять перед несколькими воинами. Глаз имел отравляющую силу.
\end{quotation}

Но чего я про Ольгу вспомнил? А вот в книжке Семенского «Предания и легенды польские, русские и литовские» читаем\cite[стр. 39]{siem01}:

\begin{quotation}
Вал Ольги\\

В окрестностях Зборова на Подолии, Колтова, народ показывает вал, который тянется с перерывами, называемый валом Ольги. Та княгиня, уходя от Батыя, вождя татар, обратилась в мышь и шла под землею; везде, где она рылась, насыпался такой вал. Безопасное убежище она нашла только в городище, называемом Плениско, вблизи Подгорец, где, запершись в замке, дала отпор ордам татарским. На том месте теперь видно несколько могил, а вблизи стоит монастырь базилиан\footnote{\begin{otherlanguage}{polish}28. Wał Olgi

W okolicach Zborowa (na Podolu), Kołtowa, pokazuje lud ciągnący się przerwami wał, nazywany wałem Olgi. Księżniczka ta uciekając przed Batyjem wodzem Tatarów, przemieniała się w mysz i szła pod ziemią, i wszędzie, gdzie nurtowała, wysypywał się taki wał. Bespieczne schronienie znalazła dopiero na Horodyszczu zwaném Pleśnisko, w pobliżu Podhorzec, gdzie zamknąwszy się w grodzie, dała odpór hordom tatarskim. Na tém miejscu widać dziś kilkaset mogił; a w pobliżu stoi monasterek księży Bazylianów.\end{otherlanguage}}. 
\end{quotation}

%чина святого Василия Великого

Зборов\footnote{49°39'25.6"N 25°08'49.6"E} – такой городок в Тернопольской области. Чуть южнее его, рядом с деревней Красная, находится Олеговский пруд\footnote{49°35'46"N 25°10'37"E}, но я не ведаю, с каких пор он так называется, и местные его вроде под таким именем не знают. А ежели он и вал Ольги связаны, равно как и село Ольшанка в пяти километрах от пруда на северо-восток, и Олесино в 12 километрах к югу?

Однако мало ли что можно высосать из пальца? Мне более любопытно, указание предания на то, что вал – Ольгин, и что княгиня Ольга переместилась потом в Подгорцы.

Эти Подгорцы\footnote{Подгорцы, Львовская область, 49°56'28.0"N 24°58'55.4"E} – от Зборова в 32 километрах на северо-запад. 

На южной околице Подгорцев, восточнее Благовещенского базилианского монастыря, находится большое, около 160 гектаров, Плиснеское городище. Считается, что его оборонные валы и рвы относятся к городу Плиснеску, существовавшему в 7-13 веках и разрушенному Батыем в 1341 году. Откуда взяли взяли про Батыя – из книжки Семенского?

Но слово «Плениск» странным образом похоже на «Плесков» – летописное имя Пскова. А в пяти километрах от Подгорцев на запад лежит селение Олеско с Олесским (Олежским) замком. На ум приходит созвучность с Ольгой и Олегом.

Роман Заклинский в работе «Пояснене одного темного місця в Слові о полку Ігоревім»\footnote{Львов, 1906.} писал:

\begin{quotation}
Место, где был замок и город в Плеснеске, называют теперь подгорецкие селяне «Городисько». Народное предание говорит, что «там мала свій замок цариця Єлена і він потому запався».
\end{quotation}

Как мы помним, Елена – это второе, христианское имя Ольги. Легенда из книги Семеновского связывает с Ольгой вал у Зборова и Подгорцы. А другое предание, «подгорецкое», говорит уже о царице Елене!

Но про близлежащий Благовещенский монастырь тоже ходила молва, что основала его княгиня Елена!

В стене каменной монастырской церкви, построенной в первой половине 18 века, была доска с надписью на латыни, что церковь основана в 1180 году великой принцессой Еленой (Helena) дочерью Всеволода\footnote{\begin{otherlanguage}{latin}Celsissima principissa Helena М. Ducis Wsewoldi filia anno 1180 hoc monasterium primo fundavit. Quod postmodum ill. Stanislaus Koniecpolski Palatin. Sand, resuscitavit ac tandem Serenissimus Joannes III. Poloniarum rex amplissimis beneficiis ditavit confirmavit privilegiis instruxit. Idemque fecerunt regii Principes Constantinus et Jacobus. Unde monachihuius monasterii memores tantorum beneficiorum hoc monumentum posuere.\end{otherlanguage}}.

Но Елена названа дочерью Всеволода... Насколько вообще можно доверять этой надписи и есть ли какие-то дополнительные, более ранние источники?

В начале 20 века в сборниках «Записок чину святого Василия Великого» опубликовали хронику – Синопсис Плиснеско-Подгорецкого монастыря, составленный в 1699 году\footnote{Синопсис или Краткое собрание историй, и создания Святыя Обители обще жительныя Подгорецкыя, древле Именуемыя Плесныцкыя.}. Начинается она с 1583 года, когда около места давнего, прежнего монастыря, а точнее церкви, заложенной Еленой, поселился некий старец Симон или Зосима, родом из Белого Камня, принявший монашество на Афоне. Зосима ископал себе пещерку при кринице и стал усердно молиться. Постепенно возник монастырь.

А про монастырь прежний в том месте, хроника приводит лишь краткое предание. Делаю выдержку, опуская славословие\cite[102]{zapvas01}:

\begin{quotation}
%Гди сурове з загневанного декрету бозского страшные неприятеле креста Святаго, мечем християнские край пустошили, огнем околичние места и Церкви божии руйновали. Сию же Церков от посполитого декрету и пожару, яко от обще земнаго потопа ковчег Ноев сохранити изволил Господь. Панства, Провинций, Места и Села околичние зостали зруйнованние; а тая Церковъ Святая, яко саламандрав огни, яко трие отроци во пещи вавилонской живи; яко Сигор въ Пентаполи, и яко Лот от пламенно текущей уволнений Содоми;

Що дивнейша все местце Града Плесницкаго; которое негдись на том местцу стояло, й до сего дня зовется Плесницко; яко окопы и вали значние досих час дают ведомость, тий так зруйнованые, иж дубы високие, дерева дубравние на том же местцу повирастали, а сия Церков Святая невредно соблюдаема єсть. [...]

Так и зде исполнил Христос слова свои, непоставил бо вем той светилник, си есть Церков свою Святую подъспудом, не затаил нам фундатора тоей Церкви Святой, але показал на свещнице; – и обявил от як многих лет есть збудованна;

А збудованна есть от княжни Елени, которая княжна знать была вери Православно Християнской, и прежде бо сих времен, на главах Российского народу княжие сияли Митри. Дает того певный и виразний документ над Царскими врати внутр олтаря, втие слова напис: 

1000 100 80 Року: Гони Батый Елену Княжну: аже и Церков сию вто время таяжде сооружи Княжна, Град Пленицк тогда бывшу разорену, яве ест; ибо тую Церков невредимо Вышний Господь сохранил доселе, и прославил, же ю от посполитого гневу своего и запаления уволнил. 
\end{quotation}

Итак, внутри алтаря над царскими вратами была приведенная выше надпись, что в 1180 году Батый гнал, преследовал княжну Елену. В то же время, как утверждает хроника, Елена соорудила и церковь сию.

Ни о каких Всеволодах не сказано. Церковь, где находилась эта надпись, предшествовала каменной церкви 18 века (с латинской памятной доской), и отголосок сей славянской надписи находим в грамоте Луцкого епископа Гедеона Святополк-Четвертинского, 1663 года:

\begin{quotation}
обитель Святая Плесницкая под Именем храму Преображения Господня (а тераз названная Подгорецкая) уфундована єсть, от Святыя памяти благочестивой Княжни Елени, от лета вочеловечения Господа нашего Иисус Христа тисячасто осмъдесятого: и для частых неприятелских нахождений запустела зостала до сих лет.
\end{quotation}

Значит, в 1663 году церковь стояла в запустении, однако про нее было известно лишь, что построена Еленой в 1180 году. Само число указывает, кажется, на позднейшее датирование, поскольку в 1180 году датировали бы скорее «от сотворения мира», нежели от «рождества Христова».

Хроника – 1699 год, грамота – 1663, в них даты стоят уже от рождества Христова. Не 17-го ли века надпись с переводом даты из «от сотворения мира» в «от рождества Христова»?

%По Słownik geograficzny Królestwa Polskiego i innych krajów słowiańskich на странице 393 можно догадаться, что речь идет о монастырской деревянной церкви во имя св. Михала, поставленной княжной Еленой

Рассмотрим даты и имена с точки зрения официальной науки. 

1180 год. Княгиня Ольга-Елена, жена Игоря Рюриковича – мертва. Батый жил в 1209-1256 годах. Дочь Всеволода? Ну можно подобрать князя с именем Всеволод, жившего в то время – это Всеволод Юрьевич Большое Гнездо, но согласно летописям, у него не было дочери Елены, равно как после княгини Ольги-Елены неизвестно, чтобы какая-то другая княгиня, даже под другими именем, вела сражения и перемещалась по исторической карте. Княгини, по летописям, только и делали, что основывали храмы да умирали, чтобы быть похороненными в церкви, «юже бе сама создала». А Ольга – явление исключительное, и узнать ее можно не только по двойному имени – Ольга и Елена, но и по деяниям, частично известным по летописям, частично в преданиях.

А ежели вместо Батыя был Буняк? Батыя сочинители 17-18 веков, толкуя давние сказания, любили предполагать вместо Буняка. Вспомним Галятовского:

\begin{quotation}
Прадедов своих чували, же Буняк (яко они мовять) але рачей Батий Царь Татарский
\end{quotation}

Рачей Батий... А если не рачей? Прадеды же про Буняка говорили, не про Батыя.

%Почитаем еще одно предание, из другой книжки.

%Примеч. С.Б. Подобное в другой статье Семенский вставляет в уста крестьянина на Волыни (см. выдержку в Вестнике Европы, 1886 г., ноябрь, стр. 330).

Садок Баронч во втором издании «Байки, предания, пословицы и песни на Руси»\footnote{«Bajki, fraszki, podania, przyslowia i piesni na Rusi», Lwow. 1886} на странице 76 приводит обширное предание о Подгорцах о «царице Елене», которая, кроме прочего, уходя от Буняка, насыпает горы, «которые тянутся от Львова до Киева» – стало быть, речь идет о таки о киевской княгине Ольге, Елене во христианстве, а не какой-либо другой Ольге.
 
Перевод Драгоманова:

\begin{quotation}
Царица Елена какими-то тайными средствами выкормила вошь до величины кабана и держала ее в большой стеклянной банке. При этом она объявила, что кто отгадает, какой это зверь, тому она отдаст руку вместе с своим царством; но кто возьмется, да не отгадает, будет казнен смертью. 

Много молодых людей, прельщенных красотою царицы и жаждою власти, пробовали счастья, но ни один не мог отгадать и все казнились на смерть. Наконец появился шолудивый Буняк\footnote{Одно из примечаний Баронча к преданию: «В Чернице под Бродами есть остаток рва, тянущегося от российской границы через Поникву; народ зовет его Буняков шлях» – село Черница (49°58'02.3"N 25°14'24.1"E) Львовской области, Пониква (49°58'54.8"N 25°07'58.5"E) в 7 километрах на запад от Черницы. А в 17 километрах на запад от Черницы – уже известные вам Подгорцы!}, дивное создание, которого глаза имели такие веки, что двое человек поднимали их вилами, если он хотел что-нибудь видеть, и тогда он все и всюду видел на сто миль. 

Пришедши к царице Елене и приказав поднять себе веки вилами, взглянул он на зверя в стеклянной банке и сказал: «вот так диво! да это же вошь!».

Царица огорченная, что она должна взять себе мужем такого гадкого человека, ушла с людьми, оставивши ему царство свое во владение.

Но Буняк, приказав поднять себе веки, увидел, где она скрывается и погнался за нею с быстротою ветра, так как желал больше царицы, чем царства. Но царица, уходя, приказывала сыпать позади себя горы, которые Буняк, хотя с трудом, переходил.

Все те горы, которые тянутся от Львова до Киева, насыпала царица Елена против Буняка.

Наконец ставши в Подгорцах, окопалась она недалеко города Плениска. В это время шел польский королевич с большим войском, чтобы завоевать какую-нибудь землю и найти себе жену, и, узнав о затруднении царицы Елены, пришел с войском до Подгорец, чтоб стать ей на защиту.

Царица скоро полюбила королевича и обещала отдать ему руку, как только он победит злого напасника\footnote{Невольно вспоминается, как княгиня Ольга на предложение руки и сердца византийского императора отвечала – мол, сначала сам крести меня!}.

Но Буняк, стремительно пробежав под горы Подгорецкие и узнавши, что там для обороны стоит королевич, так крикнул, что даже земля затряслась: «Ты, королевич, хочешь войском боронить вероломную и отнять у меня жену! Отец твой, высылая тебя в свет с войском, наказал тебе держаться правого пути, если ты не хочешь погубить себя и войско».

Перепуганная царица хотела с царевичем и войском его уходить далее, но Буняк так окружил их чарами, что они не могли двинуться с места, видя перед собою горы, далеко высшие, чем прежде.

И так как царица, держась за королевича, ни за что не хотела идти к Буняку, то разгневанный Буняк заклял их обоих словами: «Ты, царица, со своим дворцом, казною и людьми провалишься; раз только в год, в ночь перед Пасхой, и то на минутку только, ты будешь выходить на верх земли с твоей пышностью и богатствами. Ты ж, королевич, провалишься со всем твоим войском и будешь терпеть кару. Когда же Польша погибнет, тогда каждый год будет в пасхальную ночь открываться ход до твоего жилища, и кто будет так счастлив, что в эту минуту войдет и на твой вопрос: пора ли уже? ответит: «уже пора для тебя», тот станет твоим избавителем. Тогда ты выйдешь с твоим войском и отобьешь от врагов твое королевство».

Сказавши это, Буняк крикнул так страшно, что земля и горы затряслись и дворец с царицей, людьми, казною, королевичем и войском его провалился.

И действительно каждый год на великую ночь этот великолепный дворец выступает из земли, но только праведный у Бога может это видеть.

Раз двое мальчиков пастухов увидели, как он выступил из земли. Движимые любопытством, они глянули в отверстие, а там было много денег, на которых сидел панич, одетый, как немец. Сын богатого бросил туда для шутки шапку сына бедного соседа. Бедный мальчик стал плакать: «да меня же убьет отец, если прийду без шапки».

Отчаяние преодолело страх; мальчик вскочил в яму, чтоб достать шапку; тот панич насыпал ему полную шапку червонцев и помог выбраться из ямы, после чего дворец тотчас же исчез.

Мальчик, сын богатого, рассказал об этом случае отцу, который научил его на будущий год, когда заметит это видение, бросить нарочно шапку в яму, полезть за нею и возвратиться с таким же количеством червонцев, как соседский сын. На другой год мальчики опять имели то же видение.

Сын богача делал, как приказал ему отец, но панич его удушил и выкинул в окно, а шапку с дукатами дал бедному мальчику, чтобы тот отдал своей материл, которой в прошлом году отец его не дал ничего на упорядкование дома. Так тогда богач, потерявши единое дитя, умер от горя, а бедняк стал большим богачем и всегда нанимает обедни и акафисты за царицу Елену, раздает нищим милостынию за нее – и скоро окончится ея покаяние.

Раз на великую ночь старик-монах (базилианин) после святых разговен вышел из монастыря подгорецкого\footnote{Баронч также сообщает, что в монастыре «можно видеть образ, масляными красками писанный, представляющий княжну Елену громящею татарское ополчение. По одним она одержала знаменитую победу, по другим легла на поле битвы по взятии Плениска» – перевод Драгоманова.} на прогулку. Но едва он ступил несколько шагов, видит: проявился какой-то ход в горе, которого прежде там не было.

Вошел он в эту пещеру, и там нашел рыцарей в железной сбруе; сидят на конях. Впереди стоявший в светлой золотой сбруе схватил бубен и спросил базилианина, можно ли уже ударить, пора ли? Базилианин, перепуганный, убегая крикнул: «не пора!». 

Сейчас же за ним замкнулась пещера, и королевич с войском там будет сидеть до тех пор, пока счастливый у Бога человек на попадает на наго в минуту открытия пещеры и, не перепугавшись, ответит на его вопрос: «уже пора».

Тогда королевич выйдет с войском, освободит Польшу, а того, кто его избавит, поставить королем или королевой, смотря по тому, будет ли это мужчина или женщина. [...]
\end{quotation}

А вот вам из книги матери Оскара Уайлда, леди Уайлд, «Легенды, заговоры, суеверия Ирландии»\cite{wilde01} предание, связанное с островом Ратлином:

\begin{quotation}
На острове есть старые развалины, именуемые Замком Брюса, и о них ходит легенда, что Брюс и его лучшие воины спят зачарованным сном в пещере в скале, на которой стоит замок, и однажды они очнутся и присоединят остров к Шотландии.

Вход в эту пещеру становится видимым лишь раз в семь лет. Одному человеку довелось увидеть вход в это время, и, войдя внутрь, он оказался среди воинов в тяжелых доспехах. 

Он посмотрел вниз и увидел около своей ноги саблю, наполовину воткнутую в землю. Когда он попытался вытащить ее, все воины повернули к нему свои лица и взялись руками за свои мечи. Перепуганный человек сбежал из пещеры, но вслед ему свирепо неслись голоса: «Ух! Ух! Почему нам не дают поспать?». И воины лязгали своими мечами об землю с ужасным шумом, а потом всё замерло, и ворота пещеры закрылись с громким звуком, подобным раскату грома.
\end{quotation}

В прошлой главе наметилась связь рода Брюсов с Ольгой-Арлогией. Вот ее укрепление. Сходство предания о замке Ольги в Подгорцах и замком Брюса на Ратлине более чем очевидно.

Rathlin – верней произносится как «Раслин». Другие названия – Ratherin, Rauchryne. Скалистый остров этот лежит между Северной Ирландией и Шотландией. 

Считается, что в 1306 или 1037 году тут скрывался Роберт Брюс I (Robert the Bruce) один из многочисленных Робертов Брюсов из рода, известный борец за независимость Шотландии и ее король. При перенесении его останков в 1819 году в аббатство Данфермлайн (Dunfermline Abbey) Уильям Сколар сделал гипсовый снимок с черепа, что породило несколько копий, которые теперь показывают в разных музеях, а ученые пытаются по ним восстановить облик народного героя. Особенности строения черепа списывают на воздействие проказы, о которой сообщали некоторые современные Роберту Брюсу хроники.

Бегство Брюса именно на Ратлин основывают на единственном источнике – поэме «The Brus» Джона Барбора (John Barbour) 1377 года, а именно его третьей книге. Там и остров вроде бы назван по имени\footnote{Towart Rauchryne be se to far\\
That is ane ile in the se,\\
And may weill in mydwart be\\
Betuix Kyntyr and Irland,\\
Quhar als gret stremys ar rynnand\\}, и по описанию (между Kyntyr и Irland) также соответствует Ратлину.

Однако можно ли доверять исторической поэме как источнику? Как оценивают поэму историки? Они полагают, что Барбор свёл деяния трех разных Брюсов в одного и преувеличивал размеры армий. Но рассуждать можно иначе – что это ученые разделили одного Брюса на трех, и преуменьшают численность армий. 

В восточной части острова с Брюсом поныне связывают две достопримечательности. Первая это «пещера Брюса», куда можно добраться только на лодке. По одному из преданий, Брюс прятался именно в этой пещере, хотя в разных местностях Ирландии и Шотландии есть еще несколько  «пещер Брюса». Также показывают развалины «замка Брюса» – небольшой участок стены на утёсе несколько южнее «пещеры Брюса», а между ними еще несколько пещер в скалистом берегу (Caves Inannanooan). 

Согласно Барбору, Брюс вовсе не строил замок, замок уже стоял на острове, да и Барбор о пещере ничего не говорил. Пещера относится к легенде про Брюса и паука – когда после военного поражения Роберт Брюс бежал от англичан в какие-то леса и горы, где скрылся в пещере, то наблюдал за пауком, как терпеливо тот плел свою паутину. Паук неудачно пробовал закинуть одну нить за другую, однако не оставлял попыток, что сподвигло Брюса не упасть духом и снова собрать армию. Это предание скорее искусственное, а не вышло из уст народа\footnote{В 1828 году Бернард Бартон напечатал стихотворение «Брюс и паук» (Bernard Barton, Bruce and the Spider, книга «New Year's Eve and Other Poems», London, John Hatchard and Son, Piccadilly) – впрочем без географических ориентиров. В том же году Уолтер Скотт в первом томе «Дедушкиных рассказов» (Tales of a Grandfather) в главе 8, странице 109 приводит ту же историю, уже с привязкой к Ратлину. А историки полагают, что изначально предание было связано с соратником Брюса, Джеймсом Дагласом (Черным Дагласом), который прятался от англичан в пещерах в своем имении. 

Я пробовал найти источник сего мнения, но безуспешно. Херберт Максвэлл в книге «Роберт Брюс и борьба за шотландскую независимость» (Sir Robert Maxwell, «Robert the Bruce and The Struggle For Scottish Independence», London, 1909) пишет, будто неоднократно отмечалось, что в 18 веке Дэвид Хьюм в своей «Истории дома и происхождения Дагласов и Ангусов» (David Hume, «History of the House and Race of Douglas and Angus», London, издание 1820 года) приводит предание о пауке, но вместо Брюса там Джеймс (Sir Good James). Я разыскал эту книжку, однако не нашел в ней указанного предания.}.  
 
Что же, одна из пещер на Ратлине слывет Пещерой Брюса, с его же именем связывают и некие развалины, но истинные причины такой связи, скорее, выяснить уже нельзя.

Давнейшие сказания, в частности та же «Вторая битва за Маг Тьюиред» как бы между прочим говорят, что когда Туаха Дэ Дананн победили Фир Болг, то выжившие убежали к Фомойри и осели в Аннаре, Ислэе, Мэне и Ратлине\footnote{«who escaped from the battle fled to the Fomoire, and they settled in Arran and in Islay and in Man and in Rathlin».}. Можно сделать вывод, что островом тогда владели Фомойри.

В книге «Хороший народ: новые эссе о преданиях про фэйри» («The Good People: New Fairylore Essays») помещена статья Линды-Мэй Боллард «Фэйи и сверхестественное на Ричараи» (Linda-May Ballard «Fairies and the Supernatural on Reachrai), где рассматривается фольклор Ратлина, записанный самой Боллард в 20 веке. Предания о фэйри на острове сходны с представлениями других ирландцев. Фэйри описываются как небольшого роста, одетые обычно в красное и зеленое.

Одна из быличек такова:

\begin{quotation}
На дальнем краю острова жила повитуха. [...]

Однажды, в штормовую ночь, когда она уже спала, в дверь постучали – а я сказал, это была очень, очень штормовая ночь, и повитуха пошла к двери – а в то время на острове не было четырехколесных экипажей. Единственно у кого была повозка, это у Гэйджей\footnote{Gages. Gage – фамилия лэндлордов, владельцев острова (с 1746 года). Из них в узком кругу известна Катерин Гэйдж (Catherine Gage, урожденная Boyd, 1791-1852), которая написала двухтомник «История острова Ратлина» (History of Rathlin Island) с собственными иллюстрациями и картами. Книга не была опубликована. Дочь Катерин – тоже Катерин (1815-1892), ботаник и орнитолог, сделавшая акварелью иллюстрации к книге своего брата Роберта (Робертом звали и его отца) о птицах Ратлина. Катерин-младшая вместе с сестрой Барбарой занимались также зарисовкой растений Ратлина.}. И повитуха увидела за дверью четырехколесный экипаж, запряженный четверкой лошадей, и с ними был человек, и повитуха подумал это очень странно.

Сперва она решила, что спит, а затем она осознала, что есть нечто странное во всем этом и с этим человеком. Она не могла его рассмотреть в темноте.

Он сказал, что нужна ее помощь, что есть рожающая женщина. Повитуха решила, что ей опасаться нечего, и села в экипаж.

Ну, она села в экипаж и он покатила по дороге, и если бы я был на дальнем углу острова то смог бы показать это место. Ехали через болота, около стороны холма, где нет дороги. А вы понимаете, через болото, там очень топко. И экипаж на увязал. И они прибыли к этому холму позади Брокли. И когда они приблизились к холму, часть холма открылась, и экипаж с лошадьми въехал в холм.

И внутри, понимаете, было прекраснейшее место, будто дворец внутри. Там было всё, что можно вообразить.

А перед этим много островитян пропало, юноши и девушки пропадали с острова. И никто не знал, что случилось, только думали, что их похитили, ну вроде как приплыл кто на корабле и увез их, вот так решили.

Но внутри холма, они оказались там, и повитуха узнала многих из них. Ее провели к роженице. А я забыл добавить тут, что там вокруг было много фэйри, маленького народа (the wee folk). Их было очень много в холме, видите ли.

И они подошли к ней и сказали, что дадут ей что угодно, лишь бы она осталась. «Нет», – она ответила, «Нет». Она не останется.

И они предложили ей перекусить, и она согласилась, потому что была голодна, и подумала, что вреда не будет, если поест.

А около камина сидела девушка, нянча ребенка, и она пела на гаэльском языке. Сейчас я к сожалению не знаю, не помню слов на гаэльском, что именно она пела, однако в переводе на английский это было «ничего не ешь, ничего не пей, не оставайся ночевать». Она повторяла это снова и снова на гаэльском. То есть, было было предупреждение, для повитухи.

И она начала настаивать, чтобы ее вернули домой. Ее усадили снова в экипаж и довезли до самой двери родного дома. 

Повитуха была всем этим обеспокоена и не знала, что делать. Она отправилась к Гэйджам, и рассказала землевладельцу всё про это, и он предупредил ее, чтобы даже под угрозой смерти она никому про это не рассказывала, или он прогонит ее с острова. Но она видимо проболталась, раз история теперь известна. Вот как это было. [...]
\end{quotation}

От Ратлина вернемся в наши края.

Нехилая – площадью примерно со Швейцарию – часть Западной Украины. На таком огромном пространстве в 19 веке записаны предания о Шолудивом Буняке и княгине Ольге, вероятно в какой-то мере отражающие действительно произошедшие события. Нет дыма без огня.
 
%Думаю, это весомый повод, чтобы отнестись к содержимому преданий хотя бы как к отголоску действительно произошедших событий, если не буквальной их передаче.

Не всегда Буняк и Ольга совмещены в пределах одного сказания. Сложно и упорядочить предания по времени.

Для меня нетрудно предположить, что Буняк в самом деле мог жить на протяжении многих веков, от времен великокняжеских времен до «козацких», если уж я принимаю как данность большое долголетие княгини Ольги.

Я бы расположил предания на временной шкале так – сначала действуют Буняк и Ольга, потом один только Буняк – уже как прибившийся к козакам и совсем похожий на мертвеца.

Еще с летописей прослеживается странность внешнего облика Буняка, впечатавшаяся в его прозвище – Шелудивый. Затем (как я отношу предания по времени) он предстает как обладатель некоего поражающего или всепроникающего взгляда, сдерживаемого опущенными до земли веками. После – уже как существо с «печенками за спиной» или нечто костлявое, гнилое, и вместе с тем неуязвимое для обычного оружия. Как относиться к этим сообщениям, как пробиться через слова к сути?

Совокупность западноукраинских преданий позволяет судить о том, что княгиня Ольга и княгиня или царица Елена – одно лицо. На это указывает не только известное двойное имя «нашей» Ольги, но и связь ее с Киевом, откуда неприятель гнал ее на запад. И разве известна на Руси еще какая-то воинственная княгиня с двойным именем Ольга-Елена? 

Но летописи похоронили ее раньше, чем по времени Ольга действует в сагах и преданиях. Последние дают возможность заглянуть в прошлое, не освещенное летописями, и зачастую более подробно и обстоятельно, ведь ольгины камни да валы можно увидеть и потрогать руками. А в Киеве?

Около Выбутов или Завалова каждая кочка в поле имеет название и этнографы записали связанное с ней предание. Про киевские кочки, пусть не ольгины или буняковы, слыхом не слыхать – то ли память стерлась, то ли фольклористов не нашлось, чтобы записать, а может им удобнее изучать глухомань, хотя это понятно, там старина сохраннее. Живут себе люди безвылазно веками около вала и знают, что это вал царицы Ольги. Да и вал никто столетиями не трогает. 

И если на правом берегу Киева хоть что-то да осталось, то левобережье перекроено жилмассивами и, казалось бы, дело поисков старины тут безнадежно.

%Но попробуем и здесь найти городища да, пожалуй, исчезающие замки.
