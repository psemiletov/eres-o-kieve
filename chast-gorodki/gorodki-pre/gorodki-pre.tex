\chapter{Легкость обретения Городка}

%Иногда второстепенное исследование, питаемое благодатным черноземом краеведения, разрастается до важного. Один за другой цепляются новые сведения, дело пухнет, как на дрожжах и уже не помещается в главу, что я хотел отвести под Городок.

Про Городище или Городок я давно и смутно знал, что есть такое на околице Троещины. Молва краеведов гласила, что тут при Литве жили Олельковичи\footnote{Род Олельковичей повёлся от Александра (Олелька) Владимировича, князя Слуцкого, которого литовский князь Казимир посадил править Киевом в 1443 году. В генеалогических работах считается, что род Олельковичей-Слуцких пресекся в 1593 году.}. Замок князя Симеона Олельковича! Романтический восторг.

Как в случае со Змиевой пещерой, я не подозревал, к каким открытиям приведет меня исследование этой местности, и какое направление придаст всей книге.

Однако прежде чем начать беседу о Городках обстоятельно, надо разобраться с водной системой левого берега, ее прошлым и настоящим. Левый берег всегда был у историков и киевоведов чем-то второстепенным, либо местностью, про которую можно выдумывать небылицы. 

Моя задача много скромнее, чем описание всех рек, ручьев и озер Левого берега. Мне нужна обоснованная документально модель водной системы, которую я могу использовать в рассуждениях о городищах. Построение такой модели подразумевает изучение взаимосвязи водоемов на протяжении столетий, основываясь на летописях, документах, картах, полевых наблюдениях и прочих источниках.

Я постараюсь дать обзор рек и озер киевского Левого берега, какими они были и какими стали, чтобы относительно свободно рассуждать о «городках». Коренные изменения, произошедшие со здешней водной системой, и отрывочные сведения об её прошлом делают задачу очень трудной. По возможности восстановленная картина течения рек и изменений местности послужит ключом к некоторым открытиям.

Перекроенный пуще правого, левый берег Киева меньше документирован историей. Находясь долгое время на окраине Черниговской губернии, он попадал на карты отрывочно и более известен по земельным документам, нежели в изображении. 

Редким подробным планом рек левого берега является загадочная карта землемера Сноевского, коей я посвящаю целую главу.

На вооружении у меня также отменный план «Участок р. Днепра у г. Киева» 1914 года и скудная старая его версия, 1910 года. Много пригодились советская карта 1943-го, того же года немецкие аэрофотоснимки, план РККА 1930-х и конечно же трехверстовки Шуберта середины 19 столетия.

Однако не всем картам я доверяю.

%, однако разжевать её помогает лишь план Сноевского, эдакий Розеттский камень, благодаря коему многое становится ясным – но как оно прояснилось для самого Сноевского?
