\chapter{Чудеса}

Издавна в Киев шли паломники. Привычно знать, что в прошлые века стягивались они в Лавру и другие монастыри, сообща и по одиночке, ежегодно или давая себе обеты вроде «сходить пешком в Киев святым помолиться». Полагаю, обычай сей возник еще в поганские времена, и здешние холмы носили священное значение.

Кому и как именно поклонялись тогдашние славяне, судить можно лишь по отголоскам давних верований, сохранившихся в обрядах, записанных в христианское время, да по редким упоминаниям о них же в старинных источниках.

Поскольку сейчас мы исследуем местность, с которой так или иначе связаны названия Радосынь и Радунка, полезно хотя бы кратко изучить представления о народном празднике Радуницы.

Отправной точкой послужит, пожалуй, выдержка из «Сказаний русского народа, собранных И. П. Сахаровым»:

\begin{quotation}
Радуницкая неделя начинается с Фомина воскресенья и вмещает в себе старые обряды наших отцов.

Первый день сей недели называют: Красная горка, понедельник: Радуницею, вторник: Наской, или Навий день, или Усопшие радованцы. 

На этой неделе оправляют: хождение вьюнитства.

Малоруссы Фомин понедельник называют: могилками, гробками, проводами; а в Мохновском округе Киевской губернии величают сей день: бабским праздником. В Киеве наша Радуница известна под именем: проводов.
\end{quotation}

Фомино воскресенье приурочивают к первому воскресенью после Пасхи, а все праздники радуницкого цикла относятся к поминовению родителей. Судя по летописям, Радуница приходилась порой на вторник. Вообще насколько можно судить, в разных местностях Радуница под различными именами отмечалась то в понедельник, то во вторник после Пасхи.

В самом названии Красной горки заложено описание местности, где отмечают этот день – на горках, холмах.

Радуницу, понедельник, именуют также: Радунец, Радоница, Радавница, Радованица, Проводы, Гробки. В польском Варшавском повете отмечали Radonice.

И вот как ладно выходит – местность Радосынь, речка Радонка, и село Погребы. В Погребах сложно не усмотреть связь с Гробками. А над Погребами – большая гора, где я полагаю стоял Городец (куда вроде почему-то никто не заходил).

В грамоте Петра I 1700 года Митрополиту Варлааму Ясинскому на вотчины и угодья находим кстати старинное именование нынешних Погребов:

\begin{quotation}
велено ему Богомольцу нашему Преосвященному Митрополиту, и иным по нем впредь будучим Киевским Митрополитом прежними и новоданными к Святософийскому Дому вотчинами, а имянно деревнею Погребками и Зазимьем [...]\end{quotation}

Вот оно как - не Погребы, а Погребки. Почти что гробки.

Из моих рассуждений проступает священный холм, где праздновали Радуницу – гулянья и пиршества на кладбище. В главе про Щекавицу я приводил описание того, как происходило подобное в 19 столетии на Щекавице.

Но как быть с тем, что сейчас на Киевщине этот праздник именуют Гробки, а в 19 веке – Гробки и Проводы? Из географического наложения Погребов на Радосынь можно полагать, что Погребы это другое название Радосыни, а оба слова означают производные от «Гробки» и «Радоница».

Просто Радоница, кажется, слово более давнее. Оно было в ходу в летописях для датировки событий – например, пожар случился в таком-то году на Радуницу. В Стоглаве (сборник церковных уложений, составленный при царствовании Ивана Грозного) разобран отдельный вопрос:

\begin{quotation}
А о велице дне\footnote{Великдень, Пасха} окличка на родоницы не творити вьюниц. и всяких в них беснований.

Ответ: Чтобы о велице дни и на родоницы оклички не творили. и скверными речми неупрекалися.
\end{quotation}

Сведения про отмечаемый на Радуницкой неделе Навий день, упомянутый Сахаровым, я отыскал в первом томе «Русских простонародных праздников и суеверных обрядов» И. Снегирева, напечатанном в 1837 году. Там сказано, что «Радуница, празднуемая в великой и белой России во вторник на Фоминой неделе, в Рязанской губернии называется наск\'ой и навий день: что сходно с Литовским словом navie мертвец\footnote{Под Литовцами Снегирев подразумевал народ Жмудь.}; ибо в этот день сбираются на могилки на кладбище поминать родителей».

Сходный по названию праздник, только в Чистый четверг (четверг Страстной, Пасхальной недели) именовался на Украине Навським великоднем.

Поверья, с ним связанные, были отличны от Радуницы – считалось, что мертвецы трижды в год возвращаются с того света: в Чистый четверг, второй раз – «когда цветет жито», и третий – на Спаса. И вот в Навский великдень покойники на кладбище выходят из гробов и по звуку колокола собираются у церкви, заходят в нее, совершают службу и затем возвращаются в могилы. Невольных свидетелей могут задушить. Под Пинском в 19 веке бытовало представление, что в Навский Великдень русалки катают во ржи яйца.

Кого в старину называли «навами»? В разговорном языке это слово не сохранилось кажется нигде, кроме самого названия праздника «Навского великодня». Когда в 19 веке стали записывать народные предания, прилагательное «навский» уже было для крестьян бессмысленным набором звуков, никто тогда не говорил «нави» на мертвецов.

А именно таково значение слова «навь» и «навье». Проверим.

Радзивилловская летопись за лето 6598 сообщает: 

\begin{quotation}
приведе Янка митрополита Иоана скопьчину егоже видевше людье рекоша се навье пришелъ годъ пребывъ оумре
\end{quotation}

Люди видели скопца Иоана и говорили – это навье пришел. Год пробыв, он таки умер. А в Ипатьевской летописи вместо «навье» стоит просто «мертвец».

В сочинениях Иоанна Ексарха Болгарского находим:

\begin{quotation}
Яко и навь из гроба исходящи.
\end{quotation}

Уже этого достаточно, чтобы сопоставить слово «мертвец» с «навье». Корень, возможно, в слове «небо» (на санскрите «нибо», «набо») – «б» легко переходит в «в».

Ипатьевский список:

\begin{quotation}
В лето 6600\footnote{Наука вычисляет его как 1092-й нашей эры.}.
Предивно бысть чюдо в Полотьске: у мечьте и в нощи бывши тутень\footnote{По Далю: «ТУТЕН м. тутень, стар. вост. сиб. шум, гул, зык; ныне бол. конский топот».}, стонаше полунощи, яко человеци рыщуть бесы по улици; аще кто вылезяще ис хоромины, хотя видети, и абье уязвен бяше невидимо от бесов, и с того умираху, и не смеяху излазити ис хором; посем же начаша во дне являтится на конех, и не бе их видити самех, но кони их видити копыта; и тако уязьвляху люди Полотьскыя и его область. Тем и человеци глаголаху: яко навье бьють Полочаны;
\end{quotation}

Летопись рассказывает о странных явлениях в Полоцке – ночью, под конский топот (или просто некий шум), и стоны в полночь, по улицам рыскали бесы, и люди сидели по домам, а кто вылезал, тот уязвлен (покрывался язвами) был невидимо от бесов и умирал. После этого днем бесы начали являться на конях, при этом сами не были видны, замечали только копыта их лошадей – вероятно, речь идет о следах. И так уязвляли они население Полоцка и области. Люди говорили: «яко навье бьють Полочаны» – будто навье бьют Полочан.

Отметим две вещи. Невидимок называют бесами. Невидимок уподобляют навам. Последнее, выражение «яко навье бьють Полочаны» было понятно современникам летописца, однако неясно в 21 веке мне. Что значит – как навье бьют Полочан? Бьют как мертвец, мертвецы? Звучит странно. Слово «яко» употреблялось еще, кроме в значении «как», также и «так», «ибо». Тогда – «так навье бьют Полочаны».

А может, у «навье» было еще какое-то значение? 

По своим свойствам эти невидимки ближе к Туаха Дэ Дананн, незримость коих для людей обеспечивали некие «чары Фет Фиад», и представителям Чуди Белоглазой – Дивьим людям\footnote{В фольклоре того же Урала нашлось место и одноногим, будто некоторые Фомойри: «Легенды такие болтали, что где-то жили люди с одной ногой, одноногое племя («племя одноногих», – вставляет жена Домна Терентьевна), а где жило, не скажу, не знаю; передвигались так, что схватятся за руку двое и пошли. Это даже не дедушка Петр Леонтьевич, а его отец Леонтий знал». Записано Кругляшовой В. П. в д. Бочкаревой от Васнина Карпа Александровича (1905 г. рожд). «Новые записи преданий на Урале» Публикация В. П. Кругляшовой, О. Климовой, С. Арутюнян. Фольклор Урала. – Свердловск: [УрГУ], 1976, – [Вып. 1.]: Народная проза. Подобные одноногие люди со способом перемещения «сцепкой» были известны на Украине, в частности о них рассказывали в 19 веке экспедиции Павла Чубинского в Ушицком уезде, называя этих существ сыроедами.}. Вот один из рассказов начала 20 века о них\cite{onuchkov01}:

\begin{quotation}
Дивьи люди живут в Уральских горах, выходы в мир имеют через пещеры. В заводе Каслях, по Луньевской железно-дорожной ветке они выходят из гор и ходят между людьми, но люди их не видят. Культура у них величайшая, и свет у них в горах не хуже солнца. 

Дивьи люди небольшого роста, очень красивые и с приятным голосом, но слышать их могут только избранные. Они предвещают людям разные события. [...]

В Тюменской монастырь приходил один из Дивьих людей в яве. Солдаты в него стреляли, но пули не брали.
\end{quotation}

К преданиям о Чуди весьма примыкают по ощущению истории про различные селения, ушедшие в озёра или под землю, да и сами былички про Чудь насыщены подобными «городами-призраками». Как не вспомнить тут народный рассказ о деревне, сгинувшей в болоте, а прежде озере, Ковпыт? 

Однако это еще не всё. Из статьи В. Кандинского «Из материалов по этнографии сысольских и вычегородских зырян\footnote{Народ Коми.}. Национальные божества»\cite{etnoboz3}:

\begin{quotation}
Зыряне теперь не признают никакого родства с чудью\footnote{Разве что, допустим, держалась молва о том, что население деревни Большелуг (ныне в республике Коми) – потомки Чуди.}. Среди них ходят известные в литературе рассказы от крещении св. Стефаном чуди, о ея бегстве, зарывании себя и имущества в ямы, носящие и теперь названия «чудских». И старики указывают на эти ямы, но говорят, что их разрывать нельзя, так как они нечистые, в доказательство чего приводят пустынную, неплодородную местность близ станции Чукаыб по Вятскому земскому тракту, оде были некогда могилы чуди. Зыряне помнят только, что они не последовали примеру чуди и крестились и с тех пор живут «по христианскому обряду». [...]

Несколько раз, впрочем, приходилось мне слышать, что чудь была вовсе не дикий народ, а уже оседлый, земледельческий. Доказывая его высокую культуру и богатства, эти зыряне ссылались на тот ни на чем не обоснованный, факт, что чудинцы обрабатывали свои поля серебряными орудиями. [...]

Мне удалось найти всего несколько весьма слабо обозначающихся следов древней языческой религии зырян.

Первый из этих следов сквозит в пословице, распространенной среди женщин: «чурки буди эн вомзясь», что по-русски значит «не испортись». И один старик объяснил, что слово чурка (имеющее теперь измененный смысл и означающее незаконнорожденного ребенка), происходит от Чурила, главного бога чуди.

Уверяя, что он слышал это в детстве от стариков, он сообщил еще, что чудь поклонялась коровам, кошкам и другим домашним животным. Кроме этого старика, больше никто и никогда не говорил ничего подобного.
\end{quotation}

Вспомним, что Чудь – не только земледельцы, но известные рудокопы и металлурги\footnote{Мамин-Сибиряк в 1889 году писал: «Чудь существовала задолго до русской истории, и можно только удивляться высокой металлической культуре составлявших ее племен. Достаточно сказать одно то, что все наши уральские горные заводы выстроены на местах бывшей чудской работы – руду искали именно по этим чудским местам».}. А как издавна называлось поле около Гнилуши, где я нашел следы древней металлургии, и где, на другой стороне озера, такие же следы – в погребальном валу? 

Чуриловщина!!!

Быть может здесь трудились представители Чуди, с ними же и связан погребальный, многослойный вал? Или вал – явление позднейшее, а обитавшая тут некогда Чудь была уже предметом поклонения?

Выстраивается четкая картина давних священных, чудесных земель левого берега Киева.

Сначала поганский паломник попадал в Радосынь, нынешние Погребы, у подножия высокой горы, на которой теперь стоит ТЭЦ-6. В древности, с одной стороны этого холма протекала Десна, с другой было здоровенное озеро, позже ставшее болотом Ковпыт. С ним связано предание об исчезнувшем селении. А на самой горе я предполагаю летописный Городец, по крайней мере один из них.

Вероятно где-то возле Погребов начиналась и речка Радунка. По ее водам плыли к многоярусному погребальному валу около Выгуровщины, около Чуриловщины. Заметим – раскопанное Завитневичем «городище» не застраивалось по 20 век – подобно тому, как на севере опасались нарушать чудские «городища». 

Плывя дальше по Радунке или следуя берегом, давний поганский паломник вскоре достигал Лысой горы, носившей, вероятно, важное значение в те времена.
