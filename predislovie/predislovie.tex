\chapter*{Предисловие}
\markboth{\MakeUppercase{Предисловие}}{}
\addcontentsline{toc}{chapter}{Предисловие}

\section*{Се начнем повесть сию} 

Никогда не думал, что напишу такую здоровенную книгу. Потуги были. В юности я был ударенный в изучение фольклора, и раздобыл такую амбарную книгу – листы в мелкую клетку, броневая зеленая обложка, вес несколько килограммов. Начал писать в ней энциклопедию славянской демонологии, да застрял. Хотя прежде того сил хватило на четыре рукописных тома народного календаря – «Месяцеслова» – сведенного из множества источников. Приметы, поверья на каждый день.

Так вот сам по себе человек только прозу пишет, а в краеведческих книгах всегда прямо или косвенно участвуют другие люди.

Посему благодарю тех, кто так или иначе способствовали появлению и развитию книги.

Особое спасибо родителям, с которыми обсуждал свои изыскания сразу, как только оные возникали. Коле Арестову за долголетнее краеведческое сотрудничество. Сколько хожено троп, сколько преодолено склонов! Насте, часами выслушивавшей мои исторические выкладки – за некоторые фотки и ценные сведения. Свете Семеновой, Марине Чуприне – за нужные в работе книжки. Неле Арестовой, Андрею Савченко, Саше Ураловой и снова Свете за тонкости перевода с языков, в которых я не силен. Однако не всегда я прислушивался к вашим советам. Алине – за решимость лезть со мной через бурелом всё дальше. Любе – за книжки и сведения о подземле. Зое Колзуновой за некоторые сведения про кости и черепа. Даше Кононюк за совместные краеведческие вылазки.

%Всем кого я не упомянул, дабы ересью не бросать на них тень!
Также и тем, чьими работами я пользовался как источниками.

\section*{С чего вдруг}

Долгое время я не считал себя краеведом, но кажется постепенно стал им. Как увлекся?

С 2005 года мы с друзьями, объединившись под названием «студия Дрымба», снимали любительское кино – вначале на купленную в ломбарде VHS-камеру, затем на MiniDV. В 2009-м стали делать полнометражную фантастическую картину «Сваха», и по ходу запечатлели на видео много краеведческого материала. Брали интервью у местных жителей подле озера Глинка, делали кадры по течению реки Лыбеди, включая тот короткий отрезок, где воды ея бурно протекают в естественном русле под Лысой горой.

«Сваха» зависла, а накопившийся по Киеву материал сподвиг нас к созданию, частью на его основе, большого документального фильма «Киевская сюита», выложенного в сеть в 2010 году сразу после монтажа. Когда мы снимали «Сюиту», пришлось копнуть историю да краеведение на уровне несколько более глубоком, чем листание справочника. Статьи из справочников, как правило – нечто вроде слюны, которую глотаешь, когда очень хочется пить, а никакой газировки под рукой нету и купить негде. Вроде и жидкость, но жажду знания не удовлетворяет.

Спустя год после «Сюиты», весной 2011-го, меня осенило вдохновение писать краеведческую книгу. Я еще не погрузился в предмет целиком, не нырнул в него с головой, а так – опустил туда лицо и пытался открыть в мутной воде глаза. Задумал осветить спорные и загадочные сведения о городе, отчасти поэтому и название выбрал – «Ересь о Киеве», впрочем созвучное другой моей книге, про звукорежиссуру – «Ересь звукозаписи».

«Ересь о Киеве» сначала была сделана как сайт. Это позволяло мне всё время быстро вносить в книгу правки. Читатели спрашивали меня, а нельзя ли выложить её одним файлом? Я неизменно отвечал – нет, ибо мне больше подходит HTML, так удобнее верстать.

Я пользовался множеством книг, большей частью дореволюционных, и прежде выкладывал их в отдельном разделе книги-сайта. То же касалось карт. В конце 2012 года я запустил проект \href{http://semiletov.org/kievograd}{Киевоград}, куда перенес накопленную библиотеку, карты, и также завел там фотоархив и рубрику для фильмов про Киев. Задача Киевограда – предоставление краеведческих материалов по Киеву в свободном доступе.

Медленно пришел я к мысли, что надо таки сверстать электронную книгу, ибо сайт, как ни крути, штука временная. А книгу можно куда-то зафитюлить в сеть, и пойдет гулять. Поэтому в третьей редакции, «Ересь о Киеве» превратилась из книги-сайта в полноценную книгу формата PDF.

Третья редакция стала совершенно новым произведением, от старого сохранившим только название. Это не дополненный вариант прежней книги, превзошедший ее объемом в десятки раз, а именно другое содержимое, выражение иных взглядов.

Началось всё снова с кино. Весной 2013 года мы принялись снимать цикл краеведческих фильмов «Киевская амплитуда». Осенью, параллельно, я стал делать еще один цикл – «Планету Киев». Оба смотрите на Киевограде либо онлайн в Ютубе. 

Съемки расширили мой круг интересов. Я побывал в Змиевой пещере. С Колей Арестовым мы облазили все окрестные склоны. Кирилловские высоты, Логово Змиево захватили меня и привели к открытиям, ставшими стержнем новой «Ереси». Кирилловской пещере посвящаю в книге отдельную часть. Что до видео из пещеры – смотрите «Киевскую амплитуду».

Тогда, в 2013-м, я собирался завершить книгу одновременно с монтажом серии «Логово Змиево», но вот фильм уже был готов, выложен в сеть, а «Ересь» всё писалась, писалась, раскручиваясь по спирали и пуская ветки во все стороны.

Местами она получилась очень подробной, но хотелось поделиться не просто итогом размышлений, но самими размышлениями с исходными данными, источниками. Поэтому, если я в чем-то ошибаюсь, у вас есть всё для проверки или построения каких-то своих выводов.

А работа над четвертой редакцией началась сразу после выпуска третьей, осенью 2015-го. Поначалу это было исправление ошибок (отдельное спасибо за указания на них Александру Петруку) и попутно шероховатостей слога. Вообще я хотел отдохнуть от «Ереси» и стал писать совсем другую краеведческую книгу, легче, меньше.

Но вышло иначе – и появилась новая редакция, еще больше предыдущей, да еще основательно переписанная. Я даже хотел переименовать ее, но кажется, от единожды принятого названия не уйти. 

Хотя по моим ощущениям, получилась совсем другая книга, лишь сохраняющая подобие прежней. Или – книга, более ставшая собой, чем была.

Сразу по выходу четвертой редакции к «Ереси» стал притягиваться новый материал, и дополненная им книга выходит уже в этой, пятой редакции осенью 2017 года. Существенно увеличились и были переделаны главы про Зверинец, Иорданскую церковь и Лысые горы. Точечных изменений претерпели и некоторые другие главы, я уж и забыл какие.

По февраль 2018 года я время от времени вносил в книгу кое-какие изменения и сразу заливал обновленную версию в сеть, но под номером прежней версии.

В конце февраля, добавив еще разные уточнения, мне приходится как бы утвердить их уже шестой редакцией. Отличия от пятой – в точности сообщаемых сведений. 

В седьмой версии были убраны полдесятка фотографий. Иллюстративный ряд теряет в освещении местности, зато книга приобретает первозданную «лицензионную» чистоту. Зарекался использовать чужие снимки, не попавшие в общественное достояние.

К лету 2018 года седьмая редакция обновилась новыми сведениями о болоте Ковпыте и ручье Омелютинке, однако нового в книге не так много, чтобы увеличивать номер основной редакции, посему явилась версия 7.1.

В декабре 2018 года была существенно подправлена глава о летосчислении, что дало повод к выпуску редакции 7.1. В 2019 году была продолжена работа по исправлению, уточнению и дополнению текста. В 2020 оная работа продолжилась, к тому же надо было лучше синхронизировать книгу с другим моим трудом, «Словарем киеведа». 

Так рождалась редакция 8.0, которая дополнилась затем новыми материалами по Зверинецким пещерам и вообще Зверинцу. В восьмой редакции много чего я исправил, проверил приведенные координаты, а также привел их к одному только формату – градусы и минуты, причем не в «правильном» типографском виде, а с обычными двойными и одинарными кавычками, чтобы обеспечить совместимость не только с геодезическими программами, но и популярными электронными картами. 

Девятая редакция книги возникла опять же в ходе согласования с материалами "Словаря киеведа", однако простое согласование обернулось основательным пересмотром некоторых вещей и сверкой летописей по светописным копиям, если они были доступны. Например, я, кажется, таки вычислил место урочища Курган в Бабьем яру. Добавились впечатления от новых краеведческих вылазок. Мне окончательно стало понятно, что пещеру из дела Бейлиса в наше время найти уже невозможно.

Перечитывая старые главы, с горечью убедился, что описанное в них всё более приобретает историческое значение, более не существуя в природе.

Работа над десятой редакцией закрутилась опять же в связи с большой правкой Словаря, а в 2022 году затянулась на, казалось, неопределенное время, пока весной 2023 я неожиданно не вырулил к свету в конце тоннеля. В десятую редакцию добавились две новые большие главы – про Копырев конец и капище Волоса, а поднятые темы затронули, так или иначе, некоторые другие части книги. Вообще про язычество много чего добавилось. Конечно же весь текст так или иначе подвергнулся правке – иногда это касалось буквально поправке на мое текущее мировоззрение, иногда относилось к неточностям в цитировании летописей, краеведческие сведения также уточнялись сообразно развитию моих представлений.
 
Попутно у меня продвинулись исследования о взрыве Зверинецкого форта, я по разным источникам воссоздал окружающую местность по состоянию на то время, и всю эту глыбу нового материала не хотел помещать в "Ересь", дабы не застрять в правке окончательно. Править всегда труднее, чем писать заново, особенно когда это сопряжено с вёрсткой множества иллюстраций.

Поэтому я решил писать отдельную книгу про Зверинецкий форт, оставив материал на эту тему в Ереси почти нетронутым, лишь кое-что уточнив.

Десятая редакция также ознаменовалась долгими, огромными трудами по упорядочиванию исходника книги. За более чем десятилетие структура книги на диске, в виде файлов и каталогов, развивалась естественным неряшливым образом, например все иллюстрации к такой-то части могли лежать в одном каталоге, а не чтобы на каждую главу по каталогу, куда и текст главы, и картинки. Накопилось множество неиспользованных иллюстраций, их вариантов, и копий в большем разрешении – изначально была мысль выпускать две версии Ереси, с качественными картинками и обычными. Но Ересь до того распухла, что я давно не думал о втором варианте. Весь этот отработанный материал надо было вычленить из исходника, и заново упорядочить структуру, на что и ушло между прочим много месяцев. Если бы не мой редактор TEA и некоторые функции рабочей среды Plasma, то наверное потратил бы и год, а то и несколько.


\section*{Электронное издание} 

Я выкладываю «Ересь о Киеве» в сеть как общественное достояние (public domain). Это значит, что вы можете использовать книгу как угодно, в рамках приличий. Проще говоря, вот существует допустим классическое произведение, у него есть сочинитель, но произведение принадлежит человечеству. Так и с этой книгой.

Хотя у электронных книг есть минус. Они имеют судьбу чисто информационную, не привязанную к печатному экземпляру.

У меня на полке стоит книжка «Киев. Справочник-пут\-еводи\-тель» 1954 года издания, небольшая такая, в коричневой обложке, а бумага до сих пор белая. На одной из последних, пустых страниц я нашел карандашные записи:

\begin{quotation}
Справочная \textbf{0-0-9}

Оперный театр

\sout{4-71-84}
\end{quotation}

и под углом:

\begin{quotation}
5-51-34

Оперный театр
\end{quotation}

Эта заметки, судя по телефонным номерам, сделаны примерно во время, когда книга увидела свет – пятидесятые. Кто писал эти строки, чья рука, какая судьба у этого человека? 

Ведь на одной странице развернулась целая история. Сначала человек узнал номер справочной – вероятно, телефон был внове, появился в семье впервые. Затем человек получил один номер Оперного театра. И записал его спокойно, ровно. Позже (насколько?) он получил взамен неправильного номера (который поэтому был перечеркнут) новый, и записал его в спешке, в неудобном положении, наискось!

Книга побывала и в некой библиотеке, чей штамп наполовину стерся, либо был плохо проставлен. А как потом книга попала на букинистическую раскладку – в коробку на асфальте, где среди водорослей можно найти перлы, и всё по одной цене?

В том же чудесном месте – любимом читающими киевлянами и не только ими книжном рынке Петровке – я приобрел еще пару замечательных книг со следами чужих судеб.

Первая – неказистый на вид томик очерка «Киев» Шулькевича. Приехал я домой, рассматриваю – ба, да  этой книжкой владел археолог Дмитрий Яковлевич Телегин. Вот он написал год покупки – «1963». Карандашные пометки на полях, черновой набросок схемы давнего Киева, подчеркнутые строки. И я понял – раз книга Телегина попала на Петровку, сам археолог умер.


\section*{Источники} 

Конечно же, источники, приведенные мною в конце книги – лишь малая доля общего их объема, туда я поместил лишь цитируемое. Выдержки, если они на старославянском либо «роськой мове» Великого княжества Литовского, даю в подлиннике. Важнейшие цитаты на латыни или греческом стараюсь приводить в своем, по возможности точном переводе, а также в подлиннике. При цитировании летописей, для удобочитаемости помещаю из обработанных учеными текстов Полного Собрания Русских Летописей, где подлинник трактован и потому несколько искажен. Наиболее важные выдержки, однако, беру из светописных копий подлинников.

\section*{Об иллюстрациях} 

В качестве иллюстраций я использую много старинных фотографий и изображений, за давностью лет перешедших в общественное достояние. Существует много фоток, которые я хотел бы показать в этой книге, да не могу из-за авторского права и всяческих связанных с ним ограничений.

Современные снимки в этой книге сделаны, за редкими исключениями, мною и отдаются в общественное достояние.

Увы – изображения, полученные со спутниковых карт, сурово защищены всевозможными коммерческими лицензиями, поэтому я вынужден был, несмотря на всю заманчивость, отказаться от них, за редкими исключениями и с указанием копирайта. Из лицензионных соображений я использую лишь старые карты, которым больше 70 лет – они перешли в общественное достояние. Иногда я помещаю советские карты, да немецкие аэрофотоснимки 1918 и 1943 годов.

\section*{О координатах}

Поелику книгу я пишу для вечности, а городские адреса бренны да изменчивы, буду частенько давать координаты, в геодезической системе WGS-84, которая сейчас используется в GPS и большинстве электронных карт. В СССР была другая система координат, СК-42. Есть еще СК-63. Существуют формулы пересчета из одной системы в другую, программы, онлайн-конвертеры, различные макросы и тому подобное. 

Я привожу координаты в формате градусов и минут. Пример: "50°28'6.66"N 30°29'56.42"E". Буква N означает «North», или северная широта. E – «East», восточная долгота.

В старых редакциях книги я давал еще координаты в десятичных градусах. Пример: 50.468517°, 30.499005° либо без знака градуса. Сначала идет широта, потом долгота. 

\section*{Программное обеспечение} 

В работе над этой книгой мне помогало разнообразное программное обеспечение, о чем я не могу умолчать.

Вся работа уютно происходила под операционной системой Linux, в дистрибутиве Mageia и рабочей среде KDE, а с лета 2017 года – в среде Mate. С 2019 года работаю уже в Manjaro Linux, а в 2020 перебрался в Arch, а затем сменил Mate на KDE/Plasma. 

Текст я набираю и верстаю в редакторе \href{http://semiletov.org/tea}{TEA}, собственной разработки. Программа Okular всегда под рукой для чтения сканов в форматах PDF и DjView. Вёрстка подготовлена при помощи удивительного средства вёрстки \href{http://www.latex-project.org/}{\LaTeX} с движком Lua\TeX~ – с ними я могу одновременно сочинять книгу и верстать её, не разделяя текст и макет. Я использовал семейство свободных шрифтов DejaVu. Иллюстрации подготовлены в растровом редакторе GIMP и векторном Inkscape.

%Оно не только выглядит так, как мне нравится, но и позволило, наряду с поддержкой кодировки UTF-8 в Lua\TeX, сочетать на страницах текст на кириллице, греческом и других языках. 

%При этом я подключил следующие пакеты функций – не могу умолчать оные, ведь от них зависит многое, что повлияло на вид книги: protrusion, polyglossia, verse, graphicx, pdfpagelabels, bookmark, relsize, tocloft, calc, multicol, longtable.


\section*{Ересь в Сети. Пишите письма}

Новые редакции книги, по мере их выхода, будут выкладываться по следующим адресам:\\ 

\noindent
\href{http://semiletov.org/kiev}{http://semiletov.org/kiev} (мой сайт)\\
\href{https://www.facebook.com/groups/213781655750601/}{www.facebook.com/groups/213781655750601} (группа Киевоград в ФБ)\\
\href{https://t.me/kievograd}{t.me/kievograd} (группа Киевоград в Телеге)\\

Также можете написать мне письмо по электронной почте: \href{peter.semiletov@gmail.com}{peter.semiletov@gmail.com}, 
в телегу: @petersemiletov или ФБ/Мессенджер \href{https://www.facebook.com/peter.semiletov}{www.facebook.com/peter.semiletov}. Буду рад отзывам, указаниям на ошибки, дополнениям.
