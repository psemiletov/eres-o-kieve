\chapter{Троя у Гомера}

Я подошел к этажерке, безошибочно вытащил томик в дерматиновой черной обложке. «Илиада» Гомера. Порядок!

Помню, в начале девяностых посещал я книжный магазин на улице Кудри или Лумумбы, с правой стороны, ежели спускаться. Тополя, солнцем залитый асфальт, и дверь в книжный. Там я приобрел по дешевизне два экземпляра «Хоббита» на украинском языке – в коричневой, с ёлками и волками, обложке от издательства «Веселка», еще «Историю Русов» в переводе кажется Драча (подлинником разжился позже на Петровке), да «Илиаду», перевод Николая Гнедича, самый известный и классический. 

Открыв её, я прочел:\\

\noindent
\textit{Гнев, богиня, воспой Ахиллеса, Пелеева сына,\\
Грозный, который ахеянам тысячи бедствий соделал:\\
Многие души могучие славных героев низринул\\
В мрачный Аид и самих распростер их в корысть плотоядным\\}

А затем пролистал и поставил на полку собирать пыль. Я не могу читать стихи, где слова переставлены вопреки тому, как люди говорят и обычно воспринимают. Был еще другой Гнедич, Петр Петрович, Шекспира переводил – он позже жил, а Николай Иванович умер в 1833 году. Эпоха Пушкина, но – разный слог! Хотя Гнедич мог писать и привычными словами, не мостить читателю дорогу валунами.

Поэтому в качестве источника я хотел использовать «Илиаду» в переводе Викентия Вересаева. Тот же отрывок звучит у него так:\\

\noindent
\textit{Пой, богиня, про гнев Ахиллеса, Пелеева сына,\\
Гнев проклятый, страданий без счета принесший ахейцам,\\
Много сильных душ героев пославший к Аиду\\}

Много понятней!

Вересаев известен более как создатель работ «Пушкин в жизни» и «Гоголь в жизни», но писал и художественную прозу.

Однако на излете 2025 года я, погрузившись в уточнение переводов вообще с древнегреческого, понял – по другому тексту – что тексту Вересаева не могу доверять. Однако  и самостоятельно либо при помощи ИИ перевести «Илиаду» у меня нет ни сил, ни желания, а также практической пользы. Поэтому из этой главы я выкинул все цитаты, и всё же по переводу Вересаева сложу впечатление про местность около Трои у Гомера,    с оговоркой, что перевод мог исказить исходник.

«Илиада» Гомера известна в записях, на «твердом носителе», как считают историки, с 9 века нашей эры. Дошедшие до нас рукописи всплывают гораздо позже. Например, самый известный список «Илиады» – Venetus A – упомянут в частном письме 1424 года, где итальянский историк Джованни Ауриспа сообщает, что купил два тома «Илиады». Я доверяю датам, когда источник прослеживается в письмах, документах и тому подобном. 

Печатно, «Илиада» впервые была издана в 1488 году, Димитрием Халкокондилом. Некоторые думают, что он же ее и написал, однако «Илиада» была известна много раньше. Например, ее обширно комментировал Евстафий Солунский\footnote{Ευστάθιος ό θεσσαλονίκης, 1110/15-1195/96.}, церковный деятель и писатель. 

К сочинениям Гомера о Троянской войне много обращается греческий историк Страбон, чей труд «География» знаком в Европе по меньшей мере с 15 века.

«Илиада» и «Одиссея» вроде были записаны по поручению афинского тирана Писистрата (600-527 годы до нашей эры) – оттуда и пошли дальнейшие списки. Самому же Гомеру ученые отводят время жизни в восьмом веке до нашей эры, а Троянской войне вообще черт знает какое, может быть тысячу лет до нашей эры.

Писистрат в поведении подражал героям обеих поэм и полагают, при составлении рукописей отредактировал текст. Тираном же его называли за насильственный захват власти. Он вел себя поистине бесчеловечно – построил новый рынок, водопровод, основал публичную библиотеку, как простой гражданин являлся в суд, радел за справедливое налогообложение и защиту бедняков.

Я не хочу сейчас пускаться в возню с датировками. Да, мне кажется странным, что храм Зевса Олимпийского, Олимпейон, закладывают при Писистрате, а заканчивают строить спустя 700 лет, во втором веке уже нашей эры. Это противоречит здравому смыслу.

Так вот о поэмах Гомера! Такой же нелепостью мне кажется мысль, что творчество Гомера веками передавалось из уст в уста, пока его не записали. При Гомере Греки писать не умели – полагают ученые. И мол, ходили по Греции слепые певцы вроде наших кобзарей и бандуристов, да пели Гомера. 

А ведь «Илиада» – чертовски длинная поэма. Толстая книжка! Ее не сравнить с былинами или думами. Известно, как искажались былины от сказителя к сказителю. Однако, именно четкая поэтическая форма «Илиады» могла бы служить причиной того, что поэма передавалась без искажений – нарушение стихов вело к изменению, как бы сказали программисты, контрольной суммы произведения.

Примем за данность, что известны поэмы «Илиада» и «Одиссея», и сочинителем считается Гомер, сведения про коего пестры и крайне противоречивы.

«Илиада» состоит из двух основных сюжетных слоев. Один – это Троянская война, точнее, кровавое окончание почти десятилетней осады города. В этом широком слое действуют смертные персонажи. И другой слой – описание клуба олимпийцев, небожителей, богов. Они производят точечные вмешательства в ход войны, поддерживая того или иного вожака – «героя». С такой точки зрения «Илиаду» не изучали, а ведь много чего можно почерпнуть – Гомер говорит о закулисье богов, излагая оное в своем понимании или приспосабливая его к читателю.

Вообще любопытно, как «сказочная» часть отсекается исследователями. Они могут трактовать описываемые события с исторической точки зрения, с географической, но сведения о действиях богов наравне с простыми людьми просто отсекаются.

Это всё равно что, допустим – неважное сравнение, но придумалось – читатель 777 века возьмет повесть про войну 20 века, и станет читать. А там пехота сражается, но в битву вмешивается танк. Но так получилось, что читатель 777 века про танк ничего не знает, он полагает, что в 20 веке сражались на кулаках. И увидев описание вмешательства танка, читатель 777 века воспримет это как сказочную составляющую произведения.

Но и без сказочной составляющей непонятно, что представляет собой «Илиада». Искусно обработанное изложение действительных событий? Чисто художественное произведение? Или сплав обоих вариантов?

Ученые, после «находки Шлимана», стали относиться к «Илиаде» как к поэме, где в той или иной мере отражены происходившие некогда события. Почему же не возникает вопрос – если это историческая быль, то как Гомер мог ее написать? Очевидно, обладая неким обширным материалом, подвергнув его литературной обработке. И материал был \textbf{гораздо} больше поэмы.

Подумайте – Гомер сообщает подробности гибели каждого героя, словно обладает актами судебно-медицин\-ской экспертизы. У меня сложилось впечатление, что описываемые в «Илиаде» военные действия словно были кем-то тщательно записаны при помощи средств видео и звукозаписи, а затем подвергнуты тщательному анализу и росписи. 

Так сейчас криминалисты просматривают записи с камер наблюдения или видео, снятые на месте преступлений. Определяются действующие лица, составляется описание их действий, отекстовываются разговоры. Имея на руках такие «расшифровки», Гомер мог столь правдоподобно изложить сражения, разукрашивая повествование красивыми сравнениями. Если события «Илиады» – выдумка, такое можно списать на богатое воображение Гомера. Если же поэма близка к действительным событиям, откуда Гомер всё это взял?

Но обратимся непосредственно к Трое. Как в «Илиаде» описаны город и его окрестности? Мы узнаем, как представлял себе Трою Гомер, какой рисовал ее читателю в качестве места действия, и сможем затем сравнить с описаниям Трои по другим источникам.

Главный вопрос – где Гомер помещает Трою?

Ученые дают готовый ответ – между горой Идой и Хеллеспонтом, а поскольку Хеллеспонт\footnote{Ἑλλήσποντος, Hellespontos, «Море Хели» – пролив между Эгейским и Мраморным морями. Хеле переправлялась через пролив на «баране с золотым руном», но упала в воду, а баран вместе с братом Хеле –  Фриксом – уцелел и в Колхиде был принесен самим Хеле в жертву. Именно за этим золотым руном отправились потом аргонавты.} это теперь пролив Дарданеллы, а года Ида – там-то, и область между ними слыла Троадой, то конечно же основное действие поэмы происходит в Троаде, Малой Азии, Турции.

В самом деле, Гомер довольно часто упоминает Хеллеспонт в непосредственной связи с троянской равниной, а когда боги разрушают стену Трои, то направляют реки с Иды до самого Хеллеспонта. Где же еще ученым искать Трою, как не в Малой Азии?

Но в Новом Илионе ли?

Многократно Гомер именует Трою «широкоуличной». Город у него то «Троя» (Τροία), то «Илион» (Ἴλιον). По другим источникам – мы к ним еще обратимся – Илион был нечто вроде укрепленной части города, высокой крепостью, местом пребывания знати.

«Илиада» щедра на общие слова о Трое, что она построена прекрасно, и стены ее высокие. Толку мало, разве что понятно – Троя это не захолустье. Но Гомер не чуждается подробностей. Из повествования следует, что
около Трои есть равнина, достаточно большая для перемещения по ней огромного войска. И на на равнине, но вдали от города, высокий курган.
Люди называют его Батией, а боги Олимпа – могилой «проворной Мирины». Обратим внимание, что Гомер иногда приводит двойные названия, человеческие и божественные.

Равнина же располагалась ниже города, стало быть Троя была на холме. Между Троей и равниной была некая «Скея».

Недоумение возникает у меня относительно колесниц, на которых у Гомера ездят по равнине. Сомнения зародил во мне отец, поднявший сей вопрос.

Обычно мы представляем себе эти колесницы двухколесными, на них сражались и так далее. Хорошо, вот садитесь на современный велосипед, у которого камеры надувные в резиновых покрышках, рессоры, всё такое, и попробуйте прокатиться в чистом поле. По равнине! Вас будет здорово трясти, и чем больше скорость, тем более будет тряска.

А колесницы при Гомере? Допустим, колеса деревянные. Допустим, могли быть рессоры. А из чего ось для колес? Из меди. У Гомера всюду медь. Но допустим также, под медью подразумевалось то, что мы зовем бронзой – сплав покрепче. Как далеко проедет по равнине такая колесница, да еще на большой скорости? Насколько на ней вообще возможно ездить, не то, что сражаться? Тут не до жиру, быть бы живу, а не из лука стрельнуть или копьем поразить.

Быть может, мы чего-то не знаем об этих колесницах? Почему у Гомера они везде «блестящие»?

Сохранились ли изображения древних колесниц? Давайте поглядим на кусок рельефа из дворца Ксеркса в Персеполе (современный Иран). Ученые датируют его пятым веком до нашей эры, а колесницу называют «боевой», хотя ничто не указывает на ее военное предназначение.

Однако мне больше всего любопытно устройство колеса, оно прекрасно здесь показано. Рассмотрите и вы:

\begin{center}
\includegraphics[width=\textwidth]{chast-troya/gomer/kolesn.jpg}
\end{center}

Резиновая шина с хорошо заметным протектором! Колесо по виду, кроме спиц, ничуть не отличается от современного велосипедного. Покрышка заправлена в обод. А вот спицы странные – некоторые сплошные, иные же будто составные, разделены посередке.

Я люблю обсуждать с близкими мне людьми, когда пишу. И вот маме говорю про «блестящие» колесницы у Гомера. Она говорит – спицы, если металлические, при вращении блестят! А сам не допёр, хотя на велике катаюсь, замечал же. 

Вот чего мы, наверное, не знаем про колесницы древних. У них были колеса с металлическими спицами и резиновыми шинами, оснащенными протекторами. И мы воочию видим такие колеса на древних рельефах. Это расходится с привычными представлениями о тогдашней технологии. Значит, привычные представления ошибочны.

В том же Персеполе есть и другие рельефы, где колеса выглядят точно так же. Выше – иллюстрация из сборника, выпущенного в 1932 году Британским музеем (где же еще будут предметы старины, принадлежащие Ирану?). 

\begin{center}
\includegraphics[width=\textwidth]{chast-troya/gomer/protektor-01.jpg}
\end{center}

%\begin{center}
%\includegraphics[width=\textwidth]{troya/percepol-02.jpg}
%\end{center}

А вы не задумывались, как были сделаны сами рельефы? Камень. И что, долотом долбили? А если не так кусочек отобьешь, потом приклеивать?

Здесь же, на рельефе из Персеполя, в аппарате некий человек сидит в круглом отверстии, подобном месту пилота в самолетах первой половины 20 века. 

\begin{center}
\includegraphics[width=\textwidth]{chast-troya/gomer/percepol-03-chast.jpg}
\end{center}

Подобный летун (ученые называют их «магами») запечатлен, например, на иранской горе же Бехистун, рельефе, относящемуся к надписи на трех языках, где речь идет о свержении мага Гауматы царем Дарием Первым. Дарий там попирает ногами труп мага, а в «персепольском» летательном аппарате парит, как полагают, бог Ахура Мазда.  

%Здесь же, на рельефе из Персеполя, в аппарате некий человек сидит в круглом отверстии, подобном месту пилота в самолетах первой половины 20 века. 


Есть в Персеполе и знакомые по «чудским» фигуркам раскладные, веером заплечные крылья. А чего стоят верхушка колонны, с двухголовым грифоном, да и многое другое! 

%\begin{center}
%\includegraphics[width=\textwidth]{troya/percepol-04.jpg}
%\end{center}

Но вернемся к Трое. У Гомера Зевс говорит про высокие стены Трои, ворота и прилежное принесение ему, Зевсу, жертв – за что громовержец особо чтит сей город.

Гомер рассказывает, что около Илиона сливаются реки Симоент и Скамандр. По берегу Симоента богини пускают своих лошадей.  Упомянуты также Дарданские ворота. Точное указание, где происходит сражение ахейцев и троян – в поле между реками Симоент и Ксанф\footnote{Боги и люди именовали одну и ту же реку по-разному, Скамандр и Ксанф. Гомер употребляет оба слова – не значит ли это, что Гомер причисляет себя к богам и людям одновременно, то есть намекает, что является полубогом?}.

В акрополе (часть города, ядро его, расположенное на возвышении) находился храм Афины. 

От Скейских ворот, тоже расположенных, кажется, на возвышении, можно было через равнину пройти к дубу. Враги нападали на город со стороны Скейских ворот. Ахейцы гнали троянцев к их городу по равнине, мимо «могилы потомка Дарданова, древнего Ила» и смоковницы. Добежав до Скейских ворот, около которых был дуб, троянцы остановились. 

Про Ксанф (Скамандр) Гомер пишет, что это река со светлой водой, богатая водоворотами. В ней был брод. Один из берегов Скамендра – крутой. Скамандр впадает в море, но прежде сливает свои воды с Симоентом, и между обеими реками – равнина.

Симоент протекает также вдоль некой Калликолоны. Она упомянута в поэме еще раз – вероятно, будучи названием холма, отдельного от того, на коем стоит Илион. А вот что Страбон пишет о Калликолоне:

\begin{quotation} 
В 10 стадиях над селением илионцев возвышается Калликолона — нечто вроде холма, мимо которого в 5 стадиях протекает Симоент.
\end{quotation}

Так или иначе, у Гомера события «Илиады» разворачиваются в Малой Азии, на что Гомер указывает привязкой к Калликолоне, Хеллеспонту и другим местам ныне спорным либо установленным. Между тем, втиснуть Трою из «Илиады» в окрестности именно Хиссарлыка трудно, ну да рассуждать об этом толку нет.

Однако еще давние Греки, например Посидоний и Страбон, подметили, что в «Илиаде» странным образом смешана география Малой Азии и Европы.

Например, среди воюющих сторон – Мисяне или Мисийцы. В самом деле, в Малой Азии рядом с Троадой была Мисия. Но и в Европе она была, теперь там Болгария. Более того, вместе с этими Мисянами, союзниками выступают воины Гиппемолгов, Галактофагов и Абиев – которых Страбон относит к Скифам и Сарматам. Стало быть, в сражении вероятно принимают участие европейские Мисяне и Скифы!

Что нам поведают о Трое другие источники?
