\chapter{Герои троянской войны в Киеве}

В 1675 на латыни вышла книга Иоанна Гербиния  «Религиозные киевские пещеры или Киевские подземелья и нетленные тела умерших 600 лет назад богов и героев Греции и Рутении»\footnote{Не ручаюсь за правильность своего перевода.} (M. Johannes Herbinius, Religiosae kijovienses cryptae, sive Kijovia subterranea et emortua a sexcentis annis divorum atque heroum Graeco-Ruthenorum et necdum corrupta corpora.). Уроженец Силезии\footnote{Сейчас ее земли разделены между Польшей, Чехией и Германией.}, Гербиний был ректором школы протестантов в Стокгольме, а на время сочинения книги – пастором в Вильно, хотя через год его изгнали в Кенигсберг. 

Гербиний переписывался со многими церковными деятелями того времени. В числе его адресатов был Иннокентий Гизель, архимандрит Киево-Печерской лавры. Гербиний попросил у него материалы для работы над книгой, и вскоре получил планы пещер, Патерик Печерский (1661 года издания) да служебник. На основе оных, да каких-то других источников Гербиний, сам в Киеве не бывавший, и написал «Религиозные киевские пещеры».

Уже в 1677 году книга попала в католический «Индекс запрещенных книг» (Index Librorum Prohibitorum) – за их чтение отлучали от церкви.

До сих пор работа Гербиния не переведена на какой-либо язык. На русском существует краткая выжимка из нее в «Сборнике материалов для исторической топографии Киева» 1874 года, эдакие конспекты глав. Мы пройдемся по всем главам, но сначала поглядим, что же говорится о Трое.

На странице 3 предисловия мы читаем вещи, заставляющие навострить уши. Гербиний спрашивает у читателя – как сюда, на берега Борисфена, могло занести Приама, Гектора, Аякса, Ахиллеса, прочих жителей Дардании и героев древности\footnote{Quis Troja vestigia non lustraret lubens? Quis Priamos, Hectoras, Achilles, Ajaces, aliosque Dardanorum aque ac Archivorum Heroas, etiam maxima sumptuum jactura, ad Borystenem spectatum non iret peregre?}?

В главе 2, втором её разделе\footnote{\begin{otherlanguage}{latin}Caput 2:

II.
Alterum est Kijovia Urbs cis Borysthenem, elevat, Poli circiter 50, gradus sita, cujus originem investigare hic non est animus. Illud autem expendere pretium operae erit haud leve, quod inter alia de Urbe ista vulgo praedicantur: fuisse scilicet kijoviam. I. Trojam veterem, quae, ob raptam a Paride Helenam Menelai Spartanorum Regis uxorem, a Graecis olim per decennium obsessa, tandemque expugnata ac penitus excisa est. Cui fabulae istud accedit commentum, in Crypta Kijoviensi gigantea Hectoris, Priami, Achillis, \& aliorum tum Trojanorum, tum Graecorurn Heroum corpora, a tanto jam tempore, incorrupta spectari. 

Sed absurda haec atque falsissimae esse Geographia docet. Si enim Troja olim fuit, ubi nunc Kijovia existit; qui quaeso ratione classis Graecorum eo per descendentes Borysthenis adversi Cataractas penetrare potuit, cum ne scapha quidem incolumis illac transeat? Quomodo Aeneas classem per Cataractas illas transmittere potuit, cum ne trabes quidem impune ibi praecipitentur. Magna sane est ac celebris Cataracta WestroGothica, nomine Trollhetta, Latine, si cum Nonio loqui licet, et, Capitium aut mitra Diaboli, in quam trabes vel mali nautici praecipitati impune evadunt pauci admodum: Quam Cataractam cum inspectaremus Viatores, toti horrebamus. Astuna haec Cataracta cum multis Borysthenis praecipitiis nequaquamest conferenda. Sed \& Virgilius navigationem Aeneae in Italiam recensens, nullam Borysthenis, nullam Ponti Euxini, nullamque Hellesponti ab Aenea trajecti, quae omnia tamen trajicienda classi Dardanorum erant, facit mentionem.

Certe Homerus de Borysthene nostro tacet, \& nos cum eo. Quae vero \& quorum hominum cadavera in Cryptis Kijoviensibus incorrupta hactenus ostenduntur, infra specttabimus. Quapropter fabulas istas suis remittimus auctoribus.\end{otherlanguage}}, Гербиний пересказывает, как он называет, басню, то бишь выдумку о том, что Киев это старая Троя, и что в огромных киевских пещерах (Лаврских), нетленными лежат тела героев Троянской войны Гектора, Приама, Ахиллеса и прочих. Гербиний опровергает «эту выдумку» с географической точки зрения, основываясь на описанных в «Энеиде» Вергилия странствиях Энея, да и Гомер, дескать, ничего о Борисфене не говорит.

Гербиний оспаривает и другие странные представления о Киеве – благодаря такой критике мы узнаём об этих странностях. Процитирую пересказ по «Сборнику материалов для исторической топографии Киева»:

\begin{quotation}
Гл. V. Материя пещер Киевских.

Материя пещер киевских – влажная, но плотная земля, легко поддающаяся заступу и лопате и принимающая как угодно формы. Несправедливо мнение польского писателя Флора (Flor\-is Polonicus Noribergae), утверждающего\footnote{Гербиний цитирует книгу 1666 года на немецком языке, но цитата дополняет пересказанное Гербинием.} будто пещеры тянутся под руслом Днепра и простираются до Чернигова, Смоленска, Москвы и Печоры\footnote{По крайней мере в Печоре было обратное предание, что тамошние пещеры  соединены с Киевскими!}. Пещеры киевские не так глубоки, чтобы проходить под руслом Днепра; а поименованные города отстоят на таком пространстве от Киева, что поверить самой возможности существования подземного хода на таком протяжении – нелепо. Также невероятно и противно свидетельству очевидцев уверение Флора и Фрелихия\footnote{Florus et Froelichius.} (in suo Viatorio), что пещеры русские выложены медью\footnote{Aeris.}. Ибо где взять столько меди, чтобы выложить ею пещеры, простирающиеся на 100 немецких миль. Удивляюсь ученым мужьям!
\end{quotation}

А я больше удивляюсь сведениям о тоннелях с медными сводами, тоннелях, проведенных до Чернигова, Смоленска, Москвы, Новгорода и Печоры! Ни про один город такого не говорили, кроме Киева.

В шестой главе Гербиний помещает планы известных нам поныне пещер – не тех, что со сводами медными, а где лежат «мощи», да в седьмой главе показывает картинку монаха с лопатой, у входа в выкопанную им пещеру. И говорит – вот орудие, которым пещеры вырыты, никаких сводов металлических не видно, и вообще пещеры, тянущиеся к другим городам – невозможны\footnote{Совершенно справедливо поднимая вопрос, как бы оттуда выносили грунт?}! Прямо как у Даля пословица:

\begin{quotation}
Один Иван – должно, два Иван – можно; три Иван – никак не возможно, сказал немец про Ивана Ивановича Иванова.
\end{quotation}
 
Но вернемся к героям Троянской войны.

В Патерике Печерском поначалу не было жизнеописаний сорока святых, чьи нетленные тела лежали в Дальних пещерах. Про тех, кто в Ближних пещерах – есть, помещены в Патерике. А относительно Дальних пещер – как соотносились тела и имена? 

Тела лежали в нишах, без гробов, на досках, другими досками ниши заколачивались, прорезалось окошко для обозрения, а сверху окошка писалось имя почивающего в нише святого, да краткое его жизнеописание, невесть откуда взятое. Потом уже тела переложили в гробы.

В 17 веке монахи занесли эти надписи в особую книгу, претерпевшую много списков. Независимо от сего, в 1643 году иеромонах и протосингел Константинопольской церкви, Мелетий Сириг составил по тем же нишевым надписям «службу» – прославления святых поименно.

Список монашеский и Мелетия Сирига совпадают по числу имен, самим именам и их последовательности. Вот они:

Лонгин, Игнатий, Силуан, Агафон, Зинон, Макарий, Ахила, Ипатий, Паисий, Меркурий, Лаврентий, Моисей, Иларион, Дионисий (вроде бы это про него есть короткая заметка в Патерике), Арсений, Пимен, Афанасий, Сисой, Григорий, Павел, Леонтий, Геронтий, Нестор, Тит, Памво, Захария, Феодор князь Острожский, Софроний, Панкратий, Аммон, Мардарий, Руф, Вениамин, Феофил, Мартирий Диакон, Евфимий, Кассиан, Пиор, Пафнутий, Иосиф. 

Только о шести из них остались краткие сведения – о Лонгине вратаре, Агафоне чудотворце, Арсение трудолюбивом, Вениамине, Тите воине, Феофиле – епископе Новгородском. Про Тита, например, известно следующее: «преподобный Тит воин, бывши на брани, поражен бе во главу оружием, едва не смертне, и сего ради остави воинствование» – после чего ушел в монастырь.

В изданиях Патерика конца 19, начала 20 века, кратенько составили жизнеописания остальных лежащих в Дальних пещера. Эти жизнеописания очень похожи друг на другая и написаны в самых общих выражениях – за редкими исключениями.

Где-то я встречал мнение, что путешественники, видя нишу, где лежал «Ахила» (как именно оно было написано, неизвестно – однако на Руси до сих пор переводят Ахиллес как «Ахилл»), сопоставляли тело с легендарным Ахиллесом.

Что же, греческие имена были у христианских монахов в ходу, да и просто крестившиеся принимали «церковное» имя, зачастую греческое. Правда, я не слышал о других людях с именем Ахилл, да и неясно толком, что оно означает, хотя ученые трактуют его как сочетание греческих слов akhos (горе) и laos (народ, люди), то бишь в переводе нечто вроде Горелюдям.

Для очистки совести сравним, какие еще имена из списка «дальних пещер» совпадают с именами основных героев Троянской войны. Не все имена списка – греческие. Меркурий, например – римское, это римский Гермес. 

Зинон (Sinon), Леонтий (Leonteus), Нестор (Nestor), Евфимий (Euphemus). Негусто!

Однако насколько список имен погребенных соответствует действительности?

И всегда ли монахи показывали только нетленные тела своих почивших собратьев? Любопытно свидетельство Павла Алеппского за 1654 год, о пещерах в Лавре: «Эти пещеры представляют норы и келлейки, не вмещающие даже и ребенка: как же они могли вмещать кого-либо из угодников?» – быть может, ему показывали какие-то и другие пещеры в Лавре, меньше тех, куда пускают нынче? Не в них ли покоились «нетленные младенцы», упомянутые в Патерике Печерском?

И священник Иоанн Лукьянов, в 1701 году посетивший лаврские пещеры, писал: «Видехом и младенцев нетленных лежащих». 

Кого еще показывали в пещерах, и кто еще говорил о дивных подземных ходах на сотни километров?