\chapter{Пещеры до Чернигова}

О том, что Киев был не просто большим, а очень большим даже в средние века, писали в старину многие. Много говорили об его величии, как былом, так и современном писателю.

Титмар, епископ Мерзебургский, живший на стыке 10 и 11 веков, в своей Хронике\footnote{Киев в ней именуется то Kitava, то Cuiewa.}, в разделе о 1018 годе, сообщает\cite{titmar01}:

%\begin{quotation} 
%В большом этом городе, который служит столицей государства, имеется более 400 церквей и 9 торжищ, а народу несметная сила, которая – как и во всей здешней области – составляется главным образом из беглых рабов, стекающихся сюда со всех сторон, в особенности из проворных данов (норманнов), и которая доселе много боролась с вредными печенегами и побеждала другие народы.
%\end{quotation} 

\begin{quotation}
В том большом городе, который является столицей этого королевства, имеется более 400 церквей, 8 ярмарок, а людей – неведомое количество; народ, как и вся та провинция, состоит из сильных, беглых рабов, собравшихся здесь отовсюду, и, особенно, из быстрых данов; до сих пор они, успешно сопротивляясь сильно им досаждавшим печенегам, побеждали других.
\end{quotation}

Подлинник\cite[стр. 258]{titmar00}:

\begin{quotation}
\begin{otherlanguage}{latin}
In magna hac civitate, que istius regni caput est, plus quam quadringente habentur eclesiae et mercatus VIII, populi autem ignota manus: quae, sicut omnis haec provincia ex fugitivorum robore servorum huc undique confluencium et maxime ex velocibus Danis, multum se nocentibus hactenus resistebat et alios vincebat.
\end{otherlanguage}
\end{quotation}

Прежде чем двинуться дальше, обсудим этот кусочек. Помимо 400 церквей и 8 торжищ, указывающих на воистину огромные размеры тогдашнего Киева, два вопроса возникает при прочтении. Титмар при составлении «Хроникона» так или иначе пользовался различными источниками, в том числе на греческом, и то, что писал – помимо собственных наблюдений, еще и плод обработки сведенных воедино источников.

Вопрос первый – почему «народ, как и вся та провинция, состоит из сильных, беглых рабов»? 

Если Титмар прочел об этом из источника на греческом языке, то слово «раб» там было «склав», коим Греки обозначали и рабов, и народ Славян. Посему Титмар и записал всё население Киевщины в рабов.

Вопрос второй – о проворных Данах. В примечаниях к переводам Титмара обычно добавляют – «варягов» или «норманнов». Однако с чего вдруг к Данам применять слово «проворный»?

У меня иное толкование сказанного. 

Во времена Титмара, население Дании слыло как Иудды. Потому и часть страны именуется поныне Ютландия, прежде Юдландия. Я об этом уже писал. Вот эти-то Иудды, они же Даны, явно иудеи, и обитали в Киеве в летописном урочище Жидове.

Ипатьевская летопись за лето 6632 (1124) сообщает:

\begin{quotation}
У се же лето бысть бездожгье . то погоре Подолье все, на канун святаго Рожества Ивана Крестителя и Предтеча; в утрий же день погоре Гора и моностыреве вси что их на Горе в граде, и Жидове.
\end{quotation}
 
Лаврентьевская летопись, говоря о том же событии рассказывает несколько иначе:

\begin{quotation}
Бысть пожар велик Кыеве городе, яко погоревшю ему мало не всему, по два дни, по Подолью и по Горе, яко церквий единех изгоре близ 6 сот; се же бысть месяца июня\footnote{По Румянцевскому списку – июля.} в 23 день и в 24, на рожество Иоана Предтечи.
\end{quotation}

Сгорело шесть сотен церквей. Представьте себе размеры города, в котором сгорело шесть сотен церквей.

% знаменательном пожаре 1124 года, сообщает, что «яко церквий единех изгоре близ 600».

Адам Бременский в сочинении «Gesta Gammaburhg\-ensis eccleasiae pontificum libri IV» 1072 года вдруг говорит такое:

\begin{quotation} 
%Из Юмны следуя далее, в 14-й день достигнешь до Острогарда в Руции, где метрополия город Киве (Chiwe), соревнователь константинопольского скипетра, славнейшее украшение Греции.
На 14 день пути от города под парусом добираются до Острогард Руззиэ (Ostrogard Ruzzia). В которой столичным городом есть Кивэ (Chive), соперник скипетра Константинополя, светоносного украшения Греции. 
\end{quotation} 

Александр Гваньини из итальянского города Вероны издал в 1581 году свои записки о Сарматской Европе (Guagnini Veronensis, Sarmatiae Europeae descriptio), где кроме прочего пишет:

\begin{quotation}
Киев, древнейший и обширнейший город, обнесенный деревянными оградами, некогда столица всей России, расположенная у славнейшей реки Борисфена, отстоит от Вильны на сто двадцать миль польских.

О прежнем великолепии и истинно царственном виде города свидетельствуют самые развалины и памятники, расположенные на пространстве шести миль.
\end{quotation}

А шесть польских миль – это сколько? Проклятие заключается в том, что были большая польская миля (3650 сажень), средняя (3333) и малая (2500). Возьмем самую малую и переведем 6 малых польских миль в сажени. 6*2500=15000. Теперь обратим сажени в метры. 1 сажень это 2,13 метра. 15000*2,12=31800. Или 31,8 километра. Скажем так – приблизительное расстояние от Белгородки до Троещины, то есть от одной современной окраины Киева до другой.

Гваньини продолжает:

\begin{quotation}
Доселе на соседних холмах виднеются следы церквей, монастырей и опустевших зданий. Кроме того есть некие обширнейшие подземные пещеры, прокопанные под землей на большое расстояние – как некоторые говоря на 80 миль.
\end{quotation}

Истории про чудовищно длинные пещеры, тянущиеся от Киева до Чернигова и других городов, известны поныне – такая уж ходит среди киевлян молва. 

Я еще ребенком, в восьмидесятых, услышал про это от мамы, гуляя в ботаническом саду на Зверинце. Мы проходили мимо старинного церковного погреба, переоборудованного под мастерскую. Пахло мазутом, из раскрытых ворот погреба выглядывал трехколесный мотоцикл с тележкой – «Муравей». Мы предположили, что в погребе – вход в пещеру. Тогда же мама и поведала мне, что говорят, будто пещеры из Киева тянулись до самого Чернигова. Спустя почти четыре десятилетия спрашиваю у мамы – откуда ты это знаешь, какой источник? Да вот, просто бытовало такое мнение.

Как ни опровергай, а народная молва сохраняет предание веками. Что же Гваньини пишет дальше?

\begin{quotation}
В пещерах видны многие старинные гробницы и тела некоторых знаменитых русских князей, (давно похороненные) но только истлевшие, а не распавшиеся, в особенности – кажутся целыми, как будто только что положенные, трупы каких-то двух князей в языческих одеждах, как они ходили при жизни. Так лежат в пещере тела непогребенные, которые показываются чужестранцам и пришельцам монахами русского обряда.
\end{quotation}

Гваньини не говорит о «святых», напротив, упоминает даже князей в языческих одеждах! В подлиннике – in vestibus gentili more. А что значит в языческих? Дохристианских, быть может, одетых так, как Гваньини представлял себе одежды давних для него времен.

Про пещеры до самого Новгорода и тела князей пишет в 1585 году и Станислав Сарницкий (Stanislai Sarnicii, Annales siue de origiue et rebus gestis Polonorum at Lituano\-rum libri octo 1587; Descriptio veteris et novae Polonian cum divisione ejusdem veteri et novo, 1585):

\begin{quotation}
И как некогда римляне баснословили о северных и индейских уродах и диковинах, так и русские теперь стараются уверить других о своих чудесах и героях, которых зовут богатырями т.е. полубогами.

Они погребены по русскому обычаю в горных пещерах, которые будто бы, как подземные коридоры, тянутся на огромное пространство даже до Новгорода великого. [...]

Монахи показывают там (в Киеве – прим. Семилетова) нетленные тела князей, которые благодаря какому-то особому свойству местности остаются без всякого повреждения много лет.
\end{quotation}

Рейнольд Гейденштейн, рассказывая о Киеве в 1696 году, вдруг упоминает, что непонятно, когда он был построен, «не достигает ли быть может времен Колхиды и Энея» – вновь странный отголосок Троянской войны.

Афанасий Калнофойский в «Тератургиме» 1638 года выписал многие надгробные надписи людей, сделавших на Лавру пожертвования и в ней похороненных. Наряду с христианской тематикой – много греческой мифологии – отсылки к мойрам Клоте, Лахессе, Артопе, к Пирру и Гектору, перевозчику через реку смерти Стикс – Харону, к Прозерпине. Классические троянские предания хорошо известны в киевских эпитафиях, например:

\settowidth{\versewidth}{Потому что в сию землю не положен сын Эмпузы,} 
\begin{verse}[\versewidth]
Под алтарем Филоклетета,\\
Никакая Дипса тебя не коснется;\\
Потому что в сию землю не положен сын Эмпузы,\\
С целию умножить племя пресмыкающихся гадов.
\end{verse}

По преданию, Филоклетет, глядя на гроб убитого Ахиллесом Троила, был укушен змеей. Эмпуза – Гекуба, жена Приама.

Уже в 19 веке ученые стали полагать, что таких эпитафий не было, их, мол, придумали при Петре Могиле. Античность была в моде! Максимович еще удивлялся, как Сильвестр Коссов на десятой странице Патерика 1635 года сравнивает «печерских угодников с Сатурном, Юпитером, Марсом».

А может, эпитафии указывают на существовавшую в то время культурную связь между, условно говоря, Грецией и Киевом, Русью? И так уж давно ль от 17 века отстояла «античность»?

Краковский каноник Симон Старовольский в своем описании Киева примерно за 1640 год (помещено в атласе Georaphia Blaviana) рассказывает про здешние пещеры такое:

\begin{quotation}  
Эти подкопы или подземные пещеры, как их называют, простираются на несколько десятков миль, вплоть до Московии.
\end{quotation}  

Андрей Целларий в «Описании Польши» 1659 года (Andrei Cellarii, Regni Poloniae magnique ducatus Lituauiae omnium que regionum juri polonico subjectorum novissima descriptio) цитирует Лаврентия Мюллера, советника курляндского герцога, за 1585-й – а Мюллер передает слова киевского пастора о том, что:

\begin{quotation}
от Киева до Смоленска существуют подземные проходы, и так их часть, которая проходит под самым течением Днепра, во всю его ширину, имеет литые своды; из чего можно заключить, сколько труда и сколько издержек потребовалось для подобного сооружения, и сколь велико было прежнее великолепие Киева. Говорят, что главными соорудителями этих переходов были итальянские купцы.
\end{quotation}

Не знаю, как насчет итальянских купцов, но вот в старину на Урале полагали следующее. Что металлургические предприятия стоят на местах давних разработок Чуди белоглазой. А про тамошние пещеры сочинительница и собирательница сказов Серафима Власова (1901–1972)\footnote{Завещала: «Мой труп сжечь, обязательно сжечь, а урну замуровать в одной из гор Южного Урала [...] не тлеть в земле, а вечно слушать шум сосен в горах, любоваться хребтами и далью, далью бесконечно любимого Урала», а ее похоронили в Челябинске на Успенском кладбище.} поведала:

\begin{quotation}
Услышала я недавно в старом уральском заводе, будто все пещеры, какие ни есть на Урале, сообщаются между собой. Будто таятся между ними лазы, то широкие, как Кунгурские ямы, эти провалы земные, то тонюсенькие, как золотые нити. 

Говорят также, что когда-то в стародавние времена перейти из пещеру в пещеру не составляло труда – торная дорога была. Правда, кто ее торил, неведомо – то ли человеки чудью называемые, то ли нечистая сила... Только в наше время люди, проникая в те пещеры и те ходы, где пройти можно, много следов находят: где домница поставлена, где камень аметист лежит, а где след ноги человеческой отпечатался.
\end{quotation}

Любопытно еще другое. В разных источниках, пещеры от Киева упомянуты до Чернигова, Смоленска, Новгорода. Вспомним – это путь, по которому Вольга подчинял себе земли. Вспомним также, на что я обращал уже ваше внимание – если провести по карте прямую линию с севера на юг, то по ней с относительно небольшим отклонением окажутся Питер, \textbf{Новгород, Смоленск, Чернигов, Киев}, Одесса, Стамбул (Константинополь). Прослеживается некий большой подземный путь?

%В стороне от него лежит Псковский Печерский монастырь (в Печорах, 16 километров на северо-запад от Изборска), о котором еще в конце 18 века ходило предание, что оттуда имеется подземный ход к Киевским пещерам, о чем насмешливо сообщает «Новый и полный географический словарь России» 1788 года.
