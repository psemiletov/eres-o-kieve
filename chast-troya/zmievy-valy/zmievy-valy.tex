\chapter{Валы Змиевы}

Развалины, разбросанные на протяженности пятидесяти километров. Сотни километров подземных тоннелей в другие города. Положим, первое уже не проверишь, второе еще не проверили – да и не собираются. 

А вот как насчет поныне существующих Змиевых валов Киевщины? Сходные с ними валы Поднестровья называют Трояновыми – а некоторые (например, в книге Фундуклея «Обозрение Киева» 1847 года) именуют так и Змиевы валы. Я недостаточно изучил этот вопрос, поэтому ограничусь Змиевыми валами, хотя конечно мне удобно думать, что всю систему валов, не только Киевщины, называли Трояновыми, что явно связано с Троей, а не с римским императором Марком Ульпием Нервой по прозвищу Траян, которого некоторые ученые предлагают на роль строителя по крайней мере Поднестровских валов.

Давайте лучше о Змиевых.

Змиевы почему? Ну змии, как я уже говорил, это языческие боги. А что сказано в преданиях о Трое? Кто строил оборонные сооружения для Трои, при царе Лаомедонте? А поганские боги Посейдон и Аполлон, или, как на Руси говорили, Неп­тенабушь да гуселник Тебушь.

\begin{quotation}
И Тебушь бе гyceлник и гудяше в гусли, и зидашеся Трой, где они повелеваху, а Неп­тенабушь именем идяше в море и ношаше из моря вар и камение и воду.
\end{quotation}

Вот как ладно всё получается со Змиевыми валами.

А наша родная сказка говорит про кузнецов Козьму и Демьяна, они-де, запрягли Змия и вспахали на нем землю, так и появились валы.

Что же они такое?

По Киевщине правобережной, и в гораздо меньшей степени по левобережной, на сотни километров протянулись эти валы да их остатки, доходя до 12 метров высоты, заставляя задумываться о былых масштабах. Укрепрайоны двадцатого века в сравнении с этим – мелковаты. Впрочем, рубежи Киевского укрепрайона частично проложены по Змиевым валам.

Существует несколько линий этих валов, одни ближе к Киеву, другие дальше. Тянутся они в долинах рек Роси, Ирпеня, Тетерева. Ближайшие к Киеву – около Хотова, Ходосовки, Боярки и Белогородки (у Бобрицы).

На Левобережьи Змиевы валы известны под Переславлем, а также напротив Триполья, Халепья и Витачова – от Кийлова на юго-восток. Линия валов проходит также от Лубен вдоль Сулы.

В русских летописях Змиевы валы (не именуясь так, а просто «вал» – например, «прошли вал») не несут оборонного значения, употребляются как ориентиры. Стало быть, некое значение, возможно оборонное, они играли гораздо раньше летописного времени. И значит, возвели их прежде него. Как подсчитал изучавший Змиевы валы историк архитектуры Павел Раппопорт (1913-1988), чтобы построить Постугнянско-Ирпенский участок валов, в течении года должны были работать до 52000 человек. А ведь это лишь часть валов!

Они создавались по некоему плану, а не просто – вышел князь, махнул рукой – стройте туда до леса, и строили. Нет. Это определенная система сооружений, причем разработанная с учетом местности. Валы дополняют естественные преграды. Следовательно, была точная карта. Повторю – была точная карта! Без нее нельзя начинать такое огромное строительство. Поглядите на средневековые карты – детский уровень. А задолго до этого, у кого-то были технические возможности точно картографировать местность!

Кто строил валы, руководил работами, зачем? Возможно ли, чтобы в древнее время столь организованно десятки тысяч людей направлялись на строительные работы? А ведь рабочим надо было где-то жить и чем-то питаться. Кто и как обеспечивал стройку?

Наконец, сам по себе вал легко преодолеть, если его не охраняют. Вывод – валы, вот эти сотни километров, кто-то сторожил, по крайней мере в некоторых точках, разбросанных на протяжении валов.

Отсюда предположение, справедливое в случае, если валы служили оборонным целям – в прошлом местность, охватываемая валами, подчинялась некоему центру управления – а это наверняка Киев, и у центра было достаточно человеческих сил, чтобы постоянно держать «на валах» сторожевые отряды, а также кормить их. Конечно, отряды не могли просто сидеть днями и ночами на валу, а обитали в неких крепостях, от которых до наших дней могли остаться городища.

Но вот еще вариант. Известно, что Змиевы валы так или иначе связаны с водоемами – реками, озерами, болотами – как существующими, так исчезнувшими. При возведении высокого и широкого вала, рядом безусловно образовывается ров. Ведь не возили же туда землю нарочно. А что, если валы – вторичны, побочное явление сети древних мелиоративных каналов?

Против этого есть возражение – сложная внутренняя структура самих валов, о чем я расскажу ниже. Но тогда – нельзя ли версию с мелиорацией рассматривать как дополнительную? Оборонные валы само собой, но и мелиорация.

Мы обсуждали Змиевы валы с отцом, и он предположил – а если валы сделаны не чтобы ограждать Киевщину от внешнего противника, однако наоборот, дабы удерживать кого-то внутри?

Литература по Змиевым валам отрывочна и разбросана по труднодоступным источникам. Существует книга археолога Михаила Петровича Кучеры (1922-1999) «Змиевы валы среднего Поднепровья», изданная в 1987 году смешным по советским меркам тиражом – 1440 экземпляров. Она представляет точку зрения официальной науки.

Но с пятидесятых годов по конец восьмидесятых двадцатого века валы изучал математик, доцент Киевского Педагогического Института имени Горького, Аркадий Сильвестрович Бугай (1905-1988). Существует сборник его работ – книга «Змиевы валы», выпущенная в 2011 году. В работах своих Бугай делится плодами многолетних полевых исследований, проведенных им и единомышленниками. Своими ногами исходили несколько сотен километров валов!

 %Увы, я не читал его редкую книгу «Змиевы валы», знаком только со статьями Бугая, где он делится плодами многолетних полевых исследований, проведенных им и единомышленниками. Своими ногами исходили несколько сотен километров валов!

Бугай вообще любопытно размышляет. Киевщина представлялась ему бывшим островом – если уровень воды в Днепре раньше был выше, а долины Ирпеня и Стугны – глубже, то последние могли в давнее время быть огромными заливами Днепра. И тут находилось сердце некоего развитого государства, предшествующего Киевской Руси.

Исследования экспедиций Бугая дали много важных сведений. Приведу некоторые.

В валах и по сторонам от них обнаружен древесный уголь, из чего Бугай сделал вывод, что трасса будущего вала освобождалась от кустов и деревьев выжиганием. Пробы угля, которые Бугай передал в Академию наук, там датировали примерно 8 веком нашей эры. Позже радиоуглеродный анализ угля, взятого в разных частях валов, показал иную картину – от второго века до седьмого.

Итак, если на пути вала был лес, строители его вырубали, а затем выжигали остатки. Зачем? Требовалось чистое место. Ибо основания валов состоят из слоев глины, часто обожженной. Уже поверх этой прочной основы насыпалась земля. Глиняный слой настолько прочен, что его трудно разрушить лопатой или тракторным плугом.

Бугай составил карты Змиевых валов Киевщины, и сопоставил с валами разные городища, имея основания полагать, что такой-то крепости соответствовал охраняемый отрезок вала.

Подход Бугая к загадке Змиевых валов – подход во многом истинно математический, когда на основе неких данных делается логическое умозаключение.

А книга Михаила Кучеры «Змиевы валы Поднепровья» – плод уже исследования от Академии Наук УССР. Бугай-то со студентами – энтузиаст. У официальной науки археологии иной подход, есть изначальный настрой, уместить датировку Змиевых валов во временные пределы Киевской Руси. Кучера и сотрудники провели большую работу –  раскапывали валы, тщательно зарисовывали внутреннее их строение, составляли свою карту валов, и так далее.

Кучера обильно пользуется наработками Бугая, цитирует его, но в случае разногласий говорит – Бугай ошибается! Вот датировка, выполненная способом радиоуглеродного анализа, в той же Академии Наук. Бугай ее опубликовал. Кучера говорит – ошибочная! Почему ошибочная? А вот, пишет Кучера, наши археологические исследования показали тут Киевскую Русь.

Потом Кучера отдает свои пробы на анализ. Анализ показывает – Киевская Русь, ну пару столетий туда-сюда! Но не везде. Кое-где и две тысячи лет до нашей эры. Тоже ошибочная? Кучера поясняет – датировка искажена. Мол, там подмешались материалы более ранних эпох. «Исключив результаты анализов 1981-1982 гг, ориентировочно получим древнерусский возраст Змиевых валов». Вот так. Что не вписывается – исключаем. Получается, что с выводами любых анализов наука соглашается лишь в случае, когда выводы подтверждают утверждения определенных представителей науки.

У Кучеры всё в валах древнерусское. Топор нашли – это времён Киевской Руси, строитель обронил. Такие утверждения, одно за другим, Кучера называет «совокупностью археологических данных». А разве на топоре были выбита дата его производства? Нет. Но где-то под Харьковом обнаружили похожий топор, и вот ученые его датируют временами Киевской Руси, значит и топор из вала тогда же произведен! Позвольте спросить, а топор под Харьковом имел клеймо с датой?

Кучера пишет, что некоторые валы (археолог относит их к ранним), оказывается, имели внутри каркас из бревен. И уголь от каркасов Бугай принял за уголь от выжигаемых кустов и деревьев. Но разве один вариант противоречит другому?

В «Змиевых валах Поднепровья» утверждается, что валы начал строить Владимир, строили также при другом Владимире, Мономахе, и при Ярославе Мудром, и хотя летописи об этом молчат – не беда, зато вот Энгельс писал об «иррегулярной кавалерии любого народа», да Ленин еще что-то писал. Их-то зачем трогать? Они исследовали Змиевы Валы? Надо не Ленина было цитировать, а задать вопрос – как могло обеспечиваться строительство валов хотя бы при Владимире Красном Солнышке? Сами же ученые признают – надо, скажем, чтобы двести дней подряд семьдесят тысяч человек трудились над таким-то участком вала.

Ну так представьте себе эту тьму народа, которую надо кормить, поить, охранять, расселять – пусть в каких-то шалашах, а если вал имеет каркасную основу – обеспечивать подвоз строительного леса? 

Официальная наука полагает, что в 12-13 веках население стольного града Киева составляло около 50 тысяч человек. Если мы примем, что для строительства некоего участка вала понадобилось 70 тысяч, да чтоб непрерывно горбатились, возникает совершенно фантастический образ града Киева на колесах. Вот он ехал (имея годовой запас продовольствия), делал остановки, киевляне выходили, звали на подмогу еще каких-то местных, сообща дружно строили кусок вала, на ночь забирались обратно в свой титанический сухопутный корабль, а утром трогались дальше, снова выходили, рубили ближайший лес, рыли, возводили.

К этому образу меня подталкивают ученые, официальное мнение науки. Иначе никак объяснить строительство валов я не могу. Та же наука полагает, что в 13 веке население Новгорода составляло 35 тысяч, Лондона в 11 веке – 20 тысяч. Где же киевский князь взял 70 тысяч строителей, да еще организованно их перемещал, давал харчи, охранял «от диких половцев»? Вопрос питания тоже важен – хотя ученые могут придумать, что 70 тысяч строителей питались лебедой, просто выходя в поле. Становились на четвереньки и поедали лебеду.

Хорошо, не надо год горбатиться, давайте уменьшим количество строителей, но растянем само строительство на долгие годы. Допустим, при тех же раскладах, что ежели на участок вала надо, условно говоря, 70 тысяч строителей в год, то если вал строить 10 лет, понадобится только 7 тысяч строителей. Тоже нехило. Это – по цифрам тех же традиционных историков – крупное русское войско.

Сие означает, что все 10 лет они должны трудиться в мирных условиях, иначе те же «дикие Половцы» налетят и перебьют или угонят в рабство. 10 лет надо держать отдельное войско для охраны строителей. Кормить и строителей, и войско. Что значит продержать 10 лет строителей, так сказать, на объекте? Мыслимо ли такое долговременное обеспечение? Вы сейчас такого не провернете, а нам говорят, что во времена Киевской Руси, даже не 10 тысяч, а 70 тысяч человек единовременно трудились над возведением одного лишь участка Змиевых валов!

Поистине мне более правдоподобным уж кажется, что Неп­тенабушь да гуселник Тебушь постарались!
