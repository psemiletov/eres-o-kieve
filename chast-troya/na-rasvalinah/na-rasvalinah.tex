\chapter{На развалинах}

В Турции, на полуострове Малая Азия (Asia Minor), в области Чанаккале, прежде слывшей как Троада, в шести километрах на восток от города Еникей с одноименным портом, среди возделанных полей расположен Национальный парк Троя\footnote{\textasciitilde{}39°57'27"N 26°14'19"E}.

Тут в окрестностях холма Хиссарлык\footnote{Что в переводе с турецкого значит просто Крепость.}, на плато разбросаны развалины городка, Генрихом Шлиманом в 19 столетии громко отождествленного с Троей. Проходя через селение Тевфикие, сюда отправляются туристы, чтобы за деньги побродить среди поросших травой и кустами развалин из кирпича и блоков известняка. Частью они древние, частью воссозданы уже в наше время. Живописно лежат переломленные колонны.

Ученые разделили эти руины по этапам развития, строительства города – Троя 1, Троя 2, и так до девятой\footnote{На деле почти каждый из этих номеров тоже разделен на подномера при помощи добавления букв и цифр.}, оставленной жителями примерно в 500-х годах нашей эры согласно традиционной хронологии. На этом месте с седых времен существовало поселение. Оно то подвергалось разрушению, то восстанавливалось или достраивалось. Современные археологи считают «гомеровской» Троей – Трою 6 или Трою 7, а Шлиман полагал ею то, что теперь называют Троей 2. Подробно расписано соответствие номеров Трои к годам её существования.

Для туристов изготовлены таблички и схемы, развита инфраструктура – место для парковки, музей Трои, гостиница. Под открытым небом выставлен деревянный Троянский конь в два этажа, с башенкой и квадратными окнами. Внутрь ведет лестничка. Залезай и ощути себя хитромудрым Одиссеем. В зависимости от источника предания, в легендарного коня помещалось от 100 до 9 человек.

Лежит камень с надписью на греческом языке. Археологи так и не отыскали здесь никаких письмен от Трои 1 до Трои 7b-2 включительно, кроме обнаруженной в 1995 году бронзовой бляхи с некими значками, в которых исследователи заподозрили пиктограммы хеттов. Развалины нумерованных Трой принадлежат к разным культурам – хеттской, греческой, римской (Троя-9). Это видно по стенам, очертаниям валов, посуде, колоннам и так далее.
%Возможно, часть руин относится к упомянутому на хеттских табличках государству Вилусе. 

По монетам\footnote{В 19 веке население местной деревни Калифатли собирало в руинах монеты с надписью «Ilium».} и надписям археологи установили, что здесь существовал основанный греками небольшой город Илион, входящий в состав Римской империи. Жители оного предполагали, что выстроили свой город на месте, на развалинах легендарной Трои. Слово Илион, судя по источникам – было то ли вторым названием Трои, то ли именем ее главного укрепления на возвышенности. И город, возникший на неких руинах, греки в честь того, былого Илиона, назвали тоже Илионом.  

Сей второй Илион – или, чтобы не путать его с легендарным – далее буду писать Новый Илион – вполне обоснованная находками и письменными источниками данность.

Это ко времени Нового Илиона относится театр на 6000 зрителей, известные надписи на греческом – например, сообщение о том, что на празднике накормили 3000 человек. К тому времени принадлежит храм Афины, и стены почти километр на километр вокруг города – их построили под руководством Лисимаха, полководца Александра Македонского. 

Новый Илион дожил и до появления в нем христианской церкви. Археологи предполагают, что люди обитали тут примерно по тринадцатый век нашей эры включительно, в виде крошечного посёлка на развалинах. Но с заселением Троады Турками жизнь в Новом Илионе угасла.

Неподалеку еще при Шлимане было греческое поселение Калифатли, но в 1922 году, после войны, жители покинули его. Современный Калифатли населен Турками, основавшими близко прежней деревни новую с таким же именем.

Городище Нового Илиона сосредоточилось на холме Хиссарлыка и вокруг него. Какой высоты был Хиссарлык прежде, неведомо, поскольку его срывали и в древности, уничтожая былые постройки, и в близкое к нам время – особенно много срыл Шлиман. Культурные слои купно имеют высоту около 15 метров – с некоторых давних времен холм приподнялся на такую высоту.

В стороне от Хиссарлыка, через поля извивается речушка Кючюк Мендерес\footnote{Малый Мендерас.}, сопоставленная со Скамандром – рекой, протекавшей у легендарной Трои, да Хамамлык, от которого прорыли в сторону канал. Под легендарной Троей подразумеваю ту, что упомянута в дошедших до нас древних произведениях о Троянской войне и предшествующих событиях.

Ученые написали сотни трудов, где сопоставили различные названия, упомянутые у Гомера и других сочинителей, с современными турецкими названиями. Ведь уже к 14 веку эта земля были завоевана османами. Впрочем известно историческое ее название – Бига-Троада или просто Троада. 

Оно встречается во множестве древних источников. Например, в «Деяниях святых апостолов» есть подробное описание перемещений апостола Павла через Троаду (20:7-12). Много внимания уделяет Троаде в «Истории» греческий географ и историк Страбон, живший, по представлениям нынешних ученых, на стыке нашей эры и минувшей.

Чем больше археологи раскапывают местность национального парка, тем больше возникает вопросов и споров. Они не выходит за рамки устоявшегося представления о том, что исследуемый предмет является остатками легендарного города Трои, описанной в многочисленных источниках, а самый известный это поэма «Илиада» Гомера.

Когда же и среди кого сложилось это представление, так ли было в обозримом прошлом?
