\chapter{Троя от Шлимана}

Давайте обратимся к обозримому прошлому и попытаемся выяснить, как получилось, что развалины на холме Хиссарлык стали считать остатками легендарного города Трои, описанного в древних произведениях о Троянской войне и смежных с нею событиях.

У нас есть довольно четкое сопоставление этих развалин с городом Новый Илион. Будем также, для простоты, использовать даты, принятые официальной наукой, хотя я сомневаюсь в истинности этих дат.

Считается, что примерно в 700 году до нашей эры Греки основали в развалинах на холме Хиссарлык поселение, и назвали его Илион в честь легендарного Илиона или Трои. Эти Греки-поселенцы хорошо использовали притягательную легенду. Новый Илион стал привлекать сильных мира сего, способных жертвовать деньги. 

По Геродоту\footnote{История, книга 7, раздел 42.}, Ксеркс\footnote{Ученые относят это предание к 480 году до нашей эры.}, двигаясь с войском на Грецию, прибыл сюда и в храме Афины принес в жертву тысячу быков\footnote{Число это кажется мне баснословным.}. Здесь же побывал\footnote{Считается, что в 330 году.} Александр Македонский, большой поклонник «Илиады». Он побродил по городу и взял в храме Афины некое оружие и щит, выдаваемый жрецами за то самое, с Троянской войны. После, по приказу александрового полководца, Илион обнесли стеной. Негоже, мол, Троя – и без стены.

Подробно о посещении города Александром, и вообще о Новом Илионе рассказывает Страбон в своей «Географии» (Книга XIII – Малая Азия, Троада, Лесбос, Пергам)\cite{strabon01}:

\begin{quotation}
Но даже и Ил\footnote{По преданию, основатель Илиона.} не вполне еще осмелился на это; ведь не основал он города там, где он находится в настоящее время, а почти что в 30 стадиях выше по направлению на восток и к горе Иде и к Дардании, на месте так называемой теперь «Деревни илионцев»\footnote{Страбон принимал мнение уроженца Троады, Деметрия из Скепсиса, что легендарный Илион был в 30 стадиях к востоку от Нового Илиона.}.

Современные илионцы, которые из честолюбия желают, чтобы их деревня была древним Илионом, побудили исследовать этот вопрос ученых, строящих свои выводы на поэзии Гомера; ибо это место, по-видимому, вовсе не гомеровский Илион. 

Другие ученые выяснили на основании исследований, что город несколько раз менял свое местоположение и наконец остался там, где находится теперь, около времени царствования Креза. Совершавшиеся тогда такие переселения в нижележащие местности, по моему мнению, указывают также на различные ступени образа жизни и цивилизации. Но этот вопрос подлежит исследованию в другое время.

Город современных илионцев, как говорят, прежде был селением с небольшим и незначительным святилищем Афины. Когда же Александр после победы при Гранике прибыл сюда, он украсил храм посвятительными дарами, назвал селение городом, приказал тем, кому было вверено попечение над городом, восстановить его постройки и объявить независимым и свободным от податей; впоследствии же после разгрома персов он отправил туда благосклонное послание, обещая построить великий город, сделать храм знаменитым и учредить священные игры.

После его смерти Лисимах проявил особую заботу о городе: он отстроил храм и окружил город стеной около 40 стадий длины, переселил в него жителей старых и уже разрушенных городов из окрестностей. [...]

Современный Илион также был чем-то вроде города-деревни, когда римляне впервые вступили в пределы Азии и вытеснили Антиоха Великого из области по эту сторону Тавра. Во всяком случае, согласно Деметрию из Скепсиса, который в то время подростком посетил этот город, он нашел городские дома в таком запустении, что они даже не имели черепичной кровли. По словам Гегесианакта, галаты, переправившись из Европы, пришли в город, нуждаясь в укрепленном месте, но тотчас же покинули его, так как он не имел укреплений.

Впоследствии, однако, в городе были произведены большие восстановительные работы. Затем он снова был разрушен римлянами во главе с Фимбрией, который взял его после осады во время войны с Митридатом. Фимбрия был послан в качестве квестора при консуле Валерии Флакке, назначенном главнокомандующим в войне против Митридата. Фимбрия поднял восстание, убил консула в Вифинии, сам стал во главе войска и двинулся на Илион; когда же жители Илиона не приняли его как мятежника, то он применил силу и взял город на одиннадцатый день. 

Когда Фимбрия стал хвалиться, что он на одиннадцатый день захватил этот город, который Агамемнон взял лишь с трудом на десятый год, имея флот в тысячу кораблей, причем вся Греция помогала в походе, один из илионцев заметил: «Да, но у нас не было такого защитника, как Гектор». Затем на Фимбрию напал Сулла и разбил его; Митридата он по мирному договору отпустил на родину, а илионцев утешил, оказав им большую помощь по восстановлению города.

Однако в наше время Божественный Цезарь проявил о них гораздо большую заботу, подражая Александру. Ибо Александр стал заботиться об илионцах, имея в виду восстановить древнее родство с ними и будучи в то же время поклонником Гомера. 

Во всяком случае передают об исправлении текста гомеровских поэм так называемой «редакции из Ларца», так как Александр совместно с Каллисфеном и Анаксархом просмотрел их и в некоторой части снабдил примечаниями, а затем вложил экземпляр в ларец с драгоценными инкрустациями, найденный среди сокровищ персидской казны. 

Таким образом, Александр проявлял благосклонность к илионцам в силу своего преклонения перед поэтом и по родству с Эакидами, царями молоссов, где, по рассказам, была царицей Андромаха, бывшая супруга Гектора. 

Что касается Цезаря, то он не только был поклонником Александра, но, имея более действительные доказательства родства с илионцами, смело, со всем пылом юности стал благодетельствовать им. Эти доказательства были более действительными, во-первых, потому что он был римлянин, а римляне считали Энея своим родоначальником, во-вторых, потому что имя Юлий производили от Юла, одного из его предков; последний получил свое прозвище от Юла, одного из потомков Энея. 

Поэтому Цезарь отдал им землю, сохранив свободу и освобождение от государственных повинностей; они сохраняют и до настоящего времени эти привилегии. Однако то, что древний Илион не был расположен на этом месте, если рассматривать этот вопрос согласно данным у Гомера, можно заключить на основании следующего.
\end{quotation}

Далее Страбон подробно рассказывает о местности Нового Илиона и сравнивает ее с тою, что описана у Гомера в «Илиаде». На основании этого и других сопоставлений Страбон делает вывод, что Новый Илион никак не может быть местом Илиона-Трои, выведенного в «Илиаде». 

Страбон также упоминает Гестию из Александрии, «которая написала сочинение об «Илиаде» Гомера и исследовала вопрос о том, происходила ли война около современного города Илиона и Троянской равнины, которую поэт помещает между городом и морем; ведь, по ее словам, равнина, которая видна теперь перед современным городом, является позднейшим наносом рек».

В 355 году нашей эры любитель поэзии Гомера, император Юлиан, язычник, отправился в Троаду, в Новый Илион, дабы проникнуться духом героической старины. Юлиан ожидал, что храм Афины\footnote{Между прочим, статуя Афины была в нем деревянной.} осквернен христианами, однако оказалось, что местный христианский епископ сам отправляет в нем поганские обряды и вообще старой веры держится. Также епископ показал Юлиану «склеп Гектора», где поддерживался священный огонь.

Так несколько веков существовало мнение о том, что Троя была на нынешнем холме Хиссарлыке. Мнение поддерживали жители Нового Илиона, и опровергали некоторые ученые, тот же Страбон. Но сила опровержения с годами слабела. Новый Илион с Илионом легендарным отождествляли Плиний, Дионисий Периегетес, Стефанус и другие. 

На ряде средневековых карт, в Малой Азии, примерно в области Нового Илиона, но с большой погрешностью, картографы отмечали то Трою (Troas), то Трою и Илион отдельно. Очевидно, что это отображение именно Нового Илиона, на который уже, как истину, перенесли имя легендарной Трои. Неясно лишь, была там жизнь при составлении карт, или уже прекратилась.

Между тем в карты проникало не только отражение современности, но и воссоздание былого по письменным источникам и, вероятно, картам старинным еще для того времени.

Хорошим примером изображения Трои на картах служат работы Ортелиуса.

В 16 веке, во Фламандии жил географ и картограф Абрахам Ортелиус (Abraham Ortelius, 1527 – 1598), известный в первую очередь своим великолепным атласом Theatrum Orbis Terrarum. Впервые атлас был издан в 1570 году и состоял из 53 карт, выдержал много изданий и переводов. Помимо современных на то время карт, он содержал карты исторические, что иллюстрировали некий легендарный либо исторический сюжет. Например, путешествие Авраама, святого Павла, Одиссея, Александра Македонского и так далее. 

В предисловии к атласу, Ортелиус пишет, что при знакомстве с исторической литературой, читатель сталкивается с названиями стран, которые ничего ему не говорят. Ортелиус полагает, что было бы здорово любознательному читателю видеть перед глазами карту для того или иного известного сюжета – путешествий знаменитых императоров, великих переселений народов, странствий легендарных героев.

Часть карт – большей часть «современные» – составлены Ортелиусом лично, но в атлас вошли карты сторонних авторов, без изменений либо с поправками Ортелиуса. Имена сторонних картографов Ортелиус печатал на обратной стороне каждой карты, купно с примечаниями. Ортелиус снабжал карту также списком литературы для дальнейшего чтения, и приводил источники, на основе которых составлена карта. 

Для карт, связанных с Древней Грецией, Ортелиус отмечает, кроме прочих, работы Страбона и Николауса Софиануса. Карты Греции «по Софианусу» известны примерно с середины 16 века, их стали рисовать еще до Ортелиуса, подлинник же Софиануса относится к 1544 году. Софианус, кроме прочего, известен списком соответствий древних названий и современных\footnote{Nomina antiqua et recentia urbium graeciae descriptionis a N. Sophiano Iam Aeditae. hanc quoq[uae] paginam, quae graeciae urbium, ac locorum nomina, quibus olim apud Antiquos Nuncupabantur, n.d.} – подобные списки были тогда в ходу среди ученых.

У Ортелиуса есть и карта странствий Энея, сына Приама и богини Афродиты. О них повествует поэма «Энеида» Вергилия, охватывающая время после разрушения Трои. Карта озаглавлена «AENEAE TROIANI NAVIGATIO Ad Virgili lex prioris Aeneidios». Я знаю эту карту по изданию 1594 года и на ней обозначена Троя\footnote{Была ли она в первой редакции Атласа, не ведаю.}.

Троя присутствует и на карте «ASIAE NOVA DESCRIP\-TIO» и еще на многих. Где же? В Троаде, на полуострове нынешней турецкой провинции Чанаккале, а вот на холме ли Хиссарлык, по картам Ортелиуса не видно. 

На всех Троя обозначена как «Troia», условно нарисован город, как и многие другие города, и не указано, что Троя – только развалины. На карте «по Энеиде» Троя довольно отодвинута вглубь материка, а на других картах – ближе к берегу. На карте по странствиям святого Павла «PEREGRINATIONIS DIVI PAVLI TYPUS COPOGRAPHICUS» написано TROAS как название местности, а на ней два поселения – Ilium на севере и Treas либо Troas к югу. И еще южнее поселения Troas – Alexandria, а на восток лежит страна MYSYA, она же, на той же карте, обозначена и в современной Болгарии. Поскольку было несколько Мисий – в Малой Азии и в Болгарии.

Итак, Ортелиус создавал карты обыкновенные и, что называется, исторические. В последних он занимался реконструкцией. И неясно, то ли на «обычных» картах Троя – это Новый Илион, еще населенный во время составления карты, либо это материал, привнесенный из исторической карты.

Помимо карт Ортелиуса, есть и другие, где произошло смешение исторического материала с современным составлению карты, и теперь зачастую уже трудно разобраться в подробностях. 

А вот на карте Маттиаса Куада «Graeciae Universae Secundum Hodiernum Situm Neoterica Descriptio» 1600 года, напротив острова Лесбоса, в уже привычном месте Малой Азии, видим изображение то ли развалин, то ли лабиринта с прямоугольными очертаниями, и громадную подпись: Troia. А к югу от нее есть и поныне залив Çandarlı Körfezi (в его восточной точке – поселок Yenişakran). В залив, по карте, впадало несколько рек. На реке Sarabat был город Troia vechia, а на другой речке – Troia nova, то бишь Троя старая и Троя новая.

Кроме этих Трой, в Малой Азии была известна Троянская Александрия (Alexandria Troas), основанная Антигоном и переименованная Лисимахом. Ее развалины принимались некоторыми путешественниками 16 века за дворец Приама, царя Трои.

Что можно сказать? Уже в средние века, в ходу были карты с Троей, и любой желающий мог, путешествуя по Малой Азии, попытаться на местности подыскать какие-нибудь развалины и заподозрить в них Трою. Шлиману не надо было «учить греческий», «скрупулезно изучать Илиаду» – довольно было купить в ближайшей антикварной лавке старую карту. Впрочем, он пошел другим путем, но про это после.
 
Что же сами обитатели мест, куда картографы втулили Трою? Свидетельства жителей Нового Илиона, посе\-ленцев-греков, известны нам по отголоскам в труде Страбона. Следующее по времени свидетельство относится к 15 веку.

Испанский путешественник Педро Тафур (1410-1484) в описании своих странствий за 1436-1439 годы, посетил кроме прочего остров Киос (Chios). Конечно, название похоже на Киев, я заметил. От Киоса он поплыл к берегам Турции, к порту Foja-Vecchia, который находился на небольшом расстоянии от острова. В Foja-Vecchia было поселение генуэзцев, где путешественник нашел своего друга, знакомого по Севильи.

В сопровождении одного из людей этого товарища, наняв лошадей, Тафур отправился к месту, слывущее Троей. Тафур имел такое представление, что Троя была там-то.

Путь занял три дня. По прибытии – куда именно, понять нельзя – никто не знал ни о какой Трое, и Тафур поехал в «Илиум» («как они называли его» – говорит путешественник, под «они» вероятно разумея местных). Сей Илиум был расположен на море напротив гавани Тенедос. Тенедос (по-турецки Bozcaada) – это остров. Место, которое указывает Тафур – в 20 километрах южнее Хиссарлыка. 

Путешественник пишет, что вся эта земля наполнена руинами, но здешние жители их не сносят, а строят свои жилища неподалеку. Покрутившись там, Тафур увидел холм с развалинами большого здания (Хиссарлык?), обломки мрамора, строительного камня, и это убедило Тафура, что тут была древняя Троя. Затем он вернулся на Киос. 

Если Илиум Тафура и Новый Илион – одни и те же развалины, значит, при Тафуре они были окончательно оставлены жителями.

Чтобы понять суть записок Тафура, я пользовался английским переводом и подлинником\footnote{
Текст\cite[стр. 134]{tafur00}:

\begin{quotation}
Yo estuve en esta ysla de Exio veynte dias, en que non tenia que fazer; fizeme pasar a la Turquila, que es un pequeno estrecho, e en aquell tienen fazimiento ginoveses, e falle alli uno mi amigo, que conosci en Sevilla, e roguele, pues el tenia tanta noticia con los turcos, que embiase un onbre suyo conmigo que me levase fasta Troya e me buscase calallo-alquilado, e ansi lo fizo; e camine por tierra dos jornadas por aquel lugar que dizen que era Troya, non fallando persona que sapiese dar racon ninguna, e fui fasta llegar al Elion, que dizen; este es pegado a la mar enfrenre del puerto del Tenedon.

Toda esta tierra es poblada a caserias, e an los turcos por reliquias los edificios e non desfarian ninguno dellos, antes fazen sus casas junto con ellos; e lo que mas vi para cononscer que aquel fuese el Elion de Troya, fue ver grandes pedacos de edificios e marmoles e losas, e aquella ribera, e aquel puerto del Tenedon enfrente, e un muy grande otero como que cayda de grande edificio lo oviese fecho.

E desto non pude mas saber, e volvime a Exio, e falle mi nao adovada, e dende a dos dias fezimos vela.
\end{quotation}}, разобраться в котором мне помогли добрые подруги Света и Неля, чьи познания в испанском языке далеко превосходят мои.

%Что же? Путешественник Тафур, которому кто-то раньше рассказывал, где найти Трою, отправляется туда. Однако тамошние люди не знают про Трою. Тафур заранее убежден, что Троя именно здесь. От места, где о Трое никто не слышал, Тафур едет в Илиум (Новый Илион?), что «на море», и застает там разрушенные здания, обломки мрамора, камней – словом, развалины. Здесь Тафур понимает, что находится среди остатков древней Трои. 

В 18 и 19 веках появляются крупные ученые труды по Трое, попытки понять ее положение. Это «Описание равнины Трои» (Description Of The Plain Of Troy) Жана Батиста Шевалье (1791), «Сравнительный обзор древнего и текущего состояния Трои» Роберта Вуда (Robert Wood, A Comparative View of the ancient and present State of the Troade), «История Илиума или Трои» Ричарда Чандлера (Richard Chandler, The History of Ilium Or Troy, 1802).

Книга Шевалье позволяет понять состояние знаний о Троаде на конец 18 века. Например, никто толком не ведает, где находились упоминаемые Греками реки Скамандр и Симоис – у каждого исследователя свои представления об этом.

Шевалье лично путешествует по Малой Азии и, глядя на окрестности, пытается разобраться, вызвать к жизни прежние названия, беря за основу поэмы Гомера, сведения Страбона и так далее. Ведь всё вокруг уже переименовано Турками.

Шевалье попутно оспаривает некоторые утверждения Страбона и Деметрия относительно топографии, и смешивает Новый Илион с Илионом легендарным. У Шевалье это один и тот же город, и по Шевалье получается, что Ксеркс, Александр и так далее – все посещают ту самую, легендарную Трою и воздают ей честь. 

Устраняется всего один этап в изложении – про то, что Греки, жители построенного на развалинах города Илиона, утверждают, что их Илион возведен на месте легендарного – и история Нового Илиона получает, в передаче Шевалье, продолжение в глубину веков, туда, к событиям Троянской войны, описанным Гомером. Такое умолчание со стороны Шевалье удивительно. Ладно бы не знал, так не сказал, а то ведь знал, но промолчал. 

%А трудов предшественников, относительно Трои, у Шевалье было множество – впору составить по ним библиографию исследований о Трое и смежных предметах.

Шевалье вспоминает о Новом Илионе много дальше, когда опровергает рассуждения Страбона о том, что Новый Илион – не тот, старый Илион. Шевалье утверждает, что Страбон-де никогда не был в Троаде лично, местность знал по сведениям от Деметрия. Шевалье пишет, что имеет достаточно оснований «поместить Древнюю Трою в начале цепи холмов, и Новый Илион в её, цепи, окончании».

Шевалье затем предполагает, что древняя Троя была около деревни Bounar-Bachi (Бунарбаши, теперь Пинарбаши или Пынарбаши – Pınarbaşı), на холме Бали Даг, что в 8 километрах на юго-восток от Хиссарлыка. 

Книга Шевалье любопытна не только собственными исследованиями сочинителя, но и приведенными выдержками из других трудов. Так, мы узнаём, что в 17 веке местное население не знало гору, что ныне считается учеными Идой, как Иду. Да и теперь Ида это научное название, вызванное к жизни учеными. Турки-то именуют ее иначе. Греческие соответствия были восстановлены учеными, но насколько верно?

Еще несколько веков назад Шевалье печатно советовал Роберту Вуду, что тот с равной пользой может составлять  карты Троады или рвать их на куски. Сам Шевалье отождествлял легендарную реку Симоис с современным ему Мендере (теперь Кучюк-мендерес или Карамендерес), а уже спустя сотню лет ученые решили, что Мендере раньше называлась именем другой легендарной реки – Скамандром.

С тех пор мнение «научного большинства» постепенно выработало некие общие представления о соответствии древних греческих названий и современных турецких, однако споры не утихают поныне.

Вернемся к вехам изучения Троянского вопроса.

Чандлер в 1802 году своей книгой вступается перед Шевалье за авторитет Страбона и Деметрия из Скепсиса, и смешивает Новый Илион с Илионом легендарным, для него это один город. Именно из книги Чандлера можно узнать много про Новый Илион, ибо Чандлер скрупулезно разложил по времени всё что отыскал по Илиону, не важно, Новому или старому. «История Илиума» касается более именно истории, не топографии, и Чандлер не делает особых уточнений относительно места, где стояла Троя.

На холм Хиссарлык, Новый Илион – как на вероятную Трою – указал шотландский геолог и журналист Чарлз Макларен (Charles MacLaren, 1782-1866) в 1822 году в диссертации «Dissertation on the Topography of the Plain of Troy», позже переработанной и изданной под заглавием «The Plain of Troy Described».

Очень обстоятельный труд. Подобно Шевалье, Макларен, вооруженный материалом предшественников, лично исследует местность и прикидывает топографические сведения Гомера и Страбона к современности. При этом Макларен прибегает к следующим доводам. 

Вот, пишет он, речушка Мендерес, легендарный Скамандр. Но ведь это в Троаде самая большая река. Да и по меркам древних Греков это просто здоровущая река. Но у Гомера она описана глубоководной. Так и есть, говорит Макларен – во время паводка Мендерес можно назвать глубоким. Что там Гомер еще писал? «Большая река с водоворотами». Макларен подтверждает – во время паводка конечно и большая, и с водоворотами! Но в обычном состоянии местные жители переходят его вброд. И это тоже Макларен усматривает в «Илиаде» – Греки часто пересекают Скамандр туда-сюда, будто ничего им не мешает. Наличие моста Макларен не предполагает. Ему ведь нужно доказать, что мелкий, тщедушный Мендерес – это глубокий, большой Скамандр. Надо углубить Мендерес? Что же, наводнение. Надо, чтобы Греки его без трудности переходили? А вот, Мендерес мелкий как лужа.

Чандлер, Шевалье, Макларен, Вуд – все приводили достаточно убедительные доводы, все ходили вокруг да около одной местности, разница во мнениях была не так уж значительной. Но именно Макларен назвал точное место.

Фрэнк Калверт (Frank Calvert, 1828-1908), британский и американский консул на османских землях восточного Средиземноморья, проявлял интерес к археологии и был знаком с работой Макларена. Собственно, было три брата Калверта – Фредерик, Джеймс и Фрэнк. Они жили в Троаде и увлекались поисками Трои. Особое рвение проявлял Фрэнк. Он постепенно восстанавливал для себя древние названия, изучал печатные работы.

Фрэнк стал искать Трою на холме Хиссарлык раньше Шлимана. Брат Фрэнка, Фредерик, даже купил ферму площадью 8 км² в Акча-Кой, залезающую на восточную часть холма. Фрэнк в 1865 году провел там раскопки. Им отрыты были часть городской стены (той самой, Лисимаха), развалины храма Афины. Обнаружил керамику, мутно датированную римским, эллинским и доэллинским периодами. Дальнейшие раскопки требовали денег, а их у Фрэнка не было.

В 1868 году Калверт показал свои археологические изыскания ловкому деловому человеку Генриху Шлиману, тоже интересовавшемуся Троей. И Шлиман, получив у Калверта и правительства Турции разрешение на раскопки, приступил к работе, о чем уведомлял миру в книгах и газетах. Легендарная Троя – найдена!

Уже в 1869 году Шлиман публикует книгу на немецком – «Ithaka, der Peloponnes und Troja», где Трое покамест отведена примерно пятая доля текста в конце. Шлиман уже знаком с Калвертом, уже вооружен трудом Макларена, уже раскапывает Хиссарлык, уничтожая на своем пути всё, что не представляет для него ценности.

Вскоре, в 1873 году, он объявляет, что нашел золотой клад царя Приама! Почему Приама? Так Шлиман решил.

И тайно вывез сокровища из Турции. Мир облетает фотография юной жены Шлимана, гречанки Софии, в украшениях из этого клада. Шлиман утверждал, что сие – драгоценности Елены Троянской.

Относительно происхождения этого клада ученые спорят до сих пор. Одни верят Шлиману. Другие говорят, что он выдал искусные подделки за древние драгоценности, третьи всё ищут, куда клад делся. Вот Россия в 1993 году объявила, что клад в 1945-м был вывезен из Берлинского музея и находится в Москве. Меня история «клада Приама» мало волнует.

В 1874 году Шлиман издает на французском двухтомник «Троянских древностей – отчет о раскопках Трои». Далее следуют «Troja und seine Ruinen'» (1875) и «Ilios, City and Country of the Trojans» (1880). Это роскошное издание объемом под тысячу страниц с картами, схемами, изображениями находок. Нашлось, в дополнениях, и скромное место Калверту, для помещения там небольшой статьи оного. Роль Калверта в тексте «Илиона», по Шлиману, была незначительна – мол, начинающий археолог Калверт показал Шлиману развалины на Хиссарлыке, причем, дескать, Калверт думал, что это древний город Гергис – и такое же мнение поначалу имел сам Шлиман.

В том же году Шлиман выпускает еще одну книгу – «Поездка в Троаду».

В «Илионе» Шлиман не именовал изучаемые им развалины «небольшим городком». Это случилось спустя три года в новой книге, «Троя», когда Шлиман столкнулся с противоречием между описанием Трои Гомером и тем, что предлагал Хиссарлык. Ведь гомеровская Троя – огромнейший город с широкими улицами! Поэтому Шлиман решил продолжать раскопки. На холму, Шлиман нашел остатки только «только одного или двух больших зданий», зато дал обоснование, что город на холме продолжался «нижним городом» вне холма, затрудняясь сказать, как далеко.

С невероятной легкостью Шлиман определял руины, а затем с такой же легкостью отказывался от былых определений:

\begin{quotation}
Пользуясь случаем, я хотел бы, полагаясь на свидетельство моих архитекторов, заверить читателя, что ошибался, полагая, что в 1873 году я разрушил храм Афины Паллады на юго-восточной стороне Гиссарлыка, и что это просто был фундамент римского портика, большую часть которого мне действительно пришлось разрушить, чтобы раскопать доисторические города под ним. 
\end{quotation}

В самом деле! Теперь-то Шлиман «нашел» этот храм в другом месте. Сходным образом Шлиман придумывает описания других находок. Например, остаткам статуи расслабленного лежащего мужчины с рогом для пития и опустошенным кувшином, он дает описание: «Речной бог, возможно Скамандр, с рогом изобилия и урной».

Что же, легендарную Трою рано или поздно признали бы в развалинах Нового Илиона. Шлиман лишь ускорил дело, вложив деньги и произведя мощную раскрутку своей деятельности. 

После смерти Шлимана ученые многократно проводили археологические работы на Хиссарлыке, открывали новое, уточняли или опровергали старое. Эти развалины внесены в охранный список Юнеско, а в тамошнем описании сказано, что Шлиман первым сделал раскопки, хотя научный мир хорошо осведомлен о Калверте, но – поддерживают миф.

Археологи да историки пишут новые книги, стало модным отождествлять Трою 6 и 7 с постройками подчиненного Хеттам города Вилусы – и отыскалось лувийское имя одного из его царей – Пиямараду, правда очень похоже на Приама? И это «Пияма» в имени значит «дар», а что значит «раду», никто не знает, ибо лувийские слова с «р» не начинались, так что еще одна загадка. Но другие ученые возражают – никакой он не был царь, а так, просто. Значит не Приам! Ведь Приам – царь, а этот Пиямараду – не царь. Есть над чем поразмыслить ученым, потолочь воду в ступе.

Но мы дошли до корня. Троя на Хиссарлыке как легендарная Троя зиждется, в представлении ученых, на нескольких основаниях. 

Основание первое – утверждение греков-поселенцев, построивших Новый Илион на развалинах некоего древнего города. Основание второе – сравнение местности с географическими данными из «Илиады». Маленькие реки принимаются за большие, и так далее – но я согласен, пусть принимаются, может в самом деле Гомер имел в виду Скамандр то в половодье, то в обычном состоянии, я не знаю. Огромный город Троя вдруг оказывается поэтическим преувеличением – ладно, почему нет?

Но почему описание Трои и местности берется учеными только из «Илиады»? Разве нет других источников, кроме Гомера?

Сначала впрочем почитаем, а что же пишет Гомер.
