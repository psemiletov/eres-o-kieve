\chapter{Кожемяка и Змей}

Самое время поговорить о Кириле (Никите) Кожемяке. С одной стороны, предание это весьма похоже на вторую половину былины о Добрыне, там, где богатырь спасает княжну. Разница в происхождении героя – Добрыня-то боярин, а Кожемяка – ремесленник, мнет по двенадцать кож за раз.

В предании, Змей не похищает дочку князя, но получает ее в жертву, как ежегодную дань, по установленному обычаю. Раньше жертвовали кем-то из простолюдинов, а тут по некой умолчанной причине пришел черед княжеской дочери.

Странно, что местное сказание это вообще сохранилось и разошлось по землям русским, ведь в прошлом Киева какая-то бездна, поглотившая почитай все народные предания.

Попробуем найти письменные истоки предания про Кожемяку. Ведь разные его варианты, кочующие из одного сборника в другой, из статьи в статью, могут быть выдумкой более-менее современных сказочников, написанной на основе всамделишной истории. Раньше всех, за 1820 год, печатно упоминает про Змея и Кирилла, без прозвища, Берлинский\cite{berl01}, когда рассказывает историю монастыря: 

\begin{quotation}
Название Кириловского произошло от некоего, споспешествовшего к восстановлению сего монастыря, по имени Кирила. При том случае успел он разорить гнездившуюся там в лесу издавна шайку разбойников, от чего и вышла баснь о Кириловском змее, коего пещеры и теперь в лесу видны.
\end{quotation}

Про разбойников я нигде более не читал. Возможно, у Берлинского был свой источник, но я, в дошедших до нас вариантах сказания, никаких супостатов кроме Змея не встречал. Змей, Кожемяка и родственница князя Владимира – персонажи постоянные. Кожемяка это прозвище по занятию его, а вот имя меняется: Кирилл, Кирил, Кирило, Никита.

Полагаю, предание отражает случай, произошедший некогда в действительности. Но имена в версиях предания разнятся. Как же понять, какой вариант стоит ближе к подлинно произошедшему? Ведь если, допустим, положительного героя звали Кириллом, то образуется связь между ним и Кирилловским монастырем. А ежели звали Никитой, то никакой во всяком случае по имени связи не было, нечего об этом и рассуждать! 

В 1847 году в изданном Фундуклеем «Обозрении Киева» кажется впервые печатно приведен «баснословный рассказ», в статье про Кирилловскую пещеру. Заодно описано её местоположение:

\begin{quotation}
Она находится в лесу, на правой или киевской стороне оврага, лежащего по сю сторону бывшаго Кириловского монастыря, что ныне городские богоугодные заведения. Происхождение Кириловской пещеры очень давнее. В простонародьи она служит и по ныне предметом баснословного рассказа о змее и о могучем киевском богатыре Кириле-кожемяке. 

Лютый змей, поселившийся в этой пещере, требовал от киевлян в дань людей, и никто не мог противиться чудовищу, кроме Кирила кожемяки т.е. кожевника. Проведав об том киевляне обратились к нему с мольбою; и как стукнули дверью, то Кирило, вздрогнув от того, разорвал пополам 12 кож, которые вместе он мял в руках. Рассердясь за то, он долго не соглашался; но наконец решился идти на змея. Борьба была продолжительна и страшна; но Кирило одолел, и после победы заснул богатырским сном в своей клети, попросив наперед свою мать, чтоб она не мешала ему спать 12 суток. Мать ждала 11 суток, но на 12-я не утерпела и стала будить. Кирило проснулся; но попрекнул мать за нетерпенье, и тот час же умер. 

Эта сказка, в разных видах, повторяется в Украине с давних времен.
\end{quotation}

В том же 1847 году Пантелеймон Кулиш издал книгу «Украинские народные предания», где поместил предание «Кырыло Кожемяка», с примечанием «доставил В. М. Белозерский, а он записал возле Борзны, Чернигов. губ». То же предание, слово в слово, Кулиш приводит во втором томе своих «Записок о Южной России» (1856-57), в разделе «Сказки и сказочники», как записанное художником Л. М. Жемчужниковым в некоем селе Пырятинского уезда, где художник отдыхал.

Цитирую по последнему изданию:

\begin{quotation}
Колись був у Києві якийся князь, лицар, і був коло Києва змій, і кожного году посилали йому дань: давали або молодого парубка, або дівчину. Ото пришла черга вже і до дочки самого князя. Нічого робить, коли давали горожане, треба й йому давать. Послав княвь свою дочку в дань змійові. А дочка була така хороша, що й сказати не можна. То змій її й полюбив. От вона до його прилестилась та й питаєтця раз у нього: «- Чи єсть», каже, «на світі такий чоловік, щоб тебе подужав?»

«Єсть», каже, «такий у Києві над Дніпром. Як затопить хату, то дим аж під небесами стелецця; а як вийде на Дніпр мочити кожи (бо він кожемяка), то не одну несе, а дванадцять разом, і як набрякнуть вони водою в Дніпрі, то я візьму та й учеплюсь за іх, чи витягне-то він іх? А йому й байдуже: як поцупить, то й мене з ними трохи на берег не витягне. Оттого чоловіка тілько мені й страшно».

Князівна і взяла собі те на думку й думає: як би їй вісточку додому подати і на волю до отця достатись? А при ій не було ні душі, – тілько один голубок. Вона згодавала його за щасливої години, ще як у Києві була. Думала-думала, а далі храп, і написала до панотця:

«Оттак і так», каже, «у вас», каже, «паноче, єсть в Києві чоловік, на ймення Кирило, на прізвище Кожемяка. Благайте ви його через старих людей\footnote{Обратите внимание на отношения между князем и ремесленником. Князь не приказывает ему прямо, а сам просит, скажем так, у совета старейшин, дабы те, в свою очередь, умоляли кожемяку помочь. Подобная «слабость» князя – а допустим, что под князем подразумевается Владимир – очевидна и в скандинавских сагах, где Владимир вынужден подчиняться действующим законам, а при нарушении оных рискует крепко получить на орехи от горожан, несмотря на то, что князь.}, чи не захоче він із змієм побитьця, чи не визволить мене бідну з неволі! Благайте його, панотченьку, й словами, й подарунками, щоб не обідивсь він за яке незвичайне слово! Я за його і за вас буду довіку Богу молитьця».

Написала так, привязала під крильцем голубові та й випустила в вікно. Голубок звився під небо та й прилетів додому, на подвір'є до князя. А діти саме бігали по надвір'ю та й побачили голубка. «Татусю, татусю!», кажуть, «чи бачиш – голубок од сестриці прилетів»?

Князь перше зрадів, а далі подумав-подумав та й засумовав: «Се ж уже проклятий Ірод згубив, видно, мою дитину!»

А далі приманив до себе голубка, глядь, аж під крильцем карточка. Він за карточку. Читає, аж дочка пише: так і так. Ото зараз покликав до себе всю старшину: «Чи є такий чоловік, що прозиваєтця Кирилом Кожемякою?»

«Єсть, князю. Живе над Дніпром.»

«Як же б до його приступитись, щоб не обідився та послухав?»

Ото сяк-так порадились та й послали до його самих старих людей. Приходять вони до його хати, одчинили помалу двері, зо страхом, да й злякались. Дивлятця, аж сидить сам кожемяка долі, до іх спиною, і мне руками дванадцять кож; тілько видно, як коливає оттакою білою бородою! От один з тих посланців: «Кахи»!

Кожемяка жахнувся, а дванадцять кож тільки трісь! трісь! Обернувся до іх, а вони йому в пояс: «Оттак і так: прислав до тебе князь із прозьбою...»

А він і не дивитця, і не слухає: розсердився, що через іх та дванадцять кож порвав.

Вони знов давай його просить, давай його благать. Стали на коліна... Шкода! Просили-просили та й пішли, понуривши голови.

Що тут робитимеш? Сумує князь, сумує і вся старшина.

«Чи не послать нам іще молодших?»

Послали молодших – нічого не вдіють і тіі. Мовчить та сопе, наче не йому й кажуть. Так розобрало його за тіі кожи.

Далі схаменувся князь і послав до його малих дітей. Тіі як прийшли, як почали просить, як стали навколішки та як заплакали, то й сам Кожемяка не витерпів, заплакав та й каже: «Ну, се ж уже для вас я роблю». 

Пішов до князя. «Давайте ж»,  каже,  «мині дванадцять бочок смоли і дванадцять возів конопель».

Обмотавсь коноплями, обсмолився смолою добре, взяв булаву таку, що, може, в ій пудов десять, да й пішов до змія.

А змій йому й каже: «А що, Кирило? Прийшов битьця чи миритьця?»

«Де вже миритьця? Битьця з тобою, з іродом проклятим!»

От і почали вони битьця – аж земля гуде. Що розбіжитьця змій да вхопить зубами Кирила, то так кусок смоли й вирве; що розбіжиться да вхопить, то так жмуток конопель і вирве. А він його здоровенною булавою як улупить, то так і вжене в землю. А змій, як огонь, горить, –  так йому жарко; і поки збігає до Дніпра, щоб напитьця, да вскочить у воду, щоб прохолодитьця трохи, то Кожемяка вже й обмотавсь коноплями і смолою обсмоливсь. Отто вискакує з води проклятий ірод, і що рожженеться проти Кожемяки, то він його булавою тілько луп! що рожженеться, то він знай його булавою тілько луп та луп! аж луна йде. Бились-бились, – аж курить, аж іскри скачуть. Розогрів Кирило змія ще лучче, як коваль леміш у горні: аж пирхає, аж захлипаєтьця проклятий, а під ним земля тілько стогне.

А тут у дзвони дзвонять, молебні правлять, а по горах народ, стоїть як неживий, зціпивши руки; жде, що то буде! Коли ж зміюка бубух! Аж земля затряслась. Народ, стоячи на горах, так і сплеснув руками: «Слава тобі, Господи!»

От Кирило, вбивши змія, визволив князівну і оддав князю. Князь уже не знав, як йому й дякувать, чим його й награждать. Та вже з того-то часу і почало зваться те урочище, де він жив, Кожемяками.

От же Кирило зробив трохи й нерозумно: взяв його, спалив да й пустив по вітру попель; то з того попелу завелась вся тая погань – мошки, комарі, мухи. А як-би він узяв да закопав той попел у землю, то нічого б сього не було на світі.
\end{quotation}

Во втором томе «Трудов этнографическо-статистичес\-кой экспедиции в Западно-русский край, снаряженной Императорским Русским Географическим обществом. Юго-Западный отдел. Материалы и исследования, собранные П. П. Чубинским» (1878) приведен сокращенный вариант этого предания, записанный в Белоруси, в местечке Дрогочин. Вместо князя там господин помельче – «пан». У Шейна Павла Васильевича в «Материалах для изучения быта и языка русского населения Северо-Западного края. Том ІІ» (1893) тоже есть белорусская версия, очень близкая к опубликованной Кулишом.

Выпишу еще хрестоматийный вариант из «Народных русских сказок» (1861 год, выпуск 5-й) Афанасьева:

\begin{quotation}
№ 148 (300) Записана в г. Козлове Тамбовской губ. П. И. Якушкиным\footnote{Тем самым Павлом Ивановичем Якушкиным, что записал предание о змияке-Перуне.}\\

НИКИТА КОЖЕМЯКА\\

Около Киева проявился змей, брал он с народа поборы немалые: с каждого двора по красной девке; возьмет девку да и съест ее. Пришел черед идти к тому змею царской дочери. Схватил змей царевну и потащил ее к себе в берлогу, а есть ее не стал: красавица собой была, так за жену себе взял. Полетит змей на свои промыслы, а царевну завалит бревнами, чтоб не ушла. У той царевны была собачка, увязалась с нею из дому. Напишет, бывало, царевна записочку к батюшке с матушкой, навяжет собачке на шею; а та побежит, куда надо, да и ответ еще принесет. Вот раз царь с царицею и пишут к царевне: узнай, кто сильнее змея? Царевна стала приветливей к своему змею, стала у него допытываться, кто его сильнее. Тот долго не говорил, да раз и проболтался, что живет в городе Киеве Кожемяка – тот и его сильнее. Услыхала про то царевна, написала к батюшке: сыщите в городе Киеве Никиту Кожемяку да пошлите его меня из неволи выручать.

Царь, получивши такую весть, сыскал Никиту Кожемяку да сам пошел просить его, чтобы освободил его землю от лютого змея и выручил царевну. В ту пору Никита кожи мял, держал он в руках двенадцать кож; как увидал он, что к нему пришел сам царь, задрожал со страху, руки у него затряслись – и разорвал он те двенадцать кож. Да сколько ни упрашивал царь с царицею Кожемяку, тот не пошел супротив змея. Вот и придумали собрать пять тысяч детей малолетних, да и заставили их просить Кожемяку: авось на их слезы сжалобится! Пришли к Никите малолетние, стали со слезами просить, чтоб шел он супротив змея. Прослезился и сам Никита Кожемяка, на их слезы глядя. Взял триста пуд пеньки\footnote{Волокно из конопли. Пенька шла на веревки, канаты, парусину.}, насмолил смолою и весь-таки обмотался, чтобы змей не съел, да и пошел на него.

Подходит Никита к берлоге змеиной, а змей заперся и не выходит к нему. «Выходи лучше в чистое поле, а то и берлогу размечу!» – сказал Кожемяка и стал уже двери ломать. Змей, видя беду неминучую, вышел к нему в чистое поле. Долго ли, коротко ли бился с змеем Никита Кожемяка, только повалил змея. Тут змей стал молить Никиту: «Не бей меня до смерти, Никита Кожемяка! Сильней нас с тобой в свете нет; разделим всю землю, весь свет поровну: ты будешь жить в одной половине, а я в другой». – «Хорошо, – сказал Кожемяка, – надо межу проложить». Сделал Никита соху в триста пуд, запряг в нее змея, да и стал от Киева межу пропахивать; Никита провел борозду от Киева до моря Кавстрийского. «Ну, – говорит змей, – теперь мы всю землю разделили!» – «Землю разделили, – проговорил Никита, – давай море делить, а то ты скажешь, что твою воду берут». Взъехал змей на середину моря, Никита Кожемяка убил и утопил его в море. Эта борозда и теперь видна; вышиною та борозда двух сажен. Кругом ее пашут, а борозды не трогают, а кто не знает, от чего эта борозда, – называет ее валом\footnote{Это лишь одно из преданий о Змиевых валах, но все давние сказания говорят о возникновении этих валов в связи со «змиями».}. Никита Кожемяка, сделавши святое дело, не взял за работу ничего, пошел опять кожи мять. 
\end{quotation}

Предания, как видим, повторяют былины лишь частично. Общее между ними – вызволение пленницы (причем не бедной) из Змиева логова. Зато главные герои разные – пожилой бедняк Кожемяка и молодой да богатый Добрыня.

В былинах мы можем заметить б\'ольшую привязку к названиям местности – той же Почайне. Кстати, какие еще реки протекали на Оболони, кроме упомянутых во главе про Почайну?

Вроде бы Сетомль, хотя неясно, река это или озеро. След её теряется в истории. В летописях о ней осталась странная запись. По Ипатьевскому списку:

\begin{quotation}
В лето 6573 [...]

В та же времена бысть знамение на западе: звезда превелика, луче имуще аки кроваве, всходящи с вечера по заходе солнечнемь, и бысть за 7 дний; се же проявляющи не на добро. Посемь же быша усобице многы и нашествие поганых на Русьскую землю, си бо звезда акы кровава, проявляющи крови пролитье.

В та же времена бысть детище вьвержено в Сетомле; сего же детища выволокоша рыболове в неводе, егоже позоровахом\footnote{«Позоровахом» можно перевести как баловались, забавлялись с детищем.} и до вечера, и пакы вывергоша и в воду\footnote{И бросили его обратно в воду.}; бяше бо на лице сице срамнии удове, а иного нельзе казати срама ради\footnote{Был же таков: на лице половые члены, иного нельзя сказать стыда ради.}.

Пред симже временем солнце пременися, не бысть светло, но акы месяць бысть, егоже невегласии глаголють снедаему сущю.
\end{quotation}

И другой список прибавляет:

\begin{quotation}
В си же времена приде вълхв, прельщен бесъмь; пришьд бо Кыеву...
\end{quotation}

Итак, помимо знамений – рыбаки вытащили из Сетомли «детище», описать которое подробно летописец увы постеснялся, упомянув лишь о половых членах на лице (или за них приняли нечто другое, например дыхательную трубку акваланга). Детище – некто маленький.

За эту запись ухватились уфологи – мол, выловили энлонавта! И никто не обращает внимание на окончание истории – пришел волхв, «прельщен бесъмь». Этим бесом, вероятно, и было «детище». Маленький, не человек. Это мог быть один из «змеев», бесов, живших возле Оболони, на Кирилловских высотах в пещерах.

Вот теперь отправимся на Кирилловские высоты! Но где именно искать пещерушки змеиные? Где именно то самое Логово Змиево, о котором так много говорили?
