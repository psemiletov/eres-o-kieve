\chapter{Смородинский спуск}

Любят цитировать Николая Гумилева – «Из города Киева, из логова Змиева». Это не красные слова. В Киеве есть место, где обитали некие «змеи», которые похищали людей, поедали их, пленили в «пещерушки змеиные». Предание связывает с этими пещерами нынешний Смородинский спуск. Ну а кто такие «змеи», мы попробуем разобраться в другой главе.

От окрестностей усадьбы Светославского до Смородинского спуска – один шаг. Ну, сотня. Здесь Логово Змиево. Мы туда полезем позже, подготовившись и вооружась знаниями. Не теми, что журналисты обычно пишут, а народ повторяет – мол, тут жил «змей Горыныч». В связи с этим местом припоминают еще, не вникая, былины про Добрыню Никитича и предание о Никите (Кириле) Кожемяке – как эти богатыри со змеем боролись.

Сия книга возникла благодаря пещере, спиралью ввинчивающейся в гору. Изучение мною пещеры тоже шло по спирали, с каждым витком увеличивая орбиту, отлетая от середины, начальной точки и захватывая всё новые знания, а те порождали вопросы, и приходилось искать ответы. А выводы пошли дальше, чем я мог вообразить.

В 2013 году мы затеяли съемку цикла документальных фильмов «Киевская амплитуда». Решили уделить время и так называемой Змиевой пещере. Я даже толком не знал, где она находится и что представляет.

Подольский спуск, обращенная к Днепру сторона. В некоторой точке видно через заросли другую дорогу, темный Смородинский спуск, а над ним, в осыпавшемся обрыве, несколько ходов-нор. Я подозревал, что это Змиева пещера, но ошибся – она оказалась несколько дальше. На видео мы сняли полное прохождение пещеры, и – место засело у меня в голове и больше не отпускало.

Съемочная группа возвращалась туда снова и снова. Опять посещения этой и окрестных пещер, съемка на видео, фотографирование, замеры. Попутно шло изучение источников по предмету. Это привело к созданию большого фильма «Киевская Амплитуда – Логово Змия», где высказано фантастическое предположение, что в пещере жило нечеловеческое существо небольшого роста, которому давние язычники поклонялись как божеству, вероятно Перуну.

Хотя фильм получился здоровенный, в его рамки сложно было уместить весь наработанный материал. Поэтому мы дали видеозаписи с местности да очень сжато рассказали про загадочных существ. Я хотел успеть дописать книгу одновременно к завершению монтажа фильма, однако книга ушла далеко вперед и работа затянулась, поэтому сначала был выложен фильм.

Пещерам на Смородинском спуске, одному из важнейших археологических памятников всей планеты, уделяется мало внимания. Почти нет печатных публикаций, кроме дореволюционных двух (Владимира Антоновича и Афанасия Роговича), да редких, поверхностных упоминаний в других источниках. Забыто даже название основной, уцелевшей пещеры, бытовавшее в 19 веке – Кирилловская пещера. Местное население и диггеры туда лазают, не ведая об ее возрасте, а некоторые даже утверждают, что сами выкопали оную.

Официально местность определяется как объект культурного наследия «Смородинский археологический комплекс IV-III тысячелетий до нашей эры; II-V веков нашей эры» (приказ Главного управления охраны культурного наследия от 05.02.2009 №10/10-09). Охраняется не то слово!

Когда рядом на горе возникло строительство коттеджей, рабочие поставили над пещерами халабуду и использовали вход в Кириллов\-скую пещеру как отхожее место. 

Непосредственно над ближним коридором пещеры проложен трамплин одной из двух в окрестностях трасс велосипедистов-даунхильщиков. В 2013 году туда вроде бы отправился инспектор от Управления охраны культурного наследия. Он счел, что трамплин не представляет угрозы для пещеры, а она сама, мол, засыпана с целью консервации – между тем как открытый вход был всего в нескольких шагах!

Внизу, под склоном с пещерой, через дорогу, в 2009 году тоже затеяли строительство – возводится «канализационный коллектор от Мостицкого до Главного городского, I очередь», чтобы гнать канализационные отходы от Нивок, Виноградаря и Пущи-Водицы в Главный городской коллектор. Начали с того, что летом 2010 года вырубили площадку с могучими, старыми деревьями. Это был плацдарм для наступления. Пригнали технику, стали вывозить грузовиками грунт.

Тем же летом, чуть раньше 13 июля, после сильного ливня по спуску на улицу Кирилловскую хлынули потоки грязи – случился оползень в «пещерном» склоне. Сошествие это совпало с другим строительством – коттеджей наверху, неподалеку от края склона и самих пещер. Вода лилась по Смородинскому два дня. Вероятно, нарушили водоносный горизонт.
 
А мокрым октябрем 2011 года, на повороте со стройкой, склон чуть южнее входа в Змиеву пещеру, напротив ворот стройки, был бодро раскопан рабочим в кепке, который вырыл от верха до низа траншею, проложил в ней толстый кабель или тонкую трубу, а после это дело закопал.

Археологи исследовали здешние пещеры несколько раз. В семидесятых годах 19 века этим занимались Афанасий Семенович Рогович и Владимир Бонифатьевич Антонович\footnote{Во сне 18 февраля 2016 года я прочитал в какой-то книжке еще об одних раскопках здесь – мол, провели их в усадьбе генерала Сухорука, исследовали пещеру, а на поверхности небольшое кладбище. Наяву ничего про «генерала Сухорука» я не разыскал. Однако, еще ранее предполагая, что местность принадлежала генералу Александру Федоровичу Багговуту, решил почитать про его жизнь подробнее. И вот оказалось, что Багговут был контужен ядром в локоть руки в сражении под Гроховым, 13 февраля 1831 года. Возможно, Сухорук – его прозвище? Позже я узнал, что местность в самом деле принадлежала Багговуту, и совместно с женой Елизаветой Дмитриевной он раскапывал здешние пещеры.}. Они же и оставили две печатные публикации, и по крайней мере единожды посетили пещеры вместе.

В 1930-е годы пещеру (но Змиеву ли?) исследовал И. Самойловский, но я не знаю подробностей.

Затем в 1981 году раскопки проводил Александр Авагян (Авакян)\footnote{Авагян А. Б., Дневник спелео-археологических исследований пещеры по Смородинскому спуску в Киеве в 1981 г., Архив отдела «Киев-подземный» Музея истории города Киева, фонд 1, дело 1. Мои попытки получить к нему доступ провалились – мне просто ничего не ответили.}, и в 1998-2001 – Тимур Бобровский. 

Итоги раскопок Авагяна – исследования древнейшей в Европе рукотворной пещеры – удостоились в печати ровно одного абзаца. А именно, в сборнике «Археологических открытий 1981 года», изданном в 1983-м, в заметке «Раскопки на Подоле» написано:

\begin{quotation}
Спелеологической группой отряда\footnote{Историко-стратиграфической отряд Киевской экспедиции.} обнаружены и раскопаны подземные галереи пещер на Смородинском спуске (Кирилловские высоты). Ходы выработаны в монолитном слое лесса. Стены 60-метрового прямоугольного в сечении хода (1,75х0,60 см.)\footnote{Мои замеры, как покажу в главе «Логово Змия», много ниже. Сам же Авагян в разговорах упоминал, что пещеру, мол, выкопали – передаю со слов его знакомой – «эльфы».} носят следы обработки прямоугольным шероховатым орудием шириной 20-25 см. Пещера в плане представляет собой широкую галерею в ответвлением, уходящим с понижением поворота винта. Шурфами пройдены замывы, в подошве которых обнаружены сильно обожженные камни серого гранита. Заложенный на поверхности раскоп (20 кв. м.) выявил культурный слой с керамикой киевского типа (II-V вв.). Интересна керамическая формочка (45х28х14 мм) с оттиснутым изображением трехлепестковой подвески и клеймом (вероятно, знак мастера) в виде двух концентрических кругов в торце. Тут же обнаружены фрагменты бронзового игольника и глиняного биконического пряслица.
\end{quotation}

Может быть, пещере уделено внимание в украинском журнале «Археология» за начало восьмидесятых годов? Напрасная надежда! 

Пещера появляется в печати только спустя несколько десятилетий. Бобровский в книге 2007 года «Підземні споруди Києва» кратко, на двух страницах, изложил историю изучения пещеры , да в интервью «Пещеры монахов-варягов до Чернигова доведут» («Газета по-киевски», 15/07/2010) рассказал, что нашел «сразу при входе» сосуд трипольской культуры, эпохи энеолита. В книге Бобровский упоминает любопытную ступеньчатость склона возле пещеры, чего я не замечал на местности, вероятно уступы сгладились осыпями грунта.

После раскопок 20 и начала 21 веков, входы в большинство пещер были засыпаны археологами – мол, раз археологам больше не надо, то и другим тоже. От «20 пещер» Антоновича в 2016 году остался вход в Кириллов\-скую-Змиеву, две мелкие пещерки, и еще одна совсем крошечная.

В этой части книги я по крупицам собрал имеющиеся сведения и привожу плоды собственных исследований. Когда впервые лез в Кирилловскую пещеру, Логово змия, я не знал, к чему приведет меня узкий вход, куда надобно проползать. Не имея никакого умственного багажа об истории этой пещеры, первым моим впечатлением было, что пещеру сделал карлик. Сейчас я остаюсь при той же мысли, но теперь обоснованно.

Далее изложу всё, что мне известно о Кирилловской пещере и что её касается. Наиболее полный свод данных о пещерах на Смородинском спуске.

Начнем с карты и описания на 2014 год, ибо я не предполагаю, что все читатели были на Смородинском спуске, а к тому же полезно для будущего закрепить настоящее. Эта часть книги намеренно «заморожена» мною указанным годом, лишь в отдельной главе я допишу, что плохого случилось с пещерой и окружающей местностью позже.
