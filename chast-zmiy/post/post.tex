\chapter{Взгляд из 2015}

И вот 6 июня 2015 года, день жаркий. Мы – это я, Зоя, Коля и Яна – идем от Лукьяши по Овручской к Смородинскому спуску и пещере. Я не был внутри два года, столько же не присматривался особо к окрестностям, хотя посещал их.

На Овручской у подъема к Нагорной спилены пополам тополя. Торчат обрубки в синее небо. Переваливаем через вершину. К кургану, снова на брусчатку, поворот, знакомая тропка вдоль обрывистого склона.

Иду, заглядываю вниз. Заповедный склон превращен в свалку. Выемка площадки у пещеры. Входное отверстие раскопано, стало выше.

Вооружаемся фонариками, лезем. Сразу ощущаю – звуки не те, запах – не тот. Увеличение входа нарушило микроклимат пещеры. Лёсс стал больше осыпаться, граффити теряют четкость. В передней комнатке, с «лежаком» у стены – следы еще большего обвала. Это стараются наверху велосипедисты!

Дальше по ходу натыкаюсь на какой-то новый раскоп в полу. Кто, зачем?

Цветы здесь больше не растут. Добираемся до шкурника. Он тоже раскопан, теперь ползти удобнее. Погубили пещеру!

Последняя камера, лежачок, где веками покоилась древняя ткань. От нее остался лишь темный след в слое сырой суглинистой кашицы, что обрушилась с потолка.
 
Больше здесь делать нечего.

Снаружи, внизу под склоном, напротив пещеры. Всё та же стройка коллектора, только вагончики строителей пусты, а сквозь щели в заборе просматривается бетонированное жерло шахты. Хорошо пещерный заповедник сохранили!

К левой малой пещере прорыт карниз. Зачем, кому она нужна? Разве чтобы лазать туда, когда Змиеву пещеру угробят окончательно?

Водоносный горизонт нарушен стройкой, грунтовые воды поднялись и, вымывая рыжий глинистый язык, из-под люка течет по булыжникам вперемежку с пятнами асфальта холодная вода. Молча стоят и смотрят на нас высокие тополя.

\vspace*{\fill}
\begin{center}
\includegraphics[width=\linewidth]{chast-zmiy/post/\myimgprefix IMG_20150606_160105.jpg}
\end{center}
\vspace*{\fill}