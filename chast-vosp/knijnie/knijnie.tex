\chapter{И магазины книжные}

Торговое место – серая металлическая рама. На прилавке книги, одна без названия на обложке, зато поверх прилеплена бумажка с надписью: «Открыть книгу 3 гривны». Рядом стоит продавец в пиксельном камуфляже.

Я на книжном базаре Петровке, ноябрьское воскресенье 2015-го, через неделю после того, как закончил третью редакцию «Ереси» и уже работаю над другой книгой. Небо сулит затяжной дождь, мне не страшно. Брожу по рядам один, наклоняюсь, достаю книжечки из коробок «Всё по десять гривен» или спрашиваю цены на неподписанные. 

Выбрался сюда внезапно, развеяться, ощутил потребность. Ощутив, в полдень, пытался вытащить с собой кого-то из друзей, звонил Коле, Алине, все оказались заняты. Желтая маршрутка отвезла меня по Московскому мосту с левого берега на правый, по пути я выглядывал налево в окно, всё пытаясь выяснить, мог ли Меньгукан «сглядати» весь Киев от Выгуровщины или нет. Я же всю осень собирался опять туда поехать и проверить вторично.

Оболонь, торгово-индустриальный простор, скука, на остановке выходят две культурные женщины с культурными детьми – обращали на себя внимание речью более правильной и сложной, нежели у окружающих. Культурный мальчик несколько раз наступил мне на ногу, зато употреблял слово «логично».

Я проехал еще немного, и – Петровка, плоская, людная, квадратный разворот транспорта, впереди вдоль забора железной дороги – книжный рынок, протянувшийся далее по Вербной улице обычной барахолкой, расставляющей сети свои на выходных. Толкучка эта даже больше, чем на субботней Куреневке, хотя они почти рядом, недолго дойти пешком.

Около пересечения Вербной с Московским проспектом барахолка раздваивается змеиным языком – один следует проспектом, другой – в проем забора к платформе станции «Зенит». И крутится там алого цвета патефон, звучит по-военному приглушенно, как в ладонь поётся, а пожилая тетенька говорит продавцу: «Мне нужны женские голоса», на что следует ответ почему-то: «Ну вот у меня тут есть Бернес». Товар здесь около рельс совсем плох, и рядом пахнет мочой, как в восьмидесятых в подворотнях на Крещатике.

Но это будет после, патефон и кручение в руках «Командирских» часов с синим циферблатом, за 150 гривен, пока же я хожу по букинистическим дебрям. Уже выторговал две брошюрки про Киев – продавщица сбросила цену с 50 гривен до 25-ти. Но из головы не выходит, как в одной коробке, где «всё по 10», я увидел синюю книгу Майкла Муркока «Повелители мечей», питерского издательства «Северо-Запад» 1991 года выпуска.

На обложке, герой этой книги, принц Корум Джайлин Ирси, стоит к зрителю лицом. Повернешь книгу – там он изображен со спины. С тем, каким он описан в книге, на картинке сходство лишь в особых приметах – закрытый повязкой глаз да рука с б\'ольшим количеством пальцем – это части тел богов Ринна и Кулла.

Точно такую книжку я купил сто лет назад за 15 кажется еще рублей, в маленьком книжном магазине на Никольской слободке, едва не в том же 1991-м. Это считалось очень дорого, 15 рублей. Я увидел тогда эту книгу, вернулся к маме, расписал восхищенно – целые три книги в одном томе! А там у Муркока деление было на небольшие книги, по несколько глав в каждой. Я прежде не читал Муркока, но меня впечатлила обложка. Книга оказалось несколько перекошена, будто пролежала под грузом, а бумага и печать походили на самиздатовские.

«Повелителей мечей» я проглотил одним махом. А недавно перечитал, и местами было уже нудно. Но я, чтобы вернуться ощущениями в прошлое, затеял перечитку всего Муркока, которого собирал в девяностых и около. Начал, конечно же, с цикла об Эрлике. Муркок отличный писатель, не просто фантаст, но дальше нескольких книг я забраться не смог и остановился, чтобы не портить впечатление совсем. 

Сюжеты состояли из набора приключений. Вы заранее знаете, что главный герой всех победит. Второстепенные могут погибнуть, на то они второй степени. А главный нет, разве что когда сочинитель решится поставить в серии точку. И то, после всегда остается лазейка – писать про время до его смерти, либо окажется, что он вовсе не умер, а так, отлежался и вернулся, и в этом суровом человеке со шрамом писатель вдруг являет нам старого знакомца, хотя это нечестно, в кино такой трюк не пройдет, мы сразу бы узнали героя.

В «Ереси» я уже повсюду разбросал книжные магазины, поэтому про описанные ранее, и далее после этой главы, здесь умолчу, дабы не повторяться. Но глава будет и про книги, не только про магазины.

С детства меня окружали книги. Ценные стояли на этажерках, прочие же – в кладовке, на высоких, под потолок, выкрашенных белым полках. Они пахли сыростью, и когда переселялись наружу, к свету, то всё равно безошибочно отличались по запаху. Некоторым поныне здравствующим уже больше шестидесяти лет, и всё пахнут кладовкой.

А она в нашей хрущовской квартире была знатная, протянувшаяся сбоку обеих комнат. В кладовку вела дверь из каждой. В спальне дверь белая, краской крашенная, а в большой комнате оклеенная обоями, да еще дверка антресолей сверху – там лежал тяжелый деревянный ящик с коллекцией открыток, и альбомы с фотографиями. На другие антресоли можно было забраться из самой кладовки, по книжным полкам – я не раз залезал туда, ко клумакам – тюкам с одеждой, рядом с лампочкой, торчащей из черного патрона. Настоящий подпольный штаб!

Среди тюков лежало несколько древних дамских сумочек. В них хранились квитанции от практически беспроцентных в Советском Союзе рассрочек и прочие документы, среди которых свидетельство о смерти моей прабабушки по материнской линии, Марии (Марфы) Семеновны Бородиной (в девичестве Черняевой), единственная ее фотография паспортного размера, да тридцатых годов бумага об устройстве на работу. Прабабушка некоторое время трудилась уборщицей в НКВД, а в бумаге была подпись начальника, тоже Бородина, и я некоторое время полагал, что это прадед. Позже я узнал, что прадед, Федор Максимович, был простым столяром. Прабабушка рассказывала домашним, что в кабинетах НКВД просто на столах, в открытую лежали кучи конфискованных драгоценностей.

Кладовка наша делилась на две проходные части. В передней стояло варенье, коробка с диапроектором, какие-то ложки-вилки, ручной миксер шестидесятых годов (в нем на моей памяти готовили только «гоголь-моголь») и чудесный электрический светильник в виде мраморного домика с белой мраморной же, как бы заснеженной крышей. При включении в его окошках горел свет. Прямо под ногами покоился большой мешок с игрушками.

Во втором отделении находились лыжи (санки зимовали на антресолях над коридором), сифоны для газировки, две клюшки, две пары коньков – черные и белые, деревянный чемодан с инструментами, висела старая одежда, и за ней, по опять же клумакам, можно было добраться до эдакой лестницы книжных полок.

Моя бабушка, Бородина Татьяна Федоровна, работала тогда на Полиграфкниге, и за долгие годы у нас, помимо покупных книг, собралось большое количество бракованных книг с «предприятия». Оттуда же были замечательные коробки, оклеенные иллюстрациями к сказкам Бажова – в этих коробках я держал солдатиков.

Еще подростком, до войны, бабушка пошла работать в типографию, что располагалась в Покровском монастыре. После войны она устроилась переплетчицей на Полиграфкнигу, чьи цеха находились тогда на нынешнем Майдане, по месту Дома Профсоюзов. Потом бабушка стала контролером – проверяла поступающие к ней по конвейеру книги на брак, а предприятие переехало на Шулявку, протянувшись почти от метро по улице Александра Довженко. Я бывал внутри – в цехах стоял страшный грохот и лязг, стоивший бабушке слуха, а в широких, высоких коридорах лежали громадные, больше меня, рулоны бумаги. Пахло клеем ПВА, свежей бумагой и краской. Отдельно пачками лежали обложки.

Когда я был совсем маленьким, то представлял, будто на предприятии печатают деньги, и можно найти их, оставленными вот таким рулоном, да потихоньку выкатить его куда-то во двор. Подобные фантастические мечты посещали меня и в других вариантах, с небольшим разнообразием. Например, самолет перевозит деньги, случается катастрофа, летчики выпрыгивают с парашютами, а контейнер с деньгами летит вниз и падает прямо передо мною. Либо, инкассаторы перевозят – непременно на моей родной улице Бастионной – деньги, задняя дверца сама собой открывается, деньги выпадают, в машине ничего не замечают и едут дальше. А тут я случайно прогуливаюсь мимо. Что это лежит на дороге? Батюшки светы, пачки рублей! Чьи? Не знаю, пусто, прохожих нет. Что же, полезайте денежки в карманы и авоську, с которой я пересекал улицу, чтобы спуститься в яр, где прятался хлебный магазин.

Но выкатывание рулона денег с «Полиграфкниги» во двор занимало высшее место в моих планах быстрого обогащения.

С предприятия нам иногда доставались бракованные книги, лишенные обложек. Некоторые книги давались работникам и работницам фабрики в подарок, на них ставился овальный синий штамп «Продаже не подлежит».

Когда я поступил в первый класс школы (еще не 133-й, туда я перевелся чуть позже), классная руководительница чего-то решила, будто бабушка, работая на Полиграфкниге, может способствовать «подписаться» на Дюма. Про подписку на книги я расскажу ниже. Труд на предприятии ничуть не давал доступ к «подписке», мы пояснили это училке, но она решила, что мы просто не хотим для нее постараться и обиделась. Что значит учитель, обиженный на ученика? А потом я увидел, как она одноклассника хватает за волосы и чуть не лупит головой об парту. Кроме того, за одной партой со мной сидел весьма неприятный чувак, так что вскоре я из школы математической перевелся в обычную и в пяти минутах ходьбы от дома.

Книги из кладовки привлекли мое внимание поздно, лет в 14, тогда часть их перекочевала наружу, а часть внешних книг отправилась в темноту.

Наверное лет с десяти я был одержим бюрократическим увлечением под именем «Наша библиотека» (НБ), которую сам изобрел и вовлекал в нее окружающих. 

Началось с того, что я взял синюю общую тетрадь в клетку и составил каталог нашей библиотеки, присвоив каждой книге порядковый номер и жанр. А в отдельной, тонкой тетрадке стал вести учет, кто какую книгу берет читать, что-то там подклеивал, присобачивал скрепками. Понадобилась еще одна тетрадь, возник штат должностей с копеечными зарплатами. 

Я посылал из тетради в тетрадь записки, регистрировал их прохождение, составлял приказы, оканчивающиеся так – «Утверждено и подписано, П. Семилетов». На новых книгах рисовал логотип библиотеки – «НБ» в кружке, и ставил свою роспись, как главного библиотекаря. Эта привычка на много лет пережила развернутую мною бюрократическую сеть. А сеть во время расцвета помещалась в двух папках из лигнина. Я раскладывал их содержимое на диване и углублялся в бумажную работу. Подобной фигней страдал в детстве и Станислав Лем, как он описал в своем «Высоком замке», о чем я тогда не знал.

Первой книгой, которую я прочел лично, был «Робинзон Крузо». Это было перед поступлением в школу, летом. Мы около месяца провели на ведомственной, от той же Полиграфкниги, базе отдыха «Бережок». Крутой обрыв над Десной, сосновый лес, ковер из светло-рыжей хвои, домики на сваях. С высокого правого берега видно сонный низменный левый, туманный поутру. Непривычная, глухая после Киева тишина. 

По судоходной тогда Десне ходили баржи и «ракеты». Когда проплывала баржа, то вода сначала отступала, как бы втягивалась, а потом вдоль берега сметающей рукой шел высоченный бурлящий вал, за которым вода восстанавливалась в прежнем уровне. С этим валом я бежал наперегонки. Волны от «ракеты» были иными, тоже высокими, но меньше и параллельными относительно берега. Моторные лодки не давали таких огромных волн. «Ракета» проносилась мимо в восемь вечера, ее все ждали, и я строил к тому времени песчаную, укрепленную палочками, крепость, так близко к воде, чтобы волны «ракеты» достали до стен. Но сможет ли стихия разрушить их?

В Десну, по границе села Рудни, вливался глубокий ручей, через темные воды коего перекинута была кладка из камней и досок. Теперь я знаю, что ручей этот – речка Меша, берущая начало где-то за Моровском\footnote{В «Списках населенных пунктов Черниговской губернии» 1865 года записана казенная деревня Рудня Остёрского уезда, «при болоте Меше», 152 двора, 416 мужчин, 210 женщин, часовня православная. Меша там же всюду упоминается как болото, при Меше расположены и другие деревни – Стукачи, Новоугринская Гута, Отрохи и другие. Любопытно, что Рудня стоит не «при Десне», но «при Меше». Рудня именно та, между Коропьем и Моровском, а последние указаны «при Десне».}. Неподалеку устья ее была отмель, ныне превратившаяся в остров. Сюда, напротив отмели, водили к водопою коров. В сторону лагеря берег поднимался суглинно-песочной стеной с ласточкиными норами. С этого берега меня сбросил в воду какой-то чувак, после чего я, выплыв, стал заикаться.

На 2016 год, от прежней базы отдыха «Бережок» осталась часть – когда пляж перестали намывать, вода подобралась под обрыв и теперь размывает его, обрушивая вниз домики и прочие строения. Я не видел этого разрушения, мне рассказывали. При СССР мы неоднократно ездили на «Бережок» по путевкам, которые брала на работе бабушка. 

В урочный день, рано, при слабом еще свете солнца, с предприятия прямо к дому подкатывал небольшой автобус. Собрав так по городу всех желающих отдохнуть, он ехал через дамбу на Киевском море по трассе на Чернигов, а Рудня там между Коропьем и Моровском. Если добираться на «Бережок» своим ходом, от Рудни надо было пешком шагать к дальнему бору полевой дорогой. На околице села помню ветхую, косую хату, как с картин передвижников, и гнездо аиста на сухой черной груше.

В последний раз я был там наездом, кажется, в конце девяностых. Мы с бабушкой пошли в лес по грибы, забрели на старое, спрятанное среди сосен кладбище.

Я много отвлекся. «Робинзона Крузо» в переводе Маршака я прочитал на белом песчаном пляже под обрывом. Из склона торчали корни, я хватался за них и лез наверх, покуда было возможно не сорваться. Не то, чтобы я прежде не умел читать, хотя не помню, как научился этому. Но именно самостоятельно читать мне не хотелось, я больше смотрел картинки.

Была у нас, да и сейчас есть, здоровенная, в красной обложке «Книга будущих командиров» – я любил ее рассматривать, там войска древних полководцев вроде Дария нелепо стояли друг против друга, и я мысленно играл в сражения между ними. Еще любил картинки из «Жизни на Земле», с изображениями кипящей лавы, первобытных папоротников и динозавров. Особенно мне нравились трицератопсы. Выпускались еще такие игрушки с ходящими ножками – трицератопс и черепаха. Они сами ходили по наклонной плоскости.

Классные картинки были в книге про компьютеры и кибернетику – «Быстрее мысли». Старая, шестидесятых годов, она поразила меня иллюстрацией, где некий искусственный разум восстал против «обезумевшего клерка».

Прежде чем продолжить, кратко обрисую для тех, кто не знает, государственное книгоиздание и продажу книг в Советском Союзе, каким я его помню в восьмидесятые годы и начале девяностых.

В каждом районе города, да и в каждом крупном селе был государственный книжный магазин. Выбор общедоступных книг в них казался мне скучным, как воспринимался тогда и будничного репертуар телевидения. Сейчас я оцениваю их иначе.

Почти все книги выпускали тиражами в десятки, сотни тысяч, случались и миллионы. Было много научной литературы, а среди художественной не преобладал какой-либо жанр, но ценились детективы и фантастика. Из зарубежных детективщиков у нас печатали произведения Кристи, Сименона и Конан Дойла, а фантастов – Брэдбери, Азимов, Кларк, да рассказы в разных сборниках и журналах. Когда «Вокруг света» печатал в конце номера что-нибудь Хайнлайна, журнал начинали читать с конца.

Большая часть книг поступала в свободную продажу через книжные магазины, но были более труднодоступные книги – подписные издания и «за макулатуру». Сдаешь макулатуру – получаешь талончик на покупку, скажем, томика Джека Лондона, Агаты Кристи или Пикуля. Весомо звучало – Пикуль. Еще значимость придавали сочинителям, прослывшим опальными – Ахматовой, Мандельштаму, Цветаевой, Пастернаку.

Подписные издания были многотомными, выходили  тиражами, где счет велся на сотни тысяч, а то и миллионы. Например, двухтомник Лермонтова шел тиражом 14 миллионов. В масштабах СССР и это казалось каплей в море, поэтому «подписка» распределялась между гражданами определенным образом, как благо, и затем по мере выхода очередных томов обладатели подписки получали уведомление прийти в магазин и выкупить том.

Поскольку всюду, где существует ограничение в доступе, появляются люди, способные варить на этом деньги, на подобных книгах расцвела спекуляция. Министр внутренних дел Николай Щелоков в 1975 году писал в записке для ЦК КПСС:

\begin{quotation}
За последнее время в ряде городов получила распространение спекуляция книгами, пользующимися повышенным спросом у населения. В основном это литература мемуарная, детская, приключенческого и детективного жанра, научная фантастика, а также иные книги популярных авторов. Указанную литературу спекулянты приобретают у работников книжных магазинов, складов и баз книготорговой сети, получают по подписке, покупают у граждан, а затем перепродают по повышенным ценам. Например, книга А. П. Керн «Воспоминания» перепродается по 25-30 рублей за экземпляр при номинальной цене 86 коп., роман С. и А. Галон «Анжелика» при цене 2 руб. 02 коп. – по 40-50 рублей, а однотомник М. Булгакова при цене 1 руб. 53 коп. – по 75-80 рублей. 
\end{quotation}

А 80 рублей – многие получали такие зарплаты.

Любимейшей книгой детства, кроме произведений Джералда Даррела, у меня была трилогия «Тореадоры из Васюкивки» Всеволода Нестайко – вначале мне читала ее мама, позже я сам бесчисленно перечитывал. После распада Союза, Нестайко переделал ее, «осовременил», чем существенно меня огорчил, ну да остается старая редакция.

Главные герои – школьники Ява и Павлуша из села Васюковки. В одной из частей «Тореадоров» они попадают в Киев, и подробно описывается город шестидесятых годов, склоны Днепра около Лавры и Аскольдовки, тамошняя дренажная система. Писатель «поселил» товарищей у родичей Павлуши, в дом около Печерского моста, и я подозревал в этом доме то номер первый на Бастионной, где жили мои друзья Оля и Костик, то дом напротив, на Дружбы Народов 2/34, где был огромный продуктовый магазин «Темп».

Еще одной из моих священных книг был «Остров Тамбукту» Марко Марчевского, издания конца пятидесятых, в обложке с пляшущими неграми в дикарских нарядах. Книжку мне подарил папа, рассказав историю ее обретения – что книгу ему в юношестве подарил некий бродяга, скрывавшийся на чердаке.

Сие придавало книге то самое дополнительное значение, какое имеют многие книги помимо своего содержания – такие как «Наследник из Калькутты» Штильмарка или «Мастер и Маргарита» Булгакова. У тома, что держишь в руках, появляется прошлое, и это невольно накладывает отпечаток на ощущение от книги. 

В детстве я перечитывал «Остров Тамбукту» много раз, а теперь не могу вспомнить, о чем он – на ум приходит только предложение, где туземец говорит белому Антону на своем языке: «Андо! Пакеги-гена!». И примерно знаю концовку. Снова открыть книгу боюсь – боюсь разочарования.

%Чтобы настроить себя к написанию этой главы, я решил перечитать кое-какие книги, нравящиеся мне,  когда запоем читал зарубежную фантастику. Начал с Муркока, и уже к концу «Хроник Корума» забуксовал, но положительное восприятие из прошлого помогло эту книгу проглотить. Тогда я взялся за сериал Муркока про Элрика, и на каком-то из томов отложил, чтобы не портить впечатление. Всё то же самое – отлично сохранившаяся книга, замечательное издание от «Северо-запада» – с чуток шероховатой обложкой, понюхаешь и пахнет типографией. Тот же текст, да что не так? Почему прежде нравилось? 

%За годы я перестал читать художественную литературу в переводах. Перевод это пересказ. Когда вы полагаете, что читали Диккенса или Стивена Кинга, на самом деле вы их не читали. Вы знакомы с подробным пересказом.

%Ладно, я мог бы достать Муркока в подлиннике, да не захотел, поленился. Муркок сам по себе отличный писатель, не узко фантаст, и в определенное время он наконец стал писать произведения иные по устройству, а не обычные для него прежде, где герой (с отличительными особенностями вроде камня во лбу) в очередной книге очередного цикла переживает ряд приключений в рамках ровной и единственной сюжетной линии. Вроде как в компьютерной игре-стрелялке – вы идете, на вас нападают полчища врагов, вы отстреливаете их, идете дальше, на вас нападают другие враги, вы снова их отстреливаете, снова идете дальше. «Идете дальше» разнообразится перебросками между мирами, поездками на дивных скакунах, полетами на чудесных летунах – это непременная составляющая сказочной фантастики, и Муркок пишет ее лучше, чем другие. Но мне такое скучно читать, тем более что я знаю – Муркок способен на серьезные вещи вроде «Города в осенних звездах».

%С этим произведением по настроению перекликается  

Еще одна книга, особо ценимая мною в детстве – «Похождения авантюриста Гуго фон Хабенихта» венгерского писатели Мора Йокаи. Мне подарила ее двоюродная сестра папы, Лина, когда приехала в Киев кажется из Тюмени. Черная, изданная в венгерской же «Корвине» книжка в супероболожке, где изображен рыжеволосый чувак в старинном костюме, поджигающий у пушки запал.

Таких книг я прежде не читал. Чего там только не  было! Герой попадал то к разбойникам, жившим в пещере, то к поклонникам Бафомета, то сражался в еврейском квартале с чудовищем Мулькабом. Мор Йокаи – сильный сочинитель, за внешней яркостью на время забываешь, что с самого начала говорится о гибели Гуго, и я был словно придавлен в конце его казнью.

Позже, я приобрел еще одну книгу Мора Йокаи, «Когда мы состаримся», по странной случайности тоже в черной обложке, с двумя золочеными пистолетами. Так и не дочитал ее до конца.
 
В конце восьмидесятых спокойное советское море книг стало разбавляться кооперативными течениями. 

До этого интересные с моей точки зрения книги большей частью «доставались», а не покупались с прилавка книжного магазина, хотя были исключения. Десятилетиями ценились книги, выпущенные Детгизом во вкусной серии «Библиотека научной фантастики и приключений», неуловимо выходила «Библиотека советской фантастики» в знаковых серых обложках, да в мягких «Мир приключений». Огромные по нынешним временам тиражи растворялись в еще более огромном читающем населении СССР. И помимо «подписки» и сдачи макулатуре, ценные книги можно было либо выменять на тоже ценные в отделе обмена, либо купить на руках.

Помню, как мы с мамой поехали весной, когда еще и ранний снег лежал, и на деревьях с почек свисали продолговатые капельки, а земля пахла сыростью – в какие-то дворы, за тридевять земель. Некая женщина продала нам «Энциклопедический словарь юного биолога» – здоровенную зеленую книгу.

Был еще самиздат. На пару дней в Киев из Москвы привезли отпечатанные на машинке, в зеленых обложках переплетенные, два тома «Девяти принцев Амбера» Желязны. Папа ухватил почитать и принес мне. Надо было уложиться в пару дней, потом книги отправлялись обратно в Москву. Уложились, прочли.

В ближайшем ко мне книжном на Бастионной я надыбал в букинистическом отделе кустарное «Собачье сердце» в «обложке работы Ю. П. Анненкова».

Вообще в любом книжном меня всегда интересовали больше именно букинистические отделы. Это как за грибами ходить. Иногда попадалась хорошая книжка в хорошем состоянии. Порой вообще замечательная – под стать тем, что были в отделе обмена – но зачитанная до дыр. А в отделе обмена стояли Дюма, Буссенары, Майн Риды, Ахматовы и Мандельштамы, и сосредоточенно сиял Зигмунд Фрейд, коего мама выменяла на что-то свое и подарила ко дню рождения своей подруге Лене.

Целиком букинистический магазин был на Мельникова в жилом квартале, ныне снесенном – его место заняла станция метро Лукьяновская, а раньше троллейбусы с этой стороны останавливались прямо около мрачноватых жилых домов. На первом этаже одного располагался магазин с вывеской «ОБМЕН КНИГ».

Обычные книжные помню где? Про магазин на Бастионной расскажу позже. Громадные, длинные внутри, находился в доме около остановки «Полет» на проспекте Воссоединения – теперь в нем уместилось несколько банков и военторг. Подобный же магазин был на площади Победы, по адресу проспект Победы, 12. Небольшой, похожий на прачечную книжный на Никольской слободке – вот глава и насыщается ценнейшими краеведческими сведениями.

А потом как-то всё переменилось. По радио, телевизору и в газетах замельтешили два новых слова – перестройка, гласность. Или даже так, с акцентом – пэрэстуойка, глаасност. 

Стали продаваться какие-то самопальные грузинские кепки с сетками по бокам. Не беда, что разлезались, зато считались чуть не американскими. Затрещали автоматными очередями, захлопали звуками кунгфушных ударов видеотеки. 

И случилось доселе невиданное – под солнцем и в сумраке метро появились столы с книгами. Поначалу там лежали вполне государственные издания из серий вроде «Мир приключений», или бойкие небольшие томики американской фантастики от издательства «Мир».

Вдруг оказалось, что в Союзе таки издается столь потребная народу фантастика. Саймак, Хайнлайн, Силверберг. Но хотя на заднике каждой книги стояла государственная цена – скажем, два с чем-то или трояк, спереди была прикреплена бумажка, или на внутренней стороне обложки написана цена рыночная. Десять, пятнадцать рублей. Нехило, при цене булочки в три копейки.

И сразу полезли-полезли, конкурируя с издательствами вроде «Мир», книжки от мелких новых контор, запестрели именами – Бэрроуз, Говард. Странным образом издавался и продавался Толкиен – никогда не видел все тома его «Властелина конец» вместе, пока не подарила бабушка, а то всё врозь покупал, в разных переводах.

Было какое-то время, когда новые издательства только нарождались, и люди раскупали что угодно, лишь бы под соусом фантастики. 

Помню, как ехал на отдых в поезде, и всю ночь читал две карманного формата брошюры, которые дома разрезал (продавались огромными листами) и скрепил. Серия называлась «Фантакрим-микро», была еще «Фантакрим-макро». «Фантакрим» значило – фантастика и криминал. 

Некоторые еще советские журналы пустились тогда на выпуск подобных брошюр, печатая частями старые приключенческие романы известных по середине века сочинителей – например Буссенара – с тем, чтобы читатели потом это дело сами разрезали и сшивали. Однако Буссенар и Майн Рид не выдержали соперничество с теми тоннами фантастики и детективов, которые хлынули в умирающий под пластмассовый стук драм-машин Союз.

На улице Ленина – ныне Хмельницкого, внизу, на почти углу, напротив ЦУМа, рядом с окном пирожковой открылось другое окно – книжное. Это год наверное 1992-й или раньше, потому что я стал вегетарианцем в 16 лет, в 1993-м, но в пирожковой мы с мамой покупали пирожки с сосисками. Я приезжал к маме на работу в театр, мы спускались, брали по пирожку, а если были лишние и не очень деньги, то какую-то книжку. В окне том книжном выставлялись детективы Чейза, Рекса Стаута. Подготовленные молвой, ухватили там «Казино Роял» Флеминга, про Джеймса Бонда – оказалось серо и скучно. Чейз прибалтийского кооперативного издания 1989 года стоил 10 рублей. Эта торговая точка в моей памяти совмещена со следующей, поэтому сказанное справедливо для обеих.

Когда появилась станция метро «Ленинская» – позже «Театральная» – около выходных дверей со станции, но в переходе, а потом внутри, напротив кассы – тоже была раскладка с книгами. Там мама купила мне к Новому году подарок – «Дракулу» Брэма Стоукера, и за его же подписью «Скорбь сатаны», хотя на деле ее написала Мария Корелли. Сразу же, ночью, под запах хвои с неуловимым отзвуком стекла елочных игрушек, я принялся за «Дракулу»! Ох как здорово впервые читать интересные книги!

Сейчас меня редко какая книга пронимает из тех, что читаю впервые. Как и музыка. Всё труднее удивить.

Кроме новогоднего подарка, особенно запомнилась мне покупка романа «Тигр, тигр» Алфреда Бестера. Не\-давно перечитал его на одном дыхании – отлично! Мы с мамой приобрели ее на книжной раскладке внутри станции метро «Арсенальная», как уже идешь к выходу, наверху, но еще в помещении. Один из первых компьютерных наборов, шрифт почти такой, как на пишущих машинках. На обложке человек с красной татуировкой «Номад» поперек лба.

Другое воспоминание того же времени. Был на бульваре Леси Украинки, кажется в доме номер 20/22, магазин «Военная книга». В стеклянной витрине там стояли разные картины, глобусы. Наверное, в давнее время там торговали сплошь книгами про войну, однако когда мы зачастили туда с отцом, привлекал нас отдел кооперативных книг, по сути коммерческая раскладка внутри государственного магазина. 

Такие раскладки появились во многих, например на Бастионной, и как-то мама увидела там несколько частей «Тарзана» – а «Тарзан» в СССР был трофейным фильмом, овеянным славой с пятидесятых годов. Той же славой воссияли и книги Бэрроуза про сего героя. Тарзан! Мы купили несколько томиков, вернулись домой, одолжили у соседей деньги, и я побежал докупать оставшиеся. Вот как раньше мы тянулись к чтению!

Но вернусь к «Военной книге». Заходим туда с папой, глядим в шкаф за стеклом, что же появилось интересного. Рядом крутится темноволосый, лысоватый мужчина. Разговорились конечно же про фантастику. И чувак поразил нас ударением в фамилии Саймак, на втором слоге. Мы потом когда покинули магазин, долго еще и возмущались, и презирали. Как, мол, Сайм\'ак, если надо С\'аймак!
 
Прямо напротив магазина, через дорогу, там где теперь огромные здания, рядом с десятиэтажкой номер 9 был зеленый пригорок, и левее от него, на север, темной зеленью кленов глубинел овраг Клова с долгой лестницей, по которой можно было в конце концов выйти к дворам на дне, и на улицу Мечникова.

В детстве мы с бабушкой много гуляли по краю яра и называли его «Клов», а потом я попал в эти места в 2004 году, когда он только начал застраиваться. Я шел по стройке с пленочным фотоаппаратом и неумело фотографировал, потом при проявке всё стало смазанным, но это не имело значения, ибо мои воспоминания о Клове остались четкими, а от застроенного Клова не осталось уже той темной кленовой зелени и земляных склонов.

Книжный магазин «Сяйво» на Красноармейской возле Бессарабки почти не помню. В девяностых я иногда бывал в «Знання» на Крещатике 44, он занимал два этажа, на втором продавалась компьютерная литература. Дальше по Крещатику работал магазин «Мистецтво», где над входом разросся дикий виноград. Я туда заходил кажется только раз в жизни. Помню еще большой «Книжковий світ» у площади Победы, на Дмитриевской 2 – он занимал чуть ли не весь первый этаж.

Конечно, в Киеве было куда больше магазинов, чем я описал, но ведь и я не во всех побывал, а про некоторые я говорю в других главах.

Почти вся советская сеть книжных со временем рухнула, заменилась рынком на Петровке. Поначалу ездили туда с азартом, как в лес по грибы. Каждое посещение Петровки ощущалось путешествием в чудесную страну, откуда возвращался с новыми мирами в рюкзаке. Потом там стали продавать еще музыку на дисках, фильмы, игры. С распространением быстрого интернета всё это потеряло смысл. А цены на книги росли, а книги под разными обложками становились одинаковыми. Сейчас езжу туда иногда, роюсь в букинистических раскладках, ищу новое для себя старое.

Под конец приведу выписки о послесоветских ценах из моих старых дневников.

3.2.1993, монография «Украинцы: народные верования, поверья, демонология» – 132 купона, приобретена в магазине на улице Московской, 15 (дом напротив одной из проходных Арсенала, через улицу).

1993 год, польские приключенческие книжки про Томека, на мерзкой бумаге, 200 купонов за 3 тома.

1994 год, «Необходимые вещи» Стивена Кинга, раскладка на Дружбы Народов, подарок на 6 мая от мамы. 49000 карбованцев. Тогда же, дискета на 5,25 дюйма с игрой «Electrobody» стоила 23000 карбованцев.

11 декабря 1997, «Нейромансер» Уильяма Гибсона, куплен на Петровке, 4 гривны.
