\chapter{Видеотеки}

Я считал, что видеотеки – явление, присущее лишь закатному СССР и странам, оставшимся после его распада, но в каком-то индийском фильме девяностых годов увидел точно такой, как у нас, видеозал с телевизором и рядами стульев, и крутили там тоже американский боевик.

Вот и не знаю, что думать – то ли видеотеки были придуманы нарочно, дабы нести миру американские ценности, то ли это само собой получилось. Лишь посмотрев уйму «видеотечных» фильмов я стал ценить советское кино. А тогда, в конце восьмидесятых, девяностые, все хотели смотреть что-то новое, не важно, какую мысль оно несло.

Два меня пишут эту главу – современный и тот юный, девяностых годов. Тогда мне нравились видеотеки и репертуар их, теперь я понимаю, что многие из фильмов действовали на сознание разлагающе, навязывали в качестве положительных героев убийц и мошенников.

Конечно, показывали и хорошие фильмы. Проявили себя замечательные переводчики, мастера дела – Михалёв, Горчаков, Володарский, Гаврилов и многие другие, чьи имена мы узнали много позже, а переводы стали собирать уже в цифровом виде, ибо таких более не делают. В девяностых отличали не по фамилии, а по голосу – тот культурный, а тот гнусавый – наверное, с прищепкой на носу, чтобы изменить голос, для конспирации.

Для обустройства видеотеки требовалось немного – цветной телевизор, видеомагнитофон, возможность доставать кассеты с фильмами, и помещение. Под помещение годилась любая комната, поэтому видеотеки возникали повсюду – в гостиницах, научно-исследовательских институтах, подвалах, при кинотеатрах, городках аттракционов, зоопарках, словом везде.

Сколько стоил вход, я не помню. Кажется, две гривны. Поэтому меня удивило, что когда мы поехали отдыхать под Луганск в дом отдыха «Зеленую Рощу», там сеанс местной видеотеки в помещении рядом со столовой обходился в 50 копеек, и даже вручался билет, похожий на транспортный. А в Киеве никаких билетов не давали, просто стоял дядька на входе и собирал деньги. Я привез из «Зеленой рощи» целую пачку видеотечных билетов, поныне храню.

Каждому видеозалу сопутствовали афиши. Обычно их писали от руки и расклеивали по округе. Указывался  репертуар на ближайшие дни, а ниже – адрес. В день показывали, наверное, около четырех-пяти фильмов. В афише сообщали время, название, и жанр, например «кунг-фу», «боевик», «фантастика», «эротика», «ужасы». 

Видеотек было так много, что мы с отцом, выбравшись в город под вечер, не знали заранее, на какой фильм в какой видеотеке попадаем. Не в одну, так в другую. Пешком дойти можно! До сеанса оставалось обычно минут 20. Заходим, там комната с жопным светом, который потом выключается. Или актовый зал. Все рассаживаются на свободные места. Поэтому хорошо прийти заранее и выбрать удобное. Если телевизор стоит высоко на какой-нибудь тумбе, лучше отсесть, иначе придется голову задирать.

Промежутки между фильмами заполнялись мультиками. Над ними дружно смеялись и взрослые, и дети. Обычно крутили «Том и Джерри» и другие старые американские. Мне «Том и Джерри» быстро приелись и больше нравились от Warner Brothers, про Бакса Банни, Даффи Дака, Порки Пига, Роадраннера и Койота. Были еще нудные диснеевские, но не сериалы, как потом по телеку показывали «Утиные истории» и «Talespin», а отдельные, про Микки Мауса, Дональда.

Раньше, в Союзе, о диснеевских мультфильмах, которые мало кто видел, все говорили с придыханием. Мол, какие-то сверхмультфильмы. Чудо! Чудо! И вот мы с мамой были в зоопарке, там в шапито действовала видеотека. Как раз это «чудо» и показывали. Мы с братом Сашей (вообще он мой двоюродный брат, но удобно называть его просто братом) взяли у наших мам деньги и пошли смотреть. А мамы нас ждали снаружи. Ибо было дорого, чтобы они тоже смотрели.

И я увидел плавного Микки Мауса с его масляным тонким голосом. Это невозможно выдержать. Я всё ждал, что будет хотя бы смешно. Но нет, плясал перед экраном пресный Микки Маус и не происходило ничего волшебного, к чему я был подготовлен молвой.

Ближайшая ко мне видеотека располагалась на втором этаже кинотеатра «Слава» на Бастионной, я про него расскажу позже. Больше в округе видеотек не было, как и кинотеатров. Когда появился на Печерске, ближе к Арсеналу, кинотеатр «Звездный», мы ездили туда, а раньше, до его появления, в «Краков» на Русановке. 

В «Кракове» мне запомнились не названия фильмов, а кофе-глясе (мы говорили «кофе-глисе»), что продавалось там в буфете. Холодное кофе с мороженым. Зато в «Звездном», который из-за новизны пахнул сырым бетоном, стояли игровые автоматы. Среди прочих – такой стеклянный аквариум с манипулятором и разной всячиной внутри. Теперь такую иногда выбрасывают. А тогда это были сокровища! Иностранные пластмассовые солдатики, жвачки, конфетки в ярких пачечках – вроде тех, что можно было выиграть в Лунапарке.

Да, в Киев раз в году приезжал Лунапарк, то ли польский, то ли чешский. Передвижные аттракционы. Лагерь свой он разбивал возле Республиканского стадиона, сейчас это Олимпийский. Там был тир. Стреляешь – я делал это очень метко – получаешь призы. Обычно я выигрывал пачечки фруктовых прямоугольных конфеток, но однажды выбил фиолетового пластмассового дракона!

В Лунапарке мне еще нравились невысокие американские горки. Тоже страшно, но не так, как в Гидропарке.

У нас в Гидропарке был постоянный городок аттракционов, гораздо дешевле Лунапарка. Пахнущие горелым электричеством, сыплющие от штанг искорками сталкивающиеся машины с резиновыми буферами. Такая едет, пока жмешь на правую педаль. Потом, качели-лодочки. Карусель с жесткими креслами на длинных цепях. Кресла почти как в старых трамваях, по форме тела. Игровые автоматы – особенно морской бой, где приникаешь лицом к перископу и видишь в зеленоватом пространстве корабли, а руки лежат на рукоятках, а под большим пальцем кнопка запуска торпеды, и носом чуешь резину. 

И жуткие американские горки, которые я выдержал дважды. Не такие замысловатые, как в Лунапарке, но выше и быстрее.

Поднимаешься на вышку, от вышки отходит синусоидой выгнутый деревянный желоб, изгибающийся затем в сторону, в тоннель с низкой бетонной аркой. Наверху уже ждут крепкие дядьки. Они заведуют вагонетками. Садишься в одну, пристегиваешься. Дядька вагонетку хватает, на себя потянул, потом от себя толкнул, и ты с грохотом летишь в ней по желобу. На пиках скорость замирает. И снова всё обрушивается. А на подходе к арке кажется, что начисто отобьет голову – и пригибаешься.

Тогда мы с мамой решались, катались. Сейчас я не отважился бы без весомой причины. Но и негде отваживаться – всё давно разобрали. Шашлычная культура поглотила Гидропарк.

Закончу о Лунапарке и продолжу про видеотеки. В Лунапарке был аттракцион, привлекавший меня особенно. Комната страхаааа. Павильон, внешне размалеванный всякими ужасами. Внутри – пахнущие мазутом гремящие вагонетки проносились по лабиринту, временами притормаживая около пугающих чучел, ловко подсвеченных, иногда механизированных. Там был тонкогубый бледный вампир в черном костюме и плаще. Я считал, что вампиры пьют у людей кровь через соломинку, как кофе-глясе в «Кракове». Наверное, протыкают соломинкой шею и пьют.

А потом комната страха меня обидела, я расплакался. Устроителям Лунапарка взбрело в голову запустить в лабиринт живого человека, молодого придурка. Он выскакивал сзади и касался людей. А мне отвесил подзатыльник. Я перепугался, потом разозлился, и больше в эту сволочную комнату страха не ходил.

Зато стал устраивать ее дома, в спальне, да еще продавал туда билеты за копейки, составляя таким образом капитал. На подготовку комнаты страха у нас с братом уходило, наверное, около часа. Мы делали из подручных средств оторванные головы, отрезанные руки, ну и сами наряжались, хрипели, сипели и всячески устрашали.

Видеотеки! Помню в парке... ох как же он назывался? Был, да и сейчас вроде его остатки существуют за зданием Министерства Здравоохранения, на подходе к Марьинскому дворцу. Там был отгороженный от всего детский парк аттракционов, с электрическим паровозиком, качелями-лодочками, каруселью в виде Сатурна с кольцом. В кольце сверху проделаны дырки, а там внутри сиденья. Садишься и крутишься. Хорошая карусель, мне очень нравилась!

И вот там в сарайчике открыли видеотеку. Мы с папой туда один раз зашли на фильм «Смертельный крик Черной Пантеры». Производство Гонконг. Там положительный герой сражался против приспешников преступного авторитета по кличке Черная Пантера. В конце, Черной Пантере сломали ногу, он полз и ныл – вернее, переводчик за него ныл: «Ой ногааа, ой ножка!». Это и был смертельный крик Черной Пантеры.

Другая видеотека в тех краях окопалась в подземном переходе перед пушкой и станцией метро «Арсенальная». Я проходил мимо и за стеклом помещения этой видеотеки, через щель в шторах заметил телевизор с врезавшейся мне в память сценой. В кадре был прикованный наручниками лысый чувак, у него изо лба извивалась змейка. Потом еще ездили «скорые помощи», этот мутант убивал врачей, но я спешил, да и что за кайф смотреть тайно, без звука?

Спустя пару лет я попал на этот же фильм – «Извне» Брайана Юзны, с Джеффри Комбсом в главной роли – в видеотеке дома отдыха «Зеленая Роща» под Луганском. На сей раз с братом. Мы сели в первом ряду, стали глядеть. 

Немного прошло времени, как брат сказал, что ему страшно – а на экране потусторонние существа вроде призрачных глистов летали и нападали на участников эксперимента. Мне тоже было страшно, и мы с братом прожогом выскочили из видеотеки на сумерки летнего вечера, скоренько вернулись в корпус, в номер, и потом долго всем рассказывали, что более страшного фильма еще не видели.

Десятилетия миновали, я таки посмотрел «Извне» целиком на компьютере, дважды. Есть в этой картине некая изюминка, как в первой и третьей частях «Экстро» (Xtro) – тоже запомнившийся мне фильм.

В какие еще я ходил видеотеки? В Планетарий, на первом этаже, на «Звездные войны». В хозяйственном блоке на Троещине, улице Маяковского – там мы с мамой смотрели «Муха» и еще что-то. В березняковском кинотеатре «Старт» на втором этаже тоже работала видеотека, там я с ужасом – буквально – глядел «Хищник». 

Я в то время очень погружался в происходящее на экране, как в своеобразную действительность, и различные ужасы на экране действительно вызывали во мне ужас. Помню, рядом на «Хищнике» сидел двоюродный брат Роман, и я в особо жутких сценах закрывал глаза и просил Романа сказать, когда это закончится. Сейчас, наверное, фильмы так не смотрят, не знаю – сто лет в кинотеатре не бывал. Но тогда не я один так близко воспринимал кино.

Понемногу появлялись проекторы – в Доме Офицеров завелась видеотека с оным. Мы с мамой пошли туда на сеанс «Космической Одиссеи» Кубрика, оцененной нами много позже. С проектора крутили фильмы позже в киноклубе при Могилянке (в помещении Бурсы) и в Киноклубе социального кино на территории Лавры, а также в культурном центре имени Леся Курбаса на Владимирской улице.

На улице Мечникова (бывшей Собачьей тропе) было две видеотеки. Одну, большую, мы называли Конюшней, и посещали ее часто. Это в доме номер 14, где в 2015 году – булочная. А другую открыли чуть дальше, вроде в 16-А.

Как-то с отцом не попали в Конюшню, а тут эта другая подвернулась. Поднимаемся на какой-то этаж. Должны показывать «Ад живых мертвецов» 1980 года, итальянский. Еще прежний сеанс не кончился, стоим на лестнице ждем. Там дядька, который этим заведовал. Небось научный сотрудник. Папа его спрашивает, мол, ну что за фильм этот «Ад живых мертвецов»? Заведующий отвечает – о, таких ужасов вы еще не смотрели. Папа ему замечает, что после «Кошмара на улице вязов», кажется, куда уж дальше... А заведующий, многозначительно кивает – сами убедитесь.

Все эти фильмы, хорошие ли, плохие, запомнились мне и спустя много лет я находил их в сети и пересматривал, именно в тех одноголосых любительских, высокого класса переводах.

Как сгинули видеотеки? А очень просто – появились «коммерческие» телеканалы, где стали крутить те же фильмы, что предлагались видеотеками. Надо было только подождать. Например, каждую пятницу, ночью, по седьмому каналу показывали фильмы ужасов, иногда очень поздно по причине сдвига расписания. Их предваряла занудная рекламная передача, которой предшествовала музыкальная «Terrorizer» про разный блэк и прочий металл. Поскольку я любил фильмы ужасов, то ждал пятницы. Кстати сейчас такие не снимают, всё испохабилось, вот остался золотой фонд ужастиков восьмидесятых до начала девяностых.

Поначалу «коммерческое» телевидение сильно выигрывало, в моем тогдашнем представлении, по сравнению с государственными тремя каналами. Блин, сидишь дома и смотришь то же, что в видеотеке! После «Мегапола» (с интересными самодельными передачами про новинки видеорынка, музыки, приставочных игр) на разных частотах начали пробиваться «Тонис», «Ютар», «ICTV» и другие. Видеотечный репертуар держался там в первые годы, потом схлынул, фильмы стали пресней, стоящих показывали всё реже.

А видеотеки – когда исчезла последняя и где? Этого я не знаю.
