\chapter{Боричев, Перун и Ручай}

Какие дополнительные сведения указывают на положение Боричева? Летопись упоминает Боричев в связи с безымянным «ручаем».

Идол Перуна стоял на холме, где во время Нестора была церковь святого Василия, построенная Владимиром на месте, откуда кумира сбросили да волочили затем по увозу Боричеву «к ручаю». И вот относительно «ручая» археологи с историками ломают копья. Одни утверждают, что это Почайна, другие же подыскивают на роль ручая иное.

В Повести временных лет четко прописан путь перемещения Перуна. Откроем Ипатьевский список. Князь Владимир, вернувшись в Киев

\begin{quotation}
повеле кумиры испроврещи, овыи сещи, а другыя огньви предати; Перуна же повеле привязати к коневи хвосту и влещи с горы по Боричеву на ручай, 12 мужа пристави бити жезлием. 
\end{quotation}

Прервем чтение. Владимир повелел сбросить некогда им же установленных кумиров. Одних приказывает изрубить (сещи\footnote{Сещи, сечь – старинное слово для обозначения действия «рубить». Отсюда – секира, сокира. «Сечь» Запорожская – значит место древесной вырубки. К секире близко латинское обозначение топора «securis» (сэкурис) и греческое «τσεκούρι» (цэкоури) – очевидно, все они имеют общий корень. Слово же «рубить», родственное к «робить», «работать», прежде употреблялось более к строительным работам из дерева, отсюда и название инструмента «рубанок», и глагол «срубить» – выстроить деревянное здание, и «сруб» – сам деревянный дом. «Срубить деньги» означает прямо – «заработать деньги» по былому равенству «робить» («работать») и «рубить».}), других сжечь, а Перуна привязать ко хвосту коня и тащить с горы по Боричеву на ручай.

Продолжим:

\begin{quotation}
И влекому же ему по Ручаеви к Днепру, плакахуся его невернии людье, еще бо не бяху прияли крещения; и привлекше и, вринуша и в Днепр. И пристави Володимер, рек: «аще кде пристанет, вы, то отревайте его от берега, доньдеже порогы проидеть; тогда охабитеся его»\footnote{Один из списков «Слова о том, како крестися Владимир, возмя Корсунь» излагает иначе: «и приста Владимер и рече так – аще где предстанет Аполон, и вы его отгеваите от берега». Примечательно, что в греческих мифах вместо Аполона порой используется другое имя, Παιάν - Пайан, Παιών - Пайон либо Παιήων - Пайеон. Что звучит совсем уж как Перун. Однако иногда в древнегреческой литературе Аполлон и Пайеон разделяются, как два разных бога. В позднейших источниках Пайеон приравнивается уже не к Апполону, а к богу-целителю Асклепию.}. Они же повеленая створиша. Яко пустиша и, пройде сквозь порогы, изверже и ветрь на рень, и яже и до сего дни словет Перуня рень.
\end{quotation}

И тащили идола Перуна по Ручаю к Днепру. Люди плакали. Сплавление кумиров по реке мы еще обсудим много глав спустя. Владимир дает наказ – где пристанет идол, отталкивать от берега, пока не дойдет до порогов. Так кумира провожали, покуда ветром не вынесло его на мель, при Несторе слывущую Перуновой ренью. Теперь это скрывшийся под водой остров Перун напротив одноименного села в Вольнянском районе Запорожской области.
 
Итак, путь Перуна в районе Киева был таков – с горы по Боричеву, на ручай, по ручаю к Днепру.

В прошлой главе мы познакомились с Почайной да её притоками. О Почайне летописных времен, которые сейчас обсуждаем, мы понятия не имеем. Где протекала, как? Заглянуть удается эдак по 17 век.

Если полагать, что и в летописные времена все реки и ручьи Подола вливались в Почайну, то никакой другой «ручай», кроме Почайны, на протяжении Подола в Днепр впадать просто не мог. Значит, в цепочке следования идола «с горы по Боричеву, на ручай, по ручаю к Днепру», ручаем может быть только Почайна. И описанный выше путь Перуна таков: фуникулер, через Почтовую площадь в Почайну, по Почайне в Днепр.

Но если Почайной при князе Владимире считалась только Почайна-1, то ручаем по размеру может быть наверное только Глубочица, но с Боричева-фуникулера волочить по ней идола бессмысленно, и вот здесь, среди многочисленных «если», и закрадывается мысль о переносе имени Боричева в какое-то другое место, прочь от фуникулера.

Что летописцы могли называть «ручаем»? Сколь велика или мала должна быть для этого река? Например, в Суздальской летописи Лаврентьевского списка реку Лыбедь именуют, в году 6738 – «ручай Лыбедь». Но сколь был осведомлен переписчик летописи о ширине и глубине Лыбеди?

Кажется Берлинский в «Кратком описании Киева» впервые предположил, что «ручай», по коему волокли идола Перуна, и есть Почайна. Это на долгие годы приняли за истину. Затем появились сомневающиеся, которые решили, будто «ручай» вовсе не Почайна, а другая речушка или ручеек, но при этом снова подразумевали, что «ручаем» в летописи всегда называют один и тот же водоток.

Однако, на основании чего считается, что один и тот же? Неясно. Более того, ежели летопись в некоторых случаях говорит о разных «ручаях», то места, связанные с ними, могут оказаться вовсе не на Подоле.

Давайте рассмотрим упоминания в летописях о киевском «ручае» и Почайне. Про ручаи выпишу все, про Почайну – весомые, от прочих толку в нашем вопросе нет. Берем Повесть временных лет по Ипатьевскому списку, читаем:

\begin{quotation}
И наутрея призва Игорь посли, и приде на холмы, кде стояше Перун, и покладоша оружья своя, и щиты, и золото, и ходи Игорь роте и мужи его, и елико поганыя Руси; а христьянскую Русь водиша в церковь святаго Ильи, яже есть над ручьем, конец Пасыньце беседы и Козаре; се бо бе сборная церкови, мнози бо беша Варязи христьяни.
\end{quotation}

Речь идет о показательной клятве перед послами из Царьграда. «Рота» означает божбу, ротьбу, клятву пред богами. Игорь ведет языческую часть своей дружины на холмы (в Лаврентьевском списке – холм) с Перуном, а христианскую часть – к церкви святого Ильи над ручьем, в конце Пасыньце (Пасынче) беседы и Козаре.   

Где находились церковь святого Ильи, Пасынча беседа и Козаре? Нынешняя церковь Ильи у набережной Днепра построена на месте одноименной деревянной в 1695 году. Когда там появилась деревянная, и на месте летописной ли – неведомо. Помните, Гупало пишет о древнем ручье, впадавшем в Почайну подле известной нам существующей Ильинской церкви? Берег набережной омывался в 1695 году Почайной. А «над ручьем» может в равной степени относиться к ручью, упомянутому Гупало, и к Почайне, если таковая омывала Подол не только в 17 веке, но и в летописное время.

Однако есть еще две привязки, Пасынча беседа и Козаре! Наши археологи даже сообщают о найденных там предметах, словно те лежали под табличками с названиями – это Пасынча беседа, а это Козаре. По летописи нельзя вычислить положение сих местностей. К тому же много разночтений. В некоторых списках стоит Пасыньца и Постнича. Варианты на любой вкус – и «конец Постничи беседы в Козарех», «конец Пасынча беседа и Козаря», «иже бе на ручаем конец Пасынчи беседы».

Исследователи порой отождествляют Козар с другим летописным районом – «Жидове», ведь известно, что первые исповедовали в том числе иудаизм, а вторым словом в старину обозначали иудеев. Часть города, именуемая «Жидове», во время пришествия в Киев литовско-польской власти находилась у Щекавицы. Однако кроме сходства вероисповедания, мы не имеем доводов, чтобы соотнести «Жидове» и «Козаре». Более того, получается, что Игорь ведет в Козаре на клятву не в иудейский храм, но в соборную христианскую церковь! А соборной, главной она могла являться только для всего города либо какого-то монастыря. 

Но продолжим выписывать про ручай и Почайну. Посольство ко княгине Ольге древлян, убивших её мужа Игоря:

\begin{quotation}
И послаша Деревляне лучьшии мужи свои, числом 20, в лодьи к Ользе, и приста под Боричевом в лодьи. Бе бо тогда вода текущи возле горы Кьевьскыя и на Подоле не седяхуть люде, но на горе; город же бяше Киев, идеже есть ныне двор Гордятин и Никифоров [...]
\end{quotation}

Здесь ручей не указан прямо, зато есть Боричев или Боричево – причем не увоз, не взвоз, а просто Боричев. У меня много вопросов. Рассудим постепенно. Поглядим разночтения в списках у Гилярова:

\begin{quotation}
и присташа под Боричевом: бе бо тогда вода текущи возле горы Киевския и до Винны. Седяху же люди на горе.
\end{quotation}

\begin{center}***\end{center}

\begin{quotation}
и присташа под Боричевом: бе бо тогда вода текущи возле горы Киевския и до Двины. Седяху же люди на горе.
\end{quotation}

\begin{center}***\end{center}

\begin{quotation}
и присташа под Боричевом. Бе бо тогда вода текущи подле горы Киевския, и на Подоле не седяху людие, но на горе;
\end{quotation}

\begin{center}***\end{center}

\begin{quotation}
И приплуоша под Боричев в лодьи; бе бо тогда идущи под гору Киевоу и на Подоле не седяхоу люди, но на горе.
\end{quotation}

«Седяху» значит «жили». Летописец говорит, что «тогда» вода текла возле горы Киевской. Какой временной отрезок отхватывает слово «тогда»? Дни древлянского посольства к Ольге? В том году? Во время правления Ольги?

Разночтения списков усиливают путаницу. Вода текла от горы Киевской до Винны? Двина ли Винна? А Двина Северная или Западная? А как «вода» могла течь от Днепра или Почайны до одной из Двин? Может, Винна – нечто близкое к Киеву? Но можно предположить в этой Винне и какое-то урочище по течению Почайны или Днепра в пределах Киева. Или летописец нарочно назвал дальнюю от Киева реку, чтобы подчеркнуть масштабы разлива воды? Нельзя сбрасывать со счетов и то, что Нестор говорит о прошлом, которому не был свидетелем.

К сожалению, я не читал книгу Сагайдака «Давньокиївський Поділ», где есть подробности про некий «ручей», найденный в 1975 году. Знаю по выдержкам. Мол, вдоль Старокиевской горы, между нею и Почайной протекал питаемый ручьями со склонов большой ручей, который около домов 6 и 8 на улице Сагайдачного имел ширину 28 метров. 

Какой же это ручей? Целая река, вторая Почайна! Как на таком коротком участке под горой мог вырасти столь богатырский ручей? Для этого нужна была чудовищная водоносность склонов, откуда бы сочились ручьи, которые бы просто размыли весь Подол вдоль Боричева Тока и до Почайны. 28 метров могла быть в том месте ширина Почайны или даже Глубочицы.

И это русло хорошо подходит под слова Нестора, что «бе бо тогда вода текущи возле горы Кьевьскыя». Да вот беда, не ведаю подробностей. На какой глубине от поверхности 1975 года его нашли?

При раскопках на Подоле в 1971 году, когда метро строили, разрыли до 12 метров вглубь и обнаружили культурные слои, чередуемые со слоями песка. И если, скажем, некий культурный слой (керамика, предметы, мусор) находится на глубине 9 метров от современной поверхности Подола, это значит – какое-то время назад поверхность, современная тому слою, была на 9 метров ниже, чем нынешний Подол в том же месте. Поскольку жить под водой нельзя, следует заключить, что Почайна с Днепром текли еще ниже.

Однако мы не знаем тогдашнюю глубину (расстояние от дна до поверхности воды) Днепра и объем его стока. Если сток за прошедшие века ощутимо не менялся, то, дабы поверхность воды лежала ниже современной, надо и чтобы дно реки залегало ниже. И со временем оно поднималось благодаря наносам.

Во время раскопок в первой половине 1970-х археологи нашли, на глубине 8-12 метров\footnote{Дюжина метров – почти высота пятиэтажной хрущовки, не считая крыши.}, остатки десятков рубленых домов, стоящих по улицам с заборами\footnote{См. П. П. Толочко «Массовая застройка Киева X-XIII вв.».}. Некоторые срубы были, вероятно, многоэтажными. В строительстве использовалась древесина дуба и сосны. Ученые отнесли здания к 10-13 векам по причине сходства культурных слоев со слоями верхнего «града», датированными этими веками. Доверяй я датировкам, предположил бы, что по крайней мере вскоре после княженья Ольги Подол точно был населен. Выяснение вопроса, как же до того, снова привело бы к пустопорожним рассуждениям о несторовых сведениях про воду «до Винны».

Можно сказать, что во время срубного поселения на Подоле, уровень Днепра был по меньшей мере на 13-14 метров ниже, чем в конце 20 века. Становится понятным наличие древнего кладбища в районе Почтовой площади. При постройке почтовой станции в 1854 году нашли много гробов (кстати, тогда же площадь была несколько надсыпана, для образования ровного места). В близости от воды, да еще учитывая разливы, никто бы не хоронил. Выходит, кладбище устроили в месте, слывшим как безопасное.

Чередование культурных слоев со слоями песка означает, что уровень воды в реке из века в век повышался\footnote{Уровень Днепра поднялся и со строительством на нем искусственных морей, водохранилищ. Поэтому были затоплены знаменитые пороги у Запорожья.}. Иначе бы не намывались очередные слои песка поверх новых культурных слоев. Времена заселенности Подола чередовались затоплениями, но как долго Подол оставался пустым и сколь скоро начинал обживаться вновь?

За какое время намывается трехметровый слой песка? Нужно знать условия, причины намытия. Это могло быть единоразовое наводнение, грязевой сель, наконец, долговременное покрытие суши водой.

Но примем положение, что по крайней мере при посольстве Древлян к Ольге, вода текла под горой Киевской. Летописец отмечает – лодья послов пристала именно под Боричевым. Можно подплыть непосредственно к Боричеву. Нестор нарочно поясняет, почему так удалось. «Тогда» вода текла возле горы Киевской. Значит, Боричев был на горе Киевской. Наверху ли, на склоне ли – не уточняется.

Итак, послы Древлян к Ольге пристали около Боричева. А куда Ольга советует пристать императору Византии? Вспомним: «А ще ты, рцы, такоже постоиши у мене в Почайне, якоже аз в Суду, то тогда ти вдам».

Таким образом привязывается Почайна к Боричеву, а Боричев – к Киевской горе. Если только за время, прошедшее между двумя посольствами, не изменилось течение рек. Значит, «ручай», к которому по Боричеву тащили идола Перуна, это Почайна. А поскольку на пригорке с Перуном теперь  стоят не языческие идолы, а здание МИД, соседствующее с верхней станцией фуникулера, очевидно, что Боричев это склон, где проложены рельсы оного фуникулера, а не Андреевский спуск.

В 2015-2016 годах археологи трудились над раскопками Почтовой площади и находят там, по сути, те же бревенчатые избы, что при проложении линии метро, да вроде более поздних времен остатки деревянной мостовой. Об этом я узнал в переложении журналистов из разных общедоступных газет, ибо какие-либо официальные и около электронные ресурсы про раскопки в стольном граде Киеве молчали.

Однако с удивлением ученые начали говорить, что древний Киев был больше, чем они считали. Это новое мнение бог весть еще когда станет общепринятым, и вряд ли дойдет до признания текущих границ Киева, что раскинулся на обоих берегах Днепра, за истинные границы Киева древнего.

Более легко наука восприняла фантастическую топографическая модель относительно Боричева увоза. Много лет она питает книги и статьи.

Академик Борис Рыбаков, следом за дореволюционным профессором Киевской Духовной академии Николаем Петровым и археологом уже советской эпохи Д. И. Блифельдом, Боричев увоз считал современным Андреевским спуском. Его мнение повторяют многие другие исследователи. Основу сего понимания Боричева положил Петров книгой «Историко-топографические очерки древнего Киева», и две статьи, одна Блифельда («Про питання про Боричів узвіз стародавнього Києва» во втором выпуске украинского журнала «Археология» за 1948 год), да Рыбакова («Город Кия», 5 выпуск «Вопросов истории», 1980 год).

Носитель новых краеведческих идей, Петров, только наметил путь. Блифельд привлек, помимо Петрова несколько дополнительных источников – для полемики, жонглировал цитатами, заодно под конец статьи привел мысль Рыбакова, выводившего название «Боричев» от Борисфена. Спустя полстолетия Рыбаков повторяет ту же мысль уже в «Городе Кия» – дескать, и Боричев увоз, и Боричев Ток – всё от Борисфена.

Именно статья «Город Кия» растащена на цитаты и питает современное мнение науки о Боричеве. И далее по цепи – краеведы, экскурсоводы, журналисты. Даже одна хиппи на Андреевском спуске многозначительно сказала мне – настоящий Боричев-то увоз не там, а тут, на Андреевском!

Рыбаков взял соображения из статьи Блифельда, назвав оную «специальным исследованием», развил её, да еще сочинил карту. Не привожу её здесь, соблюдая авторское право.

Карта отражает представления Рыбакова. Поскольку на там изображена Почайна, карта соответствует неким летописным временам, быть может самого Кия, Хорива и Щека.

Вот Рыбаков пишет: «Ручай протекал посереди Подола и вливался в Почайну у самого её впадения в Днепр». И рисует устье Почайны в районе нынешней Гавани. Но Рыбакову южнее нужен Днепр. Потому что у него Почайна это не ручай. Ручай он рисует в другом месте.

На карте у Рыбакова ручай впадает в Днепр, а в тексте академик пишет – «вливался в Почайну», что противоречит карте. По ней, однако, можно понять, что Рыбаков, знакомый с планом Ушакова, принял за летописный «ручай» – Борисоглебский. 

Параллельно ему и Глубочице, но юго-восточнее, эдак по улице Ильинской ученый изображает Боричев Ток, вопреки действительности. А Киянка и Глубочица сливаются по Рыбакову много восточнее, чем на деле. Не учтен и поворот Глубочицы вдоль подножия Щекавицы на северо-восток – по Рыбакову, Глубочица уже издревле занимала прямое русло Канавы 19 века. На карте есть также ориентир «Пирогоща», но перенесем его разбор в главу про «Слово о полку Игореве». 

Итак, взят современный рельеф, будто это давний, на него наложены объекты – из соображений теории, без привлечения простейших сведений о речной системе Подола хотя бы за прошлые несколько веков. И статья с картой служат основой для исторических работ! Почему не карты Средиземья от профессора Толкина?

Кратчайший путь с южной части Подола на гору – овраг фуникулера. Так было и остается, и не зря этот фуникулер был затеян. Параллельно ему, с некоторой высоты холма, к Почтовой площади нисходит улица Боричев спуск, известная по крайней мере с 19 века. Она обходит церковь Рождества с северо-запада. Однако по то же 19 столетие сохранялись остатки еще какой-то дороги, что ровно проходила вдоль церкви с юго-запада и продолжалась к зданию почтовой станции. Вот эта дорога и могла быть остатком Боричева увоза и даже частью его оврага. 

Помните про мост возле церкви, в 18 веке? «Сквозь крещатицкую башню по конец мосту, что ныне вновь построен был к церкви Рождества Христова, против Михайловского взвоза». По сопоставлению, Михайловский ввоз это Боричев увоз. «Ныне вновь построен» – значит, мост уже существовал когда-то, а затем был разрушен, и вот его выстроили снова.

Но зачем там мост? Через какой-то водный канал? Или было так – овраг с дорогой Михайловского ввоза нырял под этот мост и продолжался далее, допустим к набережной. А по мосту, над «звозом», шла какая-то другая дорога?

%18-vek-bor.png


%Давайте рассмотрим фотографию конца 19 века с видом на Подол.

%\begin{center}
%\includegraphics[width=0.90\linewidth]{osn-kiev/podol-nachalo.jpg}
%\end{center}

%Крутой склон слева – Владимирская горка. Справа, у реки – пристань. В некоторые летописныя времена там, а учитывая разницу примерно в 13 метров между нынешним уровнем воды и тогдашним – несколько правее, тоже была пристань, только вместо Днепра у нее плескались волны Почайны.

%При стыке набережной и Владимирского спуска (тогда Александровской улицы), на Почтовой площади – невысокая Рождественская церковь с круглым куполом. Справа от нее, через одно здание – почтовая станция. Перед церковью и станцией, к набережной, в 19 веке долго сохранялись следы дороги, точно напротив фуникулера. Возможно, это и были остатки низа Боричева увоза. Позади почтовой станции начинается Подол.

%Современная улица Боричев спуск, идущая параллельно с фуникулером, примыкая к нему, тоже нисходит к Почтовой площади, почти напротив заново выстроенной Рождественской церкви. А лихо закрученный Андреевский спуск – он намного дальше, к Контрактовой площади. Отсюда вне поля зрения.

%Представим воду на 13 метров ниже и примем к сведению соображение, что Подол затем «намылся» на эту высоту. Что же имел в виду Нестор, когда говорил, что вода «тогда» текла около горы Киевской? Половодье ли невиданное? Ведь даже сейчас, чтобы дойти до уровня подножия холма, 
%воде надо потрудиться, залив весь Подол. А если Днепр протекал еще тринадцатью метрами ниже? Я допускаю и русло Почайны под горой, но в то время Подол, получается, был островом.

Дореволюционная открытка: 

\begin{center}
\includegraphics[width=\linewidth]{chast-colebanie-osnov/bor-perun-ruchay/\myimgprefix mihpod.jpg}
\end{center}

Линия фуникулера. Вероятно, тут идола и тащили по Боричеву к текущей (метров на 13 ниже уровня поверхности воды 1970-х годов) где-то внизу реке. Только овраг был более углубленным в склон, менее крутым благодаря увеличенной протяженности на юго-запад.

Церковь в кадре – Рождественская. Самый большой, заметный дом в кадре (Боричев спуск, 13) и угловое здание перед ним (Сагайдачного, 3) существуют поныне, правда к дому №3 добавили этаж. Строения же, примыкающие к церкви и за нею давно снесены.

Самый простой путь, какой только можно вообразить. Не надо никакого обходного вроде полукруглого Андреевского спуска, этой рукотворной дороги, попросту прорытой между Уздыхальницей и Замковой во время заката киевского замка, когда сквозь него понадобился удобный подъем к верхнему городу взамен прежнего путища, петлявшего по склону Замковой, обращенной к Щекавице.

Но сторонники теории, что Андреевский это Боричев, выискали еще один источник, её питающий – «Слово о Полку Игореве».
