\chapter{Кий и его родичи}

Вернемся к летописному семейству основателей Киева. Нестор говорит мало – жили на холмах три брата с сестрой, с «родами» своими. Вместе построили град, читай селение огороженное, во имя брата старшего, Кия.

Однако не сказано, как братья и сестра появились на этих горах. Быша, и всё тут. Обратимся к «Приложениям» Гилярова. С шестидесятой страницы начинаются там вкусности по нашему предмету, разночтения про Кия, Хорива, Щека, Лыбедь и Лямень, выхолощенные официальной историей. Кроме Гилярова, буду заглядывать в «Кройнику» Софоновича\cite{sofonovich01}.

Касательно имен. «Второй Щех или Щок». «Младший Хорев или Кирев», а также «Корев». Лыбедь пишется как «Лыбедь», «Либедь», «Лебедь», «Лыбеда» (у Софоновича). 

Есть такое: «И с ними две сестры: Лыбедь, Лямень». Кто эта Лямень? Сестры ли они братьев или сестры между собой? В списках, где нет Лямени, Лыбедь указана братьям сестрой.

Некоторые летописные сведения о том, откуда явились братья и сестра. Для удобства буду говорить только о Лыбеди, покуда про Лямень нет данных:

\begin{quotation}
Приидоша от диких пол теже славяне три брата: Кий, Щок, Хорив и сестра их Либедь.
\end{quotation}

\begin{center}***\end{center}

\begin{quotation}
Приидоша из словенских полян 3 брата: единому имя Кий, другому мя Щок, третиему имя Щек, да с ними же прииде сестра их Лыбедь, а преж того живяше тут поляне около города их.
\end{quotation}

\begin{center}***\end{center}

\begin{quotation}
А брат его Щек живяше на другой горе, где ныне зовется Дрекович, и нарече Щековица.
\end{quotation}

\begin{center}***\end{center}

\begin{quotation}
И брат его Щек на горе, от негоже прозвася Щекот.
\end{quotation}

Софонович из своего 17 века дополняет:

\begin{quotation}
Корев будовал Коревице – две мили от Киева место, которое потом Вышгородом названо, а Либеда, трех тых братьев сестра, над рекою Лыбедю, и от еи имени названою, замок на горе высокои збудовала.
\end{quotation}

Некоторые списки, для уточнения времени повествования, говорят, что в описываемое время основания Киева правил «во Грецех Царь, именем Михаил, а мати его Ирина». В Византии было две императрицы Ирины, одна жила, по общепринятой хронологии, примерно в 752-803 годах, другая в 1088-1134. Ни у одной сына Михаила – императора – неизвестно. А Михаилов-императоров история знает троих. Михаил I (?-844), Михаил II Косноязычный (?-829) и Михаил III Пьяница (840-867).

В польской «Кронике» Стрыйковского\cite{strykron} находим дополнительные сведения о Лыбеди: «a czwarta ich siostra Lebeda albo Lebed»\footnote{«а четвертая их сестра Лебеда или Лебед».}, «A siostra ich Lubeda nad rzeka Libieda osady swoje ugruntowawszy, tamze zamek Libiec, albo Lubiec, zbuduvala na kopcu wynioslym»\footnote{«А сестра их Любеда над рекой Либиеда создала своё поселок, та же замок Либич, или Любич, построила на высоком холме».}. И еще, в одном списке Новгородской летописи сказано «такожде и сестра их согради град Либедь».

Софонович тоже говорит, что замок свой Лыбедь возвела на горе над одноименной рекой, вроде бы в честь Лыбеди и названной.

В «Чешской хронике» Вацлава Гаека есть предание, относимое им к восьмому веку, о влиятельной дочери чешского князя Крока\footnote{От него назван город Краков.}, Либусе, которая построила город и назвала в свою честь – Либеч или Либиче. Двух ее сестер звали Каса и Тетка. Повествование отводит Либусе значительное место. Но то про Чехию! Однако не повлиял ли один источник на другой?

К слову, в Черниговской области, на берегу Днепра есть поселок городского типа Любеч, упомянутый, ежели отождествление верно, еще в Повести временных лет, когда Вещий Олег захватил град Любеч. Оттуда родом были Малуша (мать Владимира Красно Солнышко) и Антоний (основатель Киево-Печерского монастыря, будущей Лавры). 

Но летописи упорно твердят о городке, построенном Лыбедью около киевской речки Лыбеди: 

\begin{quotation}
Сестра же их Лыбедь над рекой Лыбедью (Либедь над Либедью) свои осады положши, таможе и город на пригорку высоком согради от своего имени Лыбедь (Либедь).
\end{quotation}

\begin{center}***\end{center}

\begin{quotation}
Сестра их Лебедь построила у реки Лебедя на высоком холму Лебедин.
\end{quotation}

Где именно построила Лыбедь свою крепость? Что за высокий холм? На ум приходит, и не мне одному, Девич-гора – более известная, с 19 века, как Лысая гора южнее Зверинца. Но доказательств для такой привязки нет. На протяжении Лыбеди есть и другие высокие горы, например Байкова.

Есть варианты и в представлении того, что такое Лыбедь. Традиционно её истоки полагают на Отрадном и Караваевых дачах. Однако у Лыбеди есть приток Скоморох, текущий с Лукьяновки, впадающий в Лыбедь около вокзала. Но еще в 20 веке жители окрестностей того Скомороха, на улицах Володарского, Павловской, Речной, полагали что живут не у какого-то Скомороха, а у речки Лыбеди!

И если верно это представление, то многие летописные сообщения о Лыбеди могут относиться к тому, что мы ныне считаем Скоморохом.

Пойдем теперь по «новгородскому следу»:

\begin{quotation}
Сказует же, яко сии князи бяху нова-града жители; посадники и крамолы ради и нестроения выслаша их из новограда великого; тогда изыдоша они из града с племенем своим на горы брега реки днепра, начаша град и места ради тишайшаго жития и прибежища созыдати.
\end{quotation}

Более обширный вариант:

\begin{quotation}
В то время быша в великом Новогороде три брата кижики (Кий, Щек и?) Хорив и сестра их Лыбедь. И се братеники и с сестрою их люти разбойницы великую пакость Новгородцем творяще. Навгородцы же яша их тридцать челвек, вси храбры и мочни вельми, осудиша их повесити. 

Кий же з братьею своею моляша княза Ольга со слезами, дабы их отпустил, и обещастася ити, идеже несть вотчины и державы. Олех же оумилостися над ними, отпусти их. 

Они же идоша от великого Новаграда два месяца, и приидоша же на реку Непр, иже течет из Руския земли на полдни в мое теплое, по ней же живяху Варяхи, и обрете некие горы высокие и обрете на них крест, егоже постави апостол Христов Ондрей Первозванный. 

И вселися Кий на горе, яже нарицается Киевица, а еще (Щек?) вселися на другой горе близ его же, еже ныне нарицаеися Щековица, а Оурив вселися на третьей горе, яже ныне словет Хоривица; и начаша землю пахати своима рукама и славно жить, и к ним прихожаху мноние и вселяхуся тут. 

И потом созда градец, имя ему Киевец. В лето 6490-го году по убиении Кия виликий кия (князь?) Олег пришед и заложи град Киев великий и по начальником имени.
\end{quotation}

Да еще:

\begin{quotation}
Приде же (Олег) и в Киев и убив триех братов Киевских начальников: Кия, Щека, Хорива, и начал княжити в Киеве и в великом Нове граде.
\end{quotation}

А вот полностью та штука, которая в сокращенном виде зацепила меня в книге Котляра:

\begin{quotation}
Выпись из Рускаго летописца вкратце. Во времена и лета Олгова княжения князя Руского, иже княжи в великом Нове граде. О граде Киеве и о княжении.

Быша в великом Нове граде нецыи мужие воини, сии речь, разбойницы люти зело, три брата: первый Кий, вторый брат Щек, третий же брат Хорив, да у них же сестра Лыбядь такоже была храбра и велми красна. И много те мужие и сестра их зла творяху Новгородцем, разбои чиняху во граде и по селом. Новгородцы же граждане мужей Кия и братию его сестру их Лыбедь (Лыбеду) поимав и посадиша их в поруб и жены и дети их числом до тридесяти душ, вси храбры и силны велми. И седяху те мужие много лет в темнице. Новгородцы же их повелеша обесити.

Кий же и брат его начаша плакати и бити челом князю Олгу: господине княже Олег! яви милость рабом своим, вели ис темницы выпустить, и мы отидем от града сего и от твоея области, идеже, господине, держава твоя не есть. 

И умилосердился Олег князь и отпусти Кия и братию его и весь род его. Они же идяху дебрием до дву месяц и доидошу реки великия, рекома до Днепра, иже течет из Руси на полдень в море теплое, по нему ж живут Варяги. 

И приидоша на горы высокия и обретоша на тех горах крест, егоже постави при страсти Господни Первозванный Апостол Андрей, брат Петров\footnote{То бишь нашли там крест, поставленный Андреем Первозванным.}. И рече сице святый Андрей: в последная лета на сих горах возсияет благодать Божия и будет град наречен имя ему Киев, той град будет всеми градом Руским мати, и утвердтится в нем христианская вера, и прославится той град во вся страны еллинския области\footnote{Вероятно, подразумеваются все страны поганской области, все страны языческих земель.}, и поклонятся царие и князи и колена варварския господиям града сего. 

И полюби Кий место сие и вселися на горе, иже есть и доныне Киевица (Кивица), а брат его Щек вселися близ его на другой горе, иже есть Щековица, а Хорив вселися на третьей горе, иже есть Хоровица, а Лыбедь, сестра их, такоже вселися со всеми роды своими. 

И потом Кий и весь род его начаша делати землю рукама своима и начаша славно жити. И приходжаху к ним многие люди от всех стран и вселяхуся тамо с Кием, и рапространися место сие, и людей множество в нем. И тогда наченьше Кия и дружину его наимовать Древляне.

И в то время Кий з дружиной своею сотвори себе градец мал Киевец, и нача слыти первой Киев по всем странам, и болши того множество множашеся людей к месту тому приходжаху.
\end{quotation}

Вне общепринятой летописной линии, на смежном с Кием временном отрезке действуют Олег, Игорь, Аскольд и Дир:

\begin{quotation}
О Асколде и Дире, князех Киевских.

И по сем князь Олег посла из нова города послы своя ко Царю граду Греческому царю Михаилу с великою честию и со множеством царскими драгими дарами и посла к нему два мужа честна: единому имя Аскольд, а второму имя Идир. Они же посланнии мужие Олговы прийдоша до Кия, до гор тех, и дивитися наченше, зряще красоту места того. 

И пустиша посланние мужи на Кия и побиша они Кия и всю братию его и весь род его, а сами не поидоша ко Царю граду посолством и вселишася ту и создаша градец боле перваго\footnote{Отметим – Аскольд и Дир создают крепость б\'ольшую, чем крепость Киева.} и живяху во славе велице многи дни.

Слышавже то князь Олег Новгородцкий, яко тии мужи ему солгаша, ко Царю граду не поидоша и на Днепре (Непре) реце живяху, и возярися князь Олег и нача мыслити, коею бы виной мог уловити Аскольда и Идира. И собра воинство много в лодьях и на конех, а сам князь Олег поиде в лодиях в мал дружие, а Игоря княжича отпусти на конех с многим воинством. И приехав Олег князь под Киевец и призва к себе мужей тех Оскольда и Дира.

Они же, видевши со князем Олегом мало дружины и выидоша к нему на брег с великою частию и начаша весилитися; аж в то время прииде Игорь княженич Рюрикович со  множеством воинства, и повеле князь Олег взымати тех мужей Оскольда и Дира, и речем им: ни вы есте князи, ни княжеска роду, неудобно есть вам на сем месте власть держати; а се есте мне солгали. И повеле их князь Олег обесити, а сам ту нача и со Игорем княжити и повеле заложити великий град Киев\footnote{До этого, по списку, крепость «Киевец», а только Вещий Олег закладывает уже град «Киев». Так ли было, или нарочное преуменьшение прежней, до Рюриковичей, славы?}.
\end{quotation}

Еще вариант:

\begin{quotation}
И по сем Кий и весь род его нача делати землю руками своими и нача славно жити. И к ним приходжаху многие люди со всех стран, веселяхуся тамо с Кием, и распространися место сие, и людей многое множество в нем. И тогда наченьше Кия и дружину его насиловати Древляне. 

И Кия сотвори себе градец мал, и рекоша имя ему Киев. И по всем странам больши того множашася людие к месту сему распространяхуся. И начаша Кий з братьями своими разбой творити, и многи пакости те злые мужие творяще, и сестра их Лыбедь такожде творяху насилие великое Новгородцем\footnote{Истинная роксоланка, не держится в тени братьев своих.}, понеже была вельми храбра на ратех, и тако разбой велий чиняше во граде и по селом. Новгородстии же мужие Кия и братию и сестру их Лыбедь поимавше и посадивше в поруб и жены и дети их и всех числом их яко до тридесять душ, и все зело храбри и велмощнии суть. И седяще те мужие многие лета в темнице, и властели Новгородстии повелеша их всех обесити на древе.

Князь же Кии и братия его начаша плакатися горько со слезами князю Ольгу и милости у него просити, от смерти избавлени, и рече: господине Олег! помилуй и яви милость свою над нами, ибо отъидем далеко от града сего и от твоеи области, идеже, господине, державы твоея несть. 

Виде же князь Олг слезное их моление и прошение, умилосердился о них и выпусти их ис темницы и из града Кия и братию его и сестру его и весь род его. Они же дебрию идяше до двою месяцу и паки приидоша коиждо во свой град. 

И по сем князь Ольг посла послы своя из великого Нова града ко царю Михаилу Греческому с великою честию и со многими црьскими драгими дарами и посла к нему два мужа честныя: единому имя Оскольд и второму имя и Дир. Они же послании мужие Ольговы приидоша Днепром до гор тех и дивитися наченьше, зряще красоту места оного. 

И тако побиша Кию и братию его и весь род их, а сами не поидоша ко Царю граду, тут и вселишася и еще создаша себе градец больше перьваго, и живяше они во славе многи дни. Слыщав же то князь Олег, яко Ноугородстии мужие ему солгаша и ко Царю граду со многоценными дарами не поидоша и на Днепре реке живяста, и возъярився князь Олег и нача мыслити, коею бы виною могл уловити мужеи тех Оскольда и Дира. 

И нача воя собирати многия в лодиях и на конех, и сам поиде в лодиях в мал дружине своей и Игоря княжича постави на горах с великим опасным войском. И Олег князь повеле их изымати мужей тех и изменников Оскольда и Дира и рече к ним с великою яростию: о злие мужие! ни вы есте владельцы, ни роду княженецкого; неудобно вам на сем месте власти держати. Князь же Олег повеле Оскольда и Дира обесити их\footnote{Предположу и такой расклад – Оскольд с Диром были отправлены Олегом в Царьград именно чтобы те осели в Киеве, подмяв местную верхушку. Олег выждал и пришел следом, утверждая княжескую власть. Он сместил прошлых узурпаторов – Аскольда и Дира – явившись в глазах киевлян не захватчиком, сбросившим «коренное» правительство, но заняв место пришлых. Ощутите разницу.}, а сам в Киеве нача со Игорем княжити и повеле заложити великий град Киев болши перваго.
\end{quotation}

Более того – вариант, где снова говорится о двух сестрах:

\begin{quotation}
Кыев от начала не бысть тоу град. При князе игоре живоущие бяше в то время на том месте, идеже ныне град киев, три браникы быша: кый, хорив, и две у них сестры: лыбед. неи. лые. И шод игорь князь победи три братеника: кыя, щека и хорива. И оттоле заложиша град кыев.
\end{quotation}

Еще:

\begin{quotation}
В лето 865 старейшим двум Кию и Щеку без наследия изшедшим, Корева первый единовладетель с наследием своим князем Варяжским Рюриком убиен бысть...

Первый убо от тех князей Варяжский Рюрик безопасна и безоружна наехав Кореву и сотворь себе единовладетеля... 

В лето 886-е в начале первое писание от Болгар, над Волгою рекою вселяющихся, соседов Словенское, а потом, переселився за Дунай, тщанием Куроплата царя или Василиа Макидона приявше владение, познавать начаша; всех цариков и князей, войне паче, неже перу, изучившиися избив, трех из князей Варяжских избраша себе в государи: Кига или Кия, от него же Киев создася, Щека, от него же Щековице, Кореву, от него же Коревицы, ту твердыню нынче Русь Вышгород величают, и стольным градом Руских князей именоваша. Тии Татарских князцов подбива (идеже ныне Астрахань) яко подвластных по себе разделив, государства уставиша.
\end{quotation}

Итак, по Нестору – Кий, Хорив, Щек и Лыбедь вынесены из погодичной хроники, но оставлены в хронике относительной. Это дает возможность отодвигать основание Киева сколь угодно назад во времени.

По «новогородцам» и другим «альтернативным источникам», время действия тех же персонажей привязано к отрезку, в котором действуют Рюрик, Олег и Игорь, Аскольд и Дир. Причем Игорь там вовсе не младенец. По некоторым источникам, Аскольда убивает не Олег, но Игорь. Позже мы обсудим это подробно.

Еще одно важное различие вариантов – по Нестору, Кий и родственники это местные, «киевские» поляне. Другие источники утверждают пришлость Кия с родичами. В первом случае княженье получается естественным, выросшим на родной почве. Во втором же – власть Кия зиждется на основе весьма мутной, а если принимать предание, процитированное Котляром, то Кий это бывший злодей, ставший представителем власти Вещего Олега.

Кий, Хорив, Щек, Лыбедь, Лямень. Многие историки отказывают им в праве на жизнь. Выдумка! О Лямени вообще глухо молчат.  

А что есть горы киевские Киевица, Щекавица, Хоревица, речка Киянка и Лыбедь, так это, считают некоторые ученые, народ почесал затылок и выдумал себе князей, да связал имена их с урочищами.

Между тем нет ничего сказочного и тем паче удивительного в семействе «основателей». Семьи тогда были многодетными, а если принять сведения о разбойном поведении Кия с родичами, то приходит на ум ёмкое слово – шайка. Если только «новгородцы» не очерняют Кия намеренно, чтобы обосновать княженье действительно пришлого чужеземца Олега.

Означают ли что-то имена летописных братьев и сестер? Конечно да. Все имена это прозвища, клички на каком-то языке. Это нынче, когда люди языков чужих не ведают, имена стали просто красивыми сочетаниями букв. Даже говорящие, славянские имена вроде «Владимир» никто не соотносит с «владеть миром», а воспринимают тоже набором букв.

Кий, либо, как возможно говорили в старину, Кый. Слово это имеет много сходных значений – посох, дубинка, шест и так далее, вплоть до песта для ступы. «Кий» в старославянском – местоимение, означающее – какой, который. Кий ты? Ипатьевская летопись за 6579 год передает слова поборника дани Яна, сына Вышаты, обращенные к волхвам, что поклонялись своему богу, что сидел в бездне:

\begin{quotation}
то кий есть бог, седя в бездне? то есть бес, а бог есть седя на небесех и на престоле
\end{quotation}

А вот еще любопытное сопоставление, возможное, если верны сведения давних краеведов о том, что пристань на Подоле называлась «притыкой».

Эту «притыку» обычно переводят как «причал», но «притыка» значило в старину то же, что и «кий» – дубина, палка. И опровержение несторова утверждения, что Кий не был перевозчиком через Днепр, тихонько звенит колокольчиком.

Есть еще толкование Кия с привлечением давних греческих легенд, но я затрону это позже.

Теперь Хорив – не от животного ли хорька? Вариант «Корев» не от коровы ли? Есть еще в Аравийской пустыне, если сопоставление верно, библейская гора Хорив (Орив) – Синай это восточный ея отрог, либо второе название всей. Другие варианты имени киева брата: Кирев, Хорович.

В одном списке вместо Хорива написано «Оурив». А не переиначенное ли это «Юрий»? Либо прозвище от глагола «орать», то бишь пахать. Примечательно, что греческое имя Георгий (Георгиос), сопоставимое с Юрием\footnote{Как Георгий стал Юрием? Прежде у славян «г» было мягким, как в украинским. Иностранные имена вообще зачастую передавались туго, требовалось несколько веков, чтобы имя перенеслось на славянскую почву. Так, князь, известный нам как Юрий Долгорукий, в летописях именуется Гюрги. Гюрги это искаженное Георгий. А произносилось это как Хюрхи. И лишь несколько столетий спустя из Хюрхи образовался Юрий устранением «х».}, в переводе с греческого значит «земледелец». В этом же списке, после «а Оурив вселися на третьей горе» идет «и начаша землю пахати своима руками и славно жить». Неясно, относится это ко всем братьям или только к Оуриву. Если последнее, то получается подтверждение прозвища, Оурив – пашет, орет землю, «орач», «землепашец».

Щек, он же Щок – не от щеки ли? Или от щекотки? Может был проказником и щекотал? Почему историки любят переставлять буквы и делать из Щека Чеха?

Лыбедь – лебедь или лебеда? Или была улыбчивая, лыбилась, потому и прозвали ребенка Лыбедью? В старых земельных документах встречается, применительно к водоему, слово «убедь», но смысл его утерян. Быть может, речку, которую можно перейти вброд? Или убедь – рукав, убежавший в сторону?

Моя бабушка Таня, с 1930-х по шестидесятые жившая на улице Чкалова (ныне Гончара), а это недалеко от речки Лыбеди, удивляет непривычным для меня произношением ее названия, присущего тамошним жителям в те годы – «Лебедь», где в первом «е» слышится также звук «и».

Сестра Лямень. Было новгородское словцо «ляма», означающее «пасть» (в смысле «рот»). Также ляма – лямка, ремешок через плечо. У сибиряков «ухляматься» значило устать. Новгородцы говорили «хлямать» – качаться. Есть сходное «хлябать», а «расхлябанный» всем известно.
