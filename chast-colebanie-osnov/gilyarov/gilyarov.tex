\chapter{Загадки Повести временных лет}

Официальная история не любит новгородцев-летопис\-цев. Подобно тому, как в старинной пословице «Орел да Кромы – первые воры» больш\'ая часть населения этих городов приравнивается к ворам, новгородских летописцев считают обманщиками, выдумщиками.

 %Дружно хохочут историки над сочинениями новгородских летописцев. А может и не дружно, а наедине. Если в библиотеке вдруг раздается неуемный смех, он наверняка исходит от историка с третьим томом Полного Собрания Русских Летописей.

На полке у каждого киеведа зачем-то стоит зеленое, 1982 года издания четырехкнижие «История Киева», подготовленное коллективом ученых. Там сказано, что в отличие от киевской «Повести временных лет», летописцы новгородские прямо сообщают дату возникновения Киева. «Но ей нельзя доверять» – сурово поднимает палец наука. А почему? Поясняют: «Она была внесена в рассказ о Кие новгородскими летописцами XI-XII веков, пытавшимися создать свою схему исторического развития Руси, в которой не Киев, а Новгород выступил бы наиболее ранним восточнославянским городом».

Однако, на основании чего мы должны принять, что новгородцы врут? Лишь потому, что они новгородцы и у них «своя схема»?

Впрочем, некоторых историков уже несколько веков не устраивает и предание об основании Киева легендарными братьями. Красивая выдумка! Не было таких братьев, этимологический миф.

Объявите что угодно враньем и пол\'учите возможность бесконечно утверждать новые истины.

А мне кажется правдоподобным, что некогда живущие родичи Кий, Хорив, Щек и сестра их Лыбедь сообща основали укрепленное поселение на холмах нынешнего Киева. Но кто они, откуда взялись?

Котляр не указал источник выдержки о разбойном семействе. А у меня оно в голове застряло. Но погрузиться в изучение летописей я не отваживался много лет. Тем и довольствовался, что Котляр выписал. Однако сомнение было посеяно.

И наконец я понял – пора самому читать летописи. Но и в известнейших не было про этих разбойников. Нашлось в малоизвестных списках. 

Слово «список» буквально означает копию, нечто списанное с подлинника. К сожалению, из русского языка исконный смысл «списка» вытеснен прямым латинским переводом, словом «копия».

Благодаря этому излишеству, мы привыкли под «списком» разуметь другое – строчки с перечислением, например товаров либо имен. Применительно к историческим и литературным источникам, «список» это рукописная копия (порой исправленная, неполная, или дополненная) летописи либо иного произведения.

Существуют различные летописи и списки оных, отличающиеся подробностями изложения событий, но порой и в корневых предметах. Важнейшие и приемлемые различия, при издании, обычно выносят в приложения или примечания, дабы читатель мог сравнить этот список с другими.

Федор Александрович Гиляров (1841-1895) подготовил и выпустил целый сборник таких примечаний, «Предания Русской Начальной летописи (по 969 год). Прилож\-ения»\cite{gilyarov01}, взяв за основу «Повесть временных лет» по Лаврентьевскому списку, датируемому 14 веком. 

«Повесть» – сборник наиболее ранних преданий о Руси, составленный, как издавна считалось, летописцем Нестором из разных источников. Обычно включается в начало «главных» летописей. Кроме Нестора, в 1116-м над «Повестью временных лет», о чем можно судить по указаниям в ряде списков, потрудился игумен Выдубицкого монастыря Сильвестр. Диву даюсь, что двадцать лет прожил рядом с местом, где это происходило, да и до Лавры недалече.

Вопрос авторства текста, известного как «Повесть временных лет», остается открытым. Первоисточник до наших дней не дошел. У нас нет рукописи ни пера Нестора, ни Сильвестра. В позднейших копиях разнобой. Например, в Хлебниковском списке 16 века составителем значится Нестор, но не указан Сильвестр. В Лаврентьевском и еще двух списках 14-15 веков, после окончания текста записью за 1110 год, добавлено «Игумен Селивестр святого Михаила написах книгы си летописец, надеяся от Бога милость прияти, при князи Володимере\footnote(Мономахе), княжащю ему в Кыеве, а мне в то время игуменящу у Св. Михаила в 6624, Индикта 9 лета; а иже чтет книгы си, то буди ми в молитвах». Так ли это и каким образом соотносится с Нестором –  проверить нельзя. Исследователи строят разные предположения.

Для простоты я бы считал, что Нестор на основе ряда источников (и часть их точно установлена) и собственных впечатлений написал «Повесть», а затем ее правил Сильвестр. Насколько эта простота истинна, сказать не могу. Некоторые ученые, стремясь предать Нестора забвению, полагают, что Сильвестр сначала был иноком в Лавре, а после сделался игумном в Выдубичах.

Как бы ни было, чтение «Повести» приводит к мысли, что за ней стоял, исходно, некий любознательный монах Лавры, одновременно краевед, и по крайней мере в 1106 году он был еще жив. Спустя более столетия, между 1224-1231 годами, лаврский монах Поликарп писал архимандриту Акиндину про Нестора, указывая «иже написа летописец», то бишь в церковной среде в то время уже был известен «летописец» (слово это обозначало летопись) – подразумевается, положим, именно «Повесть» – и Поликарп связывает ее с Нестором.

Существуют однако кое-какие сюжетные противоречия между «Повестью» и приписываемым Нестору житием преподобного Феодосия, а также сказанием о святых князьях Борисе и Глебе. То есть, можно разбирать доводы за и против. Вот, в пещерах Лавры покоятся мощи Нестора Летописца. Именно под таким прозвищем он слывет, повторюсь, издревле. Но известен есть и другой Нестор, Некнижный, следовательно первый Нестор отличался именно своей образованностью. 

Возможно, в будущем я попробую решить задачу сочинительства «Повести временных лет», однако не сейчас.
 
Так вот Гиляров, в своей книге, для каждого раздела «Повести» привел описание тех же событий по летописям Новгородской, Псковской и прочим источникам, включая польские. Гиляров поместил много архивных редкостей, что еще более усиливает значение его труда, да и пожалуй является главной ценностью оного.

С шестидесятой страницы «Приложений» открываются чудеса про основателей Киева.

Книга Гилярова меня словно разбудила и сбила с толку. Как же, чему верить? Странны и противоречивы оказались списки. Одно дело, когда различия летописей встречаются отдельно и с перерывами. Словно редкая капля дождя – оросила лицо, ощутил, да идешь дальше. А у Гилярова это огромное собрание таких различий, уже даже не дождь, но вся небесная вода, упавшая на землю и собравшаяся в один поток, что уносит прочь от привычных представлений!

Известные князья, разведенные официальной историей по времени, здесь оказываются действующими вместе. Появляются новые их родичи. События переносятся в другое время и место, а участвуют в них совсем другие люди. 

Что же историки, археологи? Осведомлены о вариантах истории? Это у них надо спрашивать. Сами они в книжках ничего такого не говорят, а если даже что прорывается, то с оговоркой, вроде как у Котляра – забавная выдумка!

Общепринятая история – лишь один из вариантов, однако об этом молчат. Ученые решили, исходя из своих представлений о прошлом, считать истиной сведения только из определенных списков. Можно проверить таблицу умножения. Но давний летописный слой событий нельзя – начинается путаница. С нею столкнется любой, кто примется за такой труд.

А наука это храм. Строится из кирпичей. Кирпич должен быть однозначным. Чтобы всякий взял его в руку и признал – это кирпич. Кирпич клеймлён именем князя да годом рождения. Вот еще кирпич, тоже с именем и годом. Положим их рядом на цемент. Два князя вместе. Основа готова.

Однако нельзя точно сказать, что на кирпичах должны быть именно эти имена и числа. Но ученым нужно строить. И они делают кирпичи. Приходится одни имена и годы признавать истинными, другие ложными. Таков научный подход – отставив сомнения в сторону, возводить лишь одно здание. Вместо нескольких, поменьше высотой, зато с разными наборами кирпичей.

Основание огромного здания пересмотру не подлежит. Замена разрушит всё здание. Общепринятые представления об истории это цемент научных работ. Изъятие цемента приведет к распаду.

Если хочешь быть ученым в среде официальной науки, необходимо участвовать в общем строительстве единственного и нерушимого её храма. Делай новые кирпичи, клади цемент, возводи ряд за рядом новый этаж. Не подходит кирпич? Выкинуть! И не сомневайся. Правда за нами!

Вошедшее в «Приложения» Гилярова – выкинуто. Этого будто нет. Не должно быть. Иначе научные труды – в последние годы ставшие смесью взаимных ссылок – потеряют целостность, а журналисты будут вынуждены делать трудный выбор, что же писать о незнакомом предмете.

Официальная наука история и ее противники, та же «новая хронология», больше тратят сил насаждая истину, нежели выясняя оную. Утверждают – было так, а не иначе. Или, еще проще – было так. Без иначе. Умолчать. Когда не получается, обязательно указать – ошибочное мнение.

В семидесятых наконец издали первую часть «Пространной истории города Киева с топографическим его описанием» Берлинского, которую при жизни он в печати не дождался. Добро, ну так опубликуйте с уважением. Читаю.  Вот Берлинский пишет, вероятно следом за Татищевым, что вместо Аскольда и Дира был один человек, «Аскольд тирарь». Сразу примечание редактора – Берлинский конечно ошибается!

Наука, словно ползающий на коленях Хома Брут, чертит по полу колдовской круг с шептанием – Берлинский ошибается, новгородские книжники врут, этой дате безусловно нельзя доверять. И перестает быть наукой, ибо наука это изучение, а изучение прекращается, когда нет сомнений.

Но внутри науки, коли хочешь в ней обретаться, не остается ничего другого, как принимать ее правила игры. Не зря Гиляров вспоминал в своих заметках\footnote{Что печатались из номера в номер «Русского Архива» за 1904 год.}, наставление:

\begin{quotation}
«Не всегда говори, что мыслишь; знай больше, а говори меньше» и т.д. Достоинство этих истин оценено было мною, конечно, уже много лет спустя по их изучении.
\end{quotation}

Сын священника Московского Новодевичьего монастыря\footnote{Фамилию Гиляров священник получил от начальства духовного училища, как прозвище – в переводе с латыни «hilaris» значит «веселый». Александр Петрович Гиляров – старший брат Никиты Петровича Гилярова-Платонова.}, Федор Александрович Гиляров учился поначалу в Московской духовной семинарии, откуда был вытурен за статью «Материалы для физиологии общества»\footnote{«Московские Ведомости», 1859, № 62.} и пошел по светской стезе, поступив на историко-филологи\-ческий факультет Московского университета\footnote{Там же учился историк Ключевский, они с Гиляровым были приятелями еще в бытность последнего семинаристом.}. 

Окончив курс в 1866-м, пятнадцать лет преподавал русский язык в различных учреждениях, попутно выпуская одну за другой работы по истории, языку, краеведческие исследования. Среди них наиболее известным трудом была «Этимология Русского языка», составленная совместно с Александром Ивановичем Кирпичниковым. Перу Гилярова принадлежат также «Этимология Церковнославянского языка», «Русская хрестоматия для низших классов гимназий», «Исторические и поэтические сказания о Русской земле в хронологическом порядке событий», азбука для сельских школ «Школа родного языка», «15 лет крамолы» и другие. Всё это нынче днем с огнем не сыщешь.

Гиляров умел, кажется, выбирать самые неудобные предметы для своих работ. Названия его трудов порой даже боялись верно писать в библиографиях – искажали и название, и год выхода в печать, чтобы сбить с толку цензуру.

В 1883 году Гиляров выпустил, на основе своих статей в «Современных известиях»\footnote{Оные издавал дядя Гилярова, Николай Гиляров-Платонов. Федор Гиляров состоял там вторым редактором. Оба Гилярова слыли, по словам Владимира Алексеевича Гиляровского, людьми «не от мира сего». В типографии «Современных известий» вышли и «Приложения русской начальной летописи».}, книгу «15 лет крамолы (4 апреля 1866 г. – 1 марта 1881 г.)» о революционном движении в России. Отпечатанный трехтысячный тираж первой части первого тома – «Пропаганда. (4 апреля 1866 г. – 24 января 1878 г.)» – был запрещен спохватившейся цензурой под предлогом «исключительной важности предмета книги и его щекотливости»\footnote{См. Сводный каталог русской нелегальной и запрещенной печати. М., 1971. Ч. I. С. 149.}. Название оказалось пророческим. 15 лет книги лежали в закромах под арестом, от сырости 2500 штук истлели, оставшиеся поныне считаются редкостью.

В том же 1883 году выйдя в отставку, Гиляров начал печатать свою газету, «Афиши и объявления», позже переименованную в «Вестник литературный, политический, научный и художественный». Издание освещало большей частью театральную жизнь. Современник с похожей фамилией, Владимир Гиляровский вспоминал в «Москве газетной», что «Федор Александрович писал недурные театральные рецензии, а затем сам издавал какой-то театральный листок, на котором прогорел вдребезги».

Скончался Гиляров в Химках под Москвой, в 1895 году. Похоронен в самой Москве на Пятницком кладбище.

Зачем я рассказал о нем относительно подробно? Дабы за отсылками к источнику, важному в моей книге, стоял человек, а не имя. И если о человеке нынче мало помнят, стоит напомнить, что был такой, и каким он был.

Гиляров вытащил из архивов в печать наиболее яркие «отклонения» от общеизвестных списков\footnote{Подобное, но в меньшем объеме, либо пересказе, и с выраженной оценкой можно найти еще в работах Михаила Николаевича Тихомирова (1893-1965), например в  его «Кратких заметках о летописных произведениях в рукописных собраниях Москвы», изданной в 1962 году.} – то, что ученые предпочитают, если знают, замалчивать. А с несведущих и спроса нет. «Приложения» могли бы перевернуть представления о ходе истории. Но куда там! Эта книга ведь целая дорога, усыпанная камнями, о которые каждый споткнется, коли не захочет обойти.

Науке бы чего попроще! Гладкий асфальт. Но помимо редких списков, даже основные летописи задают такие загадки, ответы на которые разрушат здание современной науки истории. Посему, да по неведению, эти вопросы обычно не поднимают. А мы поднимем. 
