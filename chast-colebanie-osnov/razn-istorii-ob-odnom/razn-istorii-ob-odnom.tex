\chapter{Разные истории об одном}

Мы знаем о прошлом по источникам. В разных летописях одни и те же события могут быть описаны различно. Как именно? В иной последовательности. С другими именами. Отличаясь сюжетами. Мы не были очевидцами описываемых событий и не знаем, где правда. Может в чем-то врет летопись А, может летопись Б, а возможно врут обе, независимо друг от друга либо основываясь на третьей, ложной летописи.

Относительно основания Киева Кием и его родственниками. Существует, по большому счету, два изложения событий. По киевским летописям и по другим. С давних пор «киевские» признаны за честные, а расхождения с ними автоматически попадают в разряд выдумок.

Я уже рассказывал о книге Гилярова «Предания Русской Начальной летописи (по 969 год). Приложения», полной нестыковок с общепринятыми положениями. Оттуда, а также из выпусков Полного Собрания Русских летописей, и будем черпать сведения. По книге Гилярова главные варианты хода истории явственно отслеживаются.

Хронология – привязка событий к датам – штука запутанная. Но воспользуемся хронологией относительной, то бишь последовательностью перечисления событий. Без дат. Иван пошел в поле. Марья съела яблоко. Тимофей косил траву. Или иначе. Сначала Марья съела яблоко, потом Тимофей траву косил, а Иван в поле пошел. Как же было на самом деле?

По киевской версии истории, принятой академической наукой за истинную, события, согласно летописцу Нестору, в кратком изложении развивались так:

\begin{enumerate}
\item Козаре брали дань с Полян, Северян, Вятичей. А заморские Варяги взымали дань с Чуди, Славян, Мери, Веси, Кривичей. Эти данники прогнали Варягов, однако начались междоусобицы, и для восстановления порядка призвали Варягов Русь. Править вызвались три брата: Рюрик, Синеус, Трувор. Поделили между собой земли да стали княжить. Рюрик в Новгороде, Синеус на Беле озере, Трувор в Изьборске.

\item Поляне братья Кий, Хорив, Щек и сестра их Лыбедь пришли на холмы современного Киева и поселившись тут, основали «град», назвав его по имени старшего из братьев. В некоторых списках прибавляется, что роды братьев «бяху же погани, жряху идолом в колодезем и озмом крощению, яко протчии погани». Кроме известной трактовки, что погани (язычники) приносили жертву идолам в колодцах (водных криницах), возможен иной вариант – поклонялись идолам в подземных ходах. По Далю, колодец это не только водный резервуар, но и «узкая и глубокая яма; рудная дудка, спуск в землю». Заметим это и вспомним в части о Логове Змиевом.

\item «Два мужа», «не племени его\footnote{Не родственники Рюрику.}, но боярина», отпросились у Рюрика в Царьград с родом своим. Дошли до «города» – огражденного поселения – на горе. Спрашивают, чья это крепость? А им в ответ, что тут были три брата – Кий, Щек, Хорив – построили город и погибли, а мы их род, платим дань Козарам. Аскольд и Дир (они-то, надо полагать, и есть упомянутые «два мужа») остаются в Киеве, зовут еще дополнительных Варягов и начинают владеть «Польской землей» (Полянской землей). Рюрик продолжает княжить в Новгороде.

\item Аскольд и Дир (почему-то всюду упоминаемые именно так, сначала Аскольд, потом после «и» Дир), идут воевать на Царьград, но терпят поражение.

\item Рюрик, умирая, передает княжение родственнику, Вещему Олегу, поручив ему также малолетнего сына своего Игоря.

\item Олег собирает войско из Варягов, Чуди, Словенов, Мери, Веси, Кривичей. Захватывает Смоленск, садит там своих людей. Идет вниз по Днепру, с «младенцем Игорем» прибывает в Киев, обманом убивает Аскольда и Дира, делает Киев столицей, княжит здесь до возмужания Игоря и много после.
\end{enumerate}

По летописям другого толка дела в любопытном для нас отрезке времени обстоят иначе:

\begin{enumerate}
\item По смерти Рюрика, в Новгороде княжит Олег. Там же обитает беспокойное, разбойное семейство: Кий, Хорив, Щек, да сестры их Лыбедь и Лямень. Они местные, либо прибыли с нынешней Киевщины. Оных разбойников вместе с родом их, тридцать человек, на несколько лет заточают в поруб (деревянную тюрьму), а потом решают повесить, но Кий упрашивает Олега отпустить их. Мол, мы уберемся подальше и будем везде насаждать твою власть. Олег соглашается. Разбойники два месяца «идут дебрием» до гор над Днепром, где и останавливаются. Братья в честь старшего создают городок (огороженное поселение, крепость) – Киев или, в одном списке – Киевец.

\item Олег отправляет Аскольда и Дира послами в Царьград, с дарами. По пути, послы посещают Киев, в некоторых списках убивают Кия, Хорива со Щеком, и поселяются тут, чтобы править вместо них.

\item Олег узнаёт, что его послы купно с «драгими дарами» для Царьграда осели в Киеве. Олег с Игорем, который (младенец в киевской версии) стоит во главе конного войска, отправляется в Киев. Олег обманом призывает к себе Аскольда и Дира. Те ведутся. Вылетает конница Игоря. Аскольда и Дира захватывают и вешают. По другим вариантам, их казнит Олег, не Игорь.
\end{enumerate}

Одни и те же вехи истории описаны по-разному. Где правда? А в ряде списков Аскольда и Дира посылает не Олег, но Рюрик, затем Рюрик умирает, оставив Олега регентом при Игоре. Другой список смутно повествует, без участия Аскольда, Дира и Олега, что Кий и Щек умирают сами, а вот Хорива (Корева) и его наследников убивает Рюрик, и после этого делит Русу или Роксолянию на Великую со столицей в Новгороде, Малую со столицей Киевом, и так далее:

\begin{quotation}
роксолянию или русу всю раздели на великую, еяже столный град новгород великий, на малую, еяж столный град киев, на красную, еяже стольный град галич, на белую и черную, ихже столний град мстилавль. И сподоблься века своего близ ста лет, наследника по себе остави сына игоря или георга.
\end{quotation}

%, поначалу разделивший с Коревом власть:

%\begin{quotation}
%первый убо от тех князей варяжских рюрик, безопасна и безоружна наехав кореву и сотворе себе единовладетеля, роксолянию или русу всю раздели на великую, еяже столный град новгород великий, на малую, еяж столный град киев, на красную, еяже стольный град галич, на белую и черную, ихже столний град мстилавль. И сподоблься века своего близ ста лет, наследника по себе остави сына игоря или георга.
%\end{quotation}

И всё это источники, которым надо просто верить или нет. Ведь никаких доказательств не сохранилось. От выбора источников зависит наше понимание истории.

Если с «призванием Варягов» я еще могу согласиться, то захват Олегом Киева, о чем мы еще много будем говорить в части про Вольгу, в любом случае дело мокрое, а подробность про убиение Рюриком Хорива с наследниками кажется мне весьма важной. 

Ведь согласно Нестору, Кий, Хорив и Щек это были князья Полян. А Рюрик и Олег представляли и вели за собой другой народ, Варягов Русь\footnote{В списках смутно прослеживается еще одна жертва на пути к власти – новгородский князь Вадим: 

\begin{quotation}
6377. В сии времена Славяне бежали от Рюрика из Новагорода в Киев, зане убил Вадима Храбраго, князя Славенского, иже не хотеша яко рабы быти Варягом.
\end{quotation}

Славяне, не желая быть рабами Варягам, бежали от Рюрика, который убил Вадима Храброго, князя Славенского.}. Произошла смена власти, при которой место правителей здешних, законных, заняли пришлые. Стоит принять за правду сообщение про уничтожение Хорива с потомством, и всё принимает другой оборот.

И поди догадайся, то ли в списке чудом уцелело обличение преступления, то ли оно было внесено туда нарочно, дабы опорочить Рюрика.

Перед дальнейшими рассуждениями замечу, что многие летописные датировки привязаны к царствованию в Константинополе Михаила III (прозванного Пьяницей). В истории он прославился тем, что на пирах устраивал соревнования по испусканию газов, щедро награждая победителя. В византийских хрониках император изображается с нимбом.

Как же разобраться в летописных событиях? Для их сопоставления хочется иметь даты твердые и незыблемые. Но трогать даты бессмысленно – по причинам, изложенным в главе про летосчисление. Любая из летописных дат может быть не просто ошибочной, однако итогом нагромождения ошибочных вычислений. Но давайте попробуем, что выйдет, если принять на веру числа, признанные наукой.

Мы еще будем говорить об Аскольде и Дире подробно, пока же узрим путаницу в летописи. «Повесть временных лет» (по Ипатьевскому списку) сообщает:

\begin{quotation}  
В лето 6360, индикта 15, наченшю Михаилу цесарьствовати, нача ся прозывати Руская земля. О сем бо уведахом, яко при сем цесари приходиша Русь на Цесарьград, якоже писашеть в летописании грецком.
\end{quotation}  

Что в переложении на язык современный значит:

\begin{quotation}  
В год 6360 (852), индикта 15, когда начал Михаил царствовать, стала прозываться Русская земля. Об этом известно, ибо при сем царе приходила Русь на Цесарьград, как пишется в летописи греческой.
\end{quotation}  

Какая именно Русь приходила? Трактуют как войско наше, киевское. 

Греческие (Византийские) хроники в самом деле пишут о нападении Руси на Царьград, но без имен. Просто некая Русь, и дата приведена – 18 июня 860 года. Однако по описаниям византийцев и нашего летописца, которые я здесь для сокращения опускаю, речь идет таки об одном, на Царьград напали Аскольд с Диром. И у нас уже есть две даты этого события. Если только летописец не переписал подробности из византийского источника, решив, что там сказано именно про Аскольда и Дира.

Но вот летопись сообщает о походе Аскольда с Диром отдельно, через 14 лет после начала правления Михаила:

\begin{quotation}  
В лето 6374. Иде Асколд и Дир на Грекы, и приде в 14 лето Михаила царя.
\end{quotation}  

то бишь:

\begin{quotation}  
В год 6374 (866). Шли Аскольд и Дир на Греков и пришли в четырнадцатый год царствования Михаила.
\end{quotation} 

Применим к гуманитарной науке истории точную науку математику. Добавим к летописному началу царствования Михаила 14 лет: 852+14=866. Это уже третья датировка похода Аскольда с Диром на Царьград.

В ином списке даты вообще другие – начало «земли Рускои» смещено в 854 год:

\begin{quotation}  
В лета шесть тысящ триста шестьдесят втораго (854). Начало земли Рускои. [...]

В сия же времена бысть в Греческои земли, рекше царском граде, именем Михаил царь и мати его Ирина, иже проповедает поклонение иконам в первую неделю поста. [...]

При сем же цар паки приидоша Русь на Царьград въ кораблех, безчисленно кораблеи, 200; вшедше в Суды много зла сотвориша Грекам и убииство велико христианом.
\end{quotation}  

А тут, быть может, речь идет не об «Аскольде и Дире»? Гиляров приводит \cite[стр. 102]{gilyarov01} такие сведения из другого списка:

\begin{quotation}
Лета 6360 (852) ходили Русь воиною из Новаграда князь именем Бравалин воевати на Греки на Царь град и повоеваша греческую землю от Херсона и до Коруева и до Сурожа и около Царя града и грады многие и имения бесчисленно взяша. О том же писано в чюдесех Стефана Сурожского.
\end{quotation}

Верно, события отмеченные в летописи упомянуты и в «Житии Стефана Сурожского». А именно в древнерусском его варианте, в разделе «О прихождении ратию к Сурожу князя Бравлина из Великого Новаграда».

У Гилярова приводится несколько разночтений похода новгородца Бравалина (Бравелина, Блавлина) на Царьград, в год начала царствования Михаила, а не в четырнадцатый его правления, как сказано про Аскольда и Дира. Причем Сурож (ныне – Судак) новгородцы брали десять дней, и наконец сломав железные ворота, ворвались в город. Где кроме прочего ограбили церковь Софию. 

Там покоился, в богато украшенном гробе, святой Стефан. После разорения гроба, князь Бравалин «в том часе разболеся, обратися лице его назад и лежа пены источаше». Кстати это далеко не единственный случай, когда попытки ограбить некоторых покойников, почитаемых за святых, карались странным образом – много такого описано в Патерике Печерском. 

Далее князь, чтобы поправиться, просит своих бояр вернуть всё награбленное, вывести войско из города, наконец принимает крещение, вынуждаемый к этому одному им зримым «великим человеком», «стар мужем», «святым». 

Вот подробности по одному из списков – есть более внятные для современного читателя, зато этот написан языком сильным:

\begin{quotation}
По смерти святого мало лет миноувшю\footnote{А полагают, что Стефан Сурожский, архиепископ Сурожа-Сугдеи, жил и умер в 8 веке.} приде рать велика из новагорода роусскаа князь бравлин силен поплени от корсуня и до керчева и с многою силою приде к сурожу за 10 днии приде бравленин с силою изломив железнаа врата и вниде в град и взем меч свои и вниде в святоую софию избив двери вниде идеже гроб святаго стефана и на гробе его царское одеяло женчюг и злато и камение драгое и кандила златыи ино много и пограби все и в том часе обратися лице его назад. пад тоу князь пены точаше и воспи глаголя велик человек есть сде и оудари мя по лицю. и обратися лице мое назад и рече боляром своим възвратите все назад что поимали и възвратиша и и хотеша князя подъяти от земля и взопи глаголя недейте мене да лежу изламати мя хощеть един стар муж притисноу мя к земли и давить душа ми изыти хощеть и рече им борзо выжените рать из града, да не возмоуть ничего и рать отстоупи от града а он еще не вста а что взяша сосуды церковныя в корсоуни и в керчи принесите семо и положите на гроб и положиша и рече святый аще не крестишися в церкви моей. то не изыдеши отсюдою. и взопи князь глаголя да приидуть попове и крестять мя. аще въстаноу и лице мое обрати ми ся и крещуся. прииде архиеписком филарет и ереи с ним молитву створше и крестишя и во имя отца и сына и святаго духа и обратися лице его опять и крестишася вси вящьни его и попове рекоша князю обещаися богоу от корсуни и до керча что если взял моужа и жены и дети, и повеле възвратити все. тогда повеле князь всем своим вся отпустити и идоша кождо в свояси за неделю не изыди ис церкви дондеже исшед дар дав велик святомоу и град его чтив попов и людии отиди. Се слышахоу инии не смехоу поити нань аще кто наидяше и посрамлен отхожаше.
\end{quotation}

Ученые не любят говорить о походе Бравалина. А ежели кто упоминает неудобного официальной истории предводителя новгородцев, русского князя Бравалина, то ради повода начать игру в историю альтернативную. Где и Новгород не Великий Новгород, и Бравалин не Рус, но Француз, Угр, варяг или даже семит, хотя как и «варяги», «семиты» – понятие размытое и собирательное.

Его придумал, для обозначения языков, близких к еврейскому ветхозаветному, тот самый Людвиг Шлёцер, который Славян называл дикарями. Позже слово перешло на обозначение народов. Имея, по большому счету, смутное представление о давних народах Ближнего Востока и частично Малой Азии, их записали в «семиты», точно также как народы Балтики – в «балты». Даже языковеды признают, что, допустим, «семиты» Лидийцы и Эламиты говорили на языках вовсе не «семитской» группы.

Историки взяли на вооружение библейскую хронологию, положив ее в основу научной. Ученые, на библейском же предании о принадлежности восточных стран Симу, сыну Ноя, образовали понятие о «семитах». 

А как быть с теми народами – причем описанными в Библии – которые поклонялись другим богам, не Шаддаю (Яхве), но Дагону и Астарте (Иштар)? У них были свои представления об истории, лишенные Ноя и Сима. Довольно почитать хотя бы мифы жителей Вавилона и Ассирии. Там свой Всемирный потоп, свои «ковчег» и спасенные – Пир-напиштим из города Шуриппака, с экипажем плоскодонного судна, созданного под руководством бога Эа. Над устроением же Потопа старались боги Шамаш да Нинип, а Иштар скорбела по уничтожаемому древнему народу.

Ученые ввели в обиход термин, обозначая им понятия, противоречащие определению термина! И народы, у которых в мифах нет Ноя и его сына Сима, относят к семитам.

Но про Бравалина! Опять плодится множество версий. Выбирай какая глянется. Хочешь – еще до «Аскольда и Дира» новгородцы, в качестве «Руси», могли ходить на Греков. А потом крестились! Хочешь – Бравалин не новгородец. Можем вообще закрыть на него глаза, пусть останутся только Аскольд и Дир.

Но у меня нет оснований делать выбор между источниками, которые невозможно проверить. Даты – побоку. Как хотя бы определить, что же происходило? Кто из Руси нападал на Царьград – Аскольд с Диром или Бравалин? Либо все, но в разное время? Или действия одних приписаны другому, а может наоборот?

Историки нашли очень простой путь. Не искать истину, но утверждать ее. Всё, что выпадает из навязываемой ими модели, объявляется ошибками переписчиков, переводчиков и выдумками книжников, хотя ошибаться и выдумывать могли те, кого считают правдивыми.

Если не получается, то ученые трактуют изощреннейшим образом. Один на шахматной доске глядит на коня, думает-думает, потом говорит – да это же осел! Вон морда длинная. Другой берет фигурку и, перед собой покрутив, ставит на место – слон. Определенно слон. Шея толстая. Но третий ученый муж, наблюдая за партией, молвит задумчиво – этого не может быть. Ни доски, ни коня, ни прочих фигур. Выдумка новгородского книжника. Вам всё кажется!
