\chapter{А так говорят грамоты}

Век двадцать первый, государство Литва, расположено на южном побережье Балтийского моря. Столица Литвы – город Вильнюс. Официальный язык – литовский, относимый к балтийской группе. Состоит из двух говоров – жемайтского и аукштайтского.

Века давние, Великое княжество Литовское, включает в себя земли нынешней Литвы, Беларуси, Украины. Столица княжества – город Вильнэ, Вильна, что буквально значит «вольная». Официальный язык – «руский» (с одной «с»), как он назван в тогдашних документах.

Земли государства, которое мы привыкли называть Киевской Русью, после ослабления в княжеских распрях, попали в зависимость к Орде. Затем, в 14-16 веках, вошли в состав Великого княжества Литовского, позже – Речи Посполитой.

Тартары были изгнаны из Киевской Руси в 1320 году объединенным воинством Русов и литовского князя Гедимина. Несколько позже Литовское княжество посредством браков соединилось с меньшим по размеру Королевством Польским. Влияние Польши постоянно увеличивалось, покуда в 1569 году Литва и Польша не срослись в то, что ныне именуют Речью Посполитой.

Документы – кровь государства. Оборот документов – циркуляция крови в его организме. Осталось множество, а еще больше утрачено, бумаг того времени – приказы, купчие, жалованные грамоты, законы. Писались они на нескольких языках – на латыни, польском, а во время сильного княжества Литовского – на «руском», с одной «с».

Об этом руском языке теперь много спорят. Белорусские историки величают его «старобелорусским», украинские – «староукраинским» или «древнерусским», россияне – то «северо-русским» и «западно-русским». Одни ученые доказывают, что это был чисто письменный язык, другие указывают на второе его имя – «проста мова», то есть народный, обиходный язык, третьи возражают, будто литовские князья не могли на нем говорить, а общались на другом, «балтийской группы».

Язык общения князей оставим в стороне. На руском, в Великом княжестве Литовском – в Вильне и в Киеве – составлялись завещания, писались протоколы, велись учетные книги, подавались жалобы. Огромный объем документов, известных как Литовская Метрика – по сути, архив канцелярии Великого княжества Литовского – написан большей частью на руском.

На нем был создан в 16 веке и свод законов княжества, Литовский статут, переведенный затем на польский. Однако статут и юридическая терминология Литвы вобрала много польской, а Поляки пользовались в юриспруденции латинскими выражениями, поэтому литовский «юридический» язык это такой сплав славянского и ополяченной латыни, причем чем дальше, тем больше.

Польские чиновники, с усилением влияния Польши, противодействовали использованию руского языка как официального, ибо понимали его всё хуже. Они хотели документов на польском или латыни, и хотя поначалу от этих нападок на язык удавалось отбиваться указанием на традицию – «листы писаные и выдаваны руским писмом и языком по всему панству его королевской милости великому князству литовскому», на Варшавском сейме 1696 года «руска мова» утратила статус официальной.

Неясно, в какой мере этот руский язык Литовского княжества соответствовал разговорному народному языку – говорам Литвы, Белоруси и Украины – поскольку мы имеем дело с письменными документами. Ведь если судить по польским латинским документам, то истолковывая это как прямое указание на язык народный, следует заключить, что Поляки поголовно говорили на латыни. Очевидно однако, что руским языком в княжестве Литовском пользовались и знали его хорошо.

Давайте ощутим вкус этого языка, тем более что далее в книге мы будем часто встречаться с источниками на нем. Пример из свода законов – Литовского статута, писано от имени великого князя\footnote{Здесь и далее для удобства чтения документов княжества Литовского буду отбрасывать твердые знаки, что ставились после согласных в конце слов (как в царской России), а порой и внутри. Также заменяю «і» на «и», а в письме московского князя, «іа» на «я»}:

\begin{quotation}
21. Хто бы новые мыта уставлял.

Тэж приказуем, абы жадин чоловек в панстве нашом – Великом князьстве Литовском – не смел новых мыт вымышляти ани вставлять ни на дорогах, ани на местех, ани на мостех, и на греблех, и на водах, ни на торгох в именьях своих, кром которые были з стародавна вставленые, а мели бы на то листы продков наших, великих князей, або наши. [...]
\end{quotation}

Этим приказываем, чтобы ни один человек во владениях наших – Великом княжестве Литовском – не смел новых мыт (пошлин) измышлять и устанавливать ни на дорогах, ни в городах, ни на мостах, ни на греблях (плотинах), ни на водах, ни на торгах (рынках) в имениях своих, кроме тех мыт, которые были издавна установлены, и подтверждены листами (документами) предков наших, великих князей, либо нашими листами.

Вполне понятный славянский язык. И это писалось вполне славянскими буквами, что зовутся кириллицей, а числа употреблялись тоже буквенные, как в «старославянском». Многое в этом языке было как в «старославянском» – слова с согласными на конце завершались твердым знаком, те окончания, где мы пишем «ях», здесь и в старославянском – «ех», то бишь «на санех» вместо «на санях». Однако видим и развитие письменное, если сравнивать с летописями – слова разделяются пробелами, хотя нет еще запятых, а точек мало. Но ведь летописи старше! Язык их старше! Я усматриваю в языке литовских документов развитие того же языка, который прежде был запечатлен в летописях.

Однако, это не просто развитие с течением времени, но развитие, испытавшее влияние местных говоров, ведь славяне и ныне в разных местах говорят с отличиями даже в пределах одной страны.

Еще один пример – выдержка из договорной грамоты Литовских князей Евнутия, Кестутия и Люберта Гедиминовичей, Юрия Наримундовича и Юрия Кориатовича, с Польским королем Казимиром, с Мазовецкими князьями Семовитом и братом его Казимиром, сыновьями Тройдена. Дата неизвестна – но после 1340 года. Итак, межгосударственный договор правителей. Я буду последовательно излагать его и толковать с руского на русский:

\begin{quotation}
Ведаи то каждый человек, кто на тый лист посмотрит. Оже и князь юунутий, и кистютий, и любарт, юрий наримонтович, юрий кормиантович.

чинимы мир твердый. ис королем казимиром польским. и сомовитом и с братом казимиром мазовскым, и с его землми, краковьскою и судомирскою, сирязьсклю, кумвьскою, лучичскою, добрыньскою, плотьскою. мазовскою, люблинскою, сетеховьскою, и со львовскою.

а за великого князя олькерта и за кориата и за патрикиа, и за их сыны, мы ислюбем, тот мир держати велми твердо, безо всякое хитрости.
\end{quotation}

Доводится до сведения каждого, кто на этот документ посмотрит. Далее перечисляются участники договора. Утверждаем мир твёрдый, с такими-то, и подвластными ему землями сякими-то. Мир держать очень твердо, без всякой хитрости.

\begin{quotation}
не заимати нам королевы земли, ни его людий што его слухають, королевы держати львовскую землю исполна, а нам держати володимерскую, луцкую, белзьмкую, холмьскую, берестинскую, исполна жь.
\end{quotation}

Не захватывать нам (литовским князьям) земли польского короля, ни его подданных. Королю – целиком владеть львовской землей, а нам – володимерской, луцкой и прочими, тоже полностью.

\begin{quotation}
а мир о покрова богородицы до ивана дне до копал, а о ивана дне за 2 лет,
\end{quotation}

Мир заключается от праздника Покрова Богородицы до Иванова дня, а после него два года.

\begin{quotation}
а городов оу руской земли новых не ставити, ни сожьженого не рубити, доколя мир стоит, за 2 лет, а креманец держати юрью наримоньтовичю от князий литовскых, и от короля, за 2 лет. а города не рубити. а коли мир станет, юрью князю города лишитися.
\end{quotation}

Пока держится мировое соглашение, два года, в руской земле нельзя ставить новых городов (крепостей), и сожженных не восстанавливать. Креманец два года держать совместно Юрию Наримонтовичу от литовской стороны, и от короля. А по завершении мирового соглашения, князь Юрий лишается права на «держание» города Креманца.

\begin{quotation}
аже пойдет оугорьскый король на литву, польскому королеви помагати, аже пойдет на русь што литвы слушает, королеви не помагати.
\end{quotation}

Если пойдет угорский (венгерский) король на Литву, польский король должен помогать Литве. Но если угорский король пойдет на Русь, подчиненную Литве, то польский король не должен вмешиваться.

\begin{quotation}
а поидет ли царь на ляхи, а любо князи темний, князем литовьским помагати.
\end{quotation}

Если нападет царь («а любо князи темний» – либо темный князь) на поляков, литовские князья должны помогать полякам.

\begin{quotation}
аже пойдут на русь што короля слушает, литовьским князем не помагати.
\end{quotation}

Но если нападут на Русь, подчиненную польскому королю, литовские князья не помогают Польше.

\begin{quotation}
а про любарство ятство, хочем его поставити на суде перед паны оугорьскими, по сошествии святого духа за 2 недели, литовским князем стати оу холме. а королеви оу сточьце, кде смолвять тут будет суд,\end{quotation}

А вопрос про держании в заложниках Любарта (он же Люберт, сын Гедимина) хотим поставить на суде перед правителями угорскими, через две недели после праздника Сошествия Святого Духа. Литовским князьям стать на холме, а польскому королю в «сточьце», где договорятся там и состоится суд.

\begin{quotation}
тягатися ис королем. будет ли ял его король по кривде, любарт будет прав, и я князь кистютий буду прав перед вгорьскимь королем, будет ли король прав, нам своего брата любарта дати оугорьскому королеи оу ятьство.
\end{quotation}

Короче говоря, если правда на стороне польского короля, Любарт пойдет заложником к угорскому королю.

\begin{quotation}
а коли будет по миру, кто не оусхочеть далей миру держати, тот отповесть, а по отповеденьи стояти миру за месяц.
\end{quotation}

Если во время перемирия, кто-либо не захочет его далее придерживаться, то должен уведомить об этом, и после уведомления соблюдать мир еще месяц.

\begin{quotation}
аже пойдут татарове на львовскую землю, тогда руси на львовьце не помагати. аже пойдуть татарове на ляхе, тогда руси неволя пойти и с татары.
\end{quotation}

Если пойдут Татары на львовскую землю, Русь не должна помогать львовцам. Если пойдут Татары на Поляков, Руси нельзя присоединиться к Татарам.

\begin{quotation}
а оу том перемирьи кто кому криво оучинить, надобе ся оупоминати старейшему, и оучинити тому и .. (пропуск) праву, оучинить которыи добрый человек кривду, любо воевода, а любо пан, оучинити исправу из ним.
\end{quotation}

Если в том перемирии кто кому обиду нанесет – надо его имя назвать старейшему, и вынести ему приговор по закону. Далее уточняется, что «старейший» это воевода либо «пан», словом начальство. 

\begin{quotation}
аж сам не может заплатитить тот истиньный. што же оуложат его оу вину, хочет ли сам король заплатити на нь, а его дедичество себе оузяти.
\end{quotation}

Если осужденный не в состоянии заплатить положенное приговором, то король может заплатить вместо него, а наследное имущество приговоренного взять себе.

\begin{quotation}
не оусхочет ли король сам заплатити, даст тому то дедичество, кто его потяжет.
\end{quotation}

Ежели король не хочет сам заплатить, то отдаст имущество ответчика истцу.

\begin{quotation}
а за избега можем его добыти и выдати. аже его не можем добыти, можем его искати с собою сторону. аже побегнет русин а любо руска, или во львов, или холоп чии, или роба, выдать его.
\end{quotation}

А за уклонение, побег, можем его (ответчика) поймать и выдать. Если не можем его поймать, будем искать. Если бежать будет русин (или руский – а любо руска), или во Львов, или же убежит холоп чей или раб, следует его выдать (вернуть).

\begin{quotation}
а што в тои грамоте писано, тую ж правду литовьскым князем держати. а на то есмы дали свои печати.
\end{quotation}

Что в этой грамоте записано, так закон соблюдать литовским князьям, и это подтверждаем своими печатями.

Но одни народы и языки взяли верх над другими, город Вильна стал Вильнюсом. А еще в 16 веке, когда Вильнюс был Вильной, король Жикгимонт (Сигизмунд) выносит приговор по спору городничего виленского Урлиха Гоза с Яновой Катериной. Вот документ, сохранившийся в Литовской метрике. Это подлинник:

\begin{quotation}
Вырок Улриху Гозу, городничому виленскому, з мещанкою виленскою Яновою Катериною о част дому на Немецкои улицы, куплю его у матки ее.

Жикгимонт

Смотрели есмо того дела.

Стояли перед нами очывисто, жаловал нам городничыи виленскии, минцар наш, пан Улрих Гоз на мешчанку виленскую Яновую Катерыну в том, штож деи он купил в матьке ее часть дому ее в месте н(а)шом Виленском на Немецкои улицы, и мы часть того дому потвердили ему н(а)шым прывилем водле купли его, и на то он прывилеи наш перед нами вказывал. А тая деи Катерына тую част в мене отняла и держыть оть колка леть, нет ведома которым обычаем. И тая Катерына перед нами мовила, иж она тому не сведома, которым обычаем он то купил.

Ино мы городничого пры том потвержени н(а)шом зоставили, нижли маеть он лист купчыи на часть того дому положыти на ратушы перед воитом и бурмистром, и радцы места Виленского. А воить з бурмистры и радцы мають то межы ними розознати подле части купли его, и водле права их маитбарского.

Псан у Вилни, под лет Бож нарож 1000 пятсот 22 мсца нояб 29\footnote{29 ноября 1522 года.}. Индик 11.
\end{quotation}

Городничий Урлих Гоз купил у матери мещанки Катерины Яновой часть дома на Немецкой улице, и Жикгимонт подтвердил сделку привелеем, предъявленным позже Гозом. Но Катерина ту часть дома у Гоза отняла и непонятно по какому праву (закону) удерживает несколько лет. Жикгимонту же она говорит, что «иж она тому не сведома, которым обычаем он то купил» – не знает, каким образом Гоз ту часть дома приобрел. Решение таково – если городничий имеет купчую на часть того дома, пусть предъявит ее в ратуше войту и бурмистру, и радцам (советникам) Вильны. А войт с бурмистром и радцами пусть между собой определятся по части законности купли, согласно их магдебургскому праву.

Хорошо, а на каком языке литовские чиновники обращались к чиновникам польским? Снова приведу документ из Литовской Метрики, за 12 октября 1553 года. Не буду уже заниматься его толкованием, просто хочу показать – тот же руский язык, только влияние польского усилилось. Руководство переписчикам польским, со стороны княжества Литовского, об учете скота, выпасаемого на полях королевскими (цесарскими) подданными:

\begin{quotation}
Наука писаром польским, з стороны Великого Князства Литовского пану Венцыславу, секретару, а стороны Коруны Польское пану Станиславу Вороневскому, яко ся мають справовати около пописаванья быдля которое на полях короля его милости подданые цесарские паствити будуть.

Напервеи ожидати того будуть абы сандчак белогородскийи пану воеводе бельзскому, князю воеводе киевскому отказ вчунил на тые листы которие з росказанья короля его милости до него писаны будуть, ознаймуючи ему о постановенье около тых поль межи королем. его милоцтью а цесаром турецким 

А кгды ознаимить, же вжо росказал росказаньем цесарским подданим абы без оповеданья и постанобенья дани от пашни на поля короля его милости не ездили, поедут до Белагорода и там становенье около тое пашни чинити мают старючи ся о то пильне, абы вси тые каторые паствити на полях короля его милости будут, списаными мели, и много которих стад як великих мети будут. А то все списабши до его королевское милости прислати мають.

Там же теж вже и умову вчинят, по чому от ста волов, конеи и овец платити мают абы о то при выбиранью дани трудност не была.

Будет теж того пильне догледати, жебы от подданых короля его милости шкоды жадные не делали ся тым которие умовы о паству вчинят. А которого бы такового дознали, жебы шкоды чинити мел мает его королебская милост писаным своим ознаимити стороста его милоцти бапскоми у браславскомы опобедати

А того напильнеи стеречи маете, абы во всемт згодлыве тую послугу короля его милости оправили пожитку и скарбу его королевское милости стеречи мают 

А остатокь корол его милостцноте и вере их поричити рачит.

Писан у Ломзе лет Бож нарож 1553, мсца ок 12 дня.
\end{quotation}

А вот каков язык письма московского князя Василия III к Жикгимонту в 1526 году, выдержка:

\begin{quotation}
Лист кнзя великого московского перемирныи до шести лет с королем его млстю Жикгимонтом\\

Мы, великии гсдр Василеи, Божею млстью гсдр всея Руси и великии кнзь володимерскии, новгородскии, псковскии, резаньскии, тферскии, югорскии, пермскии, болгарскии и иных. Что прысылал до нас тот брат нш, великии гсдр Жыкгимонт, Бжею млстью корол полскии, великии кнзь литовскии, рускии, кнжа пруское и жомоитскии, и мазовецкии и иных, послов своих, воеводу полоцького, старосту дорогицкого, пна Петра Станиславовича, а подскарбего земского, маршалка своего, старосту слонимского и каменецкого, пана Бгуша Бговитиновича, о миру и о доброй смолве. 

И то межы нас с тобою, братом нашым, з великим гдрем, з Жыкгимонтом, королем и великим кнзем, нне не стало ся. И послы твои, брата ншого, говорыли нам от тебе, брата ншого, от великого гсдря Жыкгимонта, короля полского и великого кнзя литовского и руского, чтобы мы взяли с тобою перемире на шест лет на то, чтобы нам в те перемирные лета межы собя рати и войны не замышляти, а слати нам в тых летах межы собя на обе стороны своих великих послов, которые межы нас то дело могуть делати. [...]

А в те перемирные лета какова учынитца обида меж ншых кнзеи в землях и в водах, и в ыных каких обидных делех, и ншы кнзи и наместьники, и волостели украинныи сослався, да тем обедным делом всим управу учынят на обе сторо­ны. А в каких обидных делех нашы кнзи и наместьники, и волостели не учынят управы и нам о том сослати судеи. И они съехався да тем обедным делом всим управу учынять на обестороне стороны без хитрости.

А татя, беглеца, холопа, робу, должника по исправе выдати; а даное, положоное, заемное, поручное отдати. [...]
\end{quotation}

\newpage

Я усматриваю две ветви развития «летописного», старославянского языка – западную и восточную. Как видим, обе они уже отличались от языка летописей, но поначалу сохраняли большое сходство. Одновременно продолжал использоваться, для церковных нужд, и устаревающий «старославянский», всё более отдаляясь от разговорного народного и светского письменного.

На каком языке говорили в Киеве, допустим, в 15 или 17 веке? На языке, отраженном в документах того времени? Или население, используя разные устные говоры (в Киеве обитали сообщества разных народов), в письменном мире пользовалось общим для княжества Литовского, руским языком? Который кстати не имел четких правил. Можно найти отличия в документах в зависимости от местности. Так или иначе, этот язык понимали даже на межгосударственном уровне. Сохранялась еще общность Славян, и язык общий еще не разделялся на ветви столь существенно, как ныне. Присущие разным славянским землям говоры были более схожи между собой, чем, например, современные русский, украинский и белорусский. 

Знакомясь с историей Киева, мы встретимся и с языком летописей, и руским (с одной «с») языком документов времени Великого княжества Литовского. 

Я не раз сталкивался в книгах с подходом, когда ученый цитату из источника приводит в подтверждение своих слов. Постараюсь действовать иначе, источник делая основой рассуждений. Поэтому разбор какого-нибудь источника порой затянется на множество страниц, ну да мне важно не подтвердить свою мысль, а докопаться до сути. Сначала надо точно выяснить, что хотел сказать например летописец, а уж затем толковать  сказанное им. Смысл слов со временем меняется, надобно всегда возвращаться туда, где и когда они сказаны, и кем сказаны, чтобы понять, о чем идет речь.
