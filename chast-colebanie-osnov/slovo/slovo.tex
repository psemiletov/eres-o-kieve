\chapter{Последний оплот – Слово о полку}

«Слово о полку Игореве» – произведение, почитаемое учеными за древнерусское. В основе его лежат отраженные в летописи события, когда Игорь, сын Святослава, предпринял неудачный поход против Половцев.

Я считаю «Слово» подделкой под старину. Ведь в самом начале его, сочинитель прямо признается в такой стилизации. Вспомним:

\settowidth{\versewidth}{трудных повестий о пълку Игореве,} 
\begin{verse}[\versewidth]
Не лепо ли ны бяшет, братие,\\
начяти старыми словесы\\
трудных повестий о пълку Игореве,\\
Игоря Святъславлича?
\end{verse}

Начяти старыми словесы... То бишь уже для рассказчика они старые! А мне, как читателю, многие вдобавок кажутся еще и несуразно выдуманными.

Неохота мне разбирать выдумку, да придется. Ведь сторонники видеть Боричев увоз в современном Андреевском спуске повторяют, словно молитву, одно предложение из «Слова о полку Игореве»: «Игорь едеть по Боричеву к Святеи Богородици Пирогощеи». Трактуют это просто – мол, Игорь спустился по нынешнему Андреевскому, внизу которого как нельзя кстати церковь Успения святой Богородицы Пирогощи.

В летописи про то ничего не говорится, там коротко – что Игорь бежит из плена, через города Донец, Новгород отправляется к брату Ярославу в Чернигов, и «оттоле же еха ко Киеву к великому князю Святославу, и рад бысть ему Святослав, также и Рюрик сват его». Всё.

Давайте разберемся с данными из «Слова». В самом деле, можно сказать, что внизу Андреевского спуска, а на деле через квартал севернее, у Контрактовой площади, к юго-западу от фонтана Самсон, напротив Гостиного двора, находится церковь Успения святой Богородицы\footnote{Археологическая история места хорошо освещена. Здесь было несколько церквей, одна из которых древняя.}. С легкой руки некоторых исследователей к ея названию добавляют – «Пирогощей».

Между тем, когда возникло краеведение Киева, в 19 веке, то краеведы не знали, какую именно церковь летопись кличет Пирогощей.

А в городе было несколько церквей, посвященных святой Богородице. 1 – Десятинная. 2 – на Печерске (Повесть временных лет глаголит, что при основании Лавры: «и повеле им Аньтоний. Они же поклонишася ему и поставиша церьквицю малу над печерою во имя святыя Богородица Успение», позже в Лавре возник и знаменитый Успенский собор). 3 – на Подоле. 4 – существовала, примерно с 1108 года, и церковь святой Богородицы Влахернского монастыря (так называемого Стефанеча), близ ручья Клов. 

Но в прошлом, обозримом по документам, нигде не называется Пирогощей именно подольская церковь Успения святой Богородицы. И в быту, в прошлые века, никто ее Пирогощей не называл. А потом вдруг ученые решили – это же Пирогоща!

Уши этого мнения растут, пожалуй, из маленькой заметки «Еще одна из древнейших церквей в Киеве» в декабрьском выпуске «Киевской старины» за 1887 год, где сочинитель за подписью «П. Л-в.»\footnote{Вероятно, Петр Александрович Лашкарев (1833-1899), профессор Киевской Духовной академии.} без всяких доводов пишет – «можно полагать», что именно к ней в «Слове о полку» едет Игорь.

Что говорят летописи, по общепринятой хронологии, да с пересчетом на годы «от нашей эры»? Для упрощения изложения не буду отвлекаться на критику датировок.
 
Свидетельство первое, в Лаврентьевском списке – о заложении церкви Мстиславом, сыном Владимира Мономаха:

\begin{quotation}
1131. В то же лето заложи церковь Мстислав святыя богородица Пирогощюю.
\end{quotation}

Свидетельство второе, в летописи Ипатьевской – о завершении возведения церкви:

\begin{quotation}
1136. Томь лете церкви Пирогоща свершена бысть. \end{quotation}

Обратите внимание – Мстислав умер в 1133 году, значит, завершал строительство другой князь. Преемник и брат Мстислава, Ярополк?

Название Пирогоща это прозвище иконы Богородицы, принесенной из Цареграда. В цитате ниже, из Ипатьевской летописи за 1155 год, речь идет о другой иконе, а Пирогоща упомянута к слову. Мол, другая икона и Пирогоща, обе были принесены ранее из Цареграда:

\begin{quotation}
Том же лете иде Андреи от отьца своего из Вышегорода в Суждаль без отне воле, и взя из Вышегорода икону святое богородици, юже принесоша с Пирогощею ис Царяграда в одином корабли, и въскова на ню боле 30 гривен золота, проче серебра, проче камени дорогого и великого жемчюга, украсив, постави ю в церкви своеи святое богородица Володимири.
\end{quotation}

С этого места начинается обычная историческая катавасия. 

Одни исследователи, потрясая Степенной книгой царского родословия, говорят о купце Пирогоще:

\begin{quotation}
принесен бысть к нему\footnote{Князю Юрию Долгорукому?} от Царяграда Пирогощею купьцем Пречистыя Богоматери чудотворный образ.
\end{quotation}

Другие исследователи раскладывают Пирогощу на греческий и спускают воображение с цепи. Тут и «пиро» – огонь, мол, огнегорящая; и «пириос» – башня. Ходит славянская трактовка, от «пир» и «гость». Гощу на пиру – вот и пирогоща. Существовала также греческая известная икона Пирготисс («Башенная» в переводе с греческого), и возможно Пирогоща являлась ее копией.

Кому и кем были принесены иконы первоначально – по спискам летописей идет чехарда с именами. То князю Георгию – вроде бы Юрию Суздальскому (Никоновская летопись), то Мстиславу, который в новгородских летописях слывет как опять же Георгий, хотя более известно другое его христианское имя Федор.

Оставим в стороне эту путаницу, обратимся к другой. Название иконы стало названием церкви или наоборот? Неведомо. Ежели в летописи сказано, мол, такой-то князь заложил церковь Пирогощей, это не значит, что во время заложения церковь так именовалась.

Далее, в двух списках, Ермолаевском и Радзивилловском, в тексте про «свершение церкви», написано не Пирогощею, а «Пироговищею», и некоторые исследователи полагают, что название связано с местностью Пирогово под Киевом. Неподалеку, в урочище Церковщина были развалины Гнилецкого монастыря с пещерами, до середины 19 века забытые, слывущие среди народа как «церковище», что дало повод считать церковь Пирогощи именно там. 

А Михаил Максимович думал, что Богородицы Пирогощей была непосредственно рядом с Михайловским монастырем, слева от Боричева увоза, если смотреть на него снизу вверх.

Что до Успенской церкви на Подоле, у Контрактовой площади – нет никаких указаний на то, кто именно её заложил и когда. Может это та самая Пирогоща, а может нет.

Но Вернемся к Игорю и допустим, что существующая церковь на Подоле – «та самая». «Слово о полку» баит: «Игорь едеть по Боричеву к Святеи Богородици Пирогощеи». В чем тут довод, что Боричев это современный Андреевский спуск? Что мешает Игорю спуститься вдоль фуникулера и свернуть на север? Около семиста метров, даже пешком можно пройти за пять минут.

И если принимать цитату про Игоря прямолинейно, что описываются его основные безостановочные перемещения из пункта А в пункт Б по кратчайшему расстоянию, то давайте, следуя этому же правилу, разберем вопрос – почему считают, что Игорь \textbf{спускался} по Боричеву? Может, он поднимался?

По любому спуску можно подниматься и спускаться. Очевидно, что направления спуска и подъема – взаимоисключающие.

У оврага с фуникулером направление подъема – юго-запад (от стороны Днепра), спуска – северо-восток (к Днепру). У Андреевского, в последней его холмистой части направление подъема – запад (от стороны Днепра), спуска – восток (к Днепру). Для времен летописных можем сделать поправку – не к Днепру, но к Почайне.

Откуда прибыл Игорь в Киев? Из Чернигова, что лежит к востоку от Киева. Как оттуда добирались? В летописях для этого пользуются перевозом киевским, около устья Десны. Наверное затем через Оболонь на Подол. Не на коне богатырском же с подкрылышками сигал. Следовательно, Игорь, направляясь в Киев с левого берега, посетил бы \textbf{подольскую} церковь Богородицы безо всякого Боричева – нет нужды переться на гору, чтобы потом с неё спускаться, если только в этом князь не находил особенную забаву.

Итак, ежели Игорь прибыл с востока, и ехал – по «Слову» – Боричевым, значит, Игорь \textbf{поднимался} на гору, ибо спускаться на запад ни по Андреевскому спуску, ни по оврагу с  фуникулером нельзя. И Пирогоща  – по данным «Слова» – была \textbf{в верхнем} городе. 

«Не Десятинная ли это?» – ухватятся сторонники Боричева-Андреевского. Вот же, Андреевский подходит к Десятинной (допустим, летописная Десятинная стояла в самом деле там), сочинитель «Слова» разумел под Пирогощей именно ее. Но и овраг фуникулера поднимался почти к ней же! Однако, Десятинная не слыла Пирогощей...

Лишено смысла судить о чем-либо по художественному произведению, создатель коего жил за много веков после описываемых событий. На доверии ученых к подобным источникам зиждется основание науки. 

Однако что для определения места Боричева дает «Слово»? Может, там Пирогощей слывет Десятинная, может нет – без разницы, поскольку к Десятинной (ее общепринятому местоположению) ведут, так или иначе, и Андреевский спуск, и фуникулер, таким образом уравнивая вероятность бытности их Боричевым в координатах «Слова».
