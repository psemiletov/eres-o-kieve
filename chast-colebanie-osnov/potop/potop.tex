\chapter{Потоп}

В каком-то сборнике баек про якутских шаманов мне попался рассказ, как шаман рожал через пупок. Верить этому или нет?

Ученые говорят, что был ледниковый период, даже несколько, и во время последнего жили мамонты и дикие люди, охотники на мамонтов. О том, что случилось такое оледенение, геологи судят по следам и показывают их нам. Это различные пески, глины, галька, щебень. Другим следам учеными придуманы названия особые – ледниковая муть, котлы, бараньи лбы.

Правда, в зависимости от научной школы, геологическое строение одного и того же холма может трактоваться по-разному. Например, холмы Киева покрыты суглинком особого рода – наука именует его лёссом\footnote{Немецкое слово «löss» в 19 веке ввели ученые, для обозначения суглинков с определенными составом и строением. У нас его называли желтозёмом, белоглазкой да просто суглинком. Но геологам оказалось мало понятных русских слов, стали пользоваться невразумительным «лёссом», еще и заменили «ё» на «е», получив уродливое подобие нашего «леса». Поди разбери! Слово это глубоко уже въелось в околонаучный язык, я же по возможности буду стараться писать «суглинок».}. Насчет происхождения лёсса существует множество предположений. Кроме прочего, одни говорят, что образовался он от ледника. А другие – лёсс надуло ветром. Обе точки зрения имеют доказательства. И сторонники ледникового лёсса говорят – был такой скандинаво-русский ледник, покрывал земли от Скандинавии до Полтавщины\footnote{По науке, Скандинавия – главный благодетель славянских земель. Дарит им то ледники, то князей.}. Украина вообще богата лёссом, что лежит под черноземом степей да под холмами. Подойдите в Киеве к любому котловану или раскопу на улице, поглядите на слои почвы. Темной земли сверху совсем мало, а ниже идет бурый или светлый суглинок. Обрывистые кручи над Днепром тоже лишь слегонца покрыты землей, под нею – суглинок.

Ледник вёл себя как-то странно. По словам геологов, он то отступал, то снова наступал, подобно обычным волнам, хотя на деле увеличивался либо уменьшался. Ученые разделяют эти приливы и отливы тысячами лет. Ведь принято считать, что природа меняется медленно.

А если вместо ледника были и в самом деле волны? Ведь то, что нам показывают как следы ледника – пески, гальки и прочее – мог принести не ледник, а наводнение. Огромное. Всемирный потоп.

В преданиях, пожалуй, всего мира есть упоминания о великом Потопе. Отражен и в Библии, и в устных рассказах уральских Манси. Приведу несколько мансийских преданий\cite{perevalova01}:

\begin{quotation}
За семь лет до начала потопа шаманам стало известно о приближающемся времени огня и воды. Шаманы били в бубны, гадая о том, как можно спастись. Люди, не умевшие плавать, стали строить плоты (пор). Только семислойный плот (лабыт лаур полет пор), сделанный из семи бревен в семь слоев, покрытый семислойным пологом из кож осетра и стерляди, мог устоять против стихии. 

В какой-то местности, выше Березова, росла священная береза с семью отростками от вершины. Однажды береза та упала и из под ее корней начала бить вода. Люди укрепляли это место, но никак не могли остановить водяного потока. Тогда люди расселись на плоты, и их понесло вниз течением Оби. Женщин и девушек на плоты не брали, их все равно вода с огнем сжирала, спасались только мужчины и «чистые» девочки. 

Семь дней вода кипела – огонь и вода вместе шли. Нижние слои плотов разбивались, верхние слои пологов сносило. Вместе с огненной водой несло множество огромных ящеров\footnote{Отметим кстати болота земли Уральской и многочисленные древние фигурки динозавров, находимые в тех краях.} и змей, которые взбирались на плоты и поедали людей. 

Много народу тогда погибло – те, кто не успел построить семислойного плота, те, кого смыло водой. Когда стихия улеглась, люди стали высаживаться на высоких островках – пугорах. Приплывших на плотах людей называли нобтын ёх – «приплывший народ» или пор ёх – «люди плотов».

[А.П.Кондин, п. Казым-мыс, р. Б.Обь, 1990].
\end{quotation}

Вот еще:

\begin{quotation}
Вода была везде, кроме Урала. Когда вода прибывала, на высоких склонах Урала собирались и люди, и животные – медведи, волки, лисы, росомахи. Целую неделю бушевала вода. Люди и животные вместе жили, и никто никого не боялся. На этой горе собрались разных родов и даже народов люди. После спада воды люди разошлись по разным юртам и стали жить родами. Один зырянин пришел из-за Урала (Нерапса ху) и во время потопа оказался с хантами на одной горе, после чего породнился с ними.

[Т.И.Хунзи (Озелова), п. Унтсыльгорт, р. М.Обь, 1989].
\end{quotation}

У Греков, при помощи Потопа, Зевс уничтожает людей, ставших ему невыносимыми. Помогают в этом Посейдон и владыка ветров Эол. Спасается на плоту лишь сын Прометея, Девкалион с женой своей Пиррой. Они, проплавав 9 дней по безбрежью, пристают к оставшейся над поверхностью воды вершиной Парнаса. «Девкалионов потоп» служит как бы основой, отправной точкой человеческой истории Греции. В греческих преданиях есть сведения и о других потопах, либо том же, но с иными действующими лицами, коим выпадает счастье выжить – например, это Дарданус, сын Зевса и Электры.

У Индусов\footnote{Подробности читайте в «Матсья Пурана», «Бхагавата Пурана» и «Махабхарата».}, бог Вишну через своего аватара Матсью предупреждает о потопе человека, правителя Дравиды (южная часть Индии) йога по имени Ману, и советует строить большой корабль, куда взять, условно говоря, всякой твари по паре, чтобы затем населить землю заново. По индийским преданиям, все люди после потопа произошли от этого Ману, в то время как христиане выводят род человеческий от уцелевшего Ноя.

О потопе людей предупреждают то шаманы, то боги. Значит, более осведомленные знали о надвигающейся беде заранее, и некоторых она не застала врасплох. Люди пытались спасаться на возвышенностях. Потом, надо думать, вода спадала, где быстрее, где медленнее. В главе про Киевское море мы еще обратимся к сведениям об якорях и обломках морских судов, находимых в степях Левобережья.

Вообще в моей книге время от времени я буду касаться Всемирного Потопа, как одного из событий прошлого, события, которое оказало влияние на формирование рельефа и на развитие наземной жизни. Но прежде чем двигаться к другой теме, порассуждаю еще.

Докучаевский переулок в Киеве. Прорезался оврагом к огромному Протасову яру от расположенной выше, чем проулок, Докучаевской улицы. Назван в честь Василия Докучаева, геолога и почвоведа. Научные работы Докучаева очень интересно читать, они написаны живым языком, Докучаев приводит взгляды, противоположные своему, и оспаривает их.

В его книге «Способы образования речных долин европейской России» речь неоднократно  заходит о мамонтах и шерстистых носорогах. Приведу пример, впрочем без иллюстрации – не могу снова найти в закромах книжку. Место этого обрывистого берега речки Качни, по описанию Докучаева

\begin{quotation}
находится на левом берегу речки, близ устья Гридневского ручья. Как видно, в нем можно различать следующие слои\footnote{Описываются сверху вниз.}:

A, высотой 6 футов (30,48*6/100 = 1.82 метра). Обыкновенная бледнокрасная, рыхлая, песчаная глина без галек и органических остатков; кверху она постепенно переходит в растительный слой,а книзу в ней заметно все более и более преобладание песку. 

B, высотой 4 фута (30,48*4/100 = 1.21 метра). Верхние три фута состояли из беспрестанно перемежающихся прослойков темной глины с пропластками песку, то желтого, то очень красного, то мелкого, то довольно крупного (гравий); 

слои постоянно выклинивались или незаметно переходили один в другой, на расстоянии нескольких десятков шагов вдоль берега; слои были так мелки, что их в 3 футах насчитывалось до 25 и более; только при самом основании этого пласта виднелся совершенно однородный, чистый, светло-красный песок с фут толщиной. Весь слой был переполнен древесными стволами, которые лежали обыкновенно горизонтально и были лишены в данном разрезе мелких сучьев. 

На самой границе с пластом С были найдены мною части ступни, атлант и несколько ребер мамонта и зуб, принадлежавший тому же животному (Elephas primigenius), кроме того, здесь же попадались еще, хотя и не особенно обильные, мелкие, совершенно уже сгнившие остатки каких-то других костей.

С, высота половина фута - около 15 сантиметров. При самом основании обрыва, подымаясь над уровнем реки едва на 1/2 фута, выступала грязно-синяя вязкая глина, но ее можно было проследить на несколько фут. под водой, откуда торчали иногда громадные древесные стволы. Замечательно, что местные крестьяне именно из этого пласта выкапывают наиболее сохранившиеся дубы, которые они и употребляют на различного рода поделки; они даже заверяли меня, что некоторые экземпляры были здоровее - плотнее ныне живущего дуба; в этом случае, вероятно, уже началось окаменение дерева 
\end{quotation}

Итак, три слоя - три горизонта. Образование верхнего, медленное, постепенное, Докучаев объяснял так:

\begin{quotation}
Благодаря тому обстоятельству, что этот пласт всюду образовался частью из отложений весенних вод, а частью - на счет мути, приносимой с соседних высот атмосферными водами, минеральный состав его, понятно, не мог быть строго определенный: это есть смесь в различных пропорциях тех горных пород, которые встречаются в бассейне данной реки;\end{quotation}

Этот верхний пласт – лёссовый. Вот Докучаев выразил одну из причин его возникновения – намыло, нанесло водой. Отложения весенних вод и наносная муть с соседних высот.

   Ага, получается, соседние высоты сами состоят из суглинка. Как же он образовался там? А там надуло ветром? 

   Что же, слой суглинка высотой 1.82 метра, под ним слой 1.21 метра, с костями мамонта и лежащими стволами деревьев. Рассуждая логически, мы придем к выводу, что кости мамонта были покрыты слоем суглинка в 1.82 метра.

   Река Качня, ныне Касня, в Смоленской области, течет по равнине. Не знаю, каковы были – и были ли – паводки на ней до устроения там водохранилища. Докучаев уделил ей отдельную работу, «О наносных образованиях уезда Смоленской губернии».

   Но исходя из геологического разреза, приведенного Докучаевым, мы должны предположить, что для образования над современным Докучаеву уровнем воды в речке, должны быть справедливы следующие положения.

   Уровень воды в реке время от времени повышался и намывал этот суглинный слой, сначала покрыв упавшие почему-то кучей деревья, покрыв останки мамонтов, а потом уже шел намыв на ранее намытую муть. 

   Уровень воды в небольшой реке должен был повыситься, однако, в сумме почти на 4 метра. И вот странно. Постепенно намываемый суглинный берег не зарастал новыми деревьями, кустами... Нет, он, получается, долгое время ждал, и лишь потом соизволил покрыться сверху слоем более-менее плодородным.

   Переулок Докучаевский, Протасов яр. Яр этот прор\'езал Батыеву и Байкову горы и спускается от района Соломенки к речке Лыбедь. Перепад высот – около 62 метра, то есть верховье яра лежит на 62 метра выше низовья. По дну яра в коллекторе протекает ручей. Сложно представить, что своим жалким течением он намыл могучие суглинные берега яра. Он их только размывал.

    А ведь в 19 веке здесь, при добыче глины для нужд кирпичного завода, тоже нашли кости мамонта. Снова закономерность – слой суглинка или глины, а под ним кости мамонта. 

Многие знают про найденного на Магадане несчастного мамонтёнка Диму. Его тело обнаружили в 1977 году на прииске имени Фрунзе, на глубине 2 метра от поверхности земли, когда бульдозером раскапывали грунт в долине ручья под названием Дима – отсюда и мамонтёнок Дима, а не от имени Дмитрий. Обычно сообщают, что нашли в вечной мерзлоте. А мамонтенок, дескать, когда-то упал в яму с водой и грязью, не смог выбраться и умер. А потом мороз заморозил тело в этом льду с грязью и так оно сохранилось до наших дней. 

   То есть, надо полагать, при падении туда мамонтёнка, вода была жидкой, но потом резко ударил мороз, заморозил ту глинистую жижу и уже не размораживался до 20 века, а к тому времени сверху опять же, намыло или надуло 2 метра некой вечной мерзлоты. Что за вечная мерзлота такая? 

   Да, кстати, про сходного мамонтёнка Любу с острова Ямал ученые тоже предполагают, что Люба задохнулась в некой в глинистой массе, а потом тело законсервировалось благодаря лактобактериям.

   Так вот про вечную мерзлоту, в которой находят мамонтят. И останки мамонтов. И поваленные деревья. Их находят не просто в какой-то почве, а в промерзшем суглинке.

   Но вернемся к киевским холмам. Много и подробно еще будем говорить о Кирилловских высотах и раскопках на них. Самые знаменитые из них - обнаруженная Викентием Хвойкой Кирилловская стоянка, от места которой поныне сохранился крутой обрыв.
    Многочисленные кости мамонтов были найдены внизу, под срытой частью горы. 
   
   По Хвойке, стоянка - а он полагал, что мамонтов убивали некие первобытные охотники и там же жили, хотя скелетов людей он не отыскал - стоянка располагалась в местности еще без гор, в полной озер котловине у большой реки, у древнего Днепра.   

Тут росли высокие кедры и ели, бродили мамонты. Но с севера, как считал Хвойка, подступал ледник, из-под которого вода несла грунт, постепенно заполняя наносами долину Днепра, пока она не переполнилась. Тогда вода выступила над уровнем долины, затапливая окрестности. Ледник надвинулся, затем начал таять, отступать к северу. Снова наносы. Река пробила себе новое русло уже в них, уровень дна снизился. Слои песка между культурными слоями, как полагал Хвойка, означают несколько затоплений местности, во время которых грунтовые наносы покрывали предшествующий культурный слой. 

   Так рассуждал Хвойка, беря на вооружение передовые для 19 века представления о леднике. Конечно, всемирный Потоп уже тогда стало модным считать сказкой. Хотя даже в рассуждениях Хвойки есть некие затопления, следами коих он считал слои песка.

   Современники возражали Хвойке –  в разрезе холма нет морены – валунной глины, которая должна присутствовать, если местность была покрыта ледником, и поэтому Кирилловской стоянке надо определить возраст не более 12 тысяч лет, когда, как считается наукой, завершился последний ледниковый период.

   Изучая потоп, вы всюду натолкнетесь на эту датировку – 12, 10 тысяч лет назад, что-то случилось. Завершился ледниковый период, случились какие-то катастрофы... Например, что вы знаете об Алтайском потопе? Ученые думают, что некие ледники создали плотину на реке Чуе, и образовалось огромное ледниковое озеро, а потом, между 12000 и 9000 годами до нашей эры, плотину прорвало и вода устремилась по Чуе в Катунь, потом в Обь и в так называемое Мансийское озеро, причем уровень воды по пути резко поднялся на 12 метров, и, дескать, этот потоп докатился даже до Черного моря.

   Следы воздействия на рельеф того, что ученые называют Алтайским потопом, хорошо видны и отражены на фотографиях и в научных работах. Однако науке проще объявить причиной потопа ледниковую плотину и ледник, нежели задуматься о всемирном потопе.

   Равно как проще объяснять места массового обнаружения останков мамонтов тем, что ледник принес суглинок и покрыл им останки. Или что мамонтёнок упал в яму с глинистой водой. Ну а большие мамонты тоже в ямы с глинистой водой падали? Нет, их убивали первобытные охотники, – пояснял археолог Хвойка. А другие археологи его спрашивали – так ведь те орудия, что вы нашли, слишком малы...

   Но главное, никто толком не объясняет, как поверх этих останков образовался слой суглинка в десятки метра высотой. Чем его нанесло?

   Оглядитесь вокруг, выйдя не природу. Вы много увидите глины? Глина почти всегда скрыта под слоем чернозема, и обнажается лишь в местах крутых обрывов, либо там, где глину добывают для производства кирпича, раскапывая склон на большую глубину. Между прочим именно так совершаются многие археологические открытия.

   Современник Докучаева, профессор Головкинский писал относительно Волги: 

\begin{quotation}
Что касается состояния страны по среднему течению Волги, то из многочисленных залежей деревьев, погребенных обширными пластами, следует заключить о хвойных лесах, покрывавших страну, все еще обитаемую мамонтом, носорогом, верблюдом и другими животными.
\end{quotation}

С чего это хвойные леса легли обширными пластами? Может их срубили и потом бросили? Нет, люди всегда рубили лес для своих нужд, потому и много лесов вырубили так, что следа не осталось. От чего же полегли хвойные леса? Почему их покрыло потом песком да суглинком? Потому и покрыло, что Потоп затопил.

   И его запомнили современники мамонтов. А от них предание через поколения дошло до нас, чтобы быть осмеянным наукой и объявленным сказкой.

   Когда вы идете по улице, и проводятся какие-то строительные работы вроде проложения труб – обратите внимание на тонкий слой чернозема сверху, и суглинок ниже. Быть может, это след Всемирного Потопа, застывшая муть, принесенная его водами и здесь осевшая.

Потоп служит одной из основных отправных точек изложения истории в летописях, хрониках, анналах и преданиях. Однако наука не верит в Потоп, при этом истово веруя в другие сведения, ею же относимые к баснословным.
