\chapter{Город кузнецов}

Из чего выплавляли железо? Из руды. Руда – порода с железистыми, бурыми (рудыми, рыжими) частицами, коими богаты Кирилловские высоты. Многие ручьи образуют здесь по течению своему красно-коричневые наносы и вязкие, ржавистые болотца у склонов. При раскопках Кирилловской стоянки, Викентий Хвойка нашел в составе склона холмового отрога полутораметровый слой рудной породы – железистого песчаника. Вот что вымывают ручьи – руду!

На подобные залежи указывают рыжие наносы ручьев из Кирилловского, Репяхового, Хрещатого яров, склонов Днепра вдоль набережной, околиц озера Глинки. Есть также целая бурая речка под Выгуровщиной, был да сплыл ручей в Быковнянском лесу – но это уже равнина. Как быть с нею? А существует понятие «болотной руды», ее мы подробно коснемся в разделе про Городок и Радунку.

Доступные залежи на Кирилловских высотах глины и руды, близость к ним руды болотной (на Оболони) делали это место важным сырьевым центром древней металлургии.

Глина нужна, чтобы делать печи (горны) по переработке руды в крицу – грубо говоря, пригодные для ковки куски железа\footnote{На деле, горячую еще крицу нужно также «обжать», освободить от шлака обивая ее деревянными молотами. На Руси, крице для продажи придавался товарный вид круглой лепешки.}.

Тут же и лес под боком, ведь в печь нужно загружать древесный уголь. Древесина сгорает в два этапа – сначала вместе с пламенем уходят летучие вещества, потом уже сам уголь горит без огня, или синим пламенем. 

Уголь-то и дает температуру более высокую, нежели «пламенный» костер, а кроме того химически участвует в образовании крицы. Дрова сначала, от пары дней до месяца в зависимости от количества сырья, пережигались на уголь в особой ямах, закрытых сверху дерном и замазанных глиной, с дырочками в крыше для притока воздуха. Полученный таким способом и остывший уголь помещался в сыродутную печь вместе с рудой. Это технология 18-19 веков, а каким именно способом выжигали уголь во времена Киевской Руси и прежде, не знает уже никто.

Однако, все три необходимые для металлургии составляющие – руда, глина, древесина – сошлись на Кирилловских высотах и около них!

Про остатки древнего горна и кусков железного шлака на южной части Лысой горы рассказывал Максимов. А в трех километрах отсюда к северу, на юго-западном берегу современного Богатырского озера\footnote{Отделено от Кирилловского озера улицей Добрынинской, расположено между улицами Богатырской и Малиновского.} тоже обнаружены следы металлургического производства – около сорока железных изделий, куски руды, шлак.

В 1965-73 годах экспедиция Анны Шовкопляс и Гопака изучала остатки большого поселка «зарубинецкой культуры», немного северней. По статье «Черный металл зарубинецкого поселения на Оболони в Киеве»\footnote{Советская Археология, 1983, номер 4.} положение его толком понять нельзя. Мол, на Оболони: 

\begin{quotation}
занимало самый высокий участок правого берега летописной р. Почайны (ур. Луг). Место это было труднодоступным из-за окружавших его низких участков широкой поймы с ее заболоченными лугами, озерами и старицами.
\end{quotation}
 
Судя по планам Шовкопляс, урочище Луг лежало к северу от Луговой улицы, вдоль улицы Богатырской. Середина поселения лежала, по сопоставлению, на месте нынешнего стадиона южнее дома на Богатырской, 2-А:

50°30'46.2"N 30°29'09.5"E

Экспедиция нашла остатки 66 жилищ, расположенных по кругу – это были полуземлянки, с поверхностными стенами из «дерева, лозы и глины», по 12-20 квадратных метров площадью. Ученые датируют селение 1 веком до нашей эры, 2-3 веками нашей. Среди остовов жилищ лежали изделия из бронзы и железа. Железные – куски пряжек, ножей, серпов, багор, рыболовный крючок, стальная бритва, шпоры. Любопытна их технологическая сторона:

\begin{quotation}
Шесть ножей этой группы\footnote{С горбатой спинкой.} – цельножелезные. В большинстве случаев для их изготовления использовалось грубое кричное железо с большим количеством шлаковых включений. Лишь в одном случае почти полное отсутствие шлака указывает на его тщательную исходную проковку. [...]

Седьмой нож был целиком откован из ста\-льной заготовки. Лезвие ножа закалено в воде. [...] Стальная заготовка для ножа получена сваркой нескольких поперечных полос стали.

Восьмой нож выполнен в технике продольной сварки стальной и железной полос. Сварка выполнена качественно. Микроскопически она прослеживается на слабовыраженной светлой полосе и наличию вытянутых включений остатков сварочного флюса вдоль границы стальной и железных зон. Лезвие ножа термообработано.
\end{quotation}

Не удивлюсь, если всё это было произведено из местной руды, рядышком, у Богатырского озера, или в окрестностях. Что погребено на Оболони гидронамывом, уже не узнает никто.

Остатки давних кузниц и следов переработки руды на соседнем Подоле – у перекрестка Волошской и Героев Триполья, на углу Волошской и Нижнего вала, на Волошской 17-19, Волошской 20, Щекавицкой 25-27, на Верхнем валу, на Межигорской.

Конечно, Гора тоже не лишена была подобного, однако в меньшем количестве. Кучность находок там – по улицам Владимирской, Чкалова (Гончара), на горе Детинке. Впрочем, обилие находок всегда зависит от археологической «раскопанности» местности. Ведь где не роют, там и не находят. Прежнее название Волошской улицы – Быдлогонная – означает, что металлургию там вытеснили пастбища.

Не был ли Киев в древности центром металлургии? Вспомним Повесть временных лет – когда после смерти Кия и братьев на Полян, которые обитают «в лесе на горах, над рекою Днепрьскою» налагают дань Козаре, Поляне платят «от дыма меч».

С чего бы это? А товар местного производства!

Нестор оборачивает сей рассказ в поучение – мол, козарские старейшины устрашились такой грозной дани: «не добра дань, княже! мы доискахомся оружьем одиноя страны, рекше саблями, а сих оружие обоюдоостро, рекше мечи; си имуть имати и на нас дань и на инех странах».

Но вот в арабском сочинении «Худуд ал-алам», которое датируют 10-м веком, читаем\cite{vostistnovosel}:

\begin{quotation}
Куйаба – город русов, ближайший к мусульманам, приятное место и резиденция царя. Из него вывозят различные меха и ценные мечи.
\end{quotation}

В Куябе сложно не признать Киев!

Выдержка из сочинения, как полагают, 12 века, ал-Идриси «Нузхат ал-муштак фи-хтирак ал-афак» («Развлечение истомленного в странствии по областям»)\cite{vostistnovosel}: 

\begin{quotation}
Русов три группы. Одна группа их называется рус, и царь их живет в городе Куйаба. Другая группа их называется ас-Славийа. И царь их в городе Славе, и этот город на вершине горы. Третья группа называется ал-Арсанийа, и царь имеет местопребывание в городе Арсе. Город Арса красивый и (расположен) на укрепленной горе между Славой и Куйабой От Куйабы до Арсы четыре перехода, и от Арсы до Славии – четыре дня. И доходят мусульманские купцы из Армении до Куйабы. Что же касается Арсы, то, согласно рассказу шейха ал-Хаукаля, туда не входит ни один чужеземец, ибо убивают там всякого чужеземца. И не позволяют никому входить с целью торговли в свою землю. Вывозят оттуда шкуры черных леопардов, черных лисиц и олово (свинец?). И вывозят это все оттуда торговцы из Куйабы. 

Русы сжигают своих мертвых, а не зарывают в землю. Некоторые русы бреют бороду, а некоторые завивают ее, как гриву лошадей. Одежда их – короткие куртки, и одежда хазар, булгар и печенегов – короткие куртки из шелка, хлопка, льна и шерсти.
\end{quotation}

Подобные арабские источники про Куябу часто уточняют размер города и его значение. Из ал-Истахри «Китаб ал-масалик ва-л-мамалик»\footnote{«Книга путей и стран», ее относят к 9 веку.}:

\begin{quotation}
Русы. Их три группы (джинс). Одна группа их ближайшая к Булгару, и царь их сидит в городе, называемом Куйаба, и он (город) больше Булгара. И самая отдаленная из них группа, называемая ас-Славийа, и (третья) группа их, называемая ал-Арсанийа, и царь их сидит в Арсе. И люди для торговли прибывают в Куйабу. [...]

И русы – народ, сжигающий своих мертвых... и одежда их – короткие куртки... и эти русы торгуют с Хазарами, Румом\footnote{Римской империей, того времени именуемой наукой как Византия.} и Булгаром Великим, и они граничат с северными пределами Рума, их так много и они столь сильны, что наложили дань на пограничные им районы Рума, внутренние булгары же христиане\footnote{«и мусульмане» – добавляет Ибн Хаукаль в «Китаб ал-масалик ва-л-мамалик».}.
\end{quotation}

Эти арабские сочинения – окна в прошлое, но в какое? Времен Кия и глубже? Но тогда здесь жили вроде бы не Русы, но Поляне. Времен Вещего Олега? Но при нем и после него Киев будто уже не славился производством мечей. Понятие Русов у Арабов весьма разнится от источника к источнику, однако в приведенных выше отрывках Русы делятся на три группы – Русы (живут в Куябе), ас-Славия – в Славии, и ал-Арсанийа – в Арсе.

Увы, я знаком с этими сочинениями лишь по переводным выдержкам и не могу полноценно рассуждать о написанном вне более широко описываемой картины.

Но мы остановились около Богуславского спуска. Пора двигаться вдоль Кирилловских высот дальше, до самого Логова Змиева. По дороге к нему еще много чудес и загадок.

Шагая по Кирилловской улице на северо-запад, мы словно уходим всё глубже в прошлое, ведь совсем рядом с усадьбой Марр, за Богуславским спуском – знаменитая Кирилловская стоянка той эпохи, когда по будущему Киеву ходили мамонты и шерстистые носороги. А это, как считается, тысячелетия до нашей эры.
