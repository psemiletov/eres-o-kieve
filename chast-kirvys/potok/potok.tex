\chapter{Юрков ставок}

Про ручей Юрковицу считают, что так называли ручей, текущий в овраге между Щекавицей и Юрковицей, то бишь вдоль нынешней улицы Нижнеюрковской, а потом Юрковской. Этот ручей с 1908 года заключен в коллектор.

Когда я задался целью исследовать появление названия «Юрковица» в письменных источниках – имею в виду не ручей, а гору – то обнаружил, что сего названия применительно к горе в доступных мне источниках до 20 века не существует.

Двигаясь от прошлого к настоящему, я попытался узнать историю возникновения этого названия по источникам, ибо на карте Юрковица – гора – отмечается безо всяких пояснений.

Первым, что я встретил в документах, был Юрков ставок.

В 1701 году гетман Мазепа создал комиссию, призванную решить спор при разграничении земель, принадлежащих Киеву (магистрату) и Кирилловскому монастырю. Межей издавна был обозначен Юрков ставок. Но во времена Мазепы само местоположение ставка уже являлось вопросом.

И вот 14 марта того года комиссия в составе «енеральных особ» войска Запорожского – Василия Кочубея, Иоанна Скоропадского, Михайла Гамалеи, а также князей, генералов, сотников и представителей обоих сторон спора – войта с людьми из магистрата, да руководства Кирилловского монастыря – отправилась на местность выяснять, где же находится Юрков ставок.

Выехав из Нижнего города через Воскресенскую башню, добрались они на дорогу от башни Бископской к Кирилловскому монастырю. Дорога эта нынче – улица Кирилловская.

По пути, войт Дмитрий Полоцкий и магистратские доказывали комиссии, что ставок находится «за монастырцем девичьим Иорданским» – там была гребля на потоке, текущем из яра, и остатки старинного става. Двигаясь по Кирилловской в сторону Кирилловского монастыря, указание «за монастырцем» вероятно значит, что поток и став были к северо-западу от монастыря. А это, насколько я понимаю, однозначно ручей, текущий между отрогом «Кожевника» и отрогом с Кирилловской стоянкой. Иначе, если бы речь шла об известном ныне ручье Юрковице по улице Нижнеюрковской, было бы – «перед монастырцем», а не «за монастырцем».

Однако никаких документальных доказательств, что именно этот став является Юрковым, войт предоставить не мог, а его слова сразу опровергались наместником и экономом Кирилловского монастыря, утверждавшими противное – ставок был не тут, а под горой Щекавицей.

Но магистратские приводили такой довод – церковь Николая Иорданского «будучи прежде мирской церковью до Киева общество духовенства светцкого належала, так и люде, поселение свое тут маючии, до Киевского меского прислушали права».

В самом деле – если принять границы по версии руководства Кирилловского монастыря, то Иорданская церковь и окрестности тоже ему принадлежали, но Иорданская церковь, как говорит магистрат, относилась к городу Киеву, а не Кирилловскому монастырю. И люди, что тут жили, посему относились к Киеву, а не монастырю. Логичны доводы магистрата – граница по, проще говоря, владениям Иорданской церкви.
  
Документ комиссии продолжает:

\begin{quotation}
А кгды вернулисмося мы, вышей мененние особы, от того потоку из за монастиря паненского Иорданского, провадили нас зъездом Лисой гори оны пановы майстратовыи Киевскии, на горы, до могилок, против тогож монастыря Иорданского, а особно против того яру, в котором тот поток и гребелку виделсмо, стоячих, показуючи, же яко ставок оный з низу, так тыи могилки на горе суть власною границею земель, межи Киевом и монастырем Кириловским будучих.
\end{quotation}

Переведу с пояснением. Осмотрев поток, магистратские люди повели комиссию так – обогнув монастырь, зайдя в его тыл, на теперешние задворки фабрики молочной кислоты («усадьба Марр»), они вышли на «зъезд Лисой гори» – старинную дорогу к Лукьяновке, между отрогом Лысой горы и отрогом дач «Кожевника».

По этой дороге они поднялись «на горы, до могилок, против тогож монастыря Иорданского» – на горы выше монастыря Иорданского. Но что значит «до могилок»?

Кое-кто трактует «могилки» как курганы. Но протокол комиссии написан «роськой мовой» Великого Княжества Литовского. Я заглянул в «Словарь живаго народнаго, письменнаго и актоваго языка русских южан Российской и Австро-Венгерской империи» Фортуната Пискунова, изданный в 1882 году, и на 138-й странице прочитал:

\begin{quotation}
Могилки = гробовище, кладовище, гробки, цвинтар.
\end{quotation}

Итак, кладбище. Мы не знаем, когда именно возникло христианское кладбище над Иорданским монастырем, однако тут известны более давние языческие погребения\footnote{Между прочим и земляные холмики на современных могилах – по сути, уменьшенные курганы. Так сохраняется древний обычай.}. Предположу, что могилы христианские постепенно возникали рядом с поганскими, а может и вместо них. Ясно, что комиссия поднялась к этому кладбищу, тогдашние границы которого неведомы. Но куда комиссия свернула потом – на Лысую гору или на гору «Кожевника»? От этого зависит дальнейшая трактовка, ведь «А особно против того яру, в котором тот поток и гребелку виделсмо» – а отдельно («особно») на гору возле яра, откуда видно поток. 

Смекнем. Указанным яром может быть один из трех, с юго-востока на северо-запад: будущий Мыльный переулок, овраг с дорогой на Лукьяновку, и овраг между мысом «Кожевника» и мысом Кирилловской стоянки. Нам бы очень помогло знание, где протекал Иорданский ручей, однако он сам по себе загадка.

Мыльный переулок, прежде примыкавший к одному из яров в Лысой горе (юго-восточная ее часть ныне срыта именно по границу этого былого яра), не дает почвы для рассуждений, поэтому за бесплодностью отложим его в сторону. 

Яр в дороге на Лукьяновку. Если бы поток бежал вдоль него, комиссия так бы и написала – мол, поток у «зъезда Лысой горы». Да и подобное соседство кажется мне сомнительным. Там просто негде уместить дорогу и ручей, да и последний размывал бы путь. Я исключаю этот овраг, между отрогом с «замковищем» и отрогом Кожевника.  

Следовательно, остается только яр по другую, северо-западную сторону мыса Кожевника, где между ним и огрызком мыса Кирилловской стоянки поныне в коллекторе течет ручей.

Значит, с северо-западного склона «дачного» отрога комиссии показывали внизу ручей, греблю и «знак става старинного» (то бишь следы старинного пруда), говоря – вот это и есть граница земель Киева и Кирилловского монастыря.

Внизу, там где автобаза, даже при раскопках Хвойки был ставок, поблизости будущего «глинища» кирпичного завода Зайцева и мыса Кирилловской стоянки. А вот «след става» как понимать – существовал заполненный пруд, или просто наблюдались некие его следы? Но комиссия точно видела греблю, плотину.

Что же далее, после осмотра?

\begin{quotation}
А от тых могилок оны, панове майстратовыи, вели нас до креста Данилового будучого край шляху, з Киева и от монастыря Кирилского до Белогородки лежачего. А от того местца вели нас назад шляхом, до верхного города Киева к пробитому валу лежачим, где именовали все, поуз той шлях будучии поля, яры и дубники, належными до места Киева, твердячи: же на тых местцах з старих часов, стада, череди, и всякии их быдла на паствиска хожовали невозбранно. Якож и в привелеи Жикгимунтовскому им, мещаном, оныи найменованы, але на тых местцах явилися нивы, и поля пахатные розных монастырей Киевских, которых тыи монастыре, здавна заживают невозбранно. Потом, за настигненем вечера, не скончивши дела, ехалисимо от пробитого валу, чрез Кожемяки в Киев до господ своих\footnote{Т.е. разъехались по домам. «Госп\'ода» – господарство, хозяйство, двор.}.
\end{quotation}

16 марта пришел черед духовенству доказывать границы. Комиссия выехала теперь к потоку между Щекавицей и Лысой – на теперешнюю улицу Нижнеюрковскую. В то время это был широкий овраг. Явился игумен Кирилловского монастыря, Пахомий Прокопович, с братией, и представил комиссии десять человек-старожилов, «подданных своих монастырских», чтобы те подтвердили права монастыря на земли.

Комиссия стала расспрашивать «старожилов», которые оказались вовсе не старожилами, а всё пришлые во время войны Богдана Хмельницкого и после нее, стекшиеся сюда на поселение с разных мест. И вот они сказали комиссии, дескать, слышали в разговорах истинных старожилов, прежде тут обитавших, что именно здесь был «ставок Юрков а не где инде», и возле потока был небольшой поселок, но теперь ни гребли ставка, ни поселка в том яру нет. Эти же свидетели сообщили, что все люди, начиная с потока и до самого Кирилловского монастыря, относились не к Киеву, а к Кирилловскому.

Двое псевдо-старожилов, Захарий Семенович и Дмитро Бабич, «от отца игумена ставленых» прибавили, что давно, над потоком с обеих сторон было по корчме, одна принадлежавшая Киеву, другая монастырю, и что хотя признаков ставка рядом нет, но это разделение принадлежности корчм служит доводом в пользу границы именно здесь.

Дальше комиссия решила не выяснять, где был Юрков ставок, то есть просто наплевала на указанную в «привелеях» межу и, вместо свидетельств «старожилов» сославшись уже на некие документы показанные им законниками монастыря (под потоком далее разумея ручей между Щекавицей и Лысой, то есть указанное духовенством Кирилловского монастыря место), постановила:

\begin{quotation}
Абы люде, на подворках Карпиловкою прозываемых, близко места Киева, от монастыря паненского Иорданского и от монастыря Кирилского мешкаючий, по этот самый поток, межи горами Шкавицею и Лисой будучи, належали непременно до монастыра Кирилского.
\end{quotation}

Земли же в другую сторону от потока, к Щекавице и иже с ней, закрепили за Киевом. Впрочем магистрату отошли также кирпичные заводы под Щекавицей, которых духовенство Кирилловское «забороняли и оноя разоряли», полагая подножие Щекавицы своим.

Следующие полстолетия земельные споры не утихали. В шестом выпуске «Исторических материалов из архива Киевского губернского правления»\cite{histmatkiev} помещена статья «Споры из-за грунтов между киевским магистратом и кирилловским монастырем» с документами 1747 когда, когда противоборство снова заострилось. 

В 1746 года монахи, по приказу архимандрита Платона, напали на хутор киевского мещанина Матвея Коробки, на реке Сырец, и разорили там винокурню. Также к Коробке был подослан некий поп, наряженный Платоном в «богатую одежду» и снабженный для важности «тростиной с сребряною оправою». Этот поп выманил в лавке Коробки материи на 140 рублей.

Через год летом, 7 июля, монахи пытались украсть сено с сеножатей близ Сырца, но местные жители сопротивлялись и не дали. Спустя три дня, монастырские подданные, вооруженные топорами, киями и цепами, купно с законниками монастырскими – наместником Серапионом, экономом Павлом, писарем Нафонаилом – напали на хутор Прокопа Котляра, на луга Кондрата Турмы, Павла Коваля, жгли сено, уничтожали нескошенную траву, ломали и палили заборы, откуда огонь перекинулся на дома.
 
Магистрат жаловался кому мог на своеволие архимандрита, дело дошло до Киевского генерал-губернатора Леонтьева. Тот снарядил двух офицеров для расследования, но Платон нанес предупреждающий удар, сам подав челобитную, где перечислил тяжкие грехи магистрата и утверждал право за Кирилловским монастырем на земли вдоль речки Сырца – кроме прочих кстати упомянут «грунт Рогостинский» – поныне там известен ручей Рогостинка. Сырец в челобитной записан как «речка Сырцы». Матвей Коробка по словам Платона выглядит полным злодеем, и верхом злодейств его было ископание русла Сырца так, что он потёк в обход монастырской мельницы, однако к винокурне Коробки. Посему-то монахи, негодуя, и отправились громить винокурню. 

Девятый пункт челобитной выглядит так:

\begin{quotation}
Поток, называемый Юрковиця, мещане, в горе неоднократно раскопав, в грунт монастырский исправилы, и за таким роскопаньем не мало прынялы земли монастырской до места\footnote{Места – города. Противопоставлены земля монастырская и земля места, мещанская, городская.}, на которой ныне поселывшихся людей к месту неналежне прытягли, а другие поддание монастырские от подтопления с онаго потока водою жылыща свои прынуждены пооставлять, а иные уже и пооставлялы.
\end{quotation}

Впрочем, он не дает нам представления о том, где именно протекал поток Юрковица, но мы впервые встречаем хотя бы письменное о нем упоминание.

Десятый пункт тоже любопытен:

\begin{quotation}
На горе Лисавице за урочыщем Глыбочыцею не мало власного монастырского грунту магистрат присвоил, и хочет вовсе завладети.
\end{quotation}

Лисавица – Лысая гора или какая-то другая? Скорее всего Лысая. Глубочицкий овраг в привычном нам понимании не соседствует с Лысой горой, между ними здоровенная Щекавица. 

А если в 18 веке под Глубочицким урочищем понимался овраг, где ныне улица Соляная? В этом случае десятый пункт становится понятным.

Но вернусь к Юрковице. Весной 1765 года возникла необходимость постройки госпиталя для первого артиллерийского канонирского полка. Место под «гошпиталь» требовалось в окрестностях Подола, такое, чтобы не заливалось в паводок. Подобное нашлось «по правую сторону от Ерданской рогатки», то бишь от той самой рогатки, которая ныне улица Нижнеюрковская и одноименный переулок. Направо это как? Если идя снизу, от Плоского, Оболони, то выходит, что мимо Лысой горы на запад. Отрог её, на месте коего, срытого, ныне воинская часть?

Магистрат выделил землю, предоставил на нее документы. Генерал-губернатор Глебов приказал приступать к постройке лазарета.

Но как всегда оказалось, что земля-то не магистратская, а Кирилловского монастыря. Игумен Тарасий Вербицкий подал Глебову об этом доношение.

Снова возник вопрос о разграничении земель. Магистрат утверждал, что земля его, а монастырь – его. Выяснить, кто прав, поручили штык-юнкеру Линеву. Тот начал деятельность и в сентябре подал не точный рапорт, а недоуменное разведение руками:

\begin{quotation}
разобрание чинено было и оная спорная земля ему с обоих сторон показываема, где напред сего якобы межи бывали, точию тех межевых признаков, которые в оных крепостях\footnote{Крепостных, т.е. закрепляющих владение, грамотах, предоставленных магистратом и монастырем.} написаны, не сыскалось и за многими тех определенных с обоих сторон спорами признать было, чье то подлинно место, непочему и не можно, и для того требовано было им Линевым от тех обеих сторон старожилов, чтоб точно могли показать, токмо оных и по ныне не сыскано, да определенные от оного Кирилловского монастыря, по многопосланным к ним известиям не бывали и за небытием их то ж и за несысканием старожилов вразмежении тои земли никакова дела непроизводится.
\end{quotation}

Далее в качестве довода к размежеванию вытащили на свет выписку из старой бумаги от уже известной вам комиссии Кочубея, где упомянуто, что Юрков ставок – граница земель магистрата и монастыря.

Но ведь при Кочубее никакого «Юркового става» между Щекавицей и Лысой горой не существовало, был только «след става» на ручье между, предполагаю, отрогами дач «Кожевника» и Кирилловской стоянкой. Монастырь же привел еще один документ про то, что в 1747 году мещане изменили русло «старого течения с того Юркова става пот\'ому\footnote{После того.}». Итак, с 1747 года по 1765 некий существующий став именовался Юрковым, а из него вытекал поток.

На карте 1752 года, «снятой и свидетельствованной по Дукту дьяка Алферьева спорных земель по иску Киево-Кириловского монастыря с Киевским Магистратом» показана «речка Юрковица», вытекающая из оврага «Хвощеватой долины». На левом берегу речки, у тогдашнего самого юго-восточ\-ного угла (позже срытого) Лысой горы отмечены «Кирпичные заводы Мещанина Ивана Григоровича, который речку Юрковицу принял на одну сажень».

На карте 1803 года эта же речка вытекает из пруда в ущелье при срытом позднее склоне Лысой горы, у западной границы воинской части. %Письменно с этим потоком связаны кирпичные заводы

И в «Статистическом описании Киевской губернии» 1852 года издания\cite{fundstat} сказано про кирпичный завод гражданина Романовского, стоявший юго-западней:

\begin{quotation}
Он устроен в урочище, называемом Юрков-проток, на усадьбе, заключающей в себе 13 десятин земли. Этот завод существовал еще до 1816 года, в котором он приобретен настоящим владельцем. [...]

Глина и песок, употребляемые для кирпича и кафель, добываются в ущельи урочища Юркова-протока, принадлежащего к усадьбе заводчика.
\end{quotation}

Как видно по плану 1752 года, в середине 18 века  ручей на современной Нижнеюрковской уже слыл  Юрковицей, однако документы того времени свидетельствуют о шаткости сего именования, по крайней мере официального.

В 1765 году решение вопроса Юркового потока было насущным, от него зависело место давнего Юркового става, а к нему-то была привязана земля, выделенная под лазарет, ибо лежала напротив земли, смежной со ставом, в сторону Днепра.

Расследование поручили вести киевскому гарнизонному капитану Козме Максимову, который принялся опрашивать старожилов от магистрата и монастыря. Задачей его было освидетельствовать, «куда изстари оной поток течение свое имел и куда ныне имеет».

Монастырь затребовал у Киевской Городской Коллегии результаты расследования. Дело о следствии было найдено, а вот само донесение Максимова – нет, хотя в этом-то донесении, как утверждал монастырь, всё было расписано, включая показания старожилов. 

Дело передали в инстанцию выше, в Малороссийскую коллегию, откуда выделили следователя, подкоморого Козелецкого повета Андрея Миткевича, однако тот для выполнения дела не являлся, и грызня между монастырем и магистратом продолжилась в 1766 году. И всё ждали скорейшего прибытия подкоморого Миткевича! Неясно, прибыл он или нет, однако весной 1767 года для разбора дела о размежевании земли назначили нового подкоморого, Александра Солонину. Путь Солонины был труден и долог – в Киев явился он лишь в сентябре, и купно с полковником Палиным начал расследование, прерванное неожиданным событием – магистрат и монастырь пошли на мировую. 

В рамках мировой был выработан документ с четким описанием межевания земель. Приведу оттуда выдержку, имеющую отношение к Юрковому ставу,  вообще познавательную для краеведов и уже частично знакомую нам по главе о Почайне. Итак, новая граница размежевания определена так:

\begin{quotation}
В начале от самой нызшей, стоящей на речке Сирце, близ подгородья меского киевского Куреневщины, Кирилловского монастыря мельница идучи вниз по той речки Сирцю чрез дорогу Куреневскую, подаваясь мало в лево пройти оною же речкой сто дватцать два с половиную треаршинных саженей, а оттоль вправо на болото багнистое, називаемое Пуниское, куда быть и течение оной речки и провесть оную до того болота рвом, а от болота в сагу нижеписенную ровом же, а из саги в Кривую Почайну или в озеро Ерданское тем местом, где перекопан валок, оставляя при магистрату по левой руки два сенокосы прилеглые: еден к подгородной меской Кореневской земле, а другой к сеножати Ковальской; чрез тоеж багнистое болото, занимая оной по левой же руки не малую часть со сенокосцами, очеретом и лозами, поуз сенокосы Кравецкую и Кушнерскую, прозиваемую Борок, и пески к урочищу саги до вигонной земли, всему тому быть за магистратом, а в правую руку за Кириловским монастырем; от тоей же саги мало низще вершины поворотясь вправо, тою ж сагою, которая граничит в правую сторону сенокосы монастыря Кириловского, а в левую сторону вигонную землю до валку (который упомянут в розиску бывшего судии генералного Кочубея посему то вигону быть до озера Иорданского, от которого для того до лоз при песках и перекопано) тем же валком до озера Ерданского; а озером Ерданским пришед к смуговине Турцу, не идучи тем Турцем, но прямо с начала оного Турца к полисаду на рог полисада, где земля вигонная окончивается, и тои вигонной земли по самие занятне плосколескими\footnote{От «Плоский лес».} жительми огороди, как оние и ныне состоят, не занимая отнюдь впредь огородами и не чим другим, далее быть свободной на вечное время: для пазбы скота как всего киевского гражданства, так и живущим на землях Кириловского монастыря и плосколесных обывателей и монастырских подданных, а от предписанного рога полисадного поуз самий же тот полисад валом, даже до ворот Ерданских, а то тех же ворот прямо чрез полисад \textbf{потоком, что зделался з Юркового ставка, между горами Лысою и Щекавицею}, которого потока по левой стороне земля и обывательския жилища киевского магистрата, а по правой стороне Кирилловского монастыря, равним же образом и полисадному валу оставаться между землями Кириловского монастыря и магистратскими границею, для чего каков от Кириловского монастыря был иск к магистрату о земле за полисадом в городе Киеве и об отводе той земли от магистрата на устроения лазарета и прочего, также же за отнятие ис под владения монастырского подданного Назаренка, то и тот <неразборчиво> Кирилловский монастырь уничтожает вечно и всему тому оставаться за магистратом киевским, (а подданоному Назаренку имеючеесь на той земле строение свесть и поселиться ему на земле оного Кирилловского монастыря, где от того монастыря показано будет; которого подданного от магистрата в монастырь и вислать) Юрковам же потоком на гору к дороги, идучей з нижнего города Киева Чекердиным звозом\footnote{Из этого следует заключить, что Чекердин звоз того времени проходил несколько иначе, чем улица Чикирдин спуск 19 века, после ставшая Нижнеюрковской в крутой своей части.} по горам Шкавицы и Лысой к Белогородки, прямо до Кирилловского монастыря к свозу старинному Николаевскому\footnote{По именованию церкви святого Николая в Иорданском монастыре предположу, что Николаевский спуск и есть «старинная дорога на Лукьяновку» в овраге между Лысой горой и мысом «Кожевника».}, а оттуда поворотя на лево и оставя дорогу на лево, идучую к пробитому валу в верхний город Киев, пойти дорогою середнею и прешедши, оставя еще дорогу вправо, тоюж середнею дорогую просто до Белогородской, в верхней город Киев идучей дороги к урочищу Даниловому Кресту и земля по левую сторону магистратская, а по правую Кириловского монастыря; проходу ж скоту мескому чрез Лисую гору быть свободно и безпрепятственно, точию без шкоди пашен и около находящейся в леску монастырской пасеки; а что киевским мещанином Иваном Григоровичем отведением воды Юркова пруда от устроенного им хутора прежде бывшая поуз тот Юрков прудок дорога спорчена, оную дорогу самому ему, Григоровичу, исправить, так чтоб в проезде и в проходе и малейшей трудности быть не могло.
\end{quotation}

Согласно этому документу, по крайней мере со второй половины 18 века под Юрковым потоком и ставом уже подразумевались пруд и став между Щекавицей и Лысой горой.

Для меня очевидна связь между Оуривом (Хоривом) и Юрковицей, ибо сложно не заметить сходство имен Оурив и Юрий. Юрков ставок явно имеет отношение к Юрковице. Даже если «исконной» Юрковицей была бы Лысая гора, это мало меняет дело. Ведь на соседней к юго-востоку Щекавице обитал Щек. Общему граду Киеву больше негде находиться, кроме Лысой горы (если исконная Юрковица – отрог Щекавицы с кладбищем старообрядцев) либо (если Лысая гора это исконная Юрковица) следующего к северо-западу отрога с дачами «Кожевника».

%Однако шаткость принадлежности названия Юркова става в более далеком прошлом не стоит списывать со счетов.

%Ведь это может вбить гвоздь в предположение, что первоначальный град Киев был на Лысой горе. Но чтобы гвоздь был смертоносным, нужна истинность каждого звена цепочки рассуждения, а именно. Чтобы название Юрков став происходило от имени горы – Юрковицы (Хоревицы). Чтобы Юрков ставок был там, где указывал магистрат – грубо говоря, на автобазе около Кирилловской стоянке. Тогда исконной Юрковицей следует считать Лысую гору, а град Киев сместить на отроги с дачами «Кожевника» и Кирилловской стоянкой. 

%Если же правы «законники» Кирилловского монастыря, говоря, что Юрков ставок и поток находились между Щекавицей и Лысой горой, то правдоподобно, что Юрков став (хотя комиссия Кочубея не видела его следов, но показанный на позднейшем плане 1803 года и упомянутый в документах середины 18 века) назван по горе Юрковице, отрогу Щекавицы со старообрядческим кладбищем.

%Впрочем логично и предположение, что сначала был некий Юрков став, от которого стали именоваться ручей и гора, у которой находился став. Тут совершенно невозможно докопаться до корней. 

В земельных бумагах не указывалась гора Юрковица, а только Юрков став и поток. Если бы во  время документированных земельных споров существовало название Юрковица применительно к горе, то «законники» отсылались бы к нему, а не к исчезнувшему пруду.

А почему Юрков став так именовался? Чей став? Юрков. А гора чья? Юрковица, то есть Юркова гора. Название горы могло быть временно забыто, а после возродилось. Рассуждения о первичности названия напоминают мне докучные сказки. Наша песня хороша, начинай сначала. Как и многие краеведческие вопросы, этот – об Юрковом ставке, Юрковице, потоке – имеет свойство превращаться в сеть для разума, попав в которую начинаешь трепыхаться и запутываешься еще больше.
