\chapter{Толковник}
%\addcontentsline{toc}{chapter}{Толковник}

%\twocolumn 
\begin{multicols}{2}
\setlength{\columnsep}{1.5cm}
\setlength{\columnseprule}{0.2pt}


\textbf{А\\}
\mbox{ }\\

\textbf{Акведук} – канал или труба, по которому вода проведена в город или дом.

\textbf{Акт} – в Великом княжестве Литовском, актами называли книги присутственных мест. Документы вступали в силу, только будучи записанными в акты. Акты делились на «акты метрики Литовской и коронной» (ими заведовал канцлер), трибунальные, гродские, земские (в судах), мистские (у городов с самоуправлением, от «мисто» – «город»).

\textbf{Архимандрит} – начальник над монастырями епархии. Буквальный перевод с греческого – главный (архи) овчарни (мандрит). «Мандра» значит овчарня, стойло, так именовались монастыри в древности.

\mbox{ }\\
\textbf{Б\\}
\mbox{ }\\

\noindent\textbf{Байрак} – лиственный лес.

\textbf{Балка} – большой овраг, с более отлогими берегами, чем у яра. Впрочем, это правило не соблюдается – один берег Совской балки очень крут. 

\textbf{Белоглазка} – то же, что лёсс. 

\textbf{Берковец} – большая мера веса, 1 берковец = 10 пудов. Мера применялась обычно в оптовой торговле. Впервые документально упоминается в 12 веке в уставной грамоте князя Всеволода Гавриила Мстиславича новгородскому купечеству. Употреблялся, как берковец или берк, еще в 19 веке. Например, по сахарной свекле, на Киевщине, с десятины земли в среднем был урожай 100-120 берковцев у помещиков, и не более 60 берковцев у крестьян. 

%Киевский Берковец или Берковцы могут восходить к этой мере веса, либо, учитывая наличие там в прошлом курганов, языческого могильника – к слову «борок», означащего давнее кладбище, священную землю предков.

\textbf{Бискуп} – епископ.

\textbf{Бор} – лес сосновый.

\textbf{Бор} – побор, налог.

%\textbf{Борок} – давнее кладбище, могильник.

\textbf{Борт} – дерево с пустым стволом, где водятся дикие пчелы. Отсюда – бортничи, собиратели дикого меда.

\textbf{Брама} – крепостные или городские ворота.

\textbf{Броварня} – пивоварня.

\textbf{Буда} – шалаш, плетеная хижина.

\textbf{Буерак} – овраг.

\textbf{Бус} – аист.

\mbox{ }\\
\textbf{В\\}
\mbox{ }\\

\textbf{Вальковая постройка} - из глины, смешанной с соломой.

\textbf{Ведлуг, водлуг} – в силу чего-либо, юридического решения, права и тому подобного.

\textbf{Вербная неделя} – 6-ая неделе Великого поста.

\textbf{Весь} – обычно значит «селение», редко местечко или страну.

\textbf{Влока, волока, уволока} – мера площади земельного надела, 19 десятин. Некоторые пользующиеся землей (в зависимости от профессии или сословия) должны были платить поземельный налог, устанавливаемый из расчета на 1 волоку, смотря по качеству земли. Например, в середине 16 века за хорошую землю – 21 грош, затем несколько степеней ухудшения качества и снижения налога, до болотистой земли – 6 грошей, да овса с каждой волоки – с хорошей по 2 бочки, с худой по 1-й, а если можно платить деньгами, то по 5 грошей за бочку, а за отвоз каждой бочки еще 5 грошей. На волоку также накладывался налог – по возу сена либо по три гроша. Порой даже одной волоки для семейства крестьян было много, поэтому волоку разделяли между несколькими дворами – это называлось товариществом.

%\textbf{Внука, внук} – искаженное греческое, буквально значит «отродие», «выродок».

\textbf{Винница} – винокурня.

\textbf{Выспа, высеп, высип} – кроме болезни «оспы», еще и небольшой песчаный остров. Подобно слову «насыпь».

\textbf{Войт} – председатель магистрата; глава купцов и мещан, пожизненно избранный из их числа. Также – председатель городского суда. В селе – назначенный помещиком урядник. 

\mbox{ }\\
\textbf{Г\\}
\mbox{ }\\

\textbf{Гирло} – устье реки; водоворот.

\textbf{Глинище} – место, где добывают глину.

\textbf{Гребля} – плотина, гать.

\textbf{Громницы} – народное название праздника Сретения Господня.

\textbf{Грунт} – усадьба или земельный надел, место. 

\textbf{Гульбище} – трактир.

\textbf{Гута} – стекольный завод либо печь в стекляном заводе.

\mbox{ }\\
\textbf{Д\\}
\mbox{ }\\

\textbf{Дата} – первоначально, от слова «дача», «дано», то бишь когда был дан документ, грамота. 

\textbf{Десный} – правый, по правую сторону. Те, кто назвали реку Десну, наверное плыли снизу, против течения Днепра, и для них Десна была справа.

\textbf{Дуб} – длинная (около 10 метров) тяжелая лодка на 6 гребцов, с палубным настилом, куда вставлялась мачта для паруса.

\mbox{ }\\
\textbf{Е\\}
\mbox{ }\\

\textbf{Еханье} – проведение границы, размежевание.

\mbox{ }\\
\textbf{Ж\\}
\mbox{ }\\

\textbf{Жикгимунт} – он же Сигизмунд.

\textbf{Жолнер} – воин, рыцарь.

\mbox{ }\\
\textbf{З\\}
\mbox{ }\\

\textbf{Забора} – подводная груда камней у берега реки.

\textbf{Задница} – наследство.

\textbf{Задушный человек} – раб, освобожденный владельцем в пользу спасения своей души.

\textbf{Займище} – место, занятое садом, огородом, сенокосом.

\textbf{Законник} – юрист.

\textbf{Збыток} – преступление.

\textbf{Зводь} – очная ставка.

\textbf{Зубоеж} – укушенье.

\mbox{ }\\
\textbf{К\\}
\mbox{ }\\

\textbf{Капец} – знак на полевой меже.

\textbf{Катедра} – католический кафедральный храм, т.е. кафедральный костел.

\textbf{Копец} – земляной горб, насыпаемый иногда на границе чего-либо.

\textbf{Кгрунт} – см. «грунт».

\textbf{Клечальная неделя} – Зеленое воскресенье (неделя в значении воскресного дня), Духов день, Троицын день.

\textbf{Кляштор} – католический монастырь.

\textbf{Колокола Лаврской колокольни} – на третьем ярусе, медные, обошлись Лавре в 30415 рублей, 43 копейки (за работу над самой колокольней архитектору и рабочим было уплачено 58584 рубля, 96 копеек). Названия и вес колоколов: Успенский (1000 пудов), Орел (500 пудов), Зосимон (400 пудов 13 фунтов, был разбит и в 1825 перелит в 493 пуда, 11 фунтов), остальные поменьше: Полиелейн, Балык, Благовест, Ранний, Буденный, Скликун, Часовой. Позже к ним с башни добавили еще одного Часового. Часы же со своими колоколами установлены на втором и четвертом ярусах.

%\textbf{Коморник} – урядник, замещающий подкомория в подкоморских судах.

\textbf{Кон} – публичное место возле суда. Отсюда – на кону.

\textbf{Конвент} – католический монастырь.

\textbf{Копыл} – особый топор, которым выдалбливали из дерева посуду, ульи, корыта.

\textbf{Корчага} – глиняный сосуд с двумя ручками, отнсительно часто упоминается в источниках как емкость для хранения вина и «деревяного масла», а также зерна. Найдены корчаги объемом до шести ведер.

\textbf{Корогва} – знамя, знак. Была корогва у магистрата («золотая корогва»), у отдельных ремесленных цехов. Помимо изображений, на обоих сторонах корогвы обычно были надписи. Цеховые корогвы просуществовали по вторую половину 19 века, уже под названием «хоругв».

\textbf{Косара} – сенокос.

\textbf{Костел} – католический храм.

\textbf{Котора} – ссора.

\textbf{Красный лес} – сосновый, бор.

\textbf{Крестьянин} – христианин.

\textbf{Куга} – водоросли.

\textbf{Кудрявец} – растение пижма обыкновенная. Потому наверное и местность так называется, что там кудрявец рос, как Репяхов яр именован по репяхам.

\textbf{Кухмистер} – содержатель трактира. Другое значение – повар.

\textbf{Кучугуры (выдуй, выдма, вымет)} – сыпучие, оголенные пески на полях, которые выдуваются ветром на окрестности, засыпая полезную землю, дороги и тому подобное. 

\mbox{ }\\
\textbf{Л\\}
\mbox{ }\\

\textbf{Лайба} – длинная грузовая лодка, с парусом, обычно использовалась крестьянами для доставки продовольствия в город. Привязывали лайбу к бечеве и тянули против течения, идя по берегу. По течению же плыла сама. Лайба имела среднюю грузоподъемность до 2000 пудов.

\textbf{Лежанье} – пребывание.

\textbf{Лёс} – жребий. При склонении – лесы. «Лесами есмы метали».

\textbf{Лёсс} – немецкое слово, у нас раньше говорили – белоглазка. Это пористая известковая глина, суглинок. Благодаря порам, вода в нем не застаивается и проходит в толщу и выходит уже в нижних слоях горы источником. Известковые трубочки белого цвета в лёссе называют журавчиками. Деревья растут на лёссе, когда могут корнями достать водоносного горизонта. На лёссе хорошо растут травы, злаки. Над лёссом, на плато, образуется слой чернозема. Лёсс состоит из глинистых частиц, зерен кварца и углекислой извести.

\textbf{Листовная служба} – род занятия крестьян, которые выполняли для помещика, вместо обычных повинностей, перевозку писем и документов. Такие крестьяне назывались боярами путными. Путным боярам давали в пользование половину или 1 волоку земли.

\textbf{Лодейная мельница, лодейный млин} – водяная мельница, не требующая плотины, стоит на большой реке или при сильном течении.

\textbf{Лука} – луг в пойме.

\mbox{ }\\
\textbf{М\\}
\mbox{ }\\

\textbf{Майдан} - кроме прочего, смологонный завод.

\textbf{Малженство} – супружество.

%\textbf{Мезолит} – средний каменный век, 15-5 тысяч лет до нашей эры. Месос – средний, литос – камень. В мезолит появляются лук и стрелы, ледник вроде бы ушел.

\textbf{Мембрана} – старый документ.

\textbf{Менованый} – именуемый.

\textbf{Меской} – городской, от место, мисто – город.

\textbf{Мещанин, месчанин} – городской житель.

\textbf{Мех} – мешок.

\textbf{Мито, мыто, мыть} – денежный сбор (в казну) на мостах, дорогах, переправах.

\textbf{Млин, млын, млинок} – водяная мельница. Ветряная называется ветряком.

\textbf{Могилки} – кладбище.

\textbf{Мышеловки} – орехи, собранные белками или мышами.

\mbox{ }\\
\textbf{Н\\}
\mbox{ }\\

\textbf{Нарочитый} – именитый. Нарочитые люди.

\textbf{Невеглас} – невежда.

%\textbf{Неолит} – новый каменный век. Появление глиняной посуды. Появление топора – теперь человек способен рубить деревья и использовать бревна. Зернотёрки.

\textbf{Непра, Непр} – Днепр. Без буквы «д» его называли что за порогами, что в Сибири.

\mbox{ }\\
\textbf{О\\}
\mbox{ }\\

\textbf{Обрядить} – осудить.

\textbf{Обыклый} – обыкновенный.

\textbf{Обыклость} – обычай, обыкновение, нечто установленное.

\textbf{Осадники} – бояре осадные, прежние бояре, ставшие пахотными крестьянами. Налог с волоки платили больший, нежели тяглые крестьяне, однако не отбывавшие других повинностей перед помещиком.

\textbf{Оскорд} – долото.

\textbf{Остров} – на Киевщине и Черниговщине так называли наибольшие острова (сотни десятин), поросшие лесом. Меньшие же – смотри выспу, отоку и так далее.

\textbf{Открытый лист} – разрешение на раскопки.

\textbf{Отока} – большой песчаный остров, с обеих сторон омываемый широкими судоходными рукавами.

\textbf{Отказ} – ответ, показание (в актах).

\textbf{Отрок} – по 12 лет, «выросток».

\textbf{Отрочище} – 9 лет, до 9 лет считался младенцем.

\mbox{ }\\
\textbf{П\\}
\mbox{ }\\

\textbf{Палатинат} – воеводство, административная единица Польши и Великого Княжества Литовского, иначе говоря область. Украинские земли делились на 4 палатината – Киевский, Брацлавский, Подольский, Волынский. Палатинаты, в свою очередь, делились на поветы, или уезды. Так, Киевский палатинат состоял из поветов Киевского, Житомирского и Овручского.

\textbf{Палатин} – воевода.

%\textbf{Палеолит} – от греческих слов «палео» (древний) и «литос» (камень).

\textbf{Паперня} – фабрика по производству бумаги.

\textbf{Перекат, перевал} – место на реке, в котором глубина меньше сажени.

\textbf{Перевес} – застава.

\textbf{Перевесище} – кроме прочего: былая застава, остатки заставы.

\textbf{Пискун} – песчаная почва.

\textbf{Плес} – долина, плоское место. Сходное – пляж, несло то же значение.

\textbf{Плесо, плес} – колено реки меж двух изгибов или островов. Плесом, или бьефом, называют также пространство между двумя шлюзами.

\textbf{Плинфа} – плоский, широкий обоженный кирпич, применение его датируется 10-13 веками.

\textbf{Подкоморый} – урядник суда, разбиравшего границы межевания.

\textbf{Повет} – уезд палатина; волость, часть уезда с неким основным городом.

\textbf{Подкоморый суд} – межевой суд.

\textbf{Подкоморий} – судья в земельных спорах.

\textbf{Подворок, подварок} – дача, хутор. 

\textbf{Поколодная} – налог с продаваемого на рынке зерна, по колоде (мера).

\textbf{Полк} – как административная единица, область гетмании. Во время Богдана Хмельницкого было 16 полков.

\textbf{Повод} – истец.

\textbf{Позваный} – ответчик.

\textbf{Посконь} – конопля.

\textbf{Поток} – ручей.

\textbf{Поуз} – «повз», мимо.

\textbf{Подол} – «по долу», расположенный в долине.

\textbf{Посполюд} – народ.

\textbf{Посполитый} – народный.

\textbf{Посполитак, посполитка} – крестьянин, крестьянка.

\textbf{Потвора} – род колдовства.

\textbf{Преварка} – старинное название Приорки.

\textbf{Припять} – привязь.

\textbf{Провалье} – крутой, узкий с вертикальными стенками лёссовый овраг. Из него образуется яр.

\textbf{Просереда} – большой песчаный остров, заросший лозой.

\textbf{Пятница} – иконка (обычно святой Параскевы) на столбике, под резной крышей. Ставились обычно на перекрестках, вероятно заменив собой языческие священные столбы с костями, мимо коих проезжали непременно что-нибудь пожертвовав. В старину существовали странные, отдельные от всего дороги, украшенные пятницами. Михаил Николаевич Макаров рассказывает в первом томе «Русских преданий» о том, что в Рязанском княжении была дорога Комарина, шедшая через поля от Рязани, «не касаясь ни деревень, ни сел», и теряясь затем в Радуницких борах. Следы ее же наблюдались в лесах Московских, Владимирских. Макаров считал такие «таинственные пути» священными. Пятницы также служили местами переговоров и гаданий.

\mbox{ }\\
\textbf{Р\\}
\mbox{ }\\

\textbf{Радец} – советник.

\textbf{Рейтар} – кавалерист, всадник.

\textbf{Рень, ринь} – пески, отмель.

\textbf{Речь} – дело (судебное).

\textbf{Римарь, лимарь} – кожевник.

\textbf{Ровчак} – желобок.

\textbf{Рог} – угол. Например – рог палисада.

\textbf{Рогатка} – место, где дорога разветвляется в виде буквы Y. 

\textbf{Рота} – присяга, клятва, божба.

\textbf{Рощенье} – роща.

\textbf{Рудня} – литейный завод. От сырья – руды. А руда потому, что рыжая, рудая. Другое значение слова «руда» – кровь.

\mbox{ }\\
\textbf{С\\}
\mbox{ }\\

\textbf{Сага} – (из болота в сагу) – возможно, протока, ставшая заливом?

\textbf{Свобода} – слобода.

\textbf{Сволочь} – раздетый. «Мертвецы сволочать» – раздевают, грабят мертвецов.

\textbf{Селитьба} – усадьба.

\textbf{Синяя глина} – синяя, или спондилувая глина. Ныне ею лечат от всех болезней, а раньше делали знаменитый киевский светло-желтый кирпич, который без штукатурки выдерживал буйства природы. Из него были сложены Николаевский цепной мост, Печерская крепость, Арсенал, здание Присутственных мест. Пласт синей глины обычно – нижний водоупорный слой склона, поэтому на верхнем стыке синей глины часты выходы родников. Ключ Бусловский у саперного лагеря на Бусовой горе относится именно к такому виду источников. 

\textbf{Скифы} – культурные слои, относимые учеными к скифским, в Киеве обнаружены кроме прочего возле Лаврской колокольни. Каменные бабы, сочтенные археологами скифскими, найдены на месте Печерского ипподрома.

\textbf{Скора, скира} – шкура. Отсюда – скорняк.

\textbf{Слуп} – столп, столб.

\textbf{Сказанье} – решение. «Вчинити сказанье».

\textbf{Служба} – земельный надел около 200 десятин, полученный от князя.

\textbf{Смуга} – речка.

\textbf{Содрати} – разорвать, уничтожить. «Сорвати листы» – разорвать документы.

\textbf{Сознанье} – заявление устное.

\textbf{Спиж} – медь.

\textbf{Став} – деревянное сооружение, плотина для устройства ставка.

\textbf{Ставок} – пруд.

\textbf{Старик, старуха} – старое русло реки, ставшее заливом. Было два протока, между ними остров. Один проток становится стариком. Затем старик превращается в длинное озеро, что соединяется с рекой только в половодье.

\textbf{Стельмах} – колёсник, тележник.

\textbf{Стороник} – пришлый, со стороны.

\textbf{Стрый} – брат отца.

\textbf{Сыновец} – племянник.

\mbox{ }\\
\textbf{Т\\}
\mbox{ }\\

\textbf{Тамга} – таможенный сбор.

\textbf{Тиун} – судья.

\textbf{Тоня} – место для ловли рыбы; сеть для ловли рыбы.

\textbf{Трипольская культура} – ее следы найдены в Киеве на Кирилловских высотах (усадьба Светославского и окрестности, а также улица Верхне-Юрковская), Клове, улице Флоровской, Лукьяновском кладбище, в усадьбе Дома офицеров, на Бессарабке, на улице Прорезной, в окрестностях Исторического музея.

\textbf{Тяглые крестьяне} – поземельный оброк, бирчее и писчее платили деньгами. Что не освобождало их от даней «натурой» и барщины, работы на помещика.

\textbf{Тяжать о заднице} – вести тяжбу о наследстве.

\mbox{ }\\
\textbf{У\\}
\mbox{ }\\

\textbf{Урядник} – чиновник.

\mbox{ }\\
\textbf{Ф\\}
\mbox{ }\\

\textbf{Фарватер} – стрежень, или главный ход в реке, полоса, по которой плывут корабль.

\mbox{ }\\
\textbf{Х\\}
\mbox{ }\\

\textbf{Хутор} – посёлок в один двор, одну семью. Некоторые хутора разрослись, впрочем, до больших селений, сохранив в названии слово «хутор». Иногда же несколько разбросанных по местности однодворных хуторов считали одним селением – такие-то хутора.

\mbox{ }\\
\textbf{Ц\\}
\mbox{ }\\

\textbf{Цегельня} – кирпичный завод.

\textbf{Цына} – олово.

\mbox{ }\\
\textbf{Ч\\}
\mbox{ }\\

\textbf{Чагарник} – лес из молодого дуба и кустов.

\textbf{Череда} – стадо из коров разных хозяев.

\textbf{Через} – в юридическом смысле в актах – «вопреки». «Черес право», «черес привелей господарский».

\textbf{Чернечий, чернечь} – монашеский.

\textbf{Чернолесье} – лиственный лес.

\textbf{Чинш} – налог, оброк.

\mbox{ }\\
\textbf{Ш\\}
\mbox{ }\\

\textbf{Шелюг} – лоза.

\textbf{Шия} – смертная казнь. Например, «господарь заклад заложил под шиями и под именьи».

\textbf{Шляхтич} – дворянин.

\textbf{Шопа} – сарай.

\mbox{ }\\
\textbf{Я\\}
\mbox{ }\\

\textbf{Яр, яруга} – большой составной овраг с берегами из лёсса. Присоединенные к нему меньшие яры – приярки, переярки.

\end{multicols}
%\onecolumn